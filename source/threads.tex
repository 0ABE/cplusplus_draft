%!TEX root = std.tex
\rSec0[thread]{Thread support library}

\rSec1[thread.general]{General}

\pnum
The following subclauses describe components to create and manage
threads\iref{intro.multithread}, perform mutual exclusion, and communicate conditions
and values
between threads, as summarized in \tref{thread.summary}.

\begin{libsumtab}{Thread support library summary}{thread.summary}
\ref{thread.req}      & Requirements          &                               \\ \rowsep
\ref{thread.stoptoken}& Stop tokens           & \tcode{<stop_token>}          \\ \rowsep
\ref{thread.threads}  & Threads               & \tcode{<thread>}              \\ \rowsep
\ref{thread.mutex}    & Mutual exclusion      &
  \tcode{<mutex>}, \tcode{<shared_mutex>} \\ \rowsep
\ref{thread.condition}& Condition variables   & \tcode{<condition_variable>}  \\ \rowsep
\ref{thread.sema}     & Semaphores & \tcode{<semaphore>} \\ \rowsep
\ref{thread.coord}    & Coordination types & \tcode{<latch>} \tcode{<barrier>} \\ \rowsep
\ref{futures}         & Futures               & \tcode{<future>}              \\
\end{libsumtab}

\rSec1[thread.req]{Requirements}

\rSec2[thread.req.paramname]{Template parameter names}

\pnum
Throughout this Clause, the names of template parameters are used to express type
requirements.
If a template parameter is named \tcode{Predicate}, \tcode{operator()} applied to
the template argument shall return a value that is convertible to \tcode{bool}.
If a template parameter is named \tcode{Clock},
the corresponding template argument shall be a type \tcode{C}
for which \tcode{is_clock_v<C>} is \tcode{true};
otherwise the program is ill-formed.

\rSec2[thread.req.exception]{Exceptions}

\pnum
Some functions described in this Clause are specified to throw exceptions of type
\tcode{system_error}\iref{syserr.syserr}. Such exceptions shall be thrown if
any of the function's error conditions is detected or
a call to
an operating system or other underlying API results in an error that prevents the
library function from
meeting its specifications. Failure to allocate storage shall be reported as described
in~\ref{res.on.exception.handling}.

\begin{example}
Consider a function in this clause that is specified to throw exceptions of type
\tcode{system_error} and specifies error conditions that include
\tcode{operation_not_permitted} for a thread that does not have the privilege to
perform the operation. Assume that, during the execution of this function, an \tcode{errno}
of \tcode{EPERM} is reported by a POSIX API call used by the implementation. Since POSIX
specifies an \tcode{errno} of \tcode{EPERM} when ``the caller does not have the privilege
to perform the operation'', the implementation maps \tcode{EPERM} to an
\tcode{error_condition} of \tcode{operation_not_permitted}\iref{syserr} and an exception
of type \tcode{system_error} is thrown.
\end{example}

\pnum
The \tcode{error_code} reported by such an exception's \tcode{code()} member function
shall compare equal to one of the conditions specified in the function's error condition
element.

\rSec2[thread.req.native]{Native handles}

\pnum
Several classes described in this Clause have members \tcode{native_handle_type} and
\tcode{native_handle}. The presence of these members and their semantics is
\impldef{presence and meaning of \tcode{native_handle_type} and \tcode{native_handle}}.
\begin{note}
These members allow implementations to provide access
to implementation details. Their names are specified to facilitate portable compile-time
detection. Actual use of these members is inherently non-portable.
\end{note}

\rSec2[thread.req.timing]{Timing specifications}

\pnum
Several functions described in this Clause take an argument to specify a timeout. These
timeouts are specified as either a \tcode{duration} or a \tcode{time_point} type as
specified in~\ref{time}.

\pnum
Implementations necessarily have some delay in returning from a timeout. Any overhead in
interrupt response, function return, and scheduling induces a ``quality of implementation''
delay, expressed as duration $D_i$. Ideally, this delay would be zero. Further, any contention for
processor and memory resources induces a ``quality of management'' delay, expressed as duration
$D_m$. The delay durations may vary from timeout to timeout, but in all cases shorter is better.

\pnum
The functions whose names end in \tcode{_for} take an argument that
specifies a duration. These functions produce relative timeouts. Implementations
should use a steady clock to measure time for these functions.\footnote{All
implementations for which standard time units are meaningful must necessarily
have a steady clock within their hardware implementation.} Given a duration
argument $D_t$, the real-time duration of the timeout is $D_t + D_i + D_m$.

\pnum
The functions whose names end in \tcode{_until} take an argument that specifies a time
point. These functions produce absolute timeouts. Implementations should use the clock
specified in the time point to measure time for these functions. Given a clock time point
argument $C_t$, the clock time point of the return from timeout should be $C_t + D_i + D_m$
when the clock is not adjusted during the timeout. If the clock is adjusted to the time $C_a$
during the timeout, the behavior should be as follows:

\begin{itemize}
\item
if $C_a > C_t$, the waiting function should wake as soon as possible, i.e., $C_a + D_i + D_m$,
since the timeout is already satisfied.
\begin{note}
This specification may result in the total
duration of the wait decreasing when measured against a steady clock.
\end{note}

\item
if $C_a \leq C_t$, the waiting function should not time out until \tcode{Clock::now()} returns a
time $C_n \geq C_t$, i.e., waking at $C_t + D_i + D_m$.
\begin{note}
When the clock is adjusted
backwards, this specification may result in the total duration of the wait increasing when
measured against a steady clock. When the clock is adjusted forwards, this specification may
result in the total duration of the wait decreasing when measured against a steady clock.
\end{note}
\end{itemize}

An implementation shall return from such a timeout at any point from the time specified above to
the time it would return from a steady-clock relative timeout on the difference between $C_t$
and the time point of the call to the \tcode{_until} function.
\begin{note}
Implementations
should decrease the duration of the wait when the clock is adjusted forwards.
\end{note}

\pnum
\begin{note}
If the clock is not synchronized with a steady clock, e.g., a CPU time clock, these
timeouts might not provide useful functionality.
\end{note}

\pnum
The resolution of timing provided by an implementation depends on both operating system
and hardware. The finest resolution provided by an implementation is called the
\term{native resolution}.

\pnum
Implementation-provided clocks that are used for these functions shall meet the
\oldconcept{TrivialClock} requirements\iref{time.clock.req}.

\pnum
A function that takes an argument which specifies a timeout will throw if,
during its execution, a clock, time point, or time duration throws an exception.
Such exceptions are referred to as \term{timeout-related exceptions}.
\begin{note}
Instantiations of clock, time point and duration types supplied by
the implementation as specified in~\ref{time.clock} do not throw exceptions.
\end{note}

\rSec2[thread.req.lockable]{Requirements for \oldconcept{Lockable} types}

\rSec3[thread.req.lockable.general]{In general}

\pnum
An \defn{execution agent} is an entity such as a thread that may perform work in parallel with
other execution agents.
\begin{note}
Implementations or users may introduce other kinds of
agents such as processes or thread-pool tasks.
\end{note}
The calling agent is determined by
context, e.g., the calling thread that contains the call, and so on.

\pnum
\begin{note}
Some lockable objects are ``agent oblivious'' in that they work for any
execution agent model because they do not determine or store the agent's ID (e.g., an
ordinary spin lock).
\end{note}

\pnum
The standard library templates \tcode{unique_lock}\iref{thread.lock.unique},
\tcode{shared_lock}\iref{thread.lock.shared},
\tcode{scoped_lock}\iref{thread.lock.scoped},
\tcode{lock_guard}\iref{thread.lock.guard}, \tcode{lock},
\tcode{try_lock}\iref{thread.lock.algorithm}, and
\tcode{condition_variable_any}\iref{thread.condition.condvarany} all operate on user-supplied
lockable objects. The \oldconcept{BasicLockable} requirements, the \oldconcept{Lockable} requirements,
and the \oldconcept{TimedLockable} requirements list the requirements imposed by these library types
in order to acquire or release ownership of a \tcode{lock} by a given execution agent.
\begin{note}
The nature of any lock ownership and any synchronization it may entail are not part
of these requirements.
\end{note}

\rSec3[thread.req.lockable.basic]{\oldconcept{BasicLockable} requirements}

\pnum
A type \tcode{L} meets the \oldconcept{BasicLockable} requirements if the following expressions are
well-formed and have the specified semantics (\tcode{m} denotes a value of type \tcode{L}).

\begin{itemdecl}
m.lock()
\end{itemdecl}

\begin{itemdescr}
\pnum
\effects
Blocks until a lock can be acquired for the current execution agent. If an exception
is thrown then a lock shall not have been acquired for the current execution agent.
\end{itemdescr}

\begin{itemdecl}
m.unlock()
\end{itemdecl}

\begin{itemdescr}
\pnum
\requires The current execution agent shall hold a lock on \tcode{m}.

\pnum
\effects
Releases a lock on \tcode{m} held by the current execution agent.

\pnum
\throws
Nothing.
\end{itemdescr}

\rSec3[thread.req.lockable.req]{\oldconcept{Lockable} requirements}

\pnum
A type \tcode{L} meets the \oldconcept{Lockable} requirements if it meets the \oldconcept{BasicLockable}
requirements and the following expressions are well-formed and have the specified semantics
(\tcode{m} denotes a value of type \tcode{L}).

\begin{itemdecl}
m.try_lock()
\end{itemdecl}

\begin{itemdescr}
\pnum
\effects
Attempts to acquire a lock for the current execution agent without blocking. If an
exception is thrown then a lock shall not have been acquired for the current execution agent.

\pnum
\returntype \tcode{bool}.

\pnum
\returns
\tcode{true} if the lock was acquired, \tcode{false} otherwise.
\end{itemdescr}

\rSec3[thread.req.lockable.timed]{\oldconcept{TimedLockable} requirements}

\pnum
A type \tcode{L} meets the \oldconcept{TimedLockable} requirements if it meets the \oldconcept{Lockable}
requirements and the following expressions are well-formed and have the specified semantics
(\tcode{m} denotes a value of type \tcode{L}, \tcode{rel_time} denotes a value of an
instantiation of \tcode{duration}\iref{time.duration}, and \tcode{abs_time} denotes a value
of an instantiation of \tcode{time_point}\iref{time.point}).

\begin{itemdecl}
m.try_lock_for(rel_time)
\end{itemdecl}

\begin{itemdescr}
\pnum
\effects
Attempts to acquire a lock for the current execution agent within the relative
timeout\iref{thread.req.timing} specified by \tcode{rel_time}. The function shall not return
within the timeout specified by \tcode{rel_time} unless it has obtained a lock on \tcode{m}
for the current execution agent. If an exception is thrown then a lock shall not have been
acquired for the current execution agent.

\pnum
\returntype \tcode{bool}.

\pnum
\returns
\tcode{true} if the lock was acquired, \tcode{false} otherwise.
\end{itemdescr}

\begin{itemdecl}
m.try_lock_until(abs_time)
\end{itemdecl}

\begin{itemdescr}
\pnum
\effects
Attempts to acquire a lock for the current execution agent before the absolute
timeout\iref{thread.req.timing} specified by \tcode{abs_time}. The function shall not return
before the timeout specified by \tcode{abs_time} unless it has obtained a lock on \tcode{m} for
the current execution agent. If an exception is thrown then a lock shall not have been acquired
for the current execution agent.

\pnum
\returntype \tcode{bool}.

\pnum
\returns
\tcode{true} if the lock was acquired, \tcode{false} otherwise.
\end{itemdescr}

\rSec1[thread.stoptoken]{Stop tokens}

\rSec2[thread.stoptoken.intro]{Introduction}

\pnum
This clause describes components that can be used
to asynchonously request that an operation stops execution in a timely manner,
typically because the result is no longer required.
Such a request is called a \defn{stop request}.

\pnum
\tcode{stop_source}, \tcode{stop_token}, and \tcode{stop_callback}
implement semantics of shared ownership of a \defn{stop state}.
Any \tcode{stop_source}, \tcode{stop_token}, or \tcode{stop_callback}
that shares ownership of the same stop state is an \defn{associated}
\tcode{stop_source}, \tcode{stop_token}, or \tcode{stop_callback}, respectively.
The last remaining owner of the stop state automatically
releases the resources associated with the stop state.

\pnum
A \tcode{stop_token} can be passed to an operation which can either
\begin{itemize}
 \item actively poll the token to check if there has been a stop request, or
 \item register a callback using the \tcode{stop_callback} class template which
        will be called in the event that a stop request is made.
\end{itemize}
A stop request made via a \tcode{stop_source} will be visible to all
associated \tcode{stop_token} and \tcode{stop_source} objects.
Once a stop request has been made it cannot be withdrawn
(a subsequent stop request has no effect).

\pnum
Callbacks registered via a \tcode{stop_callback} object are called when
a stop request is first made by any associated \tcode{stop_source} object.

\pnum
Calls to the functions \tcode{request_stop}, \tcode{stop_requested},
and \tcode{stop_possible}
do not introduce data races.
A call to \tcode{request_stop} that returns \tcode{true}
synchronizes with a call to \tcode{stop_requested}
on an associated \tcode{stop_token} or \tcode{stop_source} object
that returns \tcode{true}.
Registration of a callback synchronizes with the invocation of that callback.


\rSec2[thread.stoptoken.syn]{Header \tcode{<stop_token>} synopsis}

\indexheader{stop_token}%
\begin{codeblock}
namespace std {
  // \ref{stoptoken}, class \tcode{stop_token}
  class stop_token;

  // \ref{stopsource}, class \tcode{stop_source}
  class stop_source;

  // no-shared-stop-state indicator
  struct nostopstate_t {
    explicit nostopstate_t() = default;
  };
  inline constexpr nostopstate_t nostopstate{};

  // \ref{stopcallback}, class \tcode{stop_callback}
  template<class Callback>
  class stop_callback;
}
\end{codeblock}


\indexlibraryglobal{stop_token}%
\rSec2[stoptoken]{Class \tcode{stop_token}}

\pnum
\indexlibraryglobal{stop_token}%
The class \tcode{stop_token} provides an interface for querying whether
a stop request has been made (\tcode{stop_requested})
or can ever be made (\tcode{stop_possible})
using an associated \tcode{stop_source} object (\ref{stopsource}).
A \tcode{stop_token} can also be passed to a
\tcode{stop_callback}\iref{stopcallback} constructor
to register a callback to be called when a stop request has been made
from an associated \tcode{stop_source}.

\begin{codeblock}
namespace std {
  class stop_token {
  public:
    // \ref{stoptoken.cons}, constructors, copy, and assignment
    stop_token() noexcept;

    stop_token(const stop_token&) noexcept;
    stop_token(stop_token&&) noexcept;
    stop_token& operator=(const stop_token&) noexcept;
    stop_token& operator=(stop_token&&) noexcept;
    ~stop_token();
    void swap(stop_token&) noexcept;

    // \ref{stoptoken.mem}, stop handling
    [[nodiscard]] bool stop_requested() const noexcept;
    [[nodiscard]] bool stop_possible() const noexcept;

    [[nodiscard]] friend bool operator==(const stop_token& lhs, const stop_token& rhs) noexcept;
    [[nodiscard]] friend bool operator!=(const stop_token& lhs, const stop_token& rhs) noexcept;
    friend void swap(stop_token& lhs, stop_token& rhs) noexcept;
  };
}
\end{codeblock}


\rSec3[stoptoken.cons]{Constructors, copy, and assignment}

\indexlibraryctor{stop_token}%
\begin{itemdecl}
stop_token() noexcept;
\end{itemdecl}

\begin{itemdescr}
\pnum
\ensures
\tcode{stop_possible()} is \tcode{false} and
\tcode{stop_requested()} is \tcode{false}.
\begin{note}
Because the created \tcode{stop_token} object can never receive a stop request,
no resources are allocated for a stop state.
\end{note}
\end{itemdescr}

\indexlibraryctor{stop_token}%
\begin{itemdecl}
stop_token(const stop_token& rhs) noexcept;
\end{itemdecl}

\begin{itemdescr}
\pnum
\ensures
\tcode{*this == rhs} is \tcode{true}.
\begin{note}
\tcode{*this} and \tcode{rhs} share the ownership of the same stop state,
if any.
\end{note}
\end{itemdescr}

\indexlibraryctor{stop_token}%
\begin{itemdecl}
stop_token(stop_token&& rhs) noexcept;
\end{itemdecl}
\begin{itemdescr}
\pnum
\ensures
\tcode{*this} contains the value of \tcode{rhs}
prior to the start of construction
and \tcode{rhs.stop_possible()} is \tcode{false}.
\end{itemdescr}

\indexlibrarydtor{stop_token}%
\begin{itemdecl}
~stop_token();
\end{itemdecl}

\begin{itemdescr}
\pnum
\effects
Releases ownership of the stop state, if any.
\end{itemdescr}

\indexlibrarymember{operator=}{stop_token}%
\begin{itemdecl}
stop_token& operator=(const stop_token& rhs) noexcept;
\end{itemdecl}

\begin{itemdescr}
\pnum
\effects
Equivalent to: \tcode{stop_token(rhs).swap(*this)}.

\pnum
\returns
\tcode{*this}.
\end{itemdescr}

\indexlibrarymember{operator=}{stop_token}%
\begin{itemdecl}
stop_token& operator=(stop_token&& rhs) noexcept;
\end{itemdecl}

\begin{itemdescr}
\pnum
\effects
Equivalent to: \tcode{stop_token(std::move(rhs)).swap(*this)}.

\pnum
\returns
\tcode{*this}.
\end{itemdescr}

\indexlibrarymember{swap}{stop_token}%
\begin{itemdecl}
void swap(stop_token& rhs) noexcept;
\end{itemdecl}

\begin{itemdescr}
\pnum
\effects
Exchanges the values of \tcode{*this} and \tcode{rhs}.
\end{itemdescr}

\rSec3[stoptoken.mem]{Members}

\indexlibrarymember{stop_requested}{stop_token}%
\begin{itemdecl}
[[nodiscard]] bool stop_requested() const noexcept;
\end{itemdecl}

\begin{itemdescr}
\pnum
\returns
\tcode{true} if \tcode{*this} has ownership of a stop state
that has received a stop request;
otherwise, \tcode{false}.
\end{itemdescr}

\indexlibrarymember{stop_possible}{stop_token}%
\begin{itemdecl}
[[nodiscard]] bool stop_possible() const noexcept;
\end{itemdecl}

\begin{itemdescr}
\pnum
\returns
\tcode{false} if:
\begin{itemize}
\item \tcode{*this} does not have ownership of a stop state, or
\item a stop request was not made
      and there are no associated \tcode{stop_source} objects;
\end{itemize}
otherwise, \tcode{true}.
\end{itemdescr}

\rSec3[stoptoken.cmp]{Comparisons}

\indexlibrarymember{operator==}{stop_token}%
\begin{itemdecl}
[[nodiscard]] bool operator==(const stop_token& lhs, const stop_token& rhs) noexcept;
\end{itemdecl}

\begin{itemdescr}
\pnum
\returns
\tcode{true} if \tcode{lhs} and \tcode{rhs} have ownership of the same stop state
or if both \tcode{lhs} and \tcode{rhs} do not have ownership of a stop state;
otherwise \tcode{false}.
\end{itemdescr}

\indexlibrarymember{operator!=}{stop_token}%
\begin{itemdecl}
[[nodiscard]] bool operator!=(const stop_token& lhs, const stop_token& rhs) noexcept;
\end{itemdecl}

\begin{itemdescr}
\pnum
\returns
\tcode{!(lhs==rhs)}.
\end{itemdescr}

\rSec3[stoptoken.special]{Specialized algorithms}

\indexlibrarymember{swap}{stop_token}%
\begin{itemdecl}
friend void swap(stop_token& x, stop_token& y) noexcept;
\end{itemdecl}

\begin{itemdescr}
\pnum
\effects
Equivalent to: \tcode{x.swap(y)}.
\end{itemdescr}

\indexlibraryglobal{stop_source}%
\rSec2[stopsource]{Class \tcode{stop_source}}

\pnum
\indexlibraryglobal{stop_source}%
The class \tcode{stop_source} implements the semantics of making a stop request.
A stop request made on a \tcode{stop_source} object is visible to all
associated \tcode{stop_source} and \tcode{stop_token} (\ref{stoptoken}) objects.
Once a stop request has been made it cannot be withdrawn
(a subsequent stop request has no effect).

\indexlibraryglobal{nostopstate_t}%
\indexlibraryglobal{nostopstate}%

\begin{codeblock}
namespace std {
  // no-shared-stop-state indicator
  struct nostopstate_t {
    explicit nostopstate_t() = default;
  };
  inline constexpr nostopstate_t nostopstate{};

  class stop_source {
  public:
    // \ref{stopsource.cons}, constructors, copy, and assignment
    stop_source();
    explicit stop_source(nostopstate_t) noexcept;

    stop_source(const stop_source&) noexcept;
    stop_source(stop_source&&) noexcept;
    stop_source& operator=(const stop_source&) noexcept;
    stop_source& operator=(stop_source&&) noexcept;
    ~stop_source();
    void swap(stop_source&) noexcept;

    // \ref{stopsource.mem}, stop handling
    [[nodiscard]] stop_token get_token() const noexcept;
    [[nodiscard]] bool stop_possible() const noexcept;
    [[nodiscard]] bool stop_requested() const noexcept;
    bool request_stop() noexcept;

    [[nodiscard]] friend bool
    operator==(const stop_source& lhs, const stop_source& rhs) noexcept;
    [[nodiscard]] friend bool
    operator!=(const stop_source& lhs, const stop_source& rhs) noexcept;
    friend void swap(stop_source& lhs, stop_source& rhs) noexcept;
  };
}
\end{codeblock}

\rSec3[stopsource.cons]{Constructors, copy, and assignment}

\indexlibraryctor{stop_source}%
\begin{itemdecl}
stop_source();
\end{itemdecl}

\begin{itemdescr}
\pnum
\effects
Initialises \tcode{*this} to have ownership of a new stop state.

\pnum
\ensures
\tcode{stop_possible()} is \tcode{true}
and \tcode{stop_requested()} is \tcode{false}.

\pnum
\throws
\tcode{bad_alloc} if memory could not be allocated for the stop state.
\end{itemdescr}

\indexlibraryctor{stop_source}%
\begin{itemdecl}
explicit stop_source(nostopstate_t) noexcept;
\end{itemdecl}

\begin{itemdescr}
\pnum
\ensures
\tcode{stop_possible()} is \tcode{false} and
\tcode{stop_requested()} is \tcode{false}.
\begin{note}
No resources are allocated for the state.
\end{note}
\end{itemdescr}

\indexlibraryctor{stop_source}%
\begin{itemdecl}
stop_source(const stop_source& rhs) noexcept;
\end{itemdecl}

\begin{itemdescr}
\pnum
\ensures
\tcode{*this == rhs} is \tcode{true}.
\begin{note}
\tcode{*this} and \tcode{rhs} share the ownership of the same stop state,
if any.
\end{note}
\end{itemdescr}

\indexlibraryctor{stop_source}%
\begin{itemdecl}
stop_source(stop_source&& rhs) noexcept;
\end{itemdecl}

\begin{itemdescr}
\pnum
\ensures
\tcode{*this} contains the value of \tcode{rhs}
prior to the start of construction
and \tcode{rhs.stop_possible()} is \tcode{false}.
\end{itemdescr}

\indexlibrarydtor{stop_source}%
\begin{itemdecl}
~stop_source();
\end{itemdecl}

\begin{itemdescr}
\pnum
\effects
Releases ownership of the stop state, if any.
\end{itemdescr}

\indexlibrarymember{operator=}{stop_source}%
\begin{itemdecl}
stop_source& operator=(const stop_source& rhs) noexcept;
\end{itemdecl}

\begin{itemdescr}
\pnum
\effects
Equivalent to: \tcode{stop_source(rhs).swap(*this)}.

\pnum
\returns
\tcode{*this}.
\end{itemdescr}

\indexlibrarymember{operator=}{stop_source}%
\begin{itemdecl}
stop_source& operator=(stop_source&& rhs) noexcept;
\end{itemdecl}

\begin{itemdescr}
\pnum
\effects
Equivalent to: \tcode{stop_source(std::move(rhs)).swap(*this)}.

\pnum
\returns
\tcode{*this}.
\end{itemdescr}

\indexlibrarymember{swap}{stop_source}%
\begin{itemdecl}
void swap(stop_source& rhs) noexcept;
\end{itemdecl}

\begin{itemdescr}
\pnum
\effects
Exchanges the values of \tcode{*this} and \tcode{rhs}.
\end{itemdescr}

\rSec3[stopsource.mem]{Members}

\indexlibrarymember{get_token}{stop_source sc}%
\begin{itemdecl}
[[nodiscard]] stop_token get_token() const noexcept;
\end{itemdecl}

\begin{itemdescr}
\pnum
\returns
\tcode{stop_token()} if \tcode{stop_possible()} is \tcode{false};
otherwise a new associated \tcode{stop_token} object.
\end{itemdescr}


\indexlibrarymember{stop_possible}{stop_source}%
\begin{itemdecl}
[[nodiscard]] bool stop_possible() const noexcept;
\end{itemdecl}

\begin{itemdescr}
\pnum
\returns
\tcode{true} if \tcode{*this} has ownership of a stop state;
otherwise, \tcode{false}.
\end{itemdescr}

\indexlibrarymember{stop_requested}{stop_source}%
\begin{itemdecl}
[[nodiscard]] bool stop_requested() const noexcept;
\end{itemdecl}

\begin{itemdescr}
\pnum
\returns
\tcode{true} if \tcode{*this} has ownership of a stop state
that has received a stop request;
otherwise, \tcode{false}.
\end{itemdescr}


\indexlibrarymember{request_stop}{stop_source}%
\begin{itemdecl}
bool request_stop() noexcept;
\end{itemdecl}

\begin{itemdescr}
\pnum
\effects
If \tcode{*this} does not have ownership of a stop state, returns \tcode{false}.
Otherwise, atomically determines whether the owned stop state
has received a stop request,
and if not, makes a stop request.
The determination and making of the stop request are an
atomic read-modify-write operation\iref{intro.races}.
If the request was made,
the callbacks registered by associated \tcode{stop_callback} objects
are synchronously called.
If an invocation of a callback exits via an exception
then \tcode{terminate} is called\iref{except.terminate}.
\begin{note}
A stop request includes notifying all condition variables
of type \tcode{condition_variable_any}
temporarily registered during
an interruptible wait\iref{thread.condvarany.intwait}.
\end{note}

\pnum
\ensures
\tcode{stop_possible()} is \tcode{false}
or \tcode{stop_requested()} is \tcode{true}.

\pnum
\returns
\tcode{true} if this call made a stop request;
otherwise \tcode{false}.
\end{itemdescr}

\rSec3[stopsource.cmp]{Comparisons}

\indexlibrarymember{operator==}{stop_source}%
\begin{itemdecl}
[[nodiscard]] bool operator==(const stop_source& lhs, const stop_source& rhs) noexcept;
\end{itemdecl}

\begin{itemdescr}
\pnum
\returns
\tcode{true} if \tcode{lhs} and \tcode{rhs} have ownership
of the same stop state
or if both \tcode{lhs} and \tcode{rhs} do not have ownership of a stop state;
otherwise \tcode{false}.
\end{itemdescr}

\indexlibrarymember{operator!=}{stop_source}%
\begin{itemdecl}
[[nodiscard]] bool operator!=(const stop_source& lhs, const stop_source& rhs) noexcept;
\end{itemdecl}

\begin{itemdescr}
\pnum
\returns
\tcode{!(lhs==rhs)}.
\end{itemdescr}

\rSec3[stopsource.special]{Specialized algorithms}

\indexlibrarymember{swap}{stop_source}%
\begin{itemdecl}
friend void swap(stop_source& x, stop_source& y) noexcept;
\end{itemdecl}

\begin{itemdescr}
\pnum
\effects
Equivalent to: \tcode{x.swap(y)}.
\end{itemdescr}

\indexlibraryglobal{stop_callback}%
\rSec2[stopcallback]{Class template \tcode{stop_callback}}

\pnum
\indexlibraryglobal{stop_callback}%
\begin{codeblock}
namespace std {
  template<class Callback>
  class stop_callback {
  public:
    using callback_type = Callback;

    // \ref{stopcallback.cons}, constructors and destructor
    template<class C>
    explicit stop_callback(const stop_token& st, C&& cb)
        noexcept(is_nothrow_constructible_v<Callback, C>);
    template<class C>
    explicit stop_callback(stop_token&& st, C&& cb)
        noexcept(is_nothrow_constructible_v<Callback, C>);
    ~stop_callback();

    stop_callback(const stop_callback&) = delete;
    stop_callback(stop_callback&&) = delete;
    stop_callback& operator=(const stop_callback&) = delete;
    stop_callback& operator=(stop_callback&&) = delete;

  private:
    Callback callback;      // \expos
  };

  template<class Callback>
  stop_callback(stop_token, Callback) -> stop_callback<Callback>;
}
\end{codeblock}

\pnum
\mandates
\tcode{stop_callback} is instantiated with an argument for the
template parameter \tcode{Callback}
that satisfies both \tcode{invocable}
and \tcode{destructible}.

\pnum
\expects
\tcode{stop_callback} is instantiated with an argument for the
template parameter \tcode{Callback}
that models both \tcode{invocable}
and \tcode{destructible}.


\rSec3[stopcallback.cons]{Constructors and destructor}

\indexlibraryctor{stop_callback}%
\begin{itemdecl}
template<class C>
explicit stop_callback(const stop_token& st, C&& cb)
  noexcept(is_nothrow_constructible_v<Callback, C>);
template<class C>
explicit stop_callback(stop_token&& st, C&& cb)
  noexcept(is_nothrow_constructible_v<Callback, C>);
\end{itemdecl}
\begin{itemdescr}
\pnum
\constraints
\tcode{Callback} and \tcode{C} satisfy \tcode{\libconcept{constructible_from}<Callback, C>}.

\pnum
\expects
\tcode{Callback} and \tcode{C} model \tcode{\libconcept{constructible_from}<Callback, C>}.

\pnum
\effects
Initializes \tcode{callback} with \tcode{std::forward<C>(cb)}.
If \tcode{st.stop_requested()} is \tcode{true}, then
\tcode{std::forward<Callback>(callback)()}
is evaluated in the current thread before the constructor returns.
Otherwise, if \tcode{st} has ownership of a stop state,
acquires shared ownership of that stop state and registers
the callback with that stop state
such that \tcode{std::forward<Callback>(callback)()}
is evaluated by the first call to \tcode{request_stop()}
on an associated \tcode{stop_source}.

\pnum
\remarks
If evaluating
\tcode{std::forward<Callback>(callback)()}
exits via an exception,
then \tcode{terminate} is called\iref{except.terminate}.

\pnum
\throws
Any exception thrown by the initialization of \tcode{callback}.
\end{itemdescr}

\indexlibrarydtor{stop_callback}%
\begin{itemdecl}
~stop_callback();
\end{itemdecl}

\begin{itemdescr}
\pnum
\effects
Unregisters the callback from the owned stop state, if any.
The destructor does not block waiting for the execution of another callback
registered by an associated \tcode{stop_callback}.
If \tcode{callback} is concurrently executing on another thread,
then the return from the invocation of \tcode{callback}
strongly happens before\iref{intro.races}
\tcode{callback} is destroyed.
If \tcode{callback} is executing on the current thread,
then the destructor does not block\iref{defns.block} waiting for
the return from the invocation of \tcode{callback}.
Releases ownership of the stop state, if any.
\end{itemdescr}


\rSec1[thread.threads]{Threads}

\pnum
\ref{thread.threads} describes components that can be used to create and manage threads.
\begin{note}
These threads are intended to map one-to-one with operating system threads.
\end{note}

\rSec2[thread.syn]{Header \tcode{<thread>} synopsis}

\indexheader{thread}%
\begin{codeblock}
namespace std {
  class thread;

  void swap(thread& x, thread& y) noexcept;

  // \ref{thread.jthread.class} class \tcode{jthread}
  class jthread;

  namespace this_thread {
    thread::id get_id() noexcept;

    void yield() noexcept;
    template<class Clock, class Duration>
      void sleep_until(const chrono::time_point<Clock, Duration>& abs_time);
    template<class Rep, class Period>
      void sleep_for(const chrono::duration<Rep, Period>& rel_time);
  }
}
\end{codeblock}

\rSec2[thread.thread.class]{Class \tcode{thread}}

\pnum
The class \tcode{thread} provides a mechanism to create a new thread of execution, to join with
a thread (i.e., wait for a thread to complete), and to perform other operations that manage and
query the state of a thread. A \tcode{thread} object uniquely represents a particular thread of
execution. That representation may be transferred to other \tcode{thread} objects in such a way
that no two \tcode{thread} objects simultaneously represent the same thread of execution. A
thread of execution is \term{detached} when no \tcode{thread} object represents that thread.
Objects of class \tcode{thread} can be in a state that does not represent a thread of
execution.
\begin{note}
A \tcode{thread} object does not represent a thread of execution after
default construction, after being moved from, or after a successful call to \tcode{detach} or
\tcode{join}.
\end{note}

\indexlibraryglobal{thread}%
\begin{codeblock}
namespace std {
  class thread {
  public:
    // types
    class id;
    using native_handle_type = @\impdefnc@;         // see~\ref{thread.req.native}

    // construct/copy/destroy
    thread() noexcept;
    template<class F, class... Args> explicit thread(F&& f, Args&&... args);
    ~thread();
    thread(const thread&) = delete;
    thread(thread&&) noexcept;
    thread& operator=(const thread&) = delete;
    thread& operator=(thread&&) noexcept;

    // members
    void swap(thread&) noexcept;
    bool joinable() const noexcept;
    void join();
    void detach();
    id get_id() const noexcept;
    native_handle_type native_handle();                         // see~\ref{thread.req.native}

    // static members
    static unsigned int hardware_concurrency() noexcept;
  };
}
\end{codeblock}

\rSec3[thread.thread.id]{Class \tcode{thread::id}}

\indexlibraryglobal{thread::id}%
\indexlibrarymember{thread}{id}%
\begin{codeblock}
namespace std {
  class thread::id {
  public:
    id() noexcept;
  };

  bool operator==(thread::id x, thread::id y) noexcept;
  strong_ordering operator<=>(thread::id x, thread::id y) noexcept;

  template<class charT, class traits>
    basic_ostream<charT, traits>&
      operator<<(basic_ostream<charT, traits>& out, thread::id id);

  // hash support
  template<class T> struct hash;
  template<> struct hash<thread::id>;
}
\end{codeblock}

\pnum
An object of type \tcode{thread::id} provides a unique identifier for
each thread of execution and a single distinct value for all \tcode{thread}
objects that do not represent a thread of
execution\iref{thread.thread.class}. Each thread of execution has an
associated \tcode{thread::id} object that is not equal to the
\tcode{thread::id} object of any other thread of execution and that is not
equal to the \tcode{thread::id} object of any \tcode{thread} object that
does not represent threads of execution.

\pnum
\tcode{thread::id} is a trivially copyable class\iref{class.prop}.
The library may reuse the value of a \tcode{thread::id} of a terminated thread that can no longer be joined.

\pnum
\begin{note}
Relational operators allow \tcode{thread::id} objects to be used as
keys in associative containers.
\end{note}

\indexlibraryctor{thread::id}%
\begin{itemdecl}
id() noexcept;
\end{itemdecl}

\begin{itemdescr}
\pnum
\effects
Constructs an object of type \tcode{id}.

\pnum
\ensures
The constructed object does not represent a thread of execution.
\end{itemdescr}

\indexlibrarymember{operator==}{thread::id}%
\begin{itemdecl}
bool operator==(thread::id x, thread::id y) noexcept;
\end{itemdecl}

\begin{itemdescr}
\pnum
\returns
\tcode{true} only if \tcode{x} and \tcode{y} represent the same
thread of execution or neither \tcode{x} nor \tcode{y} represents a thread of
execution.
\end{itemdescr}

\indexlibrarymember{operator<=>}{thread::id}%
\begin{itemdecl}
strong_ordering operator<=>(thread::id x, thread::id y) noexcept;
\end{itemdecl}

\begin{itemdescr}
\pnum
Let $P(\tcode{x}, \tcode{y})$ be
an unspecified total ordering over \tcode{thread::id}
as described in \ref{alg.sorting}.

\pnum
\returns
\tcode{strong_ordering::less} if $P(\tcode{x}, \tcode{y})$ is \tcode{true}.
Otherwise, \tcode{strong_ordering::greater}
if $P(\tcode{y}, \tcode{x})$ is \tcode{true}.
Otherwise, \tcode{strong_ordering::equal}.
\end{itemdescr}

\indexlibrarymember{operator<<}{thread::id}%
\begin{itemdecl}
template<class charT, class traits>
  basic_ostream<charT, traits>&
    operator<< (basic_ostream<charT, traits>& out, thread::id id);
\end{itemdecl}

\begin{itemdescr}
\pnum
\effects
Inserts an unspecified text representation of \tcode{id} into
\tcode{out}. For two objects of type \tcode{thread::id} \tcode{x} and \tcode{y},
if \tcode{x == y} the \tcode{thread::id} objects have the same text
representation and if \tcode{x != y} the \tcode{thread::id} objects have
distinct text representations.

\pnum
\returns
\tcode{out}.
\end{itemdescr}

\indexlibrarymember{hash}{thread::id}%
\begin{itemdecl}
template<> struct hash<thread::id>;
\end{itemdecl}

\begin{itemdescr}
\pnum
The specialization is enabled\iref{unord.hash}.
\end{itemdescr}

\rSec3[thread.thread.constr]{Constructors}

\indexlibraryctor{thread}%
\begin{itemdecl}
thread() noexcept;
\end{itemdecl}

\begin{itemdescr}
\pnum
\effects
Constructs a \tcode{thread} object that does not represent a thread of execution.

\pnum
\ensures
\tcode{get_id() == id()}.
\end{itemdescr}

\indexlibraryctor{thread}%
\begin{itemdecl}
template<class F, class... Args> explicit thread(F&& f, Args&&... args);
\end{itemdecl}

\begin{itemdescr}
\pnum
\requires\ \tcode{F} and each $\tcode{T}_i$ in \tcode{Args} shall meet the
\oldconcept{MoveConstructible} requirements.
\tcode{%
\placeholdernc{INVOKE}(\brk{}%
\placeholdernc{decay-copy}(std::forward<F>(f)),%
\placeholdernc{decay-copy}(std::forward<Args>({}args))...)}\iref{func.require}
shall be a valid expression.

\pnum
\remarks
This constructor shall not participate in overload resolution if \tcode{remove_cvref_t<F>}
is the same type as \tcode{std::thread}.

\pnum
\effects
Constructs an object of type \tcode{thread}. The new thread of execution executes
\tcode{%
\placeholdernc{INVOKE}(\brk{}%
\placeholdernc{decay-copy}(\brk{}%
std::forward<F>(f)),
\placeholdernc{decay-copy}(\brk{}%
std::forward<Args>(\brk{}args))...)} with the calls to
\tcode{\placeholder{decay-copy}} being evaluated in the constructing thread. Any return value from this invocation
is ignored.
\begin{note}
This implies that any exceptions not thrown from the invocation of the copy
of \tcode{f} will be thrown in the constructing thread, not the new thread.
\end{note}
If the
invocation of
\tcode{%
\placeholdernc{INVOKE}(\brk{}%
\placeholdernc{decay-copy}(\brk{}%
std::forward<F>(f)),
\placeholdernc{decay-copy}(\brk{}%
std::forward<Args>(args))...)}
termi\-nates with an uncaught exception, \tcode{terminate} shall be called.


\pnum
\sync
The completion of the invocation of the constructor
synchronizes with the beginning of the invocation of the copy of \tcode{f}.

\pnum
\ensures
\tcode{get_id() != id()}. \tcode{*this} represents the newly started thread.

\pnum
\throws
\tcode{system_error} if unable to start the new thread.

\pnum
\errors
\begin{itemize}
\item \tcode{resource_unavailable_try_again} --- the system lacked the necessary
resources to create another thread, or the system-imposed limit on the number of
threads in a process would be exceeded.
\end{itemize}
\end{itemdescr}

\indexlibraryctor{thread}%
\begin{itemdecl}
thread(thread&& x) noexcept;
\end{itemdecl}

\begin{itemdescr}
\pnum
\effects
Constructs an object of type \tcode{thread} from \tcode{x}, and sets
\tcode{x} to a default constructed state.

\pnum
\ensures
\tcode{x.get_id() == id()} and \tcode{get_id()} returns the
value of \tcode{x.get_id()} prior to the start of construction.

\end{itemdescr}

\rSec3[thread.thread.destr]{Destructor}

\indexlibrarydtor{thread}%
\begin{itemdecl}
~thread();
\end{itemdecl}

\begin{itemdescr}
\pnum
If \tcode{joinable()}, calls \tcode{terminate()}. Otherwise, has no effects.
\begin{note}
Either implicitly detaching or joining a \tcode{joinable()} thread in its
destructor could result in difficult to debug correctness (for detach) or performance
(for join) bugs encountered only when an exception is thrown. Thus the programmer must
ensure that the destructor is never executed while the thread is still joinable.
\end{note}
\end{itemdescr}

\rSec3[thread.thread.assign]{Assignment}

\indexlibrarymember{operator=}{thread}%
\begin{itemdecl}
thread& operator=(thread&& x) noexcept;
\end{itemdecl}

\begin{itemdescr}
\pnum
\effects
If \tcode{joinable()}, calls \tcode{terminate()}. Otherwise, assigns the
state of \tcode{x} to \tcode{*this} and sets \tcode{x} to a default constructed state.

\pnum
\ensures
\tcode{x.get_id() == id()} and \tcode{get_id()} returns the value of
\tcode{x.get_id()} prior to the assignment.

\pnum
\returns
\tcode{*this}.
\end{itemdescr}

\rSec3[thread.thread.member]{Members}

\indexlibrarymember{swap}{thread}%
\begin{itemdecl}
void swap(thread& x) noexcept;
\end{itemdecl}

\begin{itemdescr}
\pnum
\effects
Swaps the state of \tcode{*this} and \tcode{x}.
\end{itemdescr}

\indexlibrarymember{joinable}{thread}%
\begin{itemdecl}
bool joinable() const noexcept;
\end{itemdecl}

\begin{itemdescr}
\pnum
\returns
\tcode{get_id() != id()}.
\end{itemdescr}

\indexlibrarymember{join}{thread}%
\begin{itemdecl}
void join();
\end{itemdecl}

\begin{itemdescr}
\pnum
\effects
Blocks until the thread represented by \tcode{*this} has completed.

\pnum
\sync
The completion of the thread represented by \tcode{*this} synchronizes with\iref{intro.multithread}
the corresponding successful
\tcode{join()} return.
\begin{note}
Operations on
\tcode{*this} are not synchronized.
\end{note}

\pnum
\ensures
The thread represented by \tcode{*this} has completed. \tcode{get_id() == id()}.

\pnum
\throws
\tcode{system_error} when
an exception is required\iref{thread.req.exception}.

\pnum
\errors
\begin{itemize}
\item \tcode{resource_deadlock_would_occur} --- if deadlock is detected or
\tcode{get_id() == this_thread::\brk{}get_id()}.

\item \tcode{no_such_process} --- if the thread is not valid.

\item \tcode{invalid_argument} --- if the thread is not joinable.
\end{itemize}
\end{itemdescr}

\indexlibrarymember{detach}{thread}%
\begin{itemdecl}
void detach();
\end{itemdecl}

\begin{itemdescr}
\pnum
\effects
The thread represented by \tcode{*this} continues execution without the calling thread
blocking. When \tcode{detach()} returns, \tcode{*this} no longer represents the possibly continuing
thread of execution. When the thread previously represented by \tcode{*this} ends execution, the
implementation shall release any owned resources.

\pnum
\ensures
\tcode{get_id() == id()}.

\pnum
\throws
\tcode{system_error} when
an exception is required\iref{thread.req.exception}.

\pnum
\errors
\begin{itemize}
\item \tcode{no_such_process} --- if the thread is not valid.
\item \tcode{invalid_argument} --- if the thread is not joinable.
\end{itemize}
\end{itemdescr}

\indexlibrarymember{get_id}{thread}%
\begin{itemdecl}
id get_id() const noexcept;
\end{itemdecl}

\begin{itemdescr}
\pnum
\returns
A default constructed \tcode{id} object if \tcode{*this} does not represent a thread,
otherwise \tcode{this_thread::get_id()} for the thread of execution represented by
\tcode{*this}.
\end{itemdescr}

\rSec3[thread.thread.static]{Static members}

\indexlibrarymember{hardware_concurrency}{thread}%
\begin{itemdecl}
unsigned hardware_concurrency() noexcept;
\end{itemdecl}

\begin{itemdescr}
\pnum
\returns
The number of hardware thread contexts.
\begin{note}
This value should
only be considered to be a hint.
\end{note}
If this value is not computable or
well-defined, an implementation should return 0.
\end{itemdescr}

\rSec3[thread.thread.algorithm]{Specialized algorithms}

\indexlibrarymember{swap}{thread}%
\begin{itemdecl}
void swap(thread& x, thread& y) noexcept;
\end{itemdecl}

\begin{itemdescr}
\pnum
\effects
As if by \tcode{x.swap(y)}.
\end{itemdescr}

\rSec2[thread.jthread.class]{Class \tcode{jthread}}

\pnum
The class \tcode{jthread} provides a mechanism
to create a new thread of execution.
The functionality is the same as for
class \tcode{thread}\iref{thread.thread.class}
with the additional ability to request that the thread stops
and then joins the started thread.


\indexlibraryglobal{jthread}%
\begin{codeblock}
namespace std {
  class jthread {
  public:
    // types
    using id = thread::id;
    using native_handle_type = thread::native_handle_type;

    // \ref{thread.jthread.cons}, constructors, move, and assignment
    jthread() noexcept;
    template<class F, class... Args> explicit jthread(F&& f, Args&&... args);
    ~jthread();
    jthread(const jthread&) = delete;
    jthread(jthread&&) noexcept;
    jthread& operator=(const jthread&) = delete;
    jthread& operator=(jthread&&) noexcept;

    // \ref{thread.jthread.mem}, members
    void swap(jthread&) noexcept;
    [[nodiscard]] bool joinable() const noexcept;
    void join();
    void detach();
    [[nodiscard]] id get_id() const noexcept;
    [[nodiscard]] native_handle_type native_handle();   // see~\ref{thread.req.native}

    // \ref{thread.jthread.stop}, stop token handling
    [[nodiscard]] stop_source get_stop_source() noexcept;
    [[nodiscard]] stop_token get_stop_token() const noexcept;
    bool request_stop() noexcept;

    // \ref{thread.jthread.special}, specialized algorithms
    friend void swap(jthread& lhs, jthread& rhs) noexcept;

    // \ref{thread.jthread.static}, static members
    [[nodiscard]] static unsigned int hardware_concurrency() noexcept;

  private:
    stop_source ssource;        // \expos
  };
\end{codeblock}

\rSec3[thread.jthread.cons]{Constructors, move, and assignment}

\indexlibraryctor{jthread}%
\begin{itemdecl}
jthread() noexcept;
\end{itemdecl}

\begin{itemdescr}
\pnum
\effects
Constructs a \tcode{jthread} object that does not represent
a thread of execution.

\pnum
\ensures
\tcode{get_id() == id()} is \tcode{true}
and \tcode{ssource.stop_possible()} is \tcode{false}.
\end{itemdescr}

\indexlibraryctor{jthread}%
\begin{itemdecl}
template<class F, class... Args> explicit jthread(F&& f, Args&&... args);
\end{itemdecl}

\begin{itemdescr}
\pnum
\requires
\tcode{F} and each $\tcode{T}_i$ in \tcode{Args} meet the
\oldconcept{MoveConstructible} requirements.
Either
\begin{codeblock}
@\placeholdernc{INVOKE}@(@\placeholdernc{decay-copy}@(std::forward<F>(f)), get_stop_token(),
       @\placeholdernc{decay-copy}@(std::forward<Args>(args))...)
\end{codeblock}
is a valid expression or
\begin{codeblock}
@\placeholdernc{INVOKE}@(@\placeholdernc{decay-copy}@(std::forward<F>(f)), @\placeholdernc{decay-copy}@(std::forward<Args>(args))...)
\end{codeblock}
is a valid expression.

\pnum
\constraints
\tcode{remove_cvref_t<F>} is not the same type as \tcode{jthread}.

\pnum
\effects
Initializes \tcode{ssource} and
constructs an object of type \tcode{jthread}.
The new thread of execution executes
\begin{codeblock}
@\placeholdernc{INVOKE}@(@\placeholdernc{decay-copy}@(std::forward<F>(f)), get_stop_token(),
       @\placeholdernc{decay-copy}@(std::forward<Args>(args))...)
\end{codeblock}
if that expression is well-formed,
otherwise
\begin{codeblock}
@\placeholdernc{INVOKE}@(@\placeholdernc{decay-copy}@(std::forward<F>(f)), @\placeholdernc{decay-copy}@(std::forward<Args>(args))...)
\end{codeblock}
with the calls to
\tcode{\placeholder{decay-copy}} being evaluated in the constructing thread.
Any return value from this invocation is ignored.
\begin{note}
This implies that any exceptions not thrown from the invocation of the copy
of \tcode{f} will be thrown in the constructing thread, not the new thread.
\end{note}
If the \tcode{\placeholdernc{INVOKE}} expression exits via an exception,
\tcode{terminate} is called.

\pnum
\sync
The completion of the invocation of the constructor
synchronizes with the beginning of the invocation of the copy of \tcode{f}.

\pnum
\ensures
\tcode{get_id() != id()} is \tcode{true}
and \tcode{ssource.stop_possible()} is \tcode{true}
and \tcode{*this} represents the newly started thread.
\begin{note}
The calling thread can make a stop request only once,
because it cannot replace this stop token.
\end{note}

\pnum
\throws
\tcode{system_error} if unable to start the new thread.

\pnum
\errors
\begin{itemize}
\item \tcode{resource_unavailable_try_again} --- the system lacked
the necessary resources to create another thread,
or the system-imposed limit on the number of threads in a process
would be exceeded.
\end{itemize}
\end{itemdescr}

\indexlibraryctor{jthread}%
\begin{itemdecl}
jthread(jthread&& x) noexcept;
\end{itemdecl}

\begin{itemdescr}
\pnum
\effects
Constructs an object of type \tcode{jthread} from \tcode{x}, and sets
\tcode{x} to a default constructed state.

\pnum
\ensures
\tcode{x.get_id() == id()}
and \tcode{get_id()} returns the value of \tcode{x.get_id()}
prior to the start of construction.
\tcode{ssource} has the value of \tcode{x.ssource}
prior to the start of construction
and \tcode{x.ssource.stop_possible()} is \tcode{false}.
\end{itemdescr}

\indexlibrarydtor{jthread}%
\begin{itemdecl}
~jthread();
\end{itemdecl}

\begin{itemdescr}
\pnum
\effects
If \tcode{joinable()} is \tcode{true},
calls \tcode{request_stop()} and then \tcode{join()}.
\begin{note}
Operations on \tcode{*this} are not synchronized.
\end{note}
\end{itemdescr}

\indexlibrarymember{operator=}{jthread}%
\begin{itemdecl}
jthread& operator=(jthread&& x) noexcept;
\end{itemdecl}

\begin{itemdescr}
\pnum
\effects
If \tcode{joinable()} is \tcode{true},
calls \tcode{request_stop()} and then \tcode{join()}.
Assigns the state of \tcode{x} to \tcode{*this}
and sets \tcode{x} to a default constructed state.

\pnum
\ensures
\tcode{x.get_id() == id()}
and \tcode{get_id()} returns the value of \tcode{x.get_id()}
prior to the assignment.
\tcode{ssource} has the value of \tcode{x.ssource}
prior to the assignment
and \tcode{x.ssource.stop_possible()} is \tcode{false}.

\pnum
\returns
\tcode{*this}.
\end{itemdescr}

\rSec3[thread.jthread.mem]{Members}

\indexlibrarymember{swap}{jthread}%
\begin{itemdecl}
void swap(jthread& x) noexcept;
\end{itemdecl}

\begin{itemdescr}
\pnum
\effects
Exchanges the values of \tcode{*this} and \tcode{x}.
\end{itemdescr}


\indexlibrarymember{joinable}{jthread}%
\begin{itemdecl}
[[nodiscard]] bool joinable() const noexcept;
\end{itemdecl}

\begin{itemdescr}
\pnum
\returns
\tcode{get_id() != id()}.
\end{itemdescr}

\indexlibrarymember{join}{jthread}%
\begin{itemdecl}
void join();
\end{itemdecl}

\begin{itemdescr}
\pnum
\effects
Blocks until the thread represented by \tcode{*this} has completed.

\pnum
\sync
The completion of the thread represented by \tcode{*this}
synchronizes with\iref{intro.multithread}
the corresponding successful \tcode{join()} return.
\begin{note}
Operations on \tcode{*this} are not synchronized.
\end{note}

\pnum
\ensures
The thread represented by \tcode{*this} has completed.
\tcode{get_id() == id()}.

\pnum
\throws
\tcode{system_error} when an exception is required\iref{thread.req.exception}.

\pnum
\errors
\begin{itemize}
\item \tcode{resource_deadlock_would_occur} --- if deadlock is detected or
\tcode{get_id() == this_thread::\brk{}get_id()}.

\item \tcode{no_such_process} --- if the thread is not valid.

\item \tcode{invalid_argument} --- if the thread is not joinable.
\end{itemize}
\end{itemdescr}

\indexlibrarymember{detach}{jthread}%
\begin{itemdecl}
void detach();
\end{itemdecl}

\begin{itemdescr}
\pnum
\effects
The thread represented by \tcode{*this} continues execution
without the calling thread blocking.
When \tcode{detach()} returns,
\tcode{*this} no longer represents the possibly continuing thread of execution.
When the thread previously represented by \tcode{*this} ends execution,
the implementation shall release any owned resources.

\pnum
\ensures
\tcode{get_id() == id()}.

\pnum
\throws
\tcode{system_error} when an exception is required\iref{thread.req.exception}.

\pnum
\errors
\begin{itemize}
\item \tcode{no_such_process} --- if the thread is not valid.
\item \tcode{invalid_argument} --- if the thread is not joinable.
\end{itemize}
\end{itemdescr}

\indexlibrarymember{get_id}{jthread}%
\begin{itemdecl}
id get_id() const noexcept;
\end{itemdecl}

\begin{itemdescr}
\pnum
\returns
A default constructed \tcode{id} object
if \tcode{*this} does not represent a thread,
otherwise \tcode{this_thread::get_id()}
for the thread of execution represented by \tcode{*this}.
\end{itemdescr}

\rSec3[thread.jthread.stop]{Stop token handling}

\indexlibrarymember{get_stop_source}{jthread}%
\begin{itemdecl}
[[nodiscard]] stop_source get_stop_source() noexcept;
\end{itemdecl}

\begin{itemdescr}
\pnum
\effects
Equivalent to: \tcode{return ssource;}
\end{itemdescr}

\indexlibrarymember{get_stop_token}{jthread}%
\begin{itemdecl}
[[nodiscard]] stop_token get_stop_token() const noexcept;
\end{itemdecl}

\begin{itemdescr}
\pnum
\effects
Equivalent to: \tcode{return ssource.get_token();}
\end{itemdescr}

\indexlibrarymember{request_stop}{jthread}%
\begin{itemdecl}
bool request_stop() noexcept;
\end{itemdecl}

\begin{itemdescr}
\pnum
\effects
Equivalent to: \tcode{return ssource.request_stop();}
\end{itemdescr}


\rSec3[thread.jthread.special]{Specialized algorithms}

\indexlibrarymember{swap}{jthread}%
\begin{itemdecl}
friend void swap(jthread& x, jthread& y) noexcept;
\end{itemdecl}

\begin{itemdescr}
\pnum
\effects
Equivalent to: \tcode{x.swap(y)}.
\end{itemdescr}

\rSec3[thread.jthread.static]{Static members}

\indexlibrarymember{hardware_concurrency}{jthread}%
\begin{itemdecl}
unsigned hardware_concurrency() noexcept;
\end{itemdecl}

\begin{itemdescr}
\pnum
\returns
\tcode{thread::hardware_concurrency()}.
\end{itemdescr}


\rSec2[thread.thread.this]{Namespace \tcode{this_thread}}

\begin{codeblock}
namespace std::this_thread {
  thread::id get_id() noexcept;

  void yield() noexcept;
  template<class Clock, class Duration>
    void sleep_until(const chrono::time_point<Clock, Duration>& abs_time);
  template<class Rep, class Period>
    void sleep_for(const chrono::duration<Rep, Period>& rel_time);
}
\end{codeblock}

\indexlibrarymember{get_id}{this_thread}%
\begin{itemdecl}
thread::id this_thread::get_id() noexcept;
\end{itemdecl}

\begin{itemdescr}
\pnum
\returns
An object of type \tcode{thread::id} that uniquely identifies the current thread of
execution. No other thread of execution shall have this id and this thread of execution shall
always have this id. The object returned shall not compare equal to a default constructed
\tcode{thread::id}.
\end{itemdescr}

\indexlibrarymember{yield}{this_thread}%
\begin{itemdecl}
void this_thread::yield() noexcept;
\end{itemdecl}

\begin{itemdescr}
\pnum
\effects
Offers the implementation the opportunity to reschedule.

\pnum
\sync
None.
\end{itemdescr}

\indexlibrarymember{sleep_until}{this_thread}%
\begin{itemdecl}
template<class Clock, class Duration>
  void sleep_until(const chrono::time_point<Clock, Duration>& abs_time);
\end{itemdecl}

\begin{itemdescr}
\pnum
\effects
Blocks the calling thread for the absolute timeout\iref{thread.req.timing} specified
by \tcode{abs_time}.

\pnum
\sync
None.

\pnum
\throws
Timeout-related exceptions\iref{thread.req.timing}.
\end{itemdescr}

\indexlibrarymember{sleep_for}{this_thread}%
\begin{itemdecl}
template<class Rep, class Period>
  void sleep_for(const chrono::duration<Rep, Period>& rel_time);
\end{itemdecl}

\begin{itemdescr}
\pnum
\effects
Blocks the calling thread for the relative timeout\iref{thread.req.timing} specified
by \tcode{rel_time}.

\pnum
\sync
None.

\pnum
\throws
Timeout-related exceptions\iref{thread.req.timing}.
\end{itemdescr}

\rSec1[thread.mutex]{Mutual exclusion}

\pnum
This subclause provides mechanisms for mutual exclusion: mutexes, locks, and call
once. These mechanisms ease the production of race-free
programs\iref{intro.multithread}.

\rSec2[mutex.syn]{Header \tcode{<mutex>} synopsis}

\indexheader{mutex}%
\begin{codeblock}
namespace std {
  class mutex;
  class recursive_mutex;
  class timed_mutex;
  class recursive_timed_mutex;

  struct defer_lock_t { explicit defer_lock_t() = default; };
  struct try_to_lock_t { explicit try_to_lock_t() = default; };
  struct adopt_lock_t { explicit adopt_lock_t() = default; };

  inline constexpr defer_lock_t  defer_lock { };
  inline constexpr try_to_lock_t try_to_lock { };
  inline constexpr adopt_lock_t  adopt_lock { };

  template<class Mutex> class lock_guard;
  template<class... MutexTypes> class scoped_lock;
  template<class Mutex> class unique_lock;

  template<class Mutex>
    void swap(unique_lock<Mutex>& x, unique_lock<Mutex>& y) noexcept;

  template<class L1, class L2, class... L3> int try_lock(L1&, L2&, L3&...);
  template<class L1, class L2, class... L3> void lock(L1&, L2&, L3&...);

  struct once_flag;

  template<class Callable, class... Args>
    void call_once(once_flag& flag, Callable&& func, Args&&... args);
}
\end{codeblock}

\rSec2[shared.mutex.syn]{Header \tcode{<shared_mutex>} synopsis}

\indexheader{shared_mutex}%
\begin{codeblock}
namespace std {
  class shared_mutex;
  class shared_timed_mutex;
  template<class Mutex> class shared_lock;
  template<class Mutex>
    void swap(shared_lock<Mutex>& x, shared_lock<Mutex>& y) noexcept;
}
\end{codeblock}

\rSec2[thread.mutex.requirements]{Mutex requirements}

\rSec3[thread.mutex.requirements.general]{In general}

\pnum
A mutex object facilitates protection against data races and allows safe synchronization of
data between execution agents\iref{thread.req.lockable}.
An execution agent \term{owns} a mutex from the time it successfully calls one of the
lock functions until it calls unlock. Mutexes can be either recursive or non-recursive, and can
grant simultaneous ownership to one or many execution agents. Both
recursive and non-recursive mutexes are supplied.

\rSec3[thread.mutex.requirements.mutex]{Mutex types}

\pnum
The \defn{mutex types} are the standard library types \tcode{mutex},
\tcode{recursive_mutex}, \tcode{timed_mutex}, \tcode{recursive_timed_mutex},
\tcode{shared_mutex}, and \tcode{shared_timed_mutex}.
They shall meet the requirements set out in this subclause. In this description, \tcode{m}
denotes an object of a mutex type.

\pnum
The mutex types shall meet the \oldconcept{Lockable} requirements\iref{thread.req.lockable.req}.

\pnum
The mutex types shall be \oldconcept{DefaultConstructible} and \oldconcept{Destructible}. If
initialization of an object of a mutex type fails, an exception of type
\tcode{system_error} shall be thrown. The mutex types shall not be copyable or movable.

\pnum
The error conditions for error codes, if any, reported by member functions of the mutex types
shall be:
\begin{itemize}
\item \tcode{resource_unavailable_try_again} --- if any native handle type manipulated is not available.
\item \tcode{operation_not_permitted} --- if the thread does not have the
privilege to perform the operation.
\item \tcode{invalid_argument} --- if any native handle type manipulated as part of mutex
construction is incorrect.
\end{itemize}

\pnum
The implementation shall provide lock and unlock operations, as described below.
For purposes of determining the existence of a data race, these behave as
atomic operations\iref{intro.multithread}. The lock and unlock operations on
a single mutex shall appear to occur in a single total order.
\begin{note}
This
can be viewed as the modification order\iref{intro.multithread} of the
mutex.
\end{note}
\begin{note}
Construction and
destruction of an object of a mutex type need not be thread-safe; other
synchronization should be used to ensure that mutex objects are initialized
and visible to other threads.
\end{note}

\pnum
The expression \tcode{m.lock()} shall be well-formed and have the following semantics:

\begin{itemdescr}
\pnum
\requires If \tcode{m} is of type \tcode{mutex}, \tcode{timed_mutex},
\tcode{shared_mutex}, or \tcode{shared_timed_mutex}, the calling
thread does not own the mutex.

\pnum
\effects
Blocks the calling thread until ownership of the mutex can be obtained for the calling thread.

\pnum
\ensures
The calling thread owns the mutex.

\pnum
\returntype \tcode{void}.

\pnum
\sync
Prior \tcode{unlock()} operations on the same object shall
\term{synchronize with}\iref{intro.multithread} this operation.

\pnum
\throws
\tcode{system_error} when
an exception is required\iref{thread.req.exception}.

\pnum
\errors
\begin{itemize}
\item \tcode{operation_not_permitted} --- if the thread does not have the
privilege to perform the operation.

\item \tcode{resource_deadlock_would_occur} --- if the implementation detects
that a deadlock would occur.
\end{itemize}
\end{itemdescr}

\pnum
The expression \tcode{m.try_lock()} shall be well-formed and have the following semantics:

\begin{itemdescr}
\pnum
\requires If \tcode{m} is of type \tcode{mutex}, \tcode{timed_mutex},
\tcode{shared_mutex}, or \tcode{shared_timed_mutex}, the calling
thread does not own the mutex.

\pnum
\effects
Attempts to obtain ownership of the mutex for the calling thread without
blocking. If ownership is not obtained, there is no effect and \tcode{try_lock()}
immediately returns. An implementation may fail to obtain the lock even if it is not
held by any other thread.
\begin{note}
This spurious failure is normally uncommon, but
allows interesting implementations based on a simple
compare and exchange\iref{atomics}.
\end{note}
An implementation should ensure that \tcode{try_lock()} does not consistently return \tcode{false}
in the absence of contending mutex acquisitions.

\pnum
\returntype \tcode{bool}.

\pnum
\returns
\tcode{true} if ownership of the mutex was obtained for the calling
thread, otherwise \tcode{false}.

\pnum
\sync
If \tcode{try_lock()} returns \tcode{true}, prior \tcode{unlock()} operations
on the same object \term{synchronize with}\iref{intro.multithread} this operation.
\begin{note}
Since \tcode{lock()} does not synchronize with a failed subsequent
\tcode{try_lock()}, the visibility rules are weak enough that little would be
known about the state after a failure, even in the absence of spurious failures.
\end{note}

\pnum
\throws
Nothing.
\end{itemdescr}

\pnum
The expression \tcode{m.unlock()} shall be well-formed and have the following semantics:

\begin{itemdescr}
\pnum
\requires The calling thread shall own the mutex.

\pnum
\effects
Releases the calling thread's ownership of the mutex.

\pnum
\returntype \tcode{void}.

\pnum
\sync
This operation synchronizes with\iref{intro.multithread} subsequent
lock operations that obtain ownership on the same object.

\pnum
\throws
Nothing.
\end{itemdescr}

\rSec4[thread.mutex.class]{Class \tcode{mutex}}

\indexlibraryglobal{mutex}%
\begin{codeblock}
namespace std {
  class mutex {
  public:
    constexpr mutex() noexcept;
    ~mutex();

    mutex(const mutex&) = delete;
    mutex& operator=(const mutex&) = delete;

    void lock();
    bool try_lock();
    void unlock();

    using native_handle_type = @\impdefnc@;          // see~\ref{thread.req.native}
    native_handle_type native_handle();                         // see~\ref{thread.req.native}
  };
}
\end{codeblock}

\pnum
The class \tcode{mutex} provides a non-recursive mutex with exclusive ownership
semantics. If one thread owns a mutex object, attempts by another thread to acquire
ownership of that object will fail (for \tcode{try_lock()}) or block (for
\tcode{lock()}) until the owning thread has released ownership with a call to
\tcode{unlock()}.

\pnum
\begin{note}
After a thread \tcode{A} has called \tcode{unlock()}, releasing a mutex, it is possible for another
thread \tcode{B} to lock the same mutex, observe that it is no longer in use, unlock it, and
destroy it, before thread \tcode{A} appears to have returned from its unlock call. Implementations
are required to handle such scenarios correctly, as long as thread \tcode{A} doesn't access the
mutex after the unlock call returns. These cases typically occur when a reference-counted object
contains a mutex that is used to protect the reference count.
\end{note}

\pnum
The class \tcode{mutex} shall meet all of the mutex
requirements\iref{thread.mutex.requirements}. It shall be a standard-layout
class\iref{class.prop}.

\pnum
\begin{note}
A program may deadlock if the thread that owns a \tcode{mutex} object calls
\tcode{lock()} on that object. If the implementation can detect the deadlock,
a \tcode{resource_deadlock_would_occur} error condition may be observed.
\end{note}

\pnum
The behavior of a program is undefined if
it destroys a \tcode{mutex} object owned by any thread or
a thread terminates while owning a \tcode{mutex} object.

\rSec4[thread.mutex.recursive]{Class \tcode{recursive_mutex}}

\indexlibraryglobal{recursive_mutex}%
\begin{codeblock}
namespace std {
  class recursive_mutex {
  public:
    recursive_mutex();
    ~recursive_mutex();

    recursive_mutex(const recursive_mutex&) = delete;
    recursive_mutex& operator=(const recursive_mutex&) = delete;

    void lock();
    bool try_lock() noexcept;
    void unlock();

    using native_handle_type = @\impdefnc@;          // see~\ref{thread.req.native}
    native_handle_type native_handle();                         // see~\ref{thread.req.native}
  };
}
\end{codeblock}

\pnum
The class \tcode{recursive_mutex} provides a recursive mutex with exclusive ownership
semantics. If one thread owns a \tcode{recursive_mutex} object, attempts by another
thread to acquire ownership of that object will fail (for \tcode{try_lock()}) or block
(for \tcode{lock()}) until the first thread has completely released ownership.

\pnum
The class \tcode{recursive_mutex} shall meet all of the mutex
requirements\iref{thread.mutex.requirements}. It shall be a standard-layout
class\iref{class.prop}.

\pnum
A thread that owns a \tcode{recursive_mutex} object may acquire additional levels of
ownership by calling \tcode{lock()} or \tcode{try_lock()} on that object. It is
unspecified how many levels of ownership may be acquired by a single thread. If a thread
has already acquired the maximum level of ownership for a \tcode{recursive_mutex}
object, additional calls to \tcode{try_lock()} shall fail, and additional calls to
\tcode{lock()} shall throw an exception of type \tcode{system_error}. A thread
shall call \tcode{unlock()} once for each level of ownership acquired by calls to
\tcode{lock()} and \tcode{try_lock()}. Only when all levels of ownership have been
released may ownership be acquired by another thread.

\pnum
The behavior of a program is undefined if:

\begin{itemize}
\item it destroys a \tcode{recursive_mutex} object owned by any thread or
\item a thread terminates while owning a \tcode{recursive_mutex} object.
\end{itemize}

\rSec3[thread.timedmutex.requirements]{Timed mutex types}

\pnum
The \defn{timed mutex types} are the standard library types \tcode{timed_mutex},
\tcode{recursive_timed_mutex}, and \tcode{shared_timed_mutex}. They shall
meet the requirements set out below.
In this description, \tcode{m} denotes an object of a mutex type,
\tcode{rel_time} denotes an object of an
instantiation of \tcode{duration}\iref{time.duration}, and \tcode{abs_time} denotes an
object of an
instantiation of \tcode{time_point}\iref{time.point}.

\pnum
The timed mutex types shall meet the \oldconcept{TimedLockable}
requirements\iref{thread.req.lockable.timed}.

\pnum
The expression \tcode{m.try_lock_for(rel_time)} shall be well-formed and have the
following semantics:

\begin{itemdescr}
\pnum
\requires If \tcode{m} is of type \tcode{timed_mutex} or
\tcode{shared_timed_mutex}, the calling thread does not
own the mutex.

\pnum
\effects
The function attempts to obtain ownership of the mutex within the
relative timeout\iref{thread.req.timing}
specified by \tcode{rel_time}. If the time specified by \tcode{rel_time} is less than or
equal to \tcode{rel_time.zero()}, the function attempts to obtain ownership without blocking (as if by calling
\tcode{try_lock()}). The function shall return within the timeout specified by
\tcode{rel_time} only if it has obtained ownership of the mutex object.
\begin{note}
As
with \tcode{try_lock()}, there is no guarantee that ownership will be obtained if the
lock is available, but implementations are expected to make a strong effort to do so.
\end{note}

\pnum
\returntype \tcode{bool}.

\pnum
\returns
\tcode{true} if ownership was obtained, otherwise \tcode{false}.

\pnum
\sync
If \tcode{try_lock_for()} returns \tcode{true}, prior \tcode{unlock()} operations
on the same object \term{synchronize with}\iref{intro.multithread} this operation.

\pnum
\throws
Timeout-related exceptions\iref{thread.req.timing}.
\end{itemdescr}

\pnum
The expression \tcode{m.try_lock_until(abs_time)} shall be well-formed and have the
following semantics:

\begin{itemdescr}
\pnum
\requires If \tcode{m} is of type \tcode{timed_mutex} or
\tcode{shared_timed_mutex}, the calling thread does not own the
mutex.

\pnum
\effects
The function attempts to obtain ownership of the mutex. If
\tcode{abs_time} has already passed, the function attempts to obtain ownership
without blocking (as if by calling \tcode{try_lock()}). The function shall
return before the absolute timeout\iref{thread.req.timing} specified by
\tcode{abs_time} only if it has obtained ownership of the mutex object.
\begin{note}
As with \tcode{try_lock()}, there is no guarantee that ownership will
be obtained if the lock is available, but implementations are expected to make a
strong effort to do so.
\end{note}

\pnum
\returntype \tcode{bool}.

\pnum
\returns
\tcode{true} if ownership was obtained, otherwise \tcode{false}.

\pnum
\sync
If \tcode{try_lock_until()} returns \tcode{true}, prior \tcode{unlock()}
operations on the same object \term{synchronize with}\iref{intro.multithread}
this operation.

\pnum
\throws
Timeout-related exceptions\iref{thread.req.timing}.
\end{itemdescr}

\rSec4[thread.timedmutex.class]{Class \tcode{timed_mutex}}

\indexlibraryglobal{timed_mutex}%
\begin{codeblock}
namespace std {
  class timed_mutex {
  public:
    timed_mutex();
    ~timed_mutex();

    timed_mutex(const timed_mutex&) = delete;
    timed_mutex& operator=(const timed_mutex&) = delete;

    void lock();    // blocking
    bool try_lock();
    template<class Rep, class Period>
      bool try_lock_for(const chrono::duration<Rep, Period>& rel_time);
    template<class Clock, class Duration>
      bool try_lock_until(const chrono::time_point<Clock, Duration>& abs_time);
    void unlock();

    using native_handle_type = @\impdefnc@;          // see~\ref{thread.req.native}
    native_handle_type native_handle();                         // see~\ref{thread.req.native}
  };
}
\end{codeblock}

\pnum
The class \tcode{timed_mutex} provides a non-recursive mutex with exclusive ownership
semantics. If one thread owns a \tcode{timed_mutex} object, attempts by another thread
to acquire ownership of that object will fail (for \tcode{try_lock()}) or block
(for \tcode{lock()}, \tcode{try_lock_for()}, and \tcode{try_lock_until()}) until
the owning thread has released ownership with a call to \tcode{unlock()} or the
call to \tcode{try_lock_for()} or \tcode{try_lock_until()} times out (having
failed to obtain ownership).

\pnum
The class \tcode{timed_mutex} shall meet all of the timed mutex
requirements\iref{thread.timedmutex.requirements}. It shall be a standard-layout
class\iref{class.prop}.

\pnum
The behavior of a program is undefined if:

\begin{itemize}
\item it destroys a \tcode{timed_mutex} object owned by any thread,
\item a thread that owns a \tcode{timed_mutex} object calls \tcode{lock()},
\tcode{try_lock()}, \tcode{try_lock_for()}, or \tcode{try_lock_until()} on that object, or
\item a thread terminates while owning a \tcode{timed_mutex} object.
\end{itemize}

\rSec4[thread.timedmutex.recursive]{Class \tcode{recursive_timed_mutex}}

\indexlibraryglobal{recursive_timed_mutex}%
\begin{codeblock}
namespace std {
  class recursive_timed_mutex {
  public:
    recursive_timed_mutex();
    ~recursive_timed_mutex();

    recursive_timed_mutex(const recursive_timed_mutex&) = delete;
    recursive_timed_mutex& operator=(const recursive_timed_mutex&) = delete;

    void lock();    // blocking
    bool try_lock() noexcept;
    template<class Rep, class Period>
      bool try_lock_for(const chrono::duration<Rep, Period>& rel_time);
    template<class Clock, class Duration>
      bool try_lock_until(const chrono::time_point<Clock, Duration>& abs_time);
    void unlock();

    using native_handle_type = @\impdefnc@;          // see~\ref{thread.req.native}
    native_handle_type native_handle();                         // see~\ref{thread.req.native}
  };
}
\end{codeblock}

\pnum
The class \tcode{recursive_timed_mutex} provides a recursive mutex with exclusive
ownership semantics. If one thread owns a \tcode{recursive_timed_mutex} object,
attempts by another thread to acquire ownership of that object will fail (for
\tcode{try_lock()}) or block (for \tcode{lock()}, \tcode{try_lock_for()}, and
\tcode{try_lock_until()}) until the owning thread has completely released
ownership or the call to \tcode{try_lock_for()} or \tcode{try_lock_until()}
times out (having failed to obtain ownership).

\pnum
The class \tcode{recursive_timed_mutex} shall meet all of the timed mutex
requirements\iref{thread.timedmutex.requirements}. It shall be a standard-layout
class\iref{class.prop}.

\pnum
A thread that owns a \tcode{recursive_timed_mutex} object may acquire additional
levels of ownership by calling \tcode{lock()}, \tcode{try_lock()},
\tcode{try_lock_for()}, or \tcode{try_lock_until()} on that object. It is
unspecified how many levels of ownership may be acquired by a single thread. If
a thread has already acquired the maximum level of ownership for a
\tcode{recursive_timed_mutex} object, additional calls to \tcode{try_lock()},
\tcode{try_lock_for()}, or \tcode{try_lock_until()} shall fail, and additional
calls to \tcode{lock()} shall throw an exception of type \tcode{system_error}. A
thread shall call \tcode{unlock()} once for each level of ownership acquired by
calls to \tcode{lock()}, \tcode{try_lock()}, \tcode{try_lock_for()}, and
\tcode{try_lock_until()}. Only when all levels of ownership have been released
may ownership of the object be acquired by another thread.

\pnum
The behavior of a program is undefined if:

\begin{itemize}
\item it destroys a \tcode{recursive_timed_mutex} object owned by any thread, or
\item a thread terminates while owning a \tcode{recursive_timed_mutex} object.
\end{itemize}


\rSec3[thread.sharedmutex.requirements]{Shared mutex types}

\pnum
The standard library types \tcode{shared_mutex} and \tcode{shared_timed_mutex}
are \defn{shared mutex types}. Shared mutex types shall meet the requirements of
mutex types\iref{thread.mutex.requirements.mutex}, and additionally
shall meet the requirements set out below. In this description,
\tcode{m} denotes an object of a shared mutex type.

\pnum
In addition to the exclusive lock ownership mode specified
in~\ref{thread.mutex.requirements.mutex}, shared mutex types provide a
\defn{shared lock} ownership mode. Multiple execution agents can
simultaneously hold a shared lock ownership of a shared mutex type. But no
execution agent shall hold a shared lock while another execution agent holds an
exclusive lock on the same shared mutex type, and vice-versa. The maximum
number of execution agents which can share a shared lock on a single shared
mutex type is unspecified, but shall be at least 10000. If more than the
maximum number of execution agents attempt to obtain a shared lock, the
excess execution agents shall block until the number of shared locks are
reduced below the maximum amount by other execution agents releasing their
shared lock.

\pnum
The expression \tcode{m.lock_shared()} shall be well-formed and have the
following semantics:

\begin{itemdescr}
\pnum
\requires The calling thread has no ownership of the mutex.

\pnum
\effects
Blocks the calling thread until shared ownership of the mutex can be obtained for the calling thread.
If an exception is thrown then a shared lock shall not have been acquired for the current thread.

\pnum
\ensures
The calling thread has a shared lock on the mutex.

\pnum
\returntype \tcode{void}.

\pnum
\sync
Prior \tcode{unlock()} operations on the same object shall synchronize with\iref{intro.multithread} this operation.

\pnum
\throws
\tcode{system_error} when an exception is required\iref{thread.req.exception}.

\pnum
\errors
\begin{itemize}
\item \tcode{operation_not_permitted} --- if the thread does not have the privilege to perform the operation.
\item \tcode{resource_deadlock_would_occur} --- if the implementation detects that a deadlock would occur.
\end{itemize}
\end{itemdescr}

\pnum
The expression \tcode{m.unlock_shared()} shall be well-formed and have the following semantics:

\begin{itemdescr}
\pnum
\requires The calling thread shall hold a shared lock on the mutex.

\pnum
\effects
Releases a shared lock on the mutex held by the calling thread.

\pnum
\returntype \tcode{void}.

\pnum
\sync
This operation synchronizes with\iref{intro.multithread} subsequent
\tcode{lock()} operations that obtain ownership on the same object.

\pnum
\throws
Nothing.
\end{itemdescr}

\pnum
The expression \tcode{m.try_lock_shared()} shall be well-formed and have the following semantics:

\begin{itemdescr}
\pnum
\requires The calling thread has no ownership of the mutex.

\pnum
\effects
Attempts to obtain shared ownership of the mutex for the calling
thread without blocking. If shared ownership is not obtained, there is no
effect and \tcode{try_lock_shared()} immediately returns. An implementation
may fail to obtain the lock even if it is not held by any other thread.

\pnum
\returntype \tcode{bool}.

\pnum
\returns
\tcode{true} if the shared ownership lock was acquired, \tcode{false}
otherwise.

\pnum
\sync
If \tcode{try_lock_shared()} returns \tcode{true}, prior \tcode{unlock()}
operations on the same object synchronize with\iref{intro.multithread} this
operation.

\pnum
\throws
Nothing.
\end{itemdescr}

\rSec4[thread.sharedmutex.class]{Class \tcode{shared_mutex}}

\indexlibraryglobal{shared_mutex}%
\begin{codeblock}
namespace std {
  class shared_mutex {
  public:
    shared_mutex();
    ~shared_mutex();

    shared_mutex(const shared_mutex&) = delete;
    shared_mutex& operator=(const shared_mutex&) = delete;

    // exclusive ownership
    void lock();                // blocking
    bool try_lock();
    void unlock();

    // shared ownership
    void lock_shared();         // blocking
    bool try_lock_shared();
    void unlock_shared();

    using native_handle_type = @\impdefnc@;          // see~\ref{thread.req.native}
    native_handle_type native_handle();                         // see~\ref{thread.req.native}
  };
}
\end{codeblock}

\pnum
The class \tcode{shared_mutex} provides a non-recursive mutex
with shared ownership semantics.

\pnum
The class \tcode{shared_mutex} shall meet all of the
shared mutex requirements\iref{thread.sharedmutex.requirements}.
It shall be a standard-layout class\iref{class.prop}.

\pnum
The behavior of a program is undefined if:
\begin{itemize}
\item it destroys a \tcode{shared_mutex} object owned by any thread,
\item a thread attempts to recursively gain any ownership of a \tcode{shared_mutex}, or
\item a thread terminates while possessing any ownership of a \tcode{shared_mutex}.
\end{itemize}

\pnum
\tcode{shared_mutex} may be a synonym for \tcode{shared_timed_mutex}.

\rSec3[thread.sharedtimedmutex.requirements]{Shared timed mutex types}

\pnum
The standard library type \tcode{shared_timed_mutex} is a
\defn{shared timed mutex type}. Shared timed mutex types shall meet the requirements of
timed mutex types\iref{thread.timedmutex.requirements},
shared mutex types\iref{thread.sharedmutex.requirements}, and additionally
shall meet the requirements set out below. In this description,
\tcode{m} denotes an object of a shared timed mutex type,
\tcode{rel_type} denotes an object of an instantiation of
\tcode{duration}\iref{time.duration}, and
\tcode{abs_time} denotes an object of an instantiation of
\tcode{time_point}\iref{time.point}.

\pnum
The expression \tcode{m.try_lock_shared_for(rel_time)} shall be well-formed and
have the following semantics:

\begin{itemdescr}
\pnum
\requires The calling thread has no ownership of the mutex.

\pnum
\effects
Attempts to obtain
shared lock ownership for the calling thread within the relative
timeout\iref{thread.req.timing} specified by \tcode{rel_time}. If the time
specified by \tcode{rel_time} is less than or equal to \tcode{rel_time.zero()},
the function attempts to obtain ownership without blocking (as if by calling
\tcode{try_lock_shared()}). The function shall return within the timeout
specified by \tcode{rel_time} only if it has obtained shared ownership of the
mutex object.
\begin{note}
As with \tcode{try_lock()}, there is no guarantee that
ownership will be obtained if the lock is available, but implementations are
expected to make a strong effort to do so.
\end{note}
If an exception is thrown then a shared lock shall not have been acquired for
the current thread.

\pnum
\returntype \tcode{bool}.

\pnum
\returns
\tcode{true} if the shared lock was acquired, \tcode{false} otherwise.

\pnum
\sync
If \tcode{try_lock_shared_for()} returns \tcode{true}, prior
\tcode{unlock()} operations on the same object synchronize
with\iref{intro.multithread} this operation.

\pnum
\throws
Timeout-related exceptions\iref{thread.req.timing}.
\end{itemdescr}

\pnum
The expression \tcode{m.try_lock_shared_until(abs_time)} shall be well-formed
and have the following semantics:

\begin{itemdescr}
\pnum
\requires The calling thread has no ownership of the mutex.

\pnum
\effects
The function attempts to obtain shared ownership of the mutex. If
\tcode{abs_time} has already passed, the function attempts to obtain shared
ownership without blocking (as if by calling \tcode{try_lock_shared()}). The
function shall return before the absolute timeout\iref{thread.req.timing}
specified by \tcode{abs_time} only if it has obtained shared ownership of the
mutex object.
\begin{note}
As with \tcode{try_lock()}, there is no guarantee that
ownership will be obtained if the lock is available, but implementations are
expected to make a strong effort to do so.
\end{note}
If an exception is thrown then a shared lock shall not have been acquired for
the current thread.

\pnum
\returntype \tcode{bool}.

\pnum
\returns
\tcode{true} if the shared lock was acquired, \tcode{false} otherwise.

\pnum
\sync
If \tcode{try_lock_shared_until()} returns \tcode{true}, prior
\tcode{unlock()} operations on the same object synchronize
with\iref{intro.multithread} this operation.

\pnum
\throws
Timeout-related exceptions\iref{thread.req.timing}.
\end{itemdescr}

\rSec4[thread.sharedtimedmutex.class]{Class \tcode{shared_timed_mutex}}

\indexlibraryglobal{shared_timed_mutex}%
\begin{codeblock}
namespace std {
  class shared_timed_mutex {
  public:
    shared_timed_mutex();
    ~shared_timed_mutex();

    shared_timed_mutex(const shared_timed_mutex&) = delete;
    shared_timed_mutex& operator=(const shared_timed_mutex&) = delete;

    // exclusive ownership
    void lock();                // blocking
    bool try_lock();
    template<class Rep, class Period>
      bool try_lock_for(const chrono::duration<Rep, Period>& rel_time);
    template<class Clock, class Duration>
      bool try_lock_until(const chrono::time_point<Clock, Duration>& abs_time);
    void unlock();

    // shared ownership
    void lock_shared();         // blocking
    bool try_lock_shared();
    template<class Rep, class Period>
      bool try_lock_shared_for(const chrono::duration<Rep, Period>& rel_time);
    template<class Clock, class Duration>
      bool try_lock_shared_until(const chrono::time_point<Clock, Duration>& abs_time);
    void unlock_shared();
  };
}
\end{codeblock}

\pnum
The class \tcode{shared_timed_mutex} provides a non-recursive mutex with shared
ownership semantics.

\pnum
The class \tcode{shared_timed_mutex} shall meet all of the
shared timed mutex requirements\iref{thread.sharedtimedmutex.requirements}.
It shall be a standard-layout class\iref{class.prop}.

\pnum
The behavior of a program is undefined if:
\begin{itemize}
\item it destroys a \tcode{shared_timed_mutex} object owned by any thread,
\item a thread attempts to recursively gain any ownership of a \tcode{shared_timed_mutex}, or
\item a thread terminates while possessing any ownership of a \tcode{shared_timed_mutex}.
\end{itemize}

\rSec2[thread.lock]{Locks}

\pnum
A \term{lock} is an object that holds a reference to a lockable object and may unlock the
lockable object during the lock's destruction (such as when leaving block scope). An execution
agent may use a lock to aid in managing ownership of a lockable object in an exception safe
manner. A lock is said to \term{own} a lockable object if it is currently managing the
ownership of that lockable object for an execution agent. A lock does not manage the lifetime
of the lockable object it references.
\begin{note}
Locks are intended to ease the burden of
unlocking the lockable object under both normal and exceptional circumstances.
\end{note}

\pnum
Some lock constructors take tag types which describe what should be done with the lockable
object during the lock's construction.

\indexlibraryglobal{defer_lock_t}%
\indexlibraryglobal{try_to_lock_t}%
\indexlibraryglobal{adopt_lock_t}%
\indexlibraryglobal{defer_lock}%
\indexlibraryglobal{try_to_lock}%
\indexlibraryglobal{adopt_lock}%
\begin{codeblock}
namespace std {
  struct defer_lock_t  { };     // do not acquire ownership of the mutex
  struct try_to_lock_t { };     // try to acquire ownership of the mutex
                                // without blocking
  struct adopt_lock_t  { };     // assume the calling thread has already
                                // obtained mutex ownership and manage it

  inline constexpr defer_lock_t   defer_lock { };
  inline constexpr try_to_lock_t  try_to_lock { };
  inline constexpr adopt_lock_t   adopt_lock { };
}
\end{codeblock}

\rSec3[thread.lock.guard]{Class template \tcode{lock_guard}}

\indexlibraryglobal{lock_guard}%
\begin{codeblock}
namespace std {
  template<class Mutex>
  class lock_guard {
  public:
    using mutex_type = Mutex;

    explicit lock_guard(mutex_type& m);
    lock_guard(mutex_type& m, adopt_lock_t);
    ~lock_guard();

    lock_guard(const lock_guard&) = delete;
    lock_guard& operator=(const lock_guard&) = delete;

  private:
    mutex_type& pm;             // \expos
  };
}
\end{codeblock}

\pnum
An object of type \tcode{lock_guard} controls the ownership of a lockable object
within a scope. A \tcode{lock_guard} object maintains ownership of a lockable
object throughout the \tcode{lock_guard} object's lifetime\iref{basic.life}.
The behavior of a program is undefined if the lockable object referenced by
\tcode{pm} does not exist for the entire lifetime of the \tcode{lock_guard}
object. The supplied \tcode{Mutex} type shall meet the \oldconcept{BasicLockable}
requirements\iref{thread.req.lockable.basic}.

\indexlibraryctor{lock_guard}%
\begin{itemdecl}
explicit lock_guard(mutex_type& m);
\end{itemdecl}

\begin{itemdescr}
\pnum
\requires If \tcode{mutex_type} is not a recursive mutex,
the calling thread does not own the mutex \tcode{m}.

\pnum
\effects
Initializes \tcode{pm} with \tcode{m}. Calls \tcode{m.lock()}.
\end{itemdescr}

\indexlibraryctor{lock_guard}%
\begin{itemdecl}
lock_guard(mutex_type& m, adopt_lock_t);
\end{itemdecl}

\begin{itemdescr}
\pnum
\requires The calling thread owns the mutex \tcode{m}.

\pnum
\effects
Initializes \tcode{pm} with \tcode{m}.

\pnum
\throws
Nothing.
\end{itemdescr}

\indexlibrarydtor{lock_guard}%
\begin{itemdecl}
~lock_guard();
\end{itemdecl}

\begin{itemdescr}
\pnum
\effects
As if by \tcode{pm.unlock()}.
\end{itemdescr}


\rSec3[thread.lock.scoped]{Class template \tcode{scoped_lock}}

\indexlibraryglobal{scoped_lock}%
\begin{codeblock}
namespace std {
  template<class... MutexTypes>
  class scoped_lock {
  public:
    using mutex_type = Mutex;   // If \tcode{MutexTypes...} consists of the single type \tcode{Mutex}

    explicit scoped_lock(MutexTypes&... m);
    explicit scoped_lock(adopt_lock_t, MutexTypes&... m);
    ~scoped_lock();

    scoped_lock(const scoped_lock&) = delete;
    scoped_lock& operator=(const scoped_lock&) = delete;

  private:
    tuple<MutexTypes&...> pm;   // \expos
  };
}
\end{codeblock}

\pnum
An object of type \tcode{scoped_lock} controls the ownership of lockable objects
within a scope. A \tcode{scoped_lock} object maintains ownership of lockable
objects throughout the \tcode{scoped_lock} object's lifetime\iref{basic.life}.
The behavior of a program is undefined if the lockable objects referenced by
\tcode{pm} do not exist for the entire lifetime of the \tcode{scoped_lock}
object.
When \tcode{sizeof...(MutexTypes)} is \tcode{1},
the supplied \tcode{Mutex} type
shall meet the \oldconcept{BasicLockable} requirements\iref{thread.req.lockable.basic}.
Otherwise, each of the mutex types
shall meet the \oldconcept{Lockable} requirements\iref{thread.req.lockable.req}.

\indexlibraryctor{scoped_lock}%
\begin{itemdecl}
explicit scoped_lock(MutexTypes&... m);
\end{itemdecl}

\begin{itemdescr}
\pnum
\requires If a \tcode{MutexTypes} type is not a recursive mutex,
the calling thread does not own the corresponding mutex element of \tcode{m}.

\pnum
\effects
Initializes \tcode{pm} with \tcode{tie(m...)}.
Then if \tcode{sizeof...(MutexTypes)} is \tcode{0}, no effects.
Otherwise if \tcode{sizeof...(MutexTypes)} is \tcode{1}, then \tcode{m.lock()}.
Otherwise, \tcode{lock(m...)}.
\end{itemdescr}

\indexlibraryctor{scoped_lock}%
\begin{itemdecl}
explicit scoped_lock(adopt_lock_t, MutexTypes&... m);
\end{itemdecl}

\begin{itemdescr}
\pnum
\requires The calling thread owns all the mutexes in \tcode{m}.

\pnum
\effects
Initializes \tcode{pm} with \tcode{tie(m...)}.

\pnum
\throws
Nothing.
\end{itemdescr}

\indexlibrarydtor{scoped_lock}%
\begin{itemdecl}
~scoped_lock();
\end{itemdecl}

\begin{itemdescr}
\pnum
\effects
For all \tcode{i} in \range{0}{sizeof...(MutexTypes)},
\tcode{get<i>(pm).unlock()}.
\end{itemdescr}

\rSec3[thread.lock.unique]{Class template \tcode{unique_lock}}

\indexlibraryglobal{unique_lock}%
\begin{codeblock}
namespace std {
  template<class Mutex>
  class unique_lock {
  public:
    using mutex_type = Mutex;

    // \ref{thread.lock.unique.cons}, construct/copy/destroy
    unique_lock() noexcept;
    explicit unique_lock(mutex_type& m);
    unique_lock(mutex_type& m, defer_lock_t) noexcept;
    unique_lock(mutex_type& m, try_to_lock_t);
    unique_lock(mutex_type& m, adopt_lock_t);
    template<class Clock, class Duration>
      unique_lock(mutex_type& m, const chrono::time_point<Clock, Duration>& abs_time);
    template<class Rep, class Period>
      unique_lock(mutex_type& m, const chrono::duration<Rep, Period>& rel_time);
    ~unique_lock();

    unique_lock(const unique_lock&) = delete;
    unique_lock& operator=(const unique_lock&) = delete;

    unique_lock(unique_lock&& u) noexcept;
    unique_lock& operator=(unique_lock&& u);

    // \ref{thread.lock.unique.locking}, locking
    void lock();
    bool try_lock();

    template<class Rep, class Period>
      bool try_lock_for(const chrono::duration<Rep, Period>& rel_time);
    template<class Clock, class Duration>
      bool try_lock_until(const chrono::time_point<Clock, Duration>& abs_time);

    void unlock();

    // \ref{thread.lock.unique.mod}, modifiers
    void swap(unique_lock& u) noexcept;
    mutex_type* release() noexcept;

    // \ref{thread.lock.unique.obs}, observers
    bool owns_lock() const noexcept;
    explicit operator bool () const noexcept;
    mutex_type* mutex() const noexcept;

  private:
    mutex_type* pm;             // \expos
    bool owns;                  // \expos
  };

  template<class Mutex>
    void swap(unique_lock<Mutex>& x, unique_lock<Mutex>& y) noexcept;
}
\end{codeblock}

\pnum
An object of type \tcode{unique_lock} controls the ownership of a lockable
object within a scope. Ownership of the lockable object may be acquired at
construction or after construction, and may be transferred, after
acquisition, to another \tcode{unique_lock} object. Objects of type \tcode{unique_lock} are not
copyable but are movable. The behavior of a program is undefined if the contained pointer
\tcode{pm} is not null and the lockable object pointed
to by \tcode{pm} does not exist for the entire remaining
lifetime\iref{basic.life} of the \tcode{unique_lock} object. The supplied
\tcode{Mutex} type shall meet the \oldconcept{BasicLockable}
requirements\iref{thread.req.lockable.basic}.

\pnum
\begin{note}
\tcode{unique_lock<Mutex>} meets the \oldconcept{BasicLockable} requirements. If \tcode{Mutex}
meets the \oldconcept{Lockable} requirements\iref{thread.req.lockable.req},
\tcode{unique_lock<Mutex>} also meets the \oldconcept{Lockable} requirements;
if \tcode{Mutex}
meets the \oldconcept{TimedLockable} requirements\iref{thread.req.lockable.timed},
\tcode{unique_lock<Mutex>} also meets the \oldconcept{TimedLockable} requirements.
\end{note}

\rSec4[thread.lock.unique.cons]{Constructors, destructor, and assignment}

\indexlibraryctor{unique_lock}%
\begin{itemdecl}
unique_lock() noexcept;
\end{itemdecl}

\begin{itemdescr}
\pnum
\effects
Constructs an object of type \tcode{unique_lock}.

\pnum
\ensures
\tcode{pm == 0} and \tcode{owns == false}.
\end{itemdescr}

\indexlibraryctor{unique_lock}%
\begin{itemdecl}
explicit unique_lock(mutex_type& m);
\end{itemdecl}

\begin{itemdescr}
\pnum
\requires If \tcode{mutex_type} is not a recursive mutex the calling thread does not own the mutex.

\pnum
\effects
Constructs an object of type \tcode{unique_lock} and calls \tcode{m.lock()}.

\pnum
\ensures
\tcode{pm == addressof(m)} and \tcode{owns == true}.
\end{itemdescr}

\indexlibraryctor{unique_lock}%
\begin{itemdecl}
unique_lock(mutex_type& m, defer_lock_t) noexcept;
\end{itemdecl}

\begin{itemdescr}
\pnum
\effects
Constructs an object of type \tcode{unique_lock}.

\pnum
\ensures
\tcode{pm == addressof(m)} and \tcode{owns == false}.
\end{itemdescr}

\indexlibraryctor{unique_lock}%
\begin{itemdecl}
unique_lock(mutex_type& m, try_to_lock_t);
\end{itemdecl}

\begin{itemdescr}
\pnum
\requires
The supplied \tcode{Mutex} type shall meet the \oldconcept{Lockable}
requirements\iref{thread.req.lockable.req}.
If \tcode{mutex_type} is not a recursive mutex the calling thread does not own the mutex.

\pnum
\effects
Constructs an object of type \tcode{unique_lock} and calls \tcode{m.try_lock()}.

\pnum
\ensures
\tcode{pm == addressof(m)} and \tcode{owns == res},
where \tcode{res} is the value returned by the call to \tcode{m.try_lock()}.
\end{itemdescr}

\indexlibraryctor{unique_lock}%
\begin{itemdecl}
unique_lock(mutex_type& m, adopt_lock_t);
\end{itemdecl}

\begin{itemdescr}
\pnum
\requires The calling thread owns the mutex.

\pnum
\effects
Constructs an object of type \tcode{unique_lock}.

\pnum
\ensures
\tcode{pm == addressof(m)} and \tcode{owns == true}.

\pnum
\throws
Nothing.
\end{itemdescr}

\indexlibraryctor{unique_lock}%
\begin{itemdecl}
template<class Clock, class Duration>
  unique_lock(mutex_type& m, const chrono::time_point<Clock, Duration>& abs_time);
\end{itemdecl}

\begin{itemdescr}
\pnum
\requires If \tcode{mutex_type} is not a recursive mutex the calling thread
does not own the mutex. The supplied \tcode{Mutex} type shall meet the
\oldconcept{TimedLockable} requirements\iref{thread.req.lockable.timed}.

\pnum
\effects
Constructs an object of type \tcode{unique_lock} and calls \tcode{m.try_lock_until(abs_time)}.

\pnum
\ensures
\tcode{pm == addressof(m)} and \tcode{owns == res},
where \tcode{res} is
the value returned by the call to \tcode{m.try_lock_until(abs_time)}.
\end{itemdescr}

\indexlibraryctor{unique_lock}%
\begin{itemdecl}
template<class Rep, class Period>
  unique_lock(mutex_type& m, const chrono::duration<Rep, Period>& rel_time);
\end{itemdecl}

\begin{itemdescr}
\pnum
\requires If \tcode{mutex_type} is not a recursive mutex the calling thread does not own the mutex.
The supplied \tcode{Mutex} type shall meet the \oldconcept{TimedLockable} requirements\iref{thread.req.lockable.timed}.

\pnum
\effects
Constructs an object of type \tcode{unique_lock} and calls \tcode{m.try_lock_for(rel_time)}.

\pnum
\ensures
\tcode{pm == addressof(m)} and \tcode{owns == res},
where \tcode{res} is the value returned by the call to \tcode{m.try_lock_for(rel_time)}.
\end{itemdescr}

\indexlibraryctor{unique_lock}%
\begin{itemdecl}
unique_lock(unique_lock&& u) noexcept;
\end{itemdecl}

\begin{itemdescr}
\pnum
\ensures
\tcode{pm == u_p.pm} and \tcode{owns == u_p.owns} (where \tcode{u_p} is the state of \tcode{u} just prior to this construction),  \tcode{u.pm == 0} and \tcode{u.owns == false}.
\end{itemdescr}

\indexlibrarymember{operator=}{unique_lock}%
\begin{itemdecl}
unique_lock& operator=(unique_lock&& u);
\end{itemdecl}

\begin{itemdescr}
\pnum
\effects
If \tcode{owns} calls \tcode{pm->unlock()}.

\pnum
\ensures
\tcode{pm == u_p.pm} and \tcode{owns == u_p.owns} (where \tcode{u_p} is the state of \tcode{u} just prior to this construction),  \tcode{u.pm == 0} and \tcode{u.owns == false}.

\pnum
\begin{note}
With a recursive mutex it is possible for both \tcode{*this} and \tcode{u} to own the same mutex before the assignment. In this case, \tcode{*this} will own the mutex after the assignment and \tcode{u} will not.
\end{note}

\pnum
\throws
Nothing.
\end{itemdescr}


\indexlibrarydtor{unique_lock}%
\begin{itemdecl}
~unique_lock();
\end{itemdecl}

\begin{itemdescr}
\pnum
\effects
If \tcode{owns} calls \tcode{pm->unlock()}.
\end{itemdescr}

\rSec4[thread.lock.unique.locking]{Locking}

\indexlibrarymember{lock}{unique_lock}%
\begin{itemdecl}
void lock();
\end{itemdecl}

\begin{itemdescr}
\pnum
\effects
As if by \tcode{pm->lock()}.

\pnum
\ensures
\tcode{owns == true}.

\pnum
\throws
Any exception thrown by \tcode{pm->lock()}. \tcode{system_error} when an exception
is required\iref{thread.req.exception}.

\pnum
\errors
\begin{itemize}
\item \tcode{operation_not_permitted} --- if \tcode{pm} is \tcode{nullptr}.
\item \tcode{resource_deadlock_would_occur} --- if on entry \tcode{owns}
is \tcode{true}.
\end{itemize}
\end{itemdescr}

\indexlibrarymember{try_lock}{unique_lock}%
\begin{itemdecl}
bool try_lock();
\end{itemdecl}

\begin{itemdescr}
\pnum
\requires The supplied \tcode{Mutex} shall meet the \oldconcept{Lockable}
requirements\iref{thread.req.lockable.req}.

\pnum
\effects
As if by \tcode{pm->try_lock()}.

\pnum
\returns
The value returned by the call to \tcode{try_lock()}.

\pnum
\ensures
\tcode{owns == res}, where \tcode{res} is the value returned by
the call to \tcode{try_lock()}.

\pnum
\throws
Any exception thrown by \tcode{pm->try_lock()}. \tcode{system_error} when an exception
is required\iref{thread.req.exception}.

\pnum
\errors
\begin{itemize}
\item \tcode{operation_not_permitted} --- if \tcode{pm} is \tcode{nullptr}.
\item \tcode{resource_deadlock_would_occur} --- if on entry \tcode{owns}
is \tcode{true}.
\end{itemize}
\end{itemdescr}

\indexlibrarymember{try_lock_until}{unique_lock}%
\begin{itemdecl}
template<class Clock, class Duration>
  bool try_lock_until(const chrono::time_point<Clock, Duration>& abs_time);
\end{itemdecl}

\begin{itemdescr}
\pnum
\requires The supplied \tcode{Mutex} type shall meet the \oldconcept{TimedLockable}
requirements\iref{thread.req.lockable.timed}.

\pnum
\effects
As if by \tcode{pm->try_lock_until(abs_time)}.

\pnum
\returns
The value returned by the call to \tcode{try_lock_until(abs_time)}.

\pnum
\ensures
\tcode{owns == res}, where \tcode{res} is the value returned by
the call to \tcode{try_lock_until(abs_time)}.

\pnum
\throws
Any exception thrown by \tcode{pm->try_lock_until()}. \tcode{system_error} when an
exception is required\iref{thread.req.exception}.

\pnum
\errors
\begin{itemize}
\item \tcode{operation_not_permitted} --- if \tcode{pm} is \tcode{nullptr}.
\item \tcode{resource_deadlock_would_occur} --- if on entry \tcode{owns} is
\tcode{true}.
\end{itemize}
\end{itemdescr}

\indexlibrarymember{try_lock_for}{unique_lock}%
\begin{itemdecl}
template<class Rep, class Period>
  bool try_lock_for(const chrono::duration<Rep, Period>& rel_time);
\end{itemdecl}

\begin{itemdescr}
\pnum
\requires The supplied \tcode{Mutex} type shall meet the \oldconcept{TimedLockable} requirements\iref{thread.req.lockable.timed}.

\pnum
\effects
As if by \tcode{pm->try_lock_for(rel_time)}.

\pnum
\returns
The value returned by the call to \tcode{try_lock_until(rel_time)}.

\pnum
\ensures
\tcode{owns == res}, where \tcode{res} is the value returned by the call to \tcode{try_lock_for(rel_time)}.

\pnum
\throws
Any exception thrown by \tcode{pm->try_lock_for()}. \tcode{system_error} when an
exception is required\iref{thread.req.exception}.

\pnum
\errors
\begin{itemize}
\item \tcode{operation_not_permitted} --- if \tcode{pm} is \tcode{nullptr}.
\item \tcode{resource_deadlock_would_occur} --- if on entry \tcode{owns} is
\tcode{true}.
\end{itemize}
\end{itemdescr}

\indexlibrarymember{unlock}{unique_lock}%
\begin{itemdecl}
void unlock();
\end{itemdecl}

\begin{itemdescr}
\pnum
\effects
As if by \tcode{pm->unlock()}.

\pnum
\ensures
\tcode{owns == false}.

\pnum
\throws
\tcode{system_error} when
an exception is required\iref{thread.req.exception}.

\pnum
\errors
\begin{itemize}
\item \tcode{operation_not_permitted} --- if on entry \tcode{owns} is \tcode{false}.
\end{itemize}
\end{itemdescr}

\rSec4[thread.lock.unique.mod]{Modifiers}

\indexlibrarymember{swap}{unique_lock}%
\begin{itemdecl}
void swap(unique_lock& u) noexcept;
\end{itemdecl}

\begin{itemdescr}
\pnum
\effects
Swaps the data members of \tcode{*this} and \tcode{u}.
\end{itemdescr}

\indexlibrarymember{release}{unique_lock}%
\begin{itemdecl}
mutex_type* release() noexcept;
\end{itemdecl}

\begin{itemdescr}
\pnum
\returns
The previous value of \tcode{pm}.

\pnum
\ensures
\tcode{pm == 0} and \tcode{owns == false}.
\end{itemdescr}

\indexlibrarymember{swap}{unique_lock}%
\begin{itemdecl}
template<class Mutex>
  void swap(unique_lock<Mutex>& x, unique_lock<Mutex>& y) noexcept;
\end{itemdecl}

\begin{itemdescr}
\pnum
\effects
As if by \tcode{x.swap(y)}.
\end{itemdescr}

\rSec4[thread.lock.unique.obs]{Observers}

\indexlibrarymember{owns_lock}{unique_lock}%
\begin{itemdecl}
bool owns_lock() const noexcept;
\end{itemdecl}

\begin{itemdescr}
\pnum
\returns
\tcode{owns}.
\end{itemdescr}

\indexlibrarymember{operator bool}{unique_lock}%
\begin{itemdecl}
explicit operator bool() const noexcept;
\end{itemdecl}

\begin{itemdescr}
\pnum
\returns
\tcode{owns}.
\end{itemdescr}

\indexlibrarymember{mutex}{unique_lock}%
\begin{itemdecl}
mutex_type *mutex() const noexcept;
\end{itemdecl}

\begin{itemdescr}
\pnum
\returns
\tcode{pm}.
\end{itemdescr}

\rSec3[thread.lock.shared]{Class template \tcode{shared_lock}}

\indexlibraryglobal{shared_lock}%
\begin{codeblock}
namespace std {
  template<class Mutex>
  class shared_lock {
  public:
    using mutex_type = Mutex;

    // \ref{thread.lock.shared.cons}, construct/copy/destroy
    shared_lock() noexcept;
    explicit shared_lock(mutex_type& m);        // blocking
    shared_lock(mutex_type& m, defer_lock_t) noexcept;
    shared_lock(mutex_type& m, try_to_lock_t);
    shared_lock(mutex_type& m, adopt_lock_t);
    template<class Clock, class Duration>
      shared_lock(mutex_type& m, const chrono::time_point<Clock, Duration>& abs_time);
    template<class Rep, class Period>
      shared_lock(mutex_type& m, const chrono::duration<Rep, Period>& rel_time);
    ~shared_lock();

    shared_lock(const shared_lock&) = delete;
    shared_lock& operator=(const shared_lock&) = delete;

    shared_lock(shared_lock&& u) noexcept;
    shared_lock& operator=(shared_lock&& u) noexcept;

    // \ref{thread.lock.shared.locking}, locking
    void lock();                                // blocking
    bool try_lock();
    template<class Rep, class Period>
      bool try_lock_for(const chrono::duration<Rep, Period>& rel_time);
    template<class Clock, class Duration>
      bool try_lock_until(const chrono::time_point<Clock, Duration>& abs_time);
    void unlock();

    // \ref{thread.lock.shared.mod}, modifiers
    void swap(shared_lock& u) noexcept;
    mutex_type* release() noexcept;

    // \ref{thread.lock.shared.obs}, observers
    bool owns_lock() const noexcept;
    explicit operator bool () const noexcept;
    mutex_type* mutex() const noexcept;

  private:
    mutex_type* pm;                             // \expos
    bool owns;                                  // \expos
  };

  template<class Mutex>
    void swap(shared_lock<Mutex>& x, shared_lock<Mutex>& y) noexcept;
}
\end{codeblock}

\pnum
An object of type \tcode{shared_lock} controls the shared ownership of a
lockable object within a scope. Shared ownership of the lockable object may be
acquired at construction or after construction, and may be transferred, after
acquisition, to another \tcode{shared_lock} object. Objects of type
\tcode{shared_lock} are not copyable but are movable. The behavior of a program
is undefined if the contained pointer \tcode{pm} is not null and the lockable
object pointed to by \tcode{pm} does not exist for the entire remaining
lifetime\iref{basic.life} of the \tcode{shared_lock} object. The supplied
\tcode{Mutex} type shall meet the shared mutex
requirements\iref{thread.sharedtimedmutex.requirements}.

\pnum
\begin{note}
\tcode{shared_lock<Mutex>} meets the \oldconcept{TimedLockable}
requirements\iref{thread.req.lockable.timed}.
\end{note}

\rSec4[thread.lock.shared.cons]{Constructors, destructor, and assignment}

\indexlibraryctor{shared_lock}%
\begin{itemdecl}
shared_lock() noexcept;
\end{itemdecl}

\begin{itemdescr}
\pnum
\effects
Constructs an object of type \tcode{shared_lock}.

\pnum
\ensures
\tcode{pm == nullptr} and \tcode{owns == false}.
\end{itemdescr}

\indexlibraryctor{shared_lock}%
\begin{itemdecl}
explicit shared_lock(mutex_type& m);
\end{itemdecl}

\begin{itemdescr}
\pnum
\requires The calling thread does not own the mutex for any ownership mode.

\pnum
\effects
Constructs an object of type \tcode{shared_lock} and calls \tcode{m.lock_shared()}.

\pnum
\ensures
\tcode{pm == addressof(m)} and \tcode{owns == true}.
\end{itemdescr}

\indexlibraryctor{shared_lock}%
\begin{itemdecl}
shared_lock(mutex_type& m, defer_lock_t) noexcept;
\end{itemdecl}

\begin{itemdescr}
\pnum
\effects
Constructs an object of type \tcode{shared_lock}.

\pnum
\ensures
\tcode{pm == addressof(m)} and \tcode{owns == false}.
\end{itemdescr}

\indexlibraryctor{shared_lock}%
\begin{itemdecl}
shared_lock(mutex_type& m, try_to_lock_t);
\end{itemdecl}

\begin{itemdescr}
\pnum
\requires The calling thread does not own the mutex for any ownership mode.

\pnum
\effects
Constructs an object of type \tcode{shared_lock} and calls \tcode{m.try_lock_shared()}.

\pnum
\ensures
\tcode{pm == addressof(m)} and \tcode{owns == res}
where \tcode{res} is the
value returned by the call to \tcode{m.try_lock_shared()}.
\end{itemdescr}

\indexlibraryctor{shared_lock}%
\begin{itemdecl}
shared_lock(mutex_type& m, adopt_lock_t);
\end{itemdecl}

\begin{itemdescr}
\pnum
\requires The calling thread has shared ownership of the mutex.

\pnum
\effects
Constructs an object of type \tcode{shared_lock}.

\pnum
\ensures
\tcode{pm == addressof(m)} and \tcode{owns == true}.
\end{itemdescr}

\indexlibraryctor{shared_lock}%
\begin{itemdecl}
template<class Clock, class Duration>
  shared_lock(mutex_type& m,
              const chrono::time_point<Clock, Duration>& abs_time);
\end{itemdecl}

\begin{itemdescr}
\pnum
\requires The calling thread does not own the mutex for any ownership mode.

\pnum
\effects
Constructs an object of type \tcode{shared_lock} and calls
\tcode{m.try_lock_shared_until(abs_time)}.

\pnum
\ensures
\tcode{pm == addressof(m)} and \tcode{owns == res}
where \tcode{res}
is the value returned by the call to \tcode{m.try_lock_shared_until(abs_time)}.
\end{itemdescr}

\indexlibraryctor{shared_lock}%
\begin{itemdecl}
template<class Rep, class Period>
  shared_lock(mutex_type& m,
              const chrono::duration<Rep, Period>& rel_time);
\end{itemdecl}

\begin{itemdescr}
\pnum
\requires The calling thread does not own the mutex for any ownership mode.

\pnum
\effects
Constructs an object of type \tcode{shared_lock} and calls
\tcode{m.try_lock_shared_for(rel_time)}.

\pnum
\ensures
\tcode{pm == addressof(m)} and \tcode{owns == res}
where \tcode{res} is
the value returned by the call to \tcode{m.try_lock_shared_for(rel_time)}.
\end{itemdescr}

\indexlibrarydtor{shared_lock}%
\begin{itemdecl}
~shared_lock();
\end{itemdecl}

\begin{itemdescr}
\pnum
\effects
If \tcode{owns} calls \tcode{pm->unlock_shared()}.
\end{itemdescr}

\indexlibraryctor{shared_lock}%
\begin{itemdecl}
shared_lock(shared_lock&& sl) noexcept;
\end{itemdecl}

\begin{itemdescr}
\pnum
\ensures
\tcode{pm == sl_p.pm} and \tcode{owns == sl_p.owns} (where
\tcode{sl_p} is the state of \tcode{sl} just prior to this construction),
\tcode{sl.pm == nullptr} and \tcode{sl.owns == false}.
\end{itemdescr}

\indexlibrarymember{operator=}{shared_lock}%
\begin{itemdecl}
shared_lock& operator=(shared_lock&& sl) noexcept;
\end{itemdecl}

\begin{itemdescr}
\pnum
\effects
If \tcode{owns} calls \tcode{pm->unlock_shared()}.

\pnum
\ensures
\tcode{pm == sl_p.pm} and \tcode{owns == sl_p.owns} (where
\tcode{sl_p} is the state of \tcode{sl} just prior to this assignment),
\tcode{sl.pm == nullptr} and \tcode{sl.owns == false}.
\end{itemdescr}

\rSec4[thread.lock.shared.locking]{Locking}

\indexlibrarymember{lock}{shared_lock}%
\begin{itemdecl}
void lock();
\end{itemdecl}

\begin{itemdescr}
\pnum
\effects
As if by \tcode{pm->lock_shared()}.

\pnum
\ensures
\tcode{owns == true}.

\pnum
\throws
Any exception thrown by \tcode{pm->lock_shared()}.
\tcode{system_error} when an exception is required\iref{thread.req.exception}.

\pnum
\errors
\begin{itemize}
\item \tcode{operation_not_permitted} --- if \tcode{pm} is \tcode{nullptr}.
\item \tcode{resource_deadlock_would_occur} --- if on entry \tcode{owns} is
\tcode{true}.
\end{itemize}
\end{itemdescr}

\indexlibrarymember{try_lock}{shared_lock}%
\begin{itemdecl}
bool try_lock();
\end{itemdecl}

\begin{itemdescr}
\pnum
\effects
As if by \tcode{pm->try_lock_shared()}.

\pnum
\returns
The value returned by the call to \tcode{pm->try_lock_shared()}.

\pnum
\ensures
\tcode{owns == res}, where \tcode{res} is the value returned by
the call to \tcode{pm->try_lock_shared()}.

\pnum
\throws
Any exception thrown by \tcode{pm->try_lock_shared()}.
\tcode{system_error} when an exception is required\iref{thread.req.exception}.

\pnum
\errors
\begin{itemize}
\item \tcode{operation_not_permitted} --- if \tcode{pm} is \tcode{nullptr}.
\item \tcode{resource_deadlock_would_occur} --- if on entry \tcode{owns} is
\tcode{true}.
\end{itemize}
\end{itemdescr}

\indexlibrarymember{try_lock_until}{shared_lock}%
\begin{itemdecl}
template<class Clock, class Duration>
  bool try_lock_until(const chrono::time_point<Clock, Duration>& abs_time);
\end{itemdecl}

\begin{itemdescr}
\pnum
\effects
As if by \tcode{pm->try_lock_shared_until(abs_time)}.

\pnum
\returns
The value returned by the call to
\tcode{pm->try_lock_shared_until(abs_time)}.

\pnum
\ensures
\tcode{owns == res}, where \tcode{res} is the value returned by
the call to \tcode{pm->try_lock_shared_until(abs_time)}.

\pnum
\throws
Any exception thrown by \tcode{pm->try_lock_shared_until(abs_time)}.
\tcode{system_error} when an exception is required\iref{thread.req.exception}.

\pnum
\errors
\begin{itemize}
\item \tcode{operation_not_permitted} --- if \tcode{pm} is \tcode{nullptr}.
\item \tcode{resource_deadlock_would_occur} --- if on entry \tcode{owns} is
\tcode{true}.
\end{itemize}
\end{itemdescr}

\indexlibrarymember{try_lock_for}{shared_lock}%
\begin{itemdecl}
template<class Rep, class Period>
  bool try_lock_for(const chrono::duration<Rep, Period>& rel_time);
\end{itemdecl}

\begin{itemdescr}
\pnum
\effects
As if by \tcode{pm->try_lock_shared_for(rel_time)}.

\pnum
\returns
The value returned by the call to \tcode{pm->try_lock_shared_for(rel_time)}.

\pnum
\ensures
\tcode{owns == res}, where \tcode{res} is the value returned by the call to \tcode{pm->try_lock_shared_for(rel_time)}.

\pnum
\throws
Any exception thrown by \tcode{pm->try_lock_shared_for(rel_time)}. \tcode{system_error} when an exception is required\iref{thread.req.exception}.

\pnum
\errors
\begin{itemize}
\item \tcode{operation_not_permitted} --- if \tcode{pm} is \tcode{nullptr}.
\item \tcode{resource_deadlock_would_occur} --- if on entry \tcode{owns} is
\tcode{true}.
\end{itemize}
\end{itemdescr}

\indexlibrarymember{unlock}{shared_lock}%
\begin{itemdecl}
void unlock();
\end{itemdecl}

\begin{itemdescr}
\pnum
\effects
As if by \tcode{pm->unlock_shared()}.

\pnum
\ensures
\tcode{owns == false}.

\pnum
\throws
\tcode{system_error} when an exception is required\iref{thread.req.exception}.

\pnum
\errors
\begin{itemize}
\item \tcode{operation_not_permitted} --- if on entry \tcode{owns} is
\tcode{false}.
\end{itemize}
\end{itemdescr}

\rSec4[thread.lock.shared.mod]{Modifiers}

\indexlibrarymember{swap}{shared_lock}%
\begin{itemdecl}
void swap(shared_lock& sl) noexcept;
\end{itemdecl}

\begin{itemdescr}
\pnum
\effects
Swaps the data members of \tcode{*this} and \tcode{sl}.
\end{itemdescr}

\indexlibrarymember{release}{shared_lock}%
\begin{itemdecl}
mutex_type* release() noexcept;
\end{itemdecl}

\begin{itemdescr}
\pnum
\returns
The previous value of \tcode{pm}.

\pnum
\ensures
\tcode{pm == nullptr} and \tcode{owns == false}.
\end{itemdescr}

\indexlibrarymember{swap}{shared_lock}%
\begin{itemdecl}
template<class Mutex>
  void swap(shared_lock<Mutex>& x, shared_lock<Mutex>& y) noexcept;
\end{itemdecl}

\begin{itemdescr}
\pnum
\effects
As if by \tcode{x.swap(y)}.
\end{itemdescr}

\rSec4[thread.lock.shared.obs]{Observers}

\indexlibrarymember{owns_lock}{shared_lock}%
\begin{itemdecl}
bool owns_lock() const noexcept;
\end{itemdecl}

\begin{itemdescr}
\pnum
\returns
\tcode{owns}.
\end{itemdescr}

\indexlibrarymember{operator bool}{shared_lock}%
\begin{itemdecl}
explicit operator bool() const noexcept;
\end{itemdecl}

\begin{itemdescr}
\pnum
\returns
\tcode{owns}.
\end{itemdescr}

\indexlibrarymember{mutex}{shared_lock}%
\begin{itemdecl}
mutex_type* mutex() const noexcept;
\end{itemdecl}

\begin{itemdescr}
\pnum
\returns
\tcode{pm}.
\end{itemdescr}

\rSec2[thread.lock.algorithm]{Generic locking algorithms}

\indexlibraryglobal{try_lock}%
\begin{itemdecl}
template<class L1, class L2, class... L3> int try_lock(L1&, L2&, L3&...);
\end{itemdecl}

\begin{itemdescr}
\pnum
\requires Each template parameter type shall meet the \oldconcept{Lockable} requirements.
\begin{note}
The
\tcode{unique_lock} class template meets these requirements when suitably instantiated.
\end{note}

\pnum
\effects
Calls \tcode{try_lock()} for each argument in order beginning with the
first until all arguments have been processed or a call to \tcode{try_lock()} fails,
either by returning \tcode{false} or by throwing an exception. If a call to
\tcode{try_lock()} fails, \tcode{unlock()} is called for all prior arguments
with no further calls to \tcode{try_lock()}.

\pnum
\returns
\tcode{-1} if all calls to \tcode{try_lock()} returned \tcode{true},
otherwise a zero-based index value that indicates the argument for which \tcode{try_lock()}
returned \tcode{false}.
\end{itemdescr}

\indexlibraryglobal{lock}%
\begin{itemdecl}
template<class L1, class L2, class... L3> void lock(L1&, L2&, L3&...);
\end{itemdecl}

\begin{itemdescr}
\pnum
\requires Each template parameter type shall meet the \oldconcept{Lockable} requirements.
\begin{note}
The
\tcode{unique_lock} class template meets these requirements when suitably instantiated.
\end{note}

\pnum
\effects
All arguments are locked via a sequence of calls to \tcode{lock()},
\tcode{try_lock()}, or \tcode{unlock()} on each argument. The sequence of calls does
not result in deadlock, but is otherwise unspecified.
\begin{note}
A deadlock avoidance
algorithm such as try-and-back-off must be used, but the specific algorithm is not
specified to avoid over-constraining implementations.
\end{note}
If a call to
\tcode{lock()} or \tcode{try_lock()} throws an exception, \tcode{unlock()} is
called for any argument that had been locked by a call to \tcode{lock()} or
\tcode{try_lock()}.
\end{itemdescr}

\rSec2[thread.once]{Call once}

\rSec3[thread.once.onceflag]{Struct \tcode{once_flag}}

\indexlibraryglobal{once_flag}%
\begin{codeblock}
namespace std {
  struct once_flag {
    constexpr once_flag() noexcept;

    once_flag(const once_flag&) = delete;
    once_flag& operator=(const once_flag&) = delete;
  };
}
\end{codeblock}

\pnum
The class \tcode{once_flag} is an opaque data structure that \tcode{call_once} uses to
initialize data without causing a data race or deadlock.

\indexlibraryglobal{once_flag}%
\begin{itemdecl}
constexpr once_flag() noexcept;
\end{itemdecl}

\begin{itemdescr}
\pnum
\effects
Constructs an object of type \tcode{once_flag}.

\pnum
\sync
The construction of a \tcode{once_flag} object is not synchronized.

\pnum
\ensures
The object's internal state is set to indicate to an invocation of
\tcode{call_once} with the object as its initial argument that no function has been
called.
\end{itemdescr}

\rSec3[thread.once.callonce]{Function \tcode{call_once}}

\indexlibraryglobal{call_once}%
\begin{itemdecl}
template<class Callable, class... Args>
  void call_once(once_flag& flag, Callable&& func, Args&&... args);
\end{itemdecl}

\begin{itemdescr}
\pnum
\requires
\begin{codeblock}
@\placeholdernc{INVOKE}@(std::forward<Callable>(func), std::forward<Args>(args)...)
\end{codeblock}
(see \ref{func.require}) shall be a valid expression.

\pnum
\effects
An execution of \tcode{call_once} that does not call its \tcode{func} is a
\term{passive} execution. An execution of \tcode{call_once} that calls its \tcode{func}
is an \term{active} execution. An active execution shall call
\tcode{\placeholdernc{INVOKE}(\brk{}%
std::forward<Callable>(func),
std::forward<Args>(args)...)}. If such a call to \tcode{func}
throws an exception the execution is \term{exceptional}, otherwise it is \term{returning}.
An exceptional execution shall propagate the exception to the caller of
\tcode{call_once}. Among all executions of \tcode{call_once} for any given
\tcode{once_flag}: at most one shall be a returning execution; if there is a
returning execution, it shall be the last active execution; and there are
passive executions only if there is a returning execution.
\begin{note}
Passive
executions allow other threads to reliably observe the results produced by the
earlier returning execution.
\end{note}

\pnum
\sync
For any given \tcode{once_flag}: all active executions occur in a total
order; completion of an active execution synchronizes with\iref{intro.multithread}
the start of the next one in this total order; and the returning execution
synchronizes with the return from all passive executions.

\pnum
\throws
\tcode{system_error} when
an exception is required\iref{thread.req.exception}, or any exception thrown by \tcode{func}.

\pnum
\begin{example}
\begin{codeblock}
// global flag, regular function
void init();
std::once_flag flag;

void f() {
  std::call_once(flag, init);
}

// function static flag, function object
struct initializer {
  void operator()();
};

void g() {
  static std::once_flag flag2;
  std::call_once(flag2, initializer());
}

// object flag, member function
class information {
  std::once_flag verified;
  void verifier();
public:
  void verify() { std::call_once(verified, &information::verifier, *this); }
};
\end{codeblock}
\end{example}
\end{itemdescr}


\rSec1[thread.condition]{Condition variables}

\pnum
Condition variables provide synchronization primitives used to block a thread until
notified by some other thread that some condition is met or until a system time is
reached. Class \tcode{condition_variable} provides a condition variable that can only
wait on an object of type \tcode{unique_lock<mutex>}, allowing the implementation
to be more efficient. Class \tcode{condition_variable_any} provides a general
condition variable that can wait on objects of user-supplied lock types.

\pnum
Condition variables permit concurrent invocation of the \tcode{wait}, \tcode{wait_for},
\tcode{wait_until}, \tcode{notify_one} and \tcode{notify_all} member functions.

\pnum
The execution of \tcode{notify_one} and \tcode{notify_all} shall be atomic. The
execution of \tcode{wait}, \tcode{wait_for}, and \tcode{wait_until} shall be performed
in three atomic parts:

\begin{enumerate}
\item the release of the mutex and entry into the waiting state;
\item the unblocking of the wait; and
\item the reacquisition of the lock.
\end{enumerate}

\pnum
The implementation shall behave as if all executions of \tcode{notify_one}, \tcode{notify_all}, and each
part of the \tcode{wait}, \tcode{wait_for}, and \tcode{wait_until} executions are
executed in a single unspecified total order consistent with the "happens before" order.

\pnum
Condition variable construction and destruction need not be synchronized.

\rSec2[condition.variable.syn]{Header \tcode{<condition_variable>} synopsis}

\indexheader{condition_variable}%
\indexlibraryglobal{cv_status}%
\begin{codeblock}
namespace std {
  class condition_variable;
  class condition_variable_any;

  void notify_all_at_thread_exit(condition_variable& cond, unique_lock<mutex> lk);

  enum class cv_status { no_timeout, timeout };
}
\end{codeblock}

\rSec2[thread.condition.nonmember]{Non-member functions}
\indexlibraryglobal{notify_all_at_thread_exit}%
\begin{itemdecl}
void notify_all_at_thread_exit(condition_variable& cond, unique_lock<mutex> lk);
\end{itemdecl}

\begin{itemdescr}
\pnum
\requires \tcode{lk} is locked by the calling thread and either

\begin{itemize}
\item no other thread is waiting on \tcode{cond}, or
\item \tcode{lk.mutex()} returns the same value for each of the lock arguments
supplied by all concurrently waiting (via \tcode{wait}, \tcode{wait_for},
or \tcode{wait_until}) threads.
\end{itemize}

\pnum
\effects
Transfers ownership of the lock associated with \tcode{lk} into
internal storage and schedules \tcode{cond} to be notified when the current
thread exits, after all objects of thread storage duration associated with
the current thread have been destroyed. This notification shall be as if:
\begin{codeblock}
lk.unlock();
cond.notify_all();
\end{codeblock}

\pnum
\sync
The implied \tcode{lk.unlock()} call is sequenced after the destruction of
all objects with thread storage duration associated with the current thread.

\pnum
\begin{note}
The supplied lock will be held until the thread exits, and care
should be taken to ensure that this does not cause deadlock due to lock
ordering issues. After calling \tcode{notify_all_at_thread_exit} it is
recommended that the thread should be exited as soon as possible, and
that no blocking or time-consuming tasks are run on that thread.
\end{note}

\pnum
\begin{note}
It is the user's responsibility to ensure that waiting threads
do not erroneously assume that the thread has finished if they experience
spurious wakeups. This typically requires that the condition being waited
for is satisfied while holding the lock on \tcode{lk}, and that this lock
is not released and reacquired prior to calling \tcode{notify_all_at_thread_exit}.
\end{note}
\end{itemdescr}

\rSec2[thread.condition.condvar]{Class \tcode{condition_variable}}

\indexlibraryglobal{condition_variable}%
\begin{codeblock}
namespace std {
  class condition_variable {
  public:
    condition_variable();
    ~condition_variable();

    condition_variable(const condition_variable&) = delete;
    condition_variable& operator=(const condition_variable&) = delete;

    void notify_one() noexcept;
    void notify_all() noexcept;
    void wait(unique_lock<mutex>& lock);
    template<class Predicate>
      void wait(unique_lock<mutex>& lock, Predicate pred);
    template<class Clock, class Duration>
      cv_status wait_until(unique_lock<mutex>& lock,
                           const chrono::time_point<Clock, Duration>& abs_time);
    template<class Clock, class Duration, class Predicate>
      bool wait_until(unique_lock<mutex>& lock,
                      const chrono::time_point<Clock, Duration>& abs_time,
                      Predicate pred);
    template<class Rep, class Period>
      cv_status wait_for(unique_lock<mutex>& lock,
                         const chrono::duration<Rep, Period>& rel_time);
    template<class Rep, class Period, class Predicate>
      bool wait_for(unique_lock<mutex>& lock,
                    const chrono::duration<Rep, Period>& rel_time,
                    Predicate pred);

    using native_handle_type = @\impdefnc@;          // see~\ref{thread.req.native}
    native_handle_type native_handle();                         // see~\ref{thread.req.native}
  };
}
\end{codeblock}

\pnum
The class \tcode{condition_variable} shall be a standard-layout class\iref{class.prop}.

\indexlibraryctor{condition_variable}%
\begin{itemdecl}
condition_variable();
\end{itemdecl}

\begin{itemdescr}
\pnum
\effects
Constructs an object of type \tcode{condition_variable}.

\pnum
\throws
\tcode{system_error} when an exception is required\iref{thread.req.exception}.

\pnum
\errors
\begin{itemize}
\item \tcode{resource_unavailable_try_again} --- if some non-memory resource
limitation prevents initialization.
\end{itemize}
\end{itemdescr}

\indexlibrarydtor{condition_variable}%
\begin{itemdecl}
~condition_variable();
\end{itemdecl}

\begin{itemdescr}
\pnum
\requires There shall be no thread blocked on \tcode{*this}.
\begin{note}
That is, all
threads shall have been notified; they may subsequently block on the lock specified in the
wait.
This relaxes the usual rules, which would have required all wait calls to happen before
destruction. Only the notification to unblock the wait needs to happen before destruction.
The user should take care to ensure that no threads wait on \tcode{*this} once the destructor has
been started, especially when the waiting threads are calling the wait functions in a loop or
using the overloads of \tcode{wait}, \tcode{wait_for}, or \tcode{wait_until} that take a predicate.
\end{note}

\pnum
\effects
Destroys the object.
\end{itemdescr}

\indexlibrarymember{notify_one}{condition_variable}%
\begin{itemdecl}
void notify_one() noexcept;
\end{itemdecl}

\begin{itemdescr}
\pnum
\effects
If any threads are blocked waiting for \tcode{*this}, unblocks one of those threads.
\end{itemdescr}

\indexlibrarymember{notify_all}{condition_variable}%
\begin{itemdecl}
void notify_all() noexcept;
\end{itemdecl}

\begin{itemdescr}
\pnum
\effects
Unblocks all threads that are blocked waiting for \tcode{*this}.
\end{itemdescr}

\indexlibrarymember{wait}{condition_variable}%
\begin{itemdecl}
void wait(unique_lock<mutex>& lock);
\end{itemdecl}

\begin{itemdescr}
\pnum
\requires \tcode{lock.owns_lock()} is \tcode{true} and \tcode{lock.mutex()}
is locked by the calling thread, and either
\begin{itemize}
\item no other thread is waiting on this \tcode{condition_variable} object or
\item \tcode{lock.mutex()} returns the same value for each of the \tcode{lock}
arguments supplied by all concurrently waiting (via \tcode{wait},
\tcode{wait_for}, or \tcode{wait_until}) threads.
\end{itemize}

\pnum
\effects
\begin{itemize}
\item Atomically calls \tcode{lock.unlock()} and blocks on \tcode{*this}.
\item When unblocked, calls \tcode{lock.lock()} (possibly blocking on the lock), then returns.
\item The function will unblock when signaled by a call to \tcode{notify_one()}
or a call to \tcode{notify_all()}, or spuriously.
\end{itemize}

\pnum
\remarks
If the function fails to meet the postcondition, \tcode{terminate()}
shall be called\iref{except.terminate}.
\begin{note}
This can happen if the re-locking of the mutex throws an exception.
\end{note}

\pnum
\ensures
\tcode{lock.owns_lock()} is \tcode{true} and \tcode{lock.mutex()}
is locked by the calling thread.

\pnum
\throws
Nothing.

\end{itemdescr}

\indexlibrarymember{wait}{condition_variable}%
\begin{itemdecl}
template<class Predicate>
  void wait(unique_lock<mutex>& lock, Predicate pred);
\end{itemdecl}

\begin{itemdescr}
\pnum
\requires \tcode{lock.owns_lock()} is \tcode{true} and \tcode{lock.mutex()} is
locked by the calling thread, and either

\begin{itemize}
\item no other thread is waiting on this \tcode{condition_variable} object or
\item \tcode{lock.mutex()} returns the same value for each of the \tcode{lock}
arguments supplied by all concurrently waiting (via \tcode{wait},
\tcode{wait_for}, or \tcode{wait_until}) threads.
\end{itemize}

\pnum
\effects
Equivalent to:
\begin{codeblock}
while (!pred())
  wait(lock);
\end{codeblock}

\pnum
\remarks
If the function fails to meet the postcondition, \tcode{terminate()}
shall be called\iref{except.terminate}.
\begin{note}
This can happen if the re-locking of the mutex throws an exception.
\end{note}

\pnum
\ensures
\tcode{lock.owns_lock()} is \tcode{true} and \tcode{lock.mutex()}
is locked by the calling thread.

\pnum
\throws
Any exception thrown by \tcode{pred}.

\end{itemdescr}

\indexlibrarymember{wait_until}{condition_variable}%
\begin{itemdecl}
template<class Clock, class Duration>
  cv_status wait_until(unique_lock<mutex>& lock,
                       const chrono::time_point<Clock, Duration>& abs_time);
\end{itemdecl}

\begin{itemdescr}
\pnum
\requires \tcode{lock.owns_lock()} is \tcode{true} and \tcode{lock.mutex()}
is locked by the calling thread, and either
\begin{itemize}
\item no other thread is waiting on this \tcode{condition_variable} object or
\item \tcode{lock.mutex()} returns the same value for each of the \tcode{lock}
arguments supplied by all concurrently waiting (via \tcode{wait},
\tcode{wait_for}, or \tcode{wait_until}) threads.
\end{itemize}

\pnum
\effects
\begin{itemize}
\item
Atomically calls \tcode{lock.unlock()} and blocks on \tcode{*this}.

\item
When unblocked, calls \tcode{lock.lock()} (possibly blocking on the lock), then returns.

\item
The function will unblock when signaled by a call to \tcode{notify_one()}, a call to \tcode{notify_all()},
expiration of the absolute timeout\iref{thread.req.timing} specified by \tcode{abs_time},
or spuriously.

\item
If the function exits via an exception, \tcode{lock.lock()} shall be called prior to exiting the function.
\end{itemize}

\pnum
\remarks
If the function fails to meet the postcondition, \tcode{terminate()}
shall be called\iref{except.terminate}.
\begin{note}
This can happen if the re-locking of the mutex throws an exception.
\end{note}

\pnum
\ensures
\tcode{lock.owns_lock()} is \tcode{true} and \tcode{lock.mutex()}
is locked by the calling thread.

\pnum
\returns
\tcode{cv_status::timeout} if
the absolute timeout\iref{thread.req.timing} specified by \tcode{abs_time} expired,
otherwise \tcode{cv_status::no_timeout}.

\pnum
\throws
Timeout-related
exceptions\iref{thread.req.timing}.

\end{itemdescr}

\indexlibrarymember{wait_for}{condition_variable}%
\begin{itemdecl}
template<class Rep, class Period>
  cv_status wait_for(unique_lock<mutex>& lock,
                     const chrono::duration<Rep, Period>& rel_time);
\end{itemdecl}

\begin{itemdescr}
\pnum
\requires \tcode{lock.owns_lock()} is \tcode{true} and \tcode{lock.mutex()}
is locked by the calling thread, and either
\begin{itemize}
\item no other thread is waiting on this \tcode{condition_variable} object or
\item \tcode{lock.mutex()} returns the same value for each of the \tcode{lock} arguments
supplied by all concurrently waiting (via \tcode{wait}, \tcode{wait_for}, or
\tcode{wait_until}) threads.
\end{itemize}

\pnum
\effects
Equivalent to:
\begin{codeblock}
return wait_until(lock, chrono::steady_clock::now() + rel_time);
\end{codeblock}

\pnum
\returns
\tcode{cv_status::timeout} if
the relative timeout\iref{thread.req.timing} specified by \tcode{rel_time} expired,
otherwise \tcode{cv_status::no_timeout}.

\pnum
\remarks
If the function fails to meet the postcondition, \tcode{terminate()}
shall be called\iref{except.terminate}.
\begin{note}
This can happen if the re-locking of the mutex throws an exception.
\end{note}

\pnum
\ensures
\tcode{lock.owns_lock()} is \tcode{true} and \tcode{lock.mutex()}
is locked by the calling thread.

\pnum
\throws
Timeout-related
exceptions\iref{thread.req.timing}.

\end{itemdescr}

\indexlibrarymember{wait_until}{condition_variable}%
\begin{itemdecl}
template<class Clock, class Duration, class Predicate>
  bool wait_until(unique_lock<mutex>& lock,
                  const chrono::time_point<Clock, Duration>& abs_time,
                  Predicate pred);
\end{itemdecl}

\begin{itemdescr}
\pnum
\requires \tcode{lock.owns_lock()} is \tcode{true} and \tcode{lock.mutex()} is
locked by the calling thread, and either

\begin{itemize}
\item no other thread is waiting on this \tcode{condition_variable} object or
\item \tcode{lock.mutex()} returns the same value for each of the \tcode{lock}
arguments supplied by all concurrently waiting (via \tcode{wait},
\tcode{wait_for}, or \tcode{wait_until}) threads.
\end{itemize}

\pnum
\effects
Equivalent to:
\begin{codeblock}
while (!pred())
  if (wait_until(lock, abs_time) == cv_status::timeout)
    return pred();
return true;
\end{codeblock}

\pnum
\remarks
If the function fails to meet the postcondition, \tcode{terminate()}
shall be called\iref{except.terminate}.
\begin{note}
This can happen if the re-locking of the mutex throws an exception.
\end{note}

\pnum
\ensures
\tcode{lock.owns_lock()} is \tcode{true} and \tcode{lock.mutex()}
is locked by the calling thread.

\pnum
\begin{note}
The returned value indicates whether the predicate evaluated to
\tcode{true} regardless of whether the timeout was triggered.
\end{note}

\pnum
\throws
Timeout-related
exceptions\iref{thread.req.timing} or any exception thrown by \tcode{pred}.

\end{itemdescr}

\indexlibrarymember{wait_for}{condition_variable}%
\begin{itemdecl}
template<class Rep, class Period, class Predicate>
  bool wait_for(unique_lock<mutex>& lock,
                const chrono::duration<Rep, Period>& rel_time,
                Predicate pred);
\end{itemdecl}

\begin{itemdescr}
\pnum
\requires \tcode{lock.owns_lock()} is \tcode{true} and \tcode{lock.mutex()}
is locked by the calling thread, and either
\begin{itemize}
\item
no other thread is waiting on this \tcode{condition_variable} object or

\item
\tcode{lock.mutex()} returns the same value for each of the \tcode{lock} arguments
supplied by all concurrently waiting (via \tcode{wait}, \tcode{wait_for}, or
\tcode{wait_until}) threads.
\end{itemize}

\pnum
\effects
Equivalent to:
\begin{codeblock}
return wait_until(lock, chrono::steady_clock::now() + rel_time, std::move(pred));
\end{codeblock}

\pnum
\begin{note}
There is no blocking if \tcode{pred()} is initially \tcode{true}, even if the
timeout has already expired.
\end{note}

\pnum
\remarks
If the function fails to meet the postcondition, \tcode{terminate()}
shall be called\iref{except.terminate}.
\begin{note}
This can happen if the re-locking of the mutex throws an exception.
\end{note}

\pnum
\ensures
\tcode{lock.owns_lock()} is \tcode{true} and \tcode{lock.mutex()}
is locked by the calling thread.

\pnum
\begin{note}
The returned value indicates whether the predicate evaluates to \tcode{true}
regardless of whether the timeout was triggered.
\end{note}

\pnum
\throws
Timeout-related
exceptions\iref{thread.req.timing} or any exception thrown by \tcode{pred}.

\end{itemdescr}

\rSec2[thread.condition.condvarany]{Class \tcode{condition_variable_any}}

\pnum
A \tcode{Lock} type shall meet the \oldconcept{BasicLockable}
requirements\iref{thread.req.lockable.basic}.
\begin{note}
All of the standard
mutex types meet this requirement. If a \tcode{Lock} type other than one of the
standard mutex types or a \tcode{unique_lock} wrapper for a standard mutex type
is used with \tcode{condition_variable_any}, the user should ensure that any
necessary synchronization is in place with respect to the predicate associated
with the \tcode{condition_variable_any} instance.
\end{note}

\indexlibraryglobal{condition_variable_any}%
\begin{codeblock}
namespace std {
  class condition_variable_any {
  public:
    condition_variable_any();
    ~condition_variable_any();

    condition_variable_any(const condition_variable_any&) = delete;
    condition_variable_any& operator=(const condition_variable_any&) = delete;

    void notify_one() noexcept;
    void notify_all() noexcept;

    // \ref{thread.condvarany.wait}, noninterruptible waits
    template<class Lock>
      void wait(Lock& lock);
    template<class Lock, class Predicate>
      void wait(Lock& lock, Predicate pred);

    template<class Lock, class Clock, class Duration>
      cv_status wait_until(Lock& lock, const chrono::time_point<Clock, Duration>& abs_time);
    template<class Lock, class Clock, class Duration, class Predicate>
      bool wait_until(Lock& lock, const chrono::time_point<Clock, Duration>& abs_time,
                      Predicate pred);
    template<class Lock, class Rep, class Period>
      cv_status wait_for(Lock& lock, const chrono::duration<Rep, Period>& rel_time);
    template<class Lock, class Rep, class Period, class Predicate>
      bool wait_for(Lock& lock, const chrono::duration<Rep, Period>& rel_time, Predicate pred);

    // \ref{thread.condvarany.intwait}, interruptible waits
    template<class Lock, class Predicate>
      bool wait_until(Lock& lock, Predicate pred, stop_token stoken);
    template<class Lock, class Clock, class Duration, class Predicate>
      bool wait_until(Lock& lock, const chrono::time_point<Clock, Duration>& abs_time
                      Predicate pred, stop_token stoken);
    template<class Lock, class Rep, class Period, class Predicate>
      bool wait_for(Lock& lock, const chrono::duration<Rep, Period>& rel_time,
                    Predicate pred, stop_token stoken);
  };
}
\end{codeblock}

\indexlibraryctor{condition_variable_any}%
\begin{itemdecl}
condition_variable_any();
\end{itemdecl}

\begin{itemdescr}
\pnum
\effects
Constructs an object of type \tcode{condition_variable_any}.

\pnum
\throws
\tcode{bad_alloc} or \tcode{system_error} when an exception is
required\iref{thread.req.exception}.

\pnum
\errors
\begin{itemize}
\item \tcode{resource_unavailable_try_again} --- if some non-memory resource
limitation prevents initialization.

\item \tcode{operation_not_permitted} --- if the thread does not have the
privilege to perform the operation.
\end{itemize}
\end{itemdescr}

\indexlibrarydtor{condition_variable_any}%
\begin{itemdecl}
~condition_variable_any();
\end{itemdecl}

\begin{itemdescr}
\pnum
\requires There shall be no thread blocked on \tcode{*this}.
\begin{note}
That is, all
threads shall have been notified; they may subsequently block on the lock specified in the
wait.
This relaxes the usual rules, which would have required all wait calls to happen before
destruction. Only the notification to unblock the wait needs to happen before destruction.
The user should take care to ensure that no threads wait on \tcode{*this} once the destructor has
been started, especially when the waiting threads are calling the wait functions in a loop or
using the overloads of \tcode{wait}, \tcode{wait_for}, or \tcode{wait_until} that take a predicate.
\end{note}

\pnum
\effects
Destroys the object.
\end{itemdescr}

\indexlibrarymember{notify_one}{condition_variable_any}%
\begin{itemdecl}
void notify_one() noexcept;
\end{itemdecl}

\begin{itemdescr}
\pnum
\effects
If any threads are blocked waiting for \tcode{*this}, unblocks one of those threads.
\end{itemdescr}

\indexlibrarymember{notify_all}{condition_variable_any}%
\begin{itemdecl}
void notify_all() noexcept;
\end{itemdecl}

\begin{itemdescr}
\pnum
\effects
Unblocks all threads that are blocked waiting for \tcode{*this}.
\end{itemdescr}

\rSec3[thread.condvarany.wait]{Noninterruptible waits}

\indexlibrarymember{wait}{condition_variable_any}%
\begin{itemdecl}
template<class Lock>
  void wait(Lock& lock);
\end{itemdecl}

\begin{itemdescr}
\pnum
\effects
\begin{itemize}
\item Atomically calls \tcode{lock.unlock()} and blocks on \tcode{*this}.
\item When unblocked, calls \tcode{lock.lock()} (possibly blocking on the lock) and returns.
\item The function will unblock when signaled by a call to \tcode{notify_one()},
a call to \tcode{notify_all()}, or spuriously.
\end{itemize}

\pnum
\remarks
If the function fails to meet the postcondition, \tcode{terminate()}
shall be called\iref{except.terminate}.
\begin{note}
This can happen if the re-locking of the mutex throws an exception.
\end{note}

\pnum
\ensures
\tcode{lock} is locked by the calling thread.

\pnum
\throws
Nothing.

\end{itemdescr}

\indexlibrarymember{wait}{condition_variable_any}%
\begin{itemdecl}
template<class Lock, class Predicate>
  void wait(Lock& lock, Predicate pred);
\end{itemdecl}

\begin{itemdescr}
\pnum
\effects
Equivalent to:
\begin{codeblock}
while (!pred())
  wait(lock);
\end{codeblock}
\end{itemdescr}

\indexlibrarymember{wait_until}{condition_variable_any}%
\begin{itemdecl}
template<class Lock, class Clock, class Duration>
  cv_status wait_until(Lock& lock, const chrono::time_point<Clock, Duration>& abs_time);
\end{itemdecl}

\begin{itemdescr}
\pnum
\effects

\begin{itemize}
\item
Atomically calls \tcode{lock.unlock()} and blocks on \tcode{*this}.

\item
When unblocked, calls \tcode{lock.lock()} (possibly blocking on the lock) and returns.

\item
The function will unblock when signaled by a call to \tcode{notify_one()}, a call to \tcode{notify_all()},
expiration of the absolute timeout\iref{thread.req.timing} specified by \tcode{abs_time},
or spuriously.

\item
If the function exits via an exception, \tcode{lock.lock()} shall be called prior to exiting the function.
\end{itemize}

\pnum
\remarks
If the function fails to meet the postcondition, \tcode{terminate()}
shall be called\iref{except.terminate}.
\begin{note}
This can happen if the re-locking of the mutex throws an exception.
\end{note}

\pnum
\ensures
\tcode{lock} is locked by the calling thread.

\pnum
\returns
\tcode{cv_status::timeout} if
the absolute timeout\iref{thread.req.timing} specified by \tcode{abs_time} expired,
otherwise \tcode{cv_status::no_timeout}.

\pnum
\throws
Timeout-related
exceptions\iref{thread.req.timing}.

\end{itemdescr}

\indexlibrarymember{wait_for}{condition_variable_any}%
\begin{itemdecl}
template<class Lock, class Rep, class Period>
  cv_status wait_for(Lock& lock, const chrono::duration<Rep, Period>& rel_time);
\end{itemdecl}

\begin{itemdescr}
\pnum
\effects
Equivalent to:
\begin{codeblock}
return wait_until(lock, chrono::steady_clock::now() + rel_time);
\end{codeblock}

\pnum
\returns
\tcode{cv_status::timeout} if
the relative timeout\iref{thread.req.timing} specified by \tcode{rel_time} expired,
otherwise \tcode{cv_status::no_timeout}.

\pnum
\remarks
If the function fails to meet the postcondition, \tcode{terminate()}
shall be called\iref{except.terminate}.
\begin{note}
This can happen if the re-locking of the mutex throws an exception.
\end{note}

\pnum
\ensures
\tcode{lock} is locked by the calling thread.

\pnum
\throws
Timeout-related
exceptions\iref{thread.req.timing}.

\end{itemdescr}

\indexlibrarymember{wait_until}{condition_variable_any}%
\begin{itemdecl}
template<class Lock, class Clock, class Duration, class Predicate>
  bool wait_until(Lock& lock, const chrono::time_point<Clock, Duration>& abs_time, Predicate pred);
\end{itemdecl}

\begin{itemdescr}
\pnum
\effects
Equivalent to:
\begin{codeblock}
while (!pred())
  if (wait_until(lock, abs_time) == cv_status::timeout)
    return pred();
return true;
\end{codeblock}

\pnum
\begin{note}
There is no blocking if \tcode{pred()} is initially \tcode{true}, or
if the timeout has already expired.
\end{note}

\pnum
\begin{note}
The returned value indicates whether the predicate evaluates to \tcode{true}
regardless of whether the timeout was triggered.
\end{note}
\end{itemdescr}

\indexlibrarymember{wait_for}{condition_variable_any}%
\begin{itemdecl}
template<class Lock, class Rep, class Period, class Predicate>
  bool wait_for(Lock& lock, const chrono::duration<Rep, Period>& rel_time, Predicate pred);
\end{itemdecl}

\begin{itemdescr}
\pnum
\effects
Equivalent to:
\begin{codeblock}
return wait_until(lock, chrono::steady_clock::now() + rel_time, std::move(pred));
\end{codeblock}
\end{itemdescr}

\rSec3[thread.condvarany.intwait]{Interruptible waits}

\pnum
The following wait functions will be notified
when there is a stop request on the passed \tcode{stop_token}.
In that case the functions return immediately,
returning \tcode{false} if the predicate evaluates to \tcode{false}.

\begin{itemdecl}
template<class Lock, class Predicate>
  bool wait_until(Lock& lock, Predicate pred, stop_token stoken);
\end{itemdecl}

\begin{itemdescr}
\pnum
\effects
Registers for the duration of this call \tcode{*this}
to get notified on a stop request on \tcode{stoken}
during this call and then equivalent to:
\begin{codeblock}
while (!stoken.stop_requested()) {
  if (pred())
    return true;
  wait(lock);
}
return pred();
\end{codeblock}

\pnum
\begin{note}
The returned value indicates whether the predicate evaluated to
\tcode{true} regardless of whether there was a stop request.
\end{note}

\pnum
\ensures
\tcode{lock} is locked by the calling thread.

\pnum
\remarks
If the function fails to meet the postcondition,
\tcode{terminate} is called\iref{except.terminate}.
\begin{note}
This can happen if the re-locking of the mutex throws an exception.
\end{note}

\pnum
\throws
Any exception thrown by \tcode{pred}.
\end{itemdescr}

\begin{itemdecl}
template<class Lock, class Clock, class Duration, class Predicate>
  bool wait_until(Lock& lock, const chrono::time_point<Clock, Duration>& abs_time
                  Predicate pred, stop_token stoken);
\end{itemdecl}

\begin{itemdescr}
\pnum
\effects
Registers for the duration of this call \tcode{*this}
to get notified on a stop request on \tcode{stoken}
during this call and then equivalent to:
\begin{codeblock}
while (!stoken.stop_requested()) {
  if (pred())
    return true;
  if (cv.wait_until(lock, abs_time) == cv_status::timeout)
    return pred();
}
return pred();
\end{codeblock}

\pnum
\begin{note}
There is no blocking if \tcode{pred()} is initially \tcode{true},
\tcode{stoken.stop_requested()} was already \tcode{true}
or the timeout has already expired.
\end{note}

\pnum
\begin{note}
The returned value indicates whether the predicate evaluated to \tcode{true}
regardless of whether the timeout was triggered or a stop request was made.
\end{note}

\pnum
\ensures
\tcode{lock} is locked by the calling thread.

\pnum
\remarks
If the function fails to meet the postcondition,
\tcode{terminate} is called\iref{except.terminate}.
\begin{note}
This can happen if the re-locking of the mutex throws an exception.
\end{note}

\pnum
\throws
Timeout-related exceptions \iref{thread.req.timing},
or any exception thrown by \tcode{pred}.
\end{itemdescr}

\begin{itemdecl}
template<class Lock, class Rep, class Period, class Predicate>
  bool wait_for(Lock& lock, const chrono::duration<Rep, Period>& rel_time,
                Predicate pred, stop_token stoken);
\end{itemdecl}

\begin{itemdescr}
\pnum
\effects
Equivalent to:
\begin{codeblock}
return wait_until(lock, chrono::steady_clock::now() + rel_time, std::move(pred),
                  std::move(stoken));
\end{codeblock}
\end{itemdescr}

\rSec1[thread.sema]{Semaphore}

\pnum
Semaphores are lightweight synchronization primitives
used to constrain concurrent access to a shared resource.
They are widely used to implement other synchronization primitives and,
whenever both are applicable, can be more efficient than condition variables.

\pnum
A counting semaphore is a semaphore object
that models a non-negative resource count.
A binary semaphore is a semaphore object that has only two states.
A binary semaphore should be more efficient than
the default implementation of a counting semaphore with a unit resource count.

\rSec2[semaphore.syn]{Header \tcode{<semaphore>} synopsis}

\indexheader{semaphore}%
\begin{codeblock}
namespace std {
  template<ptrdiff_t least_max_value = @\textrm{\textit{implementation-defined}}@>
    class counting_semaphore;

  using binary_semaphore = counting_semaphore<1>;
}
\end{codeblock}

\rSec2[thread.sema.cnt]{Class template \tcode{counting_semaphore}}

\begin{codeblock}
namespace std {
  template<ptrdiff_t least_max_value = @\textrm{\textit{\impldef{value for \idxcode{least_max_value} default template argument of \idxcode{counting_semaphore}}}}@>
  class counting_semaphore {
  public:
    static constexpr ptrdiff_t max() noexcept;

    constexpr explicit counting_semaphore(ptrdiff_t desired);
    ~counting_semaphore();

    counting_semaphore(const counting_semaphore&) = delete;
    counting_semaphore& operator=(const counting_semaphore&) = delete;

    void release(ptrdiff_t update = 1);
    void acquire();
    bool try_acquire() noexcept;
    template<class Rep, class Period>
      bool try_acquire_for(const chrono::duration<Rep, Period>& rel_time);
    template<class Clock, class Duration>
      bool try_acquire_until(const chrono::time_point<Clock, Duration>& abs_time);

  private:
    ptrdiff_t counter; // \expos
  };
}
\end{codeblock}

\pnum
Class template \tcode{counting_semaphore} maintains an internal counter
that is initialized when the semaphore is created.
The counter is decremented when a thread acquires the semaphore, and
is incremented when a thread releases the semaphore.
If a thread tries to acquire the semaphore when the counter is zero,
the thread will block
until another thread increments the counter by releasing the semaphore.

\pnum
\tcode{least_max_value} shall be non-negative; otherwise the program is ill-formed.

\pnum
Concurrent invocations of the member functions of \tcode{counting_semaphore},
other than its destructor, do not introduce data races.

\indexlibrarymember{max}{counting_semaphore}%
\begin{itemdecl}
static constexpr ptrdiff_t max() noexcept;
\end{itemdecl}

\begin{itemdescr}
\pnum
\returns
The maximum value of \tcode{counter}.
This value is greater than or equal to \tcode{least_max_value}.
\end{itemdescr}

\indexlibraryctor{counting_semaphore}%
\begin{itemdecl}
constexpr explicit counting_semaphore(ptrdiff_t desired);
\end{itemdecl}

\begin{itemdescr}
\pnum
\expects
\tcode{desired >= 0} is \tcode{true}, and
\tcode{desired <= max()} is \tcode{true}.

\pnum
\effects
Initializes \tcode{counter} with \tcode{desired}.

\pnum
\throws
Nothing.
\end{itemdescr}

\indexlibrarymember{release}{counting_semaphore}%
\begin{itemdecl}
void release(ptrdiff_t update = 1);
\end{itemdecl}

\begin{itemdescr}
\pnum
\expects
\tcode{update >= 0} is \tcode{true}, and
\tcode{update <= max() - counter} is \tcode{true}.

\pnum
\effects
Atomically execute \tcode{counter += update}.
Then, unblocks any threads
that are waiting for \tcode{counter} to be greater than zero.

\pnum
\sync
Strongly happens before invocations of \tcode{try_acquire}
that observe the result of the effects.

\pnum
\throws
\tcode{system_error} when an exception is required\iref{thread.req.exception}.

\pnum
\errors
Any of the error conditions
allowed for mutex types\iref{thread.mutex.requirements.mutex}.
\end{itemdescr}

\indexlibrarymember{try_acquire}{counting_semaphore}%
\begin{itemdecl}
bool try_acquire() noexcept;
\end{itemdecl}

\begin{itemdescr}
\pnum
\effects
\begin{itemize}
\item
  With low probability, returns immediately.
  An implementation should ensure
  that \tcode{try_acquire} does not consistently return \tcode{false}
  in the absence of contending acquisitions.
\item
  Otherwise, atomically check whether \tcode{counter} is greater than zero and,
  if so, decrement \tcode{counter} by one.
\end{itemize}

\pnum
\returns
\tcode{true} if \tcode{counter} was decremented, otherwise \tcode{false}.
\end{itemdescr}

\indexlibrarymember{acquire}{counting_semaphore}%
\begin{itemdecl}
void acquire();
\end{itemdecl}

\begin{itemdescr}
\pnum
\effects
Repeatedly performs the following steps, in order:
\begin{itemize}
\item Evaluates \tcode{try_acquire}. If the result is \tcode{true}, returns.
\item Blocks on \tcode{*this} until \tcode{counter} is greater than zero.
\end{itemize}

\pnum
\throws
\tcode{system_error} when an exception is required\iref{thread.req.exception}.

\pnum
\errors
Any of the error conditions
allowed for mutex types\iref{thread.mutex.requirements.mutex}.
\end{itemdescr}

\indexlibrarymember{try_acquire_for}{counting_semaphore}%
\indexlibrarymember{try_acquire_until}{counting_semaphore}%
\begin{itemdecl}
template<class Rep, class Period>
  bool try_acquire_for(const chrono::duration<Rep, Period>& rel_time);
template<class Clock, class Duration>
  bool try_acquire_until(const chrono::time_point<Clock, Duration>& abs_time);
\end{itemdecl}

\begin{itemdescr}
\pnum
\effects
Repeatedly performs the following steps, in order:
\begin{itemize}
\item
  Evaluates \tcode{try_acquire()}.
  If the result is \tcode{true}, returns \tcode{true}.
\item
  Blocks on \tcode{*this}
  until \tcode{counter} is greater than zero or until the timeout expires.
  If it is unblocked by the timeout expiring, returns \tcode{false}.
\end{itemize}
The timeout expires\iref{thread.req.timing}
when the current time is after \tcode{abs_time} (for \tcode{try_acquire_until})
or when at least \tcode{rel_time} has passed
from the start of the function (for \tcode{try_acquire_for}).

\pnum
\throws
Timeout-related exceptions\iref{thread.req.timing}, or \tcode{system_error}
when a non-timeout-related exception is required\iref{thread.req.exception}.

\pnum
\errors
Any of the error conditions
allowed for mutex types\iref{thread.mutex.requirements.mutex}.
\end{itemdescr}

\rSec1[thread.coord]{Coordination types}

\pnum
This subclause describes various concepts related to thread coordination, and
defines the coordination types \tcode{latch} and \tcode{barrier}.
These types facilitate concurrent computation performed by a number of threads.

\rSec2[thread.latch]{Latches}

\pnum
A latch is a thread coordination mechanism
that allows any number of threads to block
until an expected number of threads arrive at the latch
(via the \tcode{count_down} function).
The expected count is set when the latch is created.
An individual latch is a single-use object;
once the expected count has been reached, the latch cannot be reused.

\rSec2[latch.syn]{Header \tcode{<latch>} synopsis}

\indexheader{latch}%
\begin{codeblock}
namespace std {
  class latch;
}
\end{codeblock}

\rSec2[thread.latch.class]{Class \tcode{latch}}

\begin{codeblock}
namespace std {
  class latch {
  public:
    constexpr explicit latch(ptrdiff_t expected);
    ~latch();

    latch(const latch&) = delete;
    latch& operator=(const latch&) = delete;

    void count_down(ptrdiff_t update = 1);
    bool try_wait() const noexcept;
    void wait() const;
    void arrive_and_wait(ptrdiff_t update = 1);

  private:
    ptrdiff_t counter;  // \expos
  };
}
\end{codeblock}

\pnum
A \tcode{latch} maintains an internal counter
that is initialized when the latch is created.
Threads can block on the latch object,
waiting for counter to be decremented to zero.

\pnum
Concurrent invocations of the member functions of \tcode{latch},
other than its destructor, do not introduce data races.

\indexlibraryctor{latch}%
\begin{itemdecl}
constexpr explicit latch(ptrdiff_t expected);
\end{itemdecl}

\begin{itemdescr}
\pnum
\expects
\tcode{expected >= 0} is \tcode{true}.

\pnum
\effects
Initializes \tcode{counter} with \tcode{expected}.

\pnum
\throws
Nothing.
\end{itemdescr}

\indexlibrarymember{count_down}{latch}%
\begin{itemdecl}
void count_down(ptrdiff_t update = 1);
\end{itemdecl}

\begin{itemdescr}
\pnum
\expects
\tcode{update >= 0} is \tcode{true}, and
\tcode{update <= counter} is \tcode{true}.

\pnum
\effects
Atomically decrements \tcode{counter} by \tcode{update}.
If \tcode{counter} is equal to zero,
unblocks all threads blocked on \tcode{*this}.

\pnum
\sync
Strongly happens before the returns from all calls that are unblocked.

\pnum
\throws
\tcode{system_error} when an exception is required\iref{thread.req.exception}.

\pnum
\errors
Any of the error conditions
allowed for mutex types\iref{thread.mutex.requirements.mutex}.
\end{itemdescr}

\indexlibrarymember{try_wait}{latch}%
\begin{itemdecl}
bool try_wait() const noexcept;
\end{itemdecl}

\begin{itemdescr}
\pnum
\returns
With very low probability \tcode{false}. Otherwise \tcode{counter == 0}.
\end{itemdescr}

\indexlibrarymember{wait}{latch}%
\begin{itemdecl}
void wait() const;
\end{itemdecl}

\begin{itemdescr}
\pnum
\effects
If \tcode{counter} equals zero, returns immediately.
Otherwise, blocks on \tcode{*this}
until a call to \tcode{count_down} that decrements \tcode{counter} to zero.

\pnum
\throws
\tcode{system_error} when an exception is required\iref{thread.req.exception}.

\pnum
\errors
Any of the error conditions
allowed for mutex types\iref{thread.mutex.requirements.mutex}.
\end{itemdescr}

\indexlibrarymember{arrive_and_wait}{latch}%
\begin{itemdecl}
void arrive_and_wait(ptrdiff_t update = 1);
\end{itemdecl}

\begin{itemdescr}
\pnum
\effects
Equivalent to:
\begin{codeblock}
count_down(update);
wait();
\end{codeblock}
\end{itemdescr}

\rSec2[thread.barrier]{Barriers}

\pnum
A barrier is a thread coordination mechanism
whose lifetime consists of a sequence of barrier phases,
where each phase allows at most an expected number of threads to block
until the expected number of threads arrive at the barrier.
\begin{note}
A barrier is useful for managing repeated tasks
that are handled by multiple threads.
\end{note}

\rSec3[barrier.syn]{Header \tcode{<barrier>} synopsis}

\indexheader{barrier}%
\begin{codeblock}
namespace std {
  template<class CompletionFunction = @\seebelow@>
    class barrier;
}
\end{codeblock}

\rSec3[thread.barrier.class]{Class template \tcode{barrier}}

\begin{codeblock}
namespace std {
  template<class CompletionFunction = @\seebelow@>
  class barrier {
  public:
    using arrival_token = @\seebelow@;

    constexpr explicit barrier(ptrdiff_t expected,
                               CompletionFunction f = CompletionFunction());
    ~barrier();

    barrier(const barrier&) = delete;
    barrier& operator=(const barrier&) = delete;

    [[nodiscard]] arrival_token arrive(ptrdiff_t update = 1);
    void wait(arrival_token&& arrival) const;

    void arrive_and_wait();
    void arrive_and_drop();

  private:
    CompletionFunction completion;      // \expos
  };
}
\end{codeblock}

\pnum
Each \defn{barrier phase} consists of the following steps:
\begin{itemize}
\item
  The expected count is decremented
  by each call to \tcode{arrive} or \tcode{arrive_and_drop}.
\item
  When the expected count reaches zero, the phase completion step is run.
  For the specialization
  with the default value of the \tcode{CompletionFunction} template parameter,
  the completion step is run
  as part of the call to \tcode{arrive} or \tcode{arrive_and_drop}
  that caused the expected count to reach zero.
  For other specializations,
  the completion step is run on one of the threads
  that arrived at the barrier during the phase.
\item
  When the completion step finishes,
  the expected count is reset
  to what was specified by the \tcode{expected} argument to the constructor,
  possibly adjusted by calls to \tcode{arrive_and_drop}, and
  the next phase starts.
\end{itemize}

\indextext{phase synchronization point|see{barrier, phase synchronization point }}%
\pnum
Each phase defines a \defnx{phase synchronization point}{barrier!phase synchronization point}.
Threads that arrive at the barrier during the phase
can block on the phase synchronization point by calling \tcode{wait}, and
will remain blocked until the phase completion step is run.

\pnum
The \defn{phase completion step}
that is executed at the end of each phase has the following effects:
\begin{itemize}
\item Invokes the completion function, equivalent to \tcode{completion()}.
\item Unblocks all threads that are blocked on the phase synchronization point.
\end{itemize}
The end of the completion step strongly happens before
the returns from all calls that were unblocked by the completion step.
For specializations that do not have
the default value of the \tcode{CompletionFunction} template parameter,
the behavior is undefined if any of the barrier object's member functions
other than \tcode{wait} are called while the completion step is in progress.

\pnum
Concurrent invocations of the member functions of \tcode{barrier},
other than its destructor, do not introduce data races.
The member functions \tcode{arrive} and \tcode{arrive_and_drop}
execute atomically.

\pnum
\tcode{CompletionFunction} shall meet the
\oldconcept{MoveConstructible} (\tref{cpp17.moveconstructible}) and
\oldconcept{Destructible} (\tref{cpp17.destructible}) requirements.
\tcode{is_nothrow_invocable_v<CompletionFunction\&>} shall be \tcode{true}.

\pnum
The default value of the \tcode{CompletionFunction} template parameter is
an unspecified type, such that,
in addition to satisfying the requirements of \tcode{CompletionFunction},
it meets the \oldconcept{DefaultConstructible}
requirements (\tref{cpp17.defaultconstructible}) and
\tcode{completion()} has no effects.

\pnum
\tcode{barrier::arrival_token} is an unspecified type,
such that it meets the
\oldconcept{MoveConstructible} (\tref{cpp17.moveconstructible}),
\oldconcept{MoveAssignable} (\tref{cpp17.moveassignable}), and
\oldconcept{Destructible} (\tref{cpp17.destructible}) requirements.

\indexlibraryctor{barrier}
\begin{itemdecl}
constexpr explicit barrier(ptrdiff_t expected,
                           CompletionFunction f = CompletionFunction());
\end{itemdecl}

\begin{itemdescr}
\pnum
\expects
\tcode{expected >= 0} is \tcode{true}.

\pnum
\effects
Sets both the initial expected count for each barrier phase and
the current expected count for the first phase to \tcode{expected}.
Initializes \tcode{completion} with \tcode{std::move(f)}.
Starts the first phase.
\begin{note}
If \tcode{expected} is 0 this object can only be destroyed.
\end{note}

\pnum
\throws
Any exception thrown by \tcode{CompletionFunction}'s move constructor.
\end{itemdescr}

\indexlibrarymember{arrive}{barrier}%
\begin{itemdecl}
[[nodiscard]] arrival_token arrive(ptrdiff_t update = 1);
\end{itemdecl}

\begin{itemdescr}
\pnum
\expects
\tcode{update > 0} is \tcode{true}, and
\tcode{update} is less than or equal to
the expected count for the current barrier phase.

\pnum
\effects
Constructs an object of type \tcode{arrival_token}
that is associated with the phase synchronization point for the current phase.
Then, decrements the expected count by \tcode{update}.

\pnum
\sync
The call to \tcode{arrive} strongly happens before
the start of the phase completion step for the current phase.

\pnum
\returns
The constructed \tcode{arrival_token} object.

\pnum
\throws
\tcode{system_error} when an exception is required\iref{thread.req.exception}.

\pnum
\errors
Any of the error conditions
allowed for mutex types\iref{thread.mutex.requirements.mutex}.

\pnum
\begin{note}
This call can cause the completion step for the current phase to start.
\end{note}
\end{itemdescr}

\indexlibrarymember{wait}{barrier}%
\begin{itemdecl}
void wait(arrival_token&& arrival) const;
\end{itemdecl}

\begin{itemdescr}
\pnum
\expects
\tcode{arrival} is associated with
the phase synchronization point for the current phase or
the immediately preceding phase of the same barrier object.

\pnum
\effects
Blocks at the synchronization point associated with \tcode{std::move(arrival)}
until the phase completion step of the synchronization point's phase is run.
\begin{note}
If \tcode{arrival} is associated with the synchronization point
for a previous phase, the call returns immediately.
\end{note}

\pnum
\throws
\tcode{system_error} when an exception is required\iref{thread.req.exception}.

\pnum
\errors
Any of the error conditions
allowed for mutex types\iref{thread.mutex.requirements.mutex}.
\end{itemdescr}

\indexlibrarymember{arrive_and_wait}{barrier}%
\begin{itemdecl}
void arrive_and_wait();
\end{itemdecl}

\begin{itemdescr}
\pnum
\effects
Equivalent to: \tcode{wait(arrive())}.
\end{itemdescr}

\indexlibrarymember{arrive_and_drop}{barrier}%
\begin{itemdecl}
void arrive_and_drop();
\end{itemdecl}

\begin{itemdescr}
\pnum
\expects
The expected count for the current barrier phase is greater than zero.

\pnum
\effects
Decrements the initial expected count for all subsequent phases by one.
Then decrements the expected count for the current phase by one.

\pnum
\sync
The call to \tcode{arrive_and_drop} strongly happens before
the start of the phase completion step for the current phase.

\pnum
\throws
\tcode{system_error} when an exception is required\iref{thread.req.exception}.

\pnum
\errors
Any of the error conditions
allowed for mutex types\iref{thread.mutex.requirements.mutex}.

\pnum
\begin{note}
This call can cause the completion step for the current phase to start.
\end{note}
\end{itemdescr}

\rSec1[futures]{Futures}

\rSec2[futures.overview]{Overview}

\pnum
\ref{futures} describes components that a \Cpp{} program can use to retrieve in one thread the
result (value or exception) from a function that has run in the same thread or another thread.
\begin{note}
These components are not restricted to multi-threaded programs but can be useful in
single-threaded programs as well.
\end{note}

\rSec2[future.syn]{Header \tcode{<future>} synopsis}

\indexheader{future}%
\indexlibraryglobal{future_errc}%
\begin{codeblock}
namespace std {
  enum class future_errc {
    broken_promise = @\impdef@,
    future_already_retrieved = @\impdef@,
    promise_already_satisfied = @\impdef@,
    no_state = @\impdef@
  };

  enum class launch : @\unspec{}@ {
    async = @\unspec{}@,
    deferred = @\unspec{}@,
    @\impdef@
  };

  enum class future_status {
    ready,
    timeout,
    deferred
  };

  template<> struct is_error_code_enum<future_errc> : public true_type { };
  error_code make_error_code(future_errc e) noexcept;
  error_condition make_error_condition(future_errc e) noexcept;

  const error_category& future_category() noexcept;

  class future_error;

  template<class R> class promise;
  template<class R> class promise<R&>;
  template<> class promise<void>;

  template<class R>
    void swap(promise<R>& x, promise<R>& y) noexcept;

  template<class R, class Alloc>
    struct uses_allocator<promise<R>, Alloc>;

  template<class R> class future;
  template<class R> class future<R&>;
  template<> class future<void>;

  template<class R> class shared_future;
  template<class R> class shared_future<R&>;
  template<> class shared_future<void>;

  template<class> class packaged_task;   // not defined
  template<class R, class... ArgTypes>
    class packaged_task<R(ArgTypes...)>;

  template<class R, class... ArgTypes>
    void swap(packaged_task<R(ArgTypes...)>&, packaged_task<R(ArgTypes...)>&) noexcept;

  template<class F, class... Args>
    [[nodiscard]] future<invoke_result_t<decay_t<F>, decay_t<Args>...>>
      async(F&& f, Args&&... args);
  template<class F, class... Args>
    [[nodiscard]] future<invoke_result_t<decay_t<F>, decay_t<Args>...>>
      async(launch policy, F&& f, Args&&... args);
}
\end{codeblock}

\pnum
The \tcode{enum} type \tcode{launch} is a bitmask type\iref{bitmask.types} with
elements \tcode{launch::async} and \tcode{launch::deferred}.
\begin{note}
Implementations can provide bitmasks to specify restrictions on task
interaction by functions launched by \tcode{async()} applicable to a
corresponding subset of available launch policies. Implementations can extend
the behavior of the first overload of \tcode{async()} by adding their extensions
to the launch policy under the ``as if'' rule.
\end{note}

\pnum
The enum values of \tcode{future_errc} are distinct and not zero.

\rSec2[futures.errors]{Error handling}

\indexlibraryglobal{future_category}%
\begin{itemdecl}
const error_category& future_category() noexcept;
\end{itemdecl}

\begin{itemdescr}
\pnum
\returns
A reference to an object of a type derived from class \tcode{error_category}.

\pnum
The object's \tcode{default_error_condition} and equivalent virtual functions shall
behave as specified for the class \tcode{error_category}. The object's \tcode{name}
virtual function shall return a pointer to the string \tcode{"future"}.
\end{itemdescr}

\indexlibrarymember{make_error_code}{future_errc}%
\begin{itemdecl}
error_code make_error_code(future_errc e) noexcept;
\end{itemdecl}

\begin{itemdescr}
\pnum
\returns
\tcode{error_code(static_cast<int>(e), future_category())}.
\end{itemdescr}

\indexlibrarymember{make_error_condition}{future_errc}%
\begin{itemdecl}
error_condition make_error_condition(future_errc e) noexcept;
\end{itemdecl}

\begin{itemdescr}
\pnum
\returns
\tcode{error_condition(static_cast<int>(e), future_category())}.
\end{itemdescr}

\rSec2[futures.future.error]{Class \tcode{future_error}}

\indexlibraryglobal{future_error}%
\begin{codeblock}
namespace std {
  class future_error : public logic_error {
  public:
    explicit future_error(future_errc e);

    const error_code& code() const noexcept;
    const char*       what() const noexcept;

  private:
    error_code ec_;             // \expos
  };
}
\end{codeblock}

\indexlibraryctor{future_error}%
\begin{itemdecl}
explicit future_error(future_errc e);
\end{itemdecl}

\begin{itemdescr}
\pnum
\effects
Constructs an object of class \tcode{future_error}
and initializes \tcode{ec_} with \tcode{make_error_code(e)}.
\end{itemdescr}

\indexlibrarymember{code}{future_error}%
\begin{itemdecl}
const error_code& code() const noexcept;
\end{itemdecl}

\begin{itemdescr}
\pnum
\returns
\tcode{ec_}.
\end{itemdescr}

\indexlibrarymember{what}{future_error}%
\begin{itemdecl}
const char* what() const noexcept;
\end{itemdecl}

\begin{itemdescr}
\pnum
\returns
An \ntbs{} incorporating \tcode{code().message()}.
\end{itemdescr}

\rSec2[futures.state]{Shared state}

\pnum
Many of the classes introduced in this subclause use some state to communicate results. This
\indextext{shared state|see{future, shared state}}
\defnx{shared state}{future!shared state} consists of some state information and some (possibly not
yet evaluated) \term{result}, which can be a (possibly void) value or an exception.
\begin{note}
Futures, promises, and tasks defined in this clause reference such shared state.
\end{note}

\pnum
\begin{note}
The result can be any kind of object including a function to compute that result,
as used by \tcode{async} when \tcode{policy} is \tcode{launch::deferred}.
\end{note}

\pnum
An \defn{asynchronous return object} is an object that reads results from a
shared state. A \defn{waiting function} of an asynchronous return object is one
that potentially blocks to wait for the shared state to be made
ready.
If a waiting function can return before the state is made ready because of a
timeout\iref{thread.req.lockable}, then it is a \term{timed waiting function}, otherwise
it is a \term{non-timed waiting function}.

\pnum
An \defn{asynchronous provider} is an object that provides a result to a shared
state.
The result of a shared state is set by
respective functions on the asynchronous provider.
\begin{note}
Such as promises or tasks.
\end{note}
The means of setting the result of a shared state is specified
in the description of those classes and functions that create such a state object.

\pnum
When an asynchronous return object or an asynchronous provider is said to release its
shared state, it means:

\begin{itemize}
\item
if the return object or provider holds the last reference to its shared state,
the shared state is destroyed; and

\item
the return object or provider gives up its reference to its shared state; and

\item these actions will not block for the shared state to become ready, except that it
may block if all of the following are true: the shared state was created by a call to
\tcode{std::async}, the shared state is not yet ready, and this was the last reference
to the shared state.
\end{itemize}

\pnum
When an asynchronous provider is said to make its shared state ready, it means:

\begin{itemize}
\item
first, the provider marks its shared state as ready; and

\item
second, the provider unblocks any execution agents waiting for its shared
state to become ready.
\end{itemize}

\pnum
When an asynchronous provider is said to abandon its shared state, it means:

\begin{itemize}
\item
first, if that state is not ready, the provider

\begin{itemize}
\item
stores an exception object of type \tcode{future_error} with an error condition of
\tcode{broken_promise} within its shared state; and then

\item
makes its shared state ready;
\end{itemize}

\item
second, the provider releases its shared state.
\end{itemize}

\pnum
A shared state is \defn{ready} only if it holds a value or an exception ready for
retrieval.
Waiting for a shared state to become ready may invoke code to compute the result on
the waiting thread if so specified in the description of the class or function that creates
the state object.

\pnum
Calls to functions that successfully set the stored result of a shared
state synchronize
with\iref{intro.multithread} calls to functions
successfully detecting the ready state resulting from that setting.
The storage of the result
(whether normal or exceptional) into the shared state
synchronizes with\iref{intro.multithread}
the successful return from a call to a waiting function on the shared state.

\pnum
Some functions (e.g., \tcode{promise::set_value_at_thread_exit}) delay making
the shared state ready until the calling thread exits. The destruction of
each of that thread's objects with thread storage duration\iref{basic.stc.thread}
is sequenced before making that shared state ready.

\pnum
Access to the result of the same shared state may conflict\iref{intro.multithread}.
\begin{note}
This explicitly specifies that the result of the shared state is
visible in the objects that reference this state in the sense of data race
avoidance\iref{res.on.data.races}. For example, concurrent accesses through
references returned by \tcode{shared_future::get()}\iref{futures.shared.future}
must either use read-only operations or provide additional synchronization.
\end{note}

\rSec2[futures.promise]{Class template \tcode{promise}}

\indexlibraryglobal{promise}%
\begin{codeblock}
namespace std {
  template<class R>
  class promise {
  public:
    promise();
    template<class Allocator>
      promise(allocator_arg_t, const Allocator& a);
    promise(promise&& rhs) noexcept;
    promise(const promise& rhs) = delete;
    ~promise();

    // assignment
    promise& operator=(promise&& rhs) noexcept;
    promise& operator=(const promise& rhs) = delete;
    void swap(promise& other) noexcept;

    // retrieving the result
    future<R> get_future();

    // setting the result
    void set_value(@\seebelow@);
    void set_exception(exception_ptr p);

    // setting the result with deferred notification
    void set_value_at_thread_exit(@\seebelow@);
    void set_exception_at_thread_exit(exception_ptr p);
  };

  template<class R>
    void swap(promise<R>& x, promise<R>& y) noexcept;

  template<class R, class Alloc>
    struct uses_allocator<promise<R>, Alloc>;
}
\end{codeblock}

\pnum
The implementation shall provide the template \tcode{promise} and two specializations,
\tcode{promise<R\&>} and \tcode{promise<\brk{}void>}. These differ only in the argument type
of the member functions \tcode{set_value} and \tcode{set_value_at_thread_exit},
as set out in their descriptions, below.

\pnum
The \tcode{set_value}, \tcode{set_exception}, \tcode{set_value_at_thread_exit},
and \tcode{set_exception_at_thread_exit} member functions behave as though
they acquire a single mutex associated with the promise object while updating the
promise object.

\indexlibrarymember{uses_allocator}{promise}%
\begin{itemdecl}
template<class R, class Alloc>
  struct uses_allocator<promise<R>, Alloc>
    : true_type { };
\end{itemdecl}

\begin{itemdescr}
\pnum
\requires \tcode{Alloc} shall meet the \oldconcept{Allocator}
requirements (\tref{cpp17.allocator}).
\end{itemdescr}


\indexlibraryctor{promise}%
\begin{itemdecl}
promise();
template<class Allocator>
  promise(allocator_arg_t, const Allocator& a);
\end{itemdecl}

\begin{itemdescr}
\pnum
\effects
Constructs a \tcode{promise} object and a shared state. The second
constructor uses the allocator \tcode{a} to allocate memory for the shared
state.
\end{itemdescr}

\indexlibraryctor{promise}%
\begin{itemdecl}
promise(promise&& rhs) noexcept;
\end{itemdecl}

\begin{itemdescr}
\pnum
\effects
Constructs a new \tcode{promise} object and transfers ownership of the shared state
of \tcode{rhs} (if any) to the newly-constructed object.

\pnum
\ensures
\tcode{rhs} has no shared state.
\end{itemdescr}

\indexlibrarydtor{promise}%
\begin{itemdecl}
~promise();
\end{itemdecl}

\begin{itemdescr}
\pnum
\effects
Abandons any shared state\iref{futures.state}.
\end{itemdescr}

\indexlibrarymember{operator=}{promise}%
\begin{itemdecl}
promise& operator=(promise&& rhs) noexcept;
\end{itemdecl}

\begin{itemdescr}
\pnum
\effects
Abandons any shared state\iref{futures.state} and then as if
\tcode{promise(std::move(rhs)).swap(*this)}.

\pnum
\returns
\tcode{*this}.
\end{itemdescr}

\indexlibrarymember{swap}{promise}%
\begin{itemdecl}
void swap(promise& other) noexcept;
\end{itemdecl}

\begin{itemdescr}
\pnum
\effects
Exchanges the shared state of \tcode{*this} and \tcode{other}.

\pnum
\ensures
\tcode{*this} has the shared state (if any) that \tcode{other} had
prior to the call to \tcode{swap}. \tcode{other} has the shared state (if any) that
\tcode{*this} had prior to the call to \tcode{swap}.
\end{itemdescr}

\indexlibrarymember{get_future}{promise}%
\begin{itemdecl}
future<R> get_future();
\end{itemdecl}

\begin{itemdescr}
\pnum
\returns
A \tcode{future<R>} object with the same shared state as
\tcode{*this}.

\pnum
\sync
Calls to this function do not introduce
data races~\iref{intro.multithread} with calls to
\tcode{set_value},
\tcode{set_exception},
\tcode{set_value_at_thread_exit}, or
\tcode{set_exception_at_thread_exit}.
\begin{note}
Such calls need not synchronize with each other.
\end{note}

\pnum
\throws
\tcode{future_error} if \tcode{*this} has no shared state or if
\tcode{get_future} has already been called on a \tcode{promise} with the same
shared state as \tcode{*this}.

\pnum
\errors

\begin{itemize}
\item \tcode{future_already_retrieved} if \tcode{get_future} has already been called on
a \tcode{promise} with the same shared state as \tcode{*this}.

\item \tcode{no_state} if \tcode{*this} has no shared state.
\end{itemize}
\end{itemdescr}

\indexlibrarymember{set_value}{promise}%
\begin{itemdecl}
void promise::set_value(const R& r);
void promise::set_value(R&& r);
void promise<R&>::set_value(R& r);
void promise<void>::set_value();
\end{itemdecl}

\begin{itemdescr}
\pnum
\effects
Atomically stores the value \tcode{r} in the shared state and
makes that state ready\iref{futures.state}.

\pnum
\throws

\begin{itemize}
\item \tcode{future_error} if its shared state
already has a stored value or exception, or
\item for the first version, any exception thrown by the constructor selected to copy an object of \tcode{R}, or
\item for the second version, any exception thrown by the constructor selected to move an object of \tcode{R}.
\end{itemize}

\pnum
\errors
\begin{itemize}
\item \tcode{promise_already_satisfied} if its shared state
already has a stored value or exception.
\item \tcode{no_state} if \tcode{*this} has no shared state.
\end{itemize}
\end{itemdescr}

\indexlibrarymember{set_exception}{promise}%
\begin{itemdecl}
void set_exception(exception_ptr p);
\end{itemdecl}

\begin{itemdescr}
\pnum
\requires \tcode{p} is not null.

\pnum
\effects
Atomically stores the exception pointer \tcode{p} in the shared state
and makes that state ready\iref{futures.state}.

\pnum
\throws
\tcode{future_error} if its shared state
already has a stored value or exception.

\pnum
\errors

\begin{itemize}
\item \tcode{promise_already_satisfied} if its shared state
already has a stored value or exception.
\item \tcode{no_state} if \tcode{*this} has no shared state.
\end{itemize}
\end{itemdescr}

\indexlibrarymember{set_value_at_thread_exit}{promise}%
\begin{itemdecl}
void promise::set_value_at_thread_exit(const R& r);
void promise::set_value_at_thread_exit(R&& r);
void promise<R&>::set_value_at_thread_exit(R& r);
void promise<void>::set_value_at_thread_exit();
\end{itemdecl}

\begin{itemdescr}
\pnum
\effects
Stores the value \tcode{r} in the shared state without making that
state ready immediately. Schedules that state to be made ready when the current
thread exits, after all objects of thread storage duration associated with the
current thread have been destroyed.

\pnum
\throws

\begin{itemize}
\item \tcode{future_error} if its shared state
already has a stored value or exception, or
\item for the first version, any exception thrown by the constructor selected to copy an object of \tcode{R}, or
\item for the second version, any exception thrown by the constructor selected to move an object of \tcode{R}.
\end{itemize}

\pnum
\errors
\begin{itemize}
\item \tcode{promise_already_satisfied} if its shared state
already has a stored value or exception.
\item \tcode{no_state} if \tcode{*this} has no shared state.
\end{itemize}
\end{itemdescr}

\indexlibrarymember{set_exception_at_thread_exit}{promise}%
\begin{itemdecl}
void set_exception_at_thread_exit(exception_ptr p);
\end{itemdecl}

\begin{itemdescr}
\pnum
\requires \tcode{p} is not null.

\pnum
\effects
Stores the exception pointer \tcode{p} in the shared state without
making that state ready immediately. Schedules that state to be made ready when
the current thread exits, after all objects of thread storage duration
associated with the current thread have been destroyed.

\pnum
\throws
\tcode{future_error} if an error condition occurs.

\pnum
\errors
\begin{itemize}
\item \tcode{promise_already_satisfied} if its shared state
already has a stored value or exception.
\item \tcode{no_state} if \tcode{*this} has no shared state.
\end{itemize}
\end{itemdescr}

\indexlibrarymember{swap}{promise}%
\begin{itemdecl}
template<class R>
  void swap(promise<R>& x, promise<R>& y) noexcept;
\end{itemdecl}

\begin{itemdescr}
\pnum
\effects
As if by \tcode{x.swap(y)}.
\end{itemdescr}

\rSec2[futures.unique.future]{Class template \tcode{future}}

\pnum
The class template \tcode{future} defines a type for asynchronous return objects which
do not share their shared state with other asynchronous return objects.
A default-constructed \tcode{future} object has no
shared state. A \tcode{future} object with shared state can be created by
functions on asynchronous providers\iref{futures.state} or by the move constructor
and shares its shared state with
the original asynchronous provider. The result (value or exception) of
a \tcode{future} object
can be
set by
calling a respective function on an
object that shares the same
shared state.

\pnum
\begin{note}
Member functions of \tcode{future} do not synchronize with themselves or with
member functions of \tcode{shared_future}.
\end{note}

\pnum
The effect of calling any member function other than the destructor, the
move-assignment operator, \tcode{share}, or \tcode{valid} on a \tcode{future} object for which
\tcode{valid() == false}
is undefined.
\begin{note}
It is valid to move from a future object for which \tcode{valid() == false}.
\end{note}
\begin{note}
Implementations should detect this case and throw an object of type
\tcode{future_error} with an error condition of \tcode{future_errc::no_state}.
\end{note}

\indexlibraryglobal{future}%
\begin{codeblock}
namespace std {
  template<class R>
  class future {
  public:
    future() noexcept;
    future(future&&) noexcept;
    future(const future& rhs) = delete;
    ~future();
    future& operator=(const future& rhs) = delete;
    future& operator=(future&&) noexcept;
    shared_future<R> share() noexcept;

    // retrieving the value
    @\seebelow@ get();

    // functions to check state
    bool valid() const noexcept;

    void wait() const;
    template<class Rep, class Period>
      future_status wait_for(const chrono::duration<Rep, Period>& rel_time) const;
    template<class Clock, class Duration>
      future_status wait_until(const chrono::time_point<Clock, Duration>& abs_time) const;
  };
}
\end{codeblock}

\pnum
The implementation shall provide the template \tcode{future} and two specializations,
\tcode{future<R\&>} and \tcode{future<\brk{}void>}. These differ only in the return type and return
value of the member function \tcode{get}, as set out in its description, below.

\indexlibraryctor{future}%
\begin{itemdecl}
future() noexcept;
\end{itemdecl}

\begin{itemdescr}
\pnum
\effects
Constructs an \defnx{empty}{empty \tcode{future} object}
\tcode{future} object that does not refer to a
shared state.

\pnum
\ensures
\tcode{valid() == false}.
\end{itemdescr}

\indexlibraryctor{future}%
\begin{itemdecl}
future(future&& rhs) noexcept;
\end{itemdecl}

\begin{itemdescr}
\pnum
\effects
Move constructs a \tcode{future} object that refers to the shared
state that
was originally referred to by \tcode{rhs} (if any).

\pnum
\ensures
\begin{itemize}
\item \tcode{valid()} returns the same value as \tcode{rhs.valid()} prior to the
constructor invocation.
\item \tcode{rhs.valid() == false}.
\end{itemize}
\end{itemdescr}

\indexlibraryctor{future}%
\begin{itemdecl}
~future();
\end{itemdecl}

\begin{itemdescr}
\pnum
\effects
\begin{itemize}
\item
Releases any shared state\iref{futures.state};
\item
destroys \tcode{*this}.
\end{itemize}
\end{itemdescr}

\indexlibrarymember{operator=}{future}%
\begin{itemdecl}
future& operator=(future&& rhs) noexcept;
\end{itemdecl}

\begin{itemdescr}
\pnum
\effects
\begin{itemize}
\item
Releases any shared state\iref{futures.state}.
\item
move assigns the contents of \tcode{rhs} to \tcode{*this}.
\end{itemize}

\pnum
\ensures
\begin{itemize}
\item
\tcode{valid()} returns the same value as \tcode{rhs.valid()} prior to the
assignment.

\item
\tcode{rhs.valid() == false}.
\end{itemize}
\end{itemdescr}

\indexlibrarymember{share}{future}%
\begin{itemdecl}
shared_future<R> share() noexcept;
\end{itemdecl}

\begin{itemdescr}
\pnum
\returns
\tcode{shared_future<R>(std::move(*this))}.

\pnum
\ensures
\tcode{valid() == false}.
\end{itemdescr}

\indexlibrarymember{get}{future}%
\begin{itemdecl}
R future::get();
R& future<R&>::get();
void future<void>::get();
\end{itemdecl}

\begin{itemdescr}
\pnum
\begin{note}
As described above, the template and its two required specializations differ only in
the return type and return value of the member function \tcode{get}.
\end{note}

\pnum
\effects
\begin{itemize}
\item \tcode{wait()}{s} until the shared state is ready, then retrieves the
value stored in the shared state;
\item releases any shared state\iref{futures.state}.
\end{itemize}

\pnum
\returns
\begin{itemize}
\item
\tcode{future::get()} returns the value \tcode{v} stored in the object's shared state as
\tcode{std::move(v)}.

\item
\tcode{future<R\&>::get()} returns the reference stored as value in the object's shared state.

\item
\tcode{future<void>::get()} returns nothing.
\end{itemize}

\pnum
\throws
The stored exception, if an exception was stored in the shared state.

\pnum
\ensures
\tcode{valid() == false}.
\end{itemdescr}

\indexlibrarymember{valid}{future}%
\begin{itemdecl}
bool valid() const noexcept;
\end{itemdecl}

\begin{itemdescr}
\pnum
\returns
\tcode{true} only if \tcode{*this} refers to a shared state.
\end{itemdescr}

\indexlibrarymember{wait}{future}%
\begin{itemdecl}
void wait() const;
\end{itemdecl}

\begin{itemdescr}
\pnum
\effects
Blocks until the shared state is ready.
\end{itemdescr}

\indexlibrarymember{wait_for}{future}%
\begin{itemdecl}
template<class Rep, class Period>
  future_status wait_for(const chrono::duration<Rep, Period>& rel_time) const;
\end{itemdecl}

\begin{itemdescr}
\pnum
\effects
None if the shared state contains a deferred function\iref{futures.async},
otherwise
blocks until the shared state is ready or until
the relative timeout\iref{thread.req.timing} specified by \tcode{rel_time} has expired.

\pnum
\returns

\begin{itemize}
\item \tcode{future_status::deferred} if the shared state contains a deferred
function.

\item \tcode{future_status::ready} if the shared state is ready.

\item \tcode{future_status::timeout} if the function is returning because the
relative timeout\iref{thread.req.timing}
specified by \tcode{rel_time} has expired.
\end{itemize}

\pnum
\throws
timeout-related exceptions\iref{thread.req.timing}.
\end{itemdescr}

\indexlibrarymember{wait_until}{future}%
\begin{itemdecl}
template<class Clock, class Duration>
  future_status wait_until(const chrono::time_point<Clock, Duration>& abs_time) const;
\end{itemdecl}

\begin{itemdescr}
\pnum
\effects
None if the shared state contains a deferred function\iref{futures.async},
otherwise
blocks until the shared state is ready or until
the absolute timeout\iref{thread.req.timing} specified by \tcode{abs_time} has expired.

\pnum
\returns

\begin{itemize}
\item \tcode{future_status::deferred} if the shared state contains a deferred
function.

\item \tcode{future_status::ready} if the shared state is ready.

\item \tcode{future_status::timeout} if the function is returning because the
absolute timeout\iref{thread.req.timing}
specified by \tcode{abs_time} has expired.
\end{itemize}

\pnum
\throws
timeout-related exceptions\iref{thread.req.timing}.
\end{itemdescr}


\rSec2[futures.shared.future]{Class template \tcode{shared_future}}

\pnum
The class template \tcode{shared_future} defines a type for asynchronous return objects
which may share their shared state with other asynchronous return
objects. A default-constructed \tcode{shared_future}
object has no shared state. A \tcode{shared_future} object with
shared state can
be created
by conversion from a \tcode{future} object and shares its shared state with the
original asynchronous provider\iref{futures.state} of the shared state.
The result (value or exception) of a \tcode{shared_future} object
can be set by
calling a respective function on an
object that shares the same shared state.

\pnum
\begin{note}
Member functions of \tcode{shared_future} do not synchronize with themselves,
but they synchronize with the shared state.
\end{note}

\pnum
The effect of calling any member function other than the destructor,
the move-assignment operator, the copy-assignment operator, or
\tcode{valid()} on a \tcode{shared_future} object for which \tcode{valid() == false} is undefined.
\begin{note}
It is valid to copy or move from a \tcode{shared_future}
object for which \tcode{valid()} is \tcode{false}.
\end{note}
\begin{note}
Implementations should detect this case and throw an object of type
\tcode{future_error} with an error condition of \tcode{future_errc::no_state}.
\end{note}

\indexlibraryglobal{shared_future}%
\begin{codeblock}
namespace std {
  template<class R>
  class shared_future {
  public:
    shared_future() noexcept;
    shared_future(const shared_future& rhs) noexcept;
    shared_future(future<R>&&) noexcept;
    shared_future(shared_future&& rhs) noexcept;
    ~shared_future();
    shared_future& operator=(const shared_future& rhs) noexcept;
    shared_future& operator=(shared_future&& rhs) noexcept;

    // retrieving the value
    @\seebelow@ get() const;

    // functions to check state
    bool valid() const noexcept;

    void wait() const;
    template<class Rep, class Period>
      future_status wait_for(const chrono::duration<Rep, Period>& rel_time) const;
    template<class Clock, class Duration>
      future_status wait_until(const chrono::time_point<Clock, Duration>& abs_time) const;
  };
}
\end{codeblock}

\pnum
The implementation shall provide the template \tcode{shared_future} and two
specializations, \tcode{shared_future<R\&>} and \tcode{shared_future<void>}. These
differ only in the return type and return value of the member function \tcode{get}, as
set out in its description, below.

\indexlibraryctor{shared_future}%
\begin{itemdecl}
shared_future() noexcept;
\end{itemdecl}

\begin{itemdescr}
\pnum
\effects
Constructs an \defnx{empty}{empty \tcode{shared_future} object}
\tcode{shared_future} object that does not refer to a
shared state.

\pnum
\ensures
\tcode{valid() == false}.
\end{itemdescr}

\indexlibraryctor{shared_future}%
\begin{itemdecl}
shared_future(const shared_future& rhs) noexcept;
\end{itemdecl}

\begin{itemdescr}
\pnum
\effects
Constructs a \tcode{shared_future} object that refers to the same
shared state as \tcode{rhs} (if any).

\pnum
\ensures
\tcode{valid()} returns the same value as \tcode{rhs.valid()}.
\end{itemdescr}

\indexlibraryctor{shared_future}%
\begin{itemdecl}
shared_future(future<R>&& rhs) noexcept;
shared_future(shared_future&& rhs) noexcept;
\end{itemdecl}

\begin{itemdescr}
\pnum
\effects
Move constructs a \tcode{shared_future} object that refers to the
shared state that was originally referred to by \tcode{rhs} (if any).

\pnum
\ensures
\begin{itemize}
\item \tcode{valid()} returns the same value as \tcode{rhs.valid()} returned prior to
the constructor invocation.
\item \tcode{rhs.valid() == false}.
\end{itemize}
\end{itemdescr}

\indexlibrarydtor{shared_future}%
\begin{itemdecl}
~shared_future();
\end{itemdecl}

\begin{itemdescr}
\pnum
\effects
\begin{itemize}
\item
Releases any shared state\iref{futures.state};
\item
destroys \tcode{*this}.
\end{itemize}
\end{itemdescr}

\indexlibrarymember{operator=}{shared_future}%
\begin{itemdecl}
shared_future& operator=(shared_future&& rhs) noexcept;
\end{itemdecl}

\begin{itemdescr}
\pnum
\effects
\begin{itemize}
\item
Releases any shared state\iref{futures.state};
\item
move assigns the contents of \tcode{rhs} to \tcode{*this}.
\end{itemize}

\pnum
\ensures
\begin{itemize}
\item
\tcode{valid()} returns the same value as \tcode{rhs.valid()} returned prior to
the assignment.

\item
\tcode{rhs.valid() == false}.
\end{itemize}
\end{itemdescr}

\indexlibrarymember{operator=}{shared_future}%
\begin{itemdecl}
shared_future& operator=(const shared_future& rhs) noexcept;
\end{itemdecl}

\begin{itemdescr}
\pnum
\effects
\begin{itemize}
\item
Releases any shared state\iref{futures.state};
\item
assigns the contents of \tcode{rhs} to \tcode{*this}.
\begin{note}
As a result,
\tcode{*this} refers to the same shared state as \tcode{rhs}
(if any).
\end{note}
\end{itemize}

\pnum
\ensures
\tcode{valid() == rhs.valid()}.
\end{itemdescr}

\indexlibrarymember{get}{shared_future}%
\begin{itemdecl}
const R& shared_future::get() const;
R& shared_future<R&>::get() const;
void shared_future<void>::get() const;
\end{itemdecl}

\begin{itemdescr}
\pnum
\begin{note}
As described above, the template and its two required specializations differ only in
the return type and return value of the member function \tcode{get}.
\end{note}

\pnum
\begin{note}
Access to a value object stored in the shared state is
unsynchronized, so programmers should apply only those operations on \tcode{R} that do not
introduce a data race\iref{intro.multithread}.
\end{note}

\pnum
\effects
\tcode{wait()}{s} until the shared state is ready, then retrieves the
value stored in the shared state.

\pnum
\returns
\begin{itemize}
\item
\tcode{shared_future::get()} returns a const reference to the value stored in the object's
shared state.
\begin{note}
Access through that reference after the shared state has been
destroyed produces undefined behavior; this can be avoided by not storing the reference in any
storage with a greater lifetime than the \tcode{shared_future} object that returned the
reference.
\end{note}

\item
\tcode{shared_future<R\&>::get()} returns the reference stored as value in the object's
shared state.

\item
\tcode{shared_future<void>::get()} returns nothing.
\end{itemize}

\pnum
\throws
The stored exception, if an exception was stored in the shared state.
\end{itemdescr}

\indexlibrarymember{valid}{shared_future}%
\begin{itemdecl}
bool valid() const noexcept;
\end{itemdecl}

\begin{itemdescr}
\pnum
\returns
\tcode{true} only if \tcode{*this} refers to a shared state.
\end{itemdescr}

\indexlibrarymember{wait}{shared_future}%
\begin{itemdecl}
void wait() const;
\end{itemdecl}

\begin{itemdescr}
\pnum
\effects
Blocks until the shared state is ready.
\end{itemdescr}

\indexlibrarymember{wait_for}{shared_future}%
\begin{itemdecl}
template<class Rep, class Period>
  future_status wait_for(const chrono::duration<Rep, Period>& rel_time) const;
\end{itemdecl}

\begin{itemdescr}
\pnum
\effects
None if the shared state contains a deferred function\iref{futures.async},
otherwise
blocks until the shared state is ready or until
the relative timeout\iref{thread.req.timing} specified by
\tcode{rel_time} has expired.

\pnum
\returns

\begin{itemize}
\item \tcode{future_status::deferred} if the shared state contains a deferred
function.

\item \tcode{future_status::ready} if the shared state is ready.

\item \tcode{future_status::timeout} if the function is returning because the
relative timeout\iref{thread.req.timing}
specified by \tcode{rel_time} has expired.
\end{itemize}

\pnum
\throws
timeout-related exceptions\iref{thread.req.timing}.
\end{itemdescr}

\indexlibrarymember{wait_until}{shared_future}%
\begin{itemdecl}
template<class Clock, class Duration>
  future_status wait_until(const chrono::time_point<Clock, Duration>& abs_time) const;
\end{itemdecl}

\begin{itemdescr}
\pnum
\effects
None if the shared state contains a deferred function\iref{futures.async},
otherwise
blocks until the shared state is ready or until the
absolute timeout\iref{thread.req.timing} specified by
\tcode{abs_time} has expired.

\pnum
\returns

\begin{itemize}
\item \tcode{future_status::deferred} if the shared state contains a deferred
function.

\item \tcode{future_status::ready} if the shared state is ready.

\item \tcode{future_status::timeout} if the function is returning because the
absolute timeout\iref{thread.req.timing}
specified by \tcode{abs_time} has expired.
\end{itemize}

\pnum
\throws
timeout-related exceptions\iref{thread.req.timing}.
\end{itemdescr}

\rSec2[futures.async]{Function template \tcode{async}}

\pnum
The function template \tcode{async} provides a mechanism to launch a function potentially
in a new thread and provides the result of the function in a \tcode{future} object with which
it shares a shared state.

\indexlibraryglobal{async}%
\begin{itemdecl}
template<class F, class... Args>
  [[nodiscard]] future<invoke_result_t<decay_t<F>, decay_t<Args>...>>
    async(F&& f, Args&&... args);
template<class F, class... Args>
  [[nodiscard]] future<invoke_result_t<decay_t<F>, decay_t<Args>...>>
    async(launch policy, F&& f, Args&&... args);
\end{itemdecl}

\begin{itemdescr}
\pnum
\requires \tcode{F} and each $\tcode{T}_i$ in \tcode{Args} shall
meet the
\oldconcept{MoveConstructible} requirements, and
\begin{codeblock}
@\placeholdernc{INVOKE}@(@\placeholdernc{decay-copy}@(std::forward<F>(f)),
       @\placeholdernc{decay-copy}@(std::forward<Args>(args))...)     // see \ref{func.require}, \ref{thread.thread.constr}
\end{codeblock}
shall be a valid expression.

\pnum
\effects
The first function
behaves the same as a call to the second function with a \tcode{policy} argument of
\tcode{launch::async | launch::deferred}
and the same arguments for \tcode{F} and \tcode{Args}.
The second function creates a shared state that is associated with
the returned \tcode{future} object.
The further behavior
of the second function depends on the \tcode{policy} argument as follows (if
more than one of these conditions applies, the implementation may choose any of
the corresponding policies):

\begin{itemize}
\item
If \tcode{launch::async} is set in \tcode{policy}, calls
\tcode{\placeholdernc{INVOKE}(\placeholdernc{decay-copy}(std::forward<F>(f)),}
\tcode{\placeholdernc{decay-\brk{}copy}(std::forward<Args>(args))...)}
(\ref{func.require}, \ref{thread.thread.constr})
as if in a new thread of execution represented by a \tcode{thread} object
with the calls to \tcode{\placeholder{decay-copy}} being evaluated in the thread that called \tcode{async}.
Any return value
is stored as the result in the
shared state. Any exception propagated from
the execution of
\tcode{\placeholdernc{INVOKE}(\placeholdernc{decay-copy}(std::forward<F>(f)),
\placeholdernc{decay-copy}(std::forward\brk{}<Args>(args))...)}
is stored as the exceptional result in the shared state.
The \tcode{thread} object is
stored in the shared state
and affects the behavior of any asynchronous return objects that
reference that state.

\item
If \tcode{launch::deferred} is set in \tcode{policy},
stores \tcode{\placeholdernc{decay-copy}(std::forward<F>(f))} and
\tcode{\placeholdernc{decay-\brk{}copy}(std::forward<Args>(args))...}
in the shared state. These copies of \tcode{f} and \tcode{args} constitute
a \defn{deferred function}. Invocation of the deferred function evaluates
\tcode{\placeholdernc{INVOKE}(std::move(g), std::move(xyz))} where \tcode{g} is the stored value of
\tcode{\placeholdernc{decay-copy}(std::forward<F>(f))} and \tcode{xyz} is the stored copy of
\tcode{\placeholdernc{decay-copy}(std::forward<Args>(args))...}.
Any return value is stored
as the result in the shared state.
Any exception propagated
from the execution
of the deferred function
is stored as the exceptional result
in the shared state.
The shared state is not
made ready until the function has completed. The first call to a
non-timed waiting function\iref{futures.state}
on an asynchronous return object referring to
this shared state shall invoke the
deferred function in the thread that called the waiting function.
Once evaluation of \tcode{\placeholdernc{INVOKE}(std::move(g), std::move(xyz))} begins, the function is no longer
considered deferred.
\begin{note}
If this policy is specified together with other policies, such as when using a
\tcode{policy} value of \tcode{launch::async | launch::deferred}, implementations should defer
invocation or the selection of the policy when no more concurrency can be effectively
exploited.
\end{note}

\item
If no value is set in the launch policy, or a value is set that is neither specified
in this document nor by the implementation, the behavior is undefined.
\end{itemize}

\pnum
\returns
An object of type
\tcode{future<invoke_result_t<decay_t<F>, decay_t<Args>...>{>}} that refers
to the shared state created by this call to \tcode{async}.
\begin{note}
If a future obtained from \tcode{async} is moved outside the local scope,
other code that uses the future should be aware that the future's destructor may
block for the shared state to become ready.
\end{note}

\pnum
\sync
Regardless of the provided \tcode{policy} argument,
\begin{itemize}
\item
the invocation of \tcode{async}
synchronizes with\iref{intro.multithread} the invocation of \tcode{f}.
\begin{note}
This statement applies even when the corresponding \tcode{future} object is moved to
another thread.
\end{note}
; and

\item
the completion of the function \tcode{f} is sequenced before\iref{intro.multithread} the
shared state is made ready.
\begin{note}
\tcode{f} might not be called at all,
so its completion might never happen.
\end{note}
\end{itemize}

If the implementation chooses the \tcode{launch::async} policy,

\begin{itemize}
\item
a call to a waiting function on an asynchronous return
object that shares the shared state created by this \tcode{async} call shall
block until the associated thread has completed, as if joined, or else time
out\iref{thread.thread.member};

\item
the associated thread completion
synchronizes with\iref{intro.multithread}
the return from
the first function
that successfully detects the ready status of the shared state or
with the return from the last
function that releases the shared state, whichever
happens first.
\end{itemize}


\pnum
\throws
\tcode{system_error} if \tcode{policy == launch::async} and the
implementation is unable to start a new thread, or
\tcode{std::bad_alloc} if memory for the internal data structures
could not be allocated.

\pnum
\errors
\begin{itemize}
\item \tcode{resource_unavailable_try_again} --- if
\tcode{policy == launch::async} and the system is unable to start a new thread.
\end{itemize}
\end{itemdescr}

\pnum
\begin{example}
\begin{codeblock}
int work1(int value);
int work2(int value);
int work(int value) {
  auto handle = std::async([=]{ return work2(value); });
  int tmp = work1(value);
  return tmp + handle.get();    // \#1
}
\end{codeblock}

\begin{note}
Line \#1 might not result in concurrency because
the \tcode{async} call uses the default policy, which may use
\tcode{launch::deferred}, in which case the lambda might not be invoked until the
\tcode{get()} call; in that case, \tcode{work1} and \tcode{work2} are called on the
same thread and there is no concurrency.
\end{note}
\end{example}

\rSec2[futures.task]{Class template \tcode{packaged_task}}

\pnum
The class template \tcode{packaged_task} defines a type for wrapping a function or
callable object so that the return value of the function or callable object is stored in
a future when it is invoked.

\pnum
When the \tcode{packaged_task} object is invoked, its stored task is invoked and the
result (whether normal or exceptional) stored in the shared state. Any futures that
share the shared state will then be able to access the stored result.

\indexlibraryglobal{packaged_task}%
\begin{codeblock}
namespace std {
  template<class> class packaged_task; // not defined

  template<class R, class... ArgTypes>
  class packaged_task<R(ArgTypes...)> {
  public:
    // construction and destruction
    packaged_task() noexcept;
    template<class F>
      explicit packaged_task(F&& f);
    ~packaged_task();

    // no copy
    packaged_task(const packaged_task&) = delete;
    packaged_task& operator=(const packaged_task&) = delete;

    // move support
    packaged_task(packaged_task&& rhs) noexcept;
    packaged_task& operator=(packaged_task&& rhs) noexcept;
    void swap(packaged_task& other) noexcept;

    bool valid() const noexcept;

    // result retrieval
    future<R> get_future();

    // execution
    void operator()(ArgTypes... );
    void make_ready_at_thread_exit(ArgTypes...);

    void reset();
  };

  template<class R, class... ArgTypes>
    void swap(packaged_task<R(ArgTypes...)>& x, packaged_task<R(ArgTypes...)>& y) noexcept;
}
\end{codeblock}

\rSec3[futures.task.members]{Member functions}

\indexlibraryctor{packaged_task}%
\begin{itemdecl}
packaged_task() noexcept;
\end{itemdecl}

\begin{itemdescr}
\pnum
\effects
Constructs a \tcode{packaged_task} object with no shared state and no stored task.
\end{itemdescr}

\indexlibraryctor{packaged_task}%
\begin{itemdecl}
template<class F>
  packaged_task(F&& f);
\end{itemdecl}

\begin{itemdescr}
\pnum
\requires
\tcode{\placeholdernc{INVOKE}<R>(f, t$_1$, t$_2$, $\dotsc$, t$_N$)}\iref{func.require}, where \tcode{t$_1$, t$_2$, $\dotsc$, t$_N$} are values
of the corresponding types in \tcode{ArgTypes...}, shall be a valid expression. Invoking
a copy of \tcode{f} shall behave the same as invoking \tcode{f}.

\pnum
\remarks
This constructor shall not participate in overload resolution if \tcode{remove_cvref_t<F>}
is the same type as \tcode{packaged_task<R(ArgTypes...)>}.

\pnum
\effects
Constructs a new \tcode{packaged_task} object with a shared state and
initializes the object's stored task with \tcode{std::forward<F>(f)}.

\pnum
\throws
Any exceptions thrown by the copy or move constructor of \tcode{f}, or
\tcode{bad_alloc} if memory for the internal data structures
could not be allocated.
\end{itemdescr}

\indexlibraryctor{packaged_task}%
\begin{itemdecl}
packaged_task(packaged_task&& rhs) noexcept;
\end{itemdecl}

\begin{itemdescr}
\pnum
\effects
Constructs a new \tcode{packaged_task} object and transfers ownership of
\tcode{rhs}'s shared state to \tcode{*this}, leaving \tcode{rhs} with no
shared state. Moves the stored task from \tcode{rhs} to \tcode{*this}.

\pnum
\ensures
\tcode{rhs} has no shared state.
\end{itemdescr}

\indexlibrarymember{operator=}{packaged_task}%
\begin{itemdecl}
packaged_task& operator=(packaged_task&& rhs) noexcept;
\end{itemdecl}

\begin{itemdescr}
\pnum
\effects
\begin{itemize}
\item
Releases any shared state\iref{futures.state};
\item
calls \tcode{packaged_task(std::move(rhs)).swap(*this)}.
\end{itemize}
\end{itemdescr}

\indexlibrarydtor{packaged_task}%
\begin{itemdecl}
~packaged_task();
\end{itemdecl}

\begin{itemdescr}
\pnum
\effects
Abandons any shared state\iref{futures.state}.
\end{itemdescr}

\indexlibrarymember{swap}{packaged_task}%
\begin{itemdecl}
void swap(packaged_task& other) noexcept;
\end{itemdecl}

\begin{itemdescr}
\pnum
\effects
Exchanges the shared states and stored tasks of \tcode{*this} and \tcode{other}.

\pnum
\ensures
\tcode{*this} has the same shared state
and stored task (if any) as \tcode{other}
prior to the call to \tcode{swap}. \tcode{other} has the same shared state
and stored task (if any)
as \tcode{*this} prior to the call to \tcode{swap}.
\end{itemdescr}

\indexlibrarymember{valid}{packaged_task}%
\begin{itemdecl}
bool valid() const noexcept;
\end{itemdecl}

\begin{itemdescr}
\pnum
\returns
\tcode{true} only if \tcode{*this} has a shared state.
\end{itemdescr}

\indexlibrarymember{get_future}{packaged_task}%
\begin{itemdecl}
future<R> get_future();
\end{itemdecl}

\begin{itemdescr}
\pnum
\returns
A \tcode{future} object that shares the same shared state as \tcode{*this}.

\pnum
\sync
Calls to this function do not introduce
data races~\iref{intro.multithread} with calls to
\tcode{operator()} or
\tcode{make_ready_at_thread_exit}.
\begin{note}
Such calls need not synchronize with each other.
\end{note}

\pnum
\throws
A \tcode{future_error} object if an error occurs.

\pnum
\errors
\begin{itemize}
\item \tcode{future_already_retrieved} if \tcode{get_future} has already been called on
a \tcode{packaged_task} object with the same shared state as \tcode{*this}.
\item \tcode{no_state} if \tcode{*this} has no shared state.
\end{itemize}
\end{itemdescr}

\indexlibrarymember{operator()}{packaged_task}%
\begin{itemdecl}
void operator()(ArgTypes... args);
\end{itemdecl}

\begin{itemdescr}
\pnum
\effects
As if by \tcode{\placeholdernc{INVOKE}<R>(f, t$_1$, t$_2$, $\dotsc$, t$_N$)}\iref{func.require},
where \tcode{f} is the
stored task of \tcode{*this} and
\tcode{t$_1$, t$_2$, $\dotsc$, t$_N$} are the values in \tcode{args...}. If the task returns normally,
the return value is stored as the asynchronous result in the shared state of
\tcode{*this}, otherwise the exception thrown by the task is stored. The
shared state of \tcode{*this} is made ready, and any threads blocked in a
function waiting for
the shared state of \tcode{*this} to become ready are unblocked.

\pnum
\throws
A \tcode{future_error} exception object if there is no shared
state or the stored task has already been invoked.

\pnum
\errors
\begin{itemize}
\item \tcode{promise_already_satisfied} if
the stored task has already been invoked.
\item \tcode{no_state} if \tcode{*this} has no shared state.
\end{itemize}
\end{itemdescr}

\indexlibrarymember{make_ready_at_thread_exit}{packaged_task}%
\begin{itemdecl}
void make_ready_at_thread_exit(ArgTypes... args);
\end{itemdecl}

\begin{itemdescr}
\pnum
\effects
As if by \tcode{\placeholdernc{INVOKE}<R>(f, t$_1$, t$_2$, $\dotsc$, t$_N$)}\iref{func.require},
where \tcode{f} is the stored task and
\tcode{t$_1$, t$_2$, $\dotsc$, t$_N$} are the values in \tcode{args...}. If the task returns normally,
the return value is stored as the asynchronous result in the shared state of
\tcode{*this}, otherwise the exception thrown by the task is stored. In either
case, this shall be done without making that state ready\iref{futures.state} immediately. Schedules
the shared state to be made ready when the current thread exits,
after all objects of thread storage duration associated with the current thread
have been destroyed.

\pnum
\throws
\tcode{future_error} if an error condition occurs.

\pnum
\errors
\begin{itemize}
\item \tcode{promise_already_satisfied} if the
stored task has already been invoked.
\item \tcode{no_state} if \tcode{*this} has no shared state.
\end{itemize}
\end{itemdescr}

\indexlibrarymember{reset}{packaged_task}%
\begin{itemdecl}
void reset();
\end{itemdecl}

\begin{itemdescr}
\pnum
\effects
As if \tcode{*this = packaged_task(std::move(f))}, where
\tcode{f} is the task stored in
\tcode{*this}.
\begin{note}
This constructs a new shared state for \tcode{*this}. The
old state is abandoned\iref{futures.state}.
\end{note}

\pnum
\throws
\begin{itemize}
\item \tcode{bad_alloc} if memory for the new shared state could not be allocated.
\item any exception thrown by the move constructor of the task stored in the shared
state.
\item \tcode{future_error} with an error condition of \tcode{no_state} if \tcode{*this}
has no shared state.
\end{itemize}
\end{itemdescr}

\rSec3[futures.task.nonmembers]{Globals}

\indexlibrarymember{swap}{packaged_task}%
\begin{itemdecl}
template<class R, class... ArgTypes>
  void swap(packaged_task<R(ArgTypes...)>& x, packaged_task<R(ArgTypes...)>& y) noexcept;
\end{itemdecl}

\begin{itemdescr}
\pnum
\effects
As if by \tcode{x.swap(y)}.
\end{itemdescr}
