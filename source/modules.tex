%!TEX root = std.tex
\rSec0[module]{Modules}%

\gramSec[gram.module]{Modules}

\rSec1[module.unit]{Module units and purviews}

\begin{bnf}
\nontermdef{module-declaration}\br
    \opt{export-keyword} module-keyword module-name \opt{module-partition} \opt{attribute-specifier-seq} \terminal{;}
\end{bnf}

\begin{bnf}
\nontermdef{module-name}\br
    \opt{module-name-qualifier} identifier
\end{bnf}

\begin{bnf}
\nontermdef{module-partition}\br
    \terminal{:} \opt{module-name-qualifier} identifier
\end{bnf}

\begin{bnf}
\nontermdef{module-name-qualifier}\br
    identifier \terminal{.}\br
    module-name-qualifier identifier \terminal{.}
\end{bnf}

\pnum
A \defn{module unit} is a translation unit that contains
a \grammarterm{module-declaration}.
A \defnadj{named}{module} is the
collection of module units with the same \grammarterm{module-name}.
The identifiers \tcode{module} and \tcode{import}
shall not appear as \grammarterm{identifier}{s}
in a \grammarterm{module-name} or \grammarterm{module-partition}.
\indextext{module!reserved name of}%
All \grammarterm{module-name}{s} either beginning with an \grammarterm{identifier}
consisting of \tcode{std} followed by zero or more \grammarterm{digit}{s} or
containing a reserved identifier\iref{lex.name}
are reserved and shall not be specified in a \grammarterm{module-declaration};
no diagnostic is required.
If any \grammarterm{identifier} in a reserved \grammarterm{module-name}
is a reserved identifier,
the module name is reserved for use by \Cpp{} implementations;
otherwise it is reserved for future standardization.
The optional \grammarterm{attribute-specifier-seq}
appertains to the \grammarterm{module-declaration}.

\pnum
A \defn{module interface unit} is a module unit whose
\grammarterm{module-declaration} starts with \grammarterm{export-keyword};
any other module unit is a \defn{module implementation unit}.
A named module shall contain exactly one module interface unit
with no \grammarterm{module-partition}, known as the
\defn{primary module interface unit} of the module;
no diagnostic is required.

\pnum
A \defn{module partition} is
a module unit whose \grammarterm{module-declaration} contains
a \grammarterm{module-partition}.
A named module shall not contain multiple module partitions with
the same \grammarterm{module-partition}.
All module partitions of a module
that are module interface units
shall be directly or indirectly exported
by the primary module interface unit\iref{module.import}.
No diagnostic is required for a violation of these rules.
\begin{note}
Module partitions can be imported only by
other module units in the same module.
The division of a module into module units
is not visible outside the module.
\end{note}

\pnum
\begin{example}
\begin{codeblocktu}{Translation unit \#1}
export module A;
export import :Foo;
export int baz();
\end{codeblocktu}

\begin{codeblocktu}{Translation unit \#2}
export module A:Foo;
import :Internals;
export int foo() { return 2 * (bar() + 1); }
\end{codeblocktu}

\begin{codeblocktu}{Translation unit \#3}
module A:Internals;
int bar();
\end{codeblocktu}

\begin{codeblocktu}{Translation unit \#4}
module A;
import :Internals;
int bar() { return baz() - 10; }
int baz() { return 30; }
\end{codeblocktu}

Module \tcode{A} contains four translation units:
\begin{itemize}
\item a primary module interface unit,
\item a module partition \tcode{A:Foo}, which is a module interface unit
forming part of the interface of module \tcode{A},
\item a module partition \tcode{A:Internals}, which does not contribute
to the external interface of module \tcode{A}, and
\item a module implementation unit providing
a definition of \tcode{bar} and \tcode{baz},
which cannot be imported because
it does not have a partition name.
\end{itemize}
\end{example}

\pnum
A \defnadj{module unit}{purview} is
the sequence of \grammarterm{token}{s}
starting at the \grammarterm{module-declaration}
and extending to the end of the translation unit.
The \defnx{purview}{purview!named module}
of a named module \tcode{M} is the set of module unit purviews
of \tcode{M}'s module units.

\pnum
The \defnadj{global}{module} is the collection of all
\grammarterm{global-module-fragment}{s}
and all translation units that are not module units.
Declarations appearing in such a context
are said to be in the \defnx{purview}{purview!global module} of the global module.
\begin{note}
The global module has no name, no module interface unit, and is not
introduced by any \grammarterm{module-declaration}.
\end{note}

\pnum
A \defn{module} is either a named module or the global module.
A declaration is \defnx{attached}{attached!declaration} to a module as follows:
\begin{itemize}
\item
If the declaration is a non-dependent friend declaration
that nominates a function with a \grammarterm{declarator-id}
that is a \grammarterm{qualified-id} or \grammarterm{template-id} or
that nominates a class
other than with an \grammarterm{elaborated-type-specifier} with neither
a \grammarterm{nested-name-specifier} nor a \grammarterm{simple-template-id},
it is attached to the module to which the friend is attached\iref{basic.link}.
\item Otherwise, if the declaration
\begin{itemize}
\item is a \grammarterm{namespace-definition} with external linkage or
\item appears within a \grammarterm{linkage-specification}\iref{dcl.link}
\end{itemize}
it is attached to the global module.

\item Otherwise, the declaration is
attached to the module in whose purview it appears.
\end{itemize}

\pnum
A \grammarterm{module-declaration}
that contains neither an \grammarterm{export-keyword}
nor a \grammarterm{module-partition}
implicitly imports the primary module interface unit of the module
as if by a \grammarterm{module-import-declaration}.
\begin{example}
\begin{codeblocktu}{Translation unit \#1}
module B:Y;                     // does not implicitly import \tcode{B}
int y();
\end{codeblocktu}

\begin{codeblocktu}{Translation unit \#2}
export module B;
import :Y;                      // OK, does not create interface dependency cycle
int n = y();
\end{codeblocktu}

\begin{codeblocktu}{Translation unit \#3}
module B:X1;                    // does not implicitly import \tcode{B}
int &a = n;                     // error: \tcode{n} not visible here
\end{codeblocktu}

\begin{codeblocktu}{Translation unit \#4}
module B:X2;                    // does not implicitly import \tcode{B}
import B;
int &b = n;                     // OK
\end{codeblocktu}

\begin{codeblocktu}{Translation unit \#5}
module B;                       // implicitly imports \tcode{B}
int &c = n;                     // OK
\end{codeblocktu}
\end{example}

\rSec1[module.interface]{Export declaration}%

\begin{bnf}
\nontermdef{export-declaration}\br
    \keyword{export} name-declaration\br
    \keyword{export} \terminal{\{} \opt{declaration-seq} \terminal{\}}\br
    export-keyword module-import-declaration
\end{bnf}

\pnum
An \grammarterm{export-declaration} shall inhabit
a namespace scope and appear in the purview of a module interface unit.
An \grammarterm{export-declaration} shall not appear directly
or indirectly within an unnamed namespace
or a \grammarterm{private-module-fragment}.
An \grammarterm{export-declaration}
has the declarative effects of its
\grammarterm{name-declaration},
\grammarterm{declaration-seq} (if any), or
\grammarterm{module-import-declaration}.
The \grammarterm{name-declaration} of an \grammarterm{export-declaration}
shall not declare a partial specialization\iref{temp.decls.general}.
The \grammarterm{declaration-seq} of
an \grammarterm{export-declaration}
shall not contain an \grammarterm{export-declaration} or
\grammarterm{module-import-declaration}.
\begin{note}
An \grammarterm{export-declaration} does not establish a scope.
\end{note}

\pnum
A declaration is \defnx{exported}{declaration!exported} if it is
declared within an \grammarterm{export-declaration} and
inhabits a namespace scope or it is
\begin{itemize}
\item a \grammarterm{namespace-definition} that contains an
      exported declaration, or
\item a declaration within a header unit\iref{module.import}
      that introduces at least one name.
\end{itemize}

\pnum
If an exported declaration is not within a header unit,
it shall not declare a name with internal linkage.

\pnum
\begin{example}
\begin{codeblocktu}{Source file \tcode{"a.h"}}
export int x;
\end{codeblocktu}

\begin{codeblocktu}{Translation unit \#1}
module;
#include "a.h"                  // error: declaration of \tcode{x} is not in the
                                // purview of a module interface unit
export module M;
export namespace {}             // error: namespace has internal linkage
namespace {
  export int a2;                // error: export of name with internal linkage
}
export static int b;            // error: b explicitly declared static
export int f();                 // OK
export namespace N { }          // OK
export using namespace N;       // OK
\end{codeblocktu}
\end{example}

\pnum
If an exported declaration is a \grammarterm{using-declaration}\iref{namespace.udecl}
and is not within a header unit,
all entities to which all of the
\grammarterm{using-declarator}{s} ultimately refer (if any)
shall have been introduced with a name having external linkage.
\begin{example}
\begin{codeblocktu}{Source file \tcode{"b.h"}}
int f();
\end{codeblocktu}

\begin{codeblocktu}{Importable header \tcode{"c.h"}}
int g();
\end{codeblocktu}

\begin{codeblocktu}{Translation unit \#1}
export module X;
export int h();
\end{codeblocktu}

\begin{codeblocktu}{Translation unit \#2}
module;
#include "b.h"
export module M;
import "c.h";
import X;
export using ::f, ::g, ::h;     // OK
struct S;
export using ::S;               // error: \tcode{S} has module linkage
namespace N {
  export int h();
  static int h(int);            // \#1
}
export using N::h;              // error: \#1 has internal linkage
\end{codeblocktu}
\end{example}
\begin{note}
These constraints do not apply to
type names introduced by \keyword{typedef} declarations
and \grammarterm{alias-declaration}{s}.
\begin{example}
\begin{codeblock}
export module M;
struct S;
export using T = S;             // OK, exports name \tcode{T} denoting type \tcode{S}
\end{codeblock}
\end{example}
\end{note}

\pnum
A redeclaration of an entity $X$
is implicitly exported
if $X$ was introduced by an exported declaration;
otherwise it shall not be exported.
\begin{example}
\begin{codeblock}
export module M;
struct S { int n; };
typedef S S;
export typedef S S;             // OK, does not redeclare an entity
export struct S;                // error: exported declaration follows non-exported declaration
\end{codeblock}
\end{example}

\pnum
\begin{note}
Names introduced by exported declarations
have either external linkage or no linkage; see \ref{basic.link}.
Namespace-scope declarations exported by a module can be found by name lookup
in any translation unit importing that module\iref{basic.lookup}.
Class and enumeration member names can be found by name lookup in any
context in which a definition of the type is reachable.
\end{note}
\begin{example}
\begin{codeblocktu}{Interface unit of \tcode{M}}
export module M;
export struct X {
  static void f();
  struct Y { };
};

namespace {
  struct S { };
}
export void f(S);               // OK
struct T { };
export T id(T);                 // OK

export struct A;                // \tcode{A} exported as incomplete

export auto rootFinder(double a) {
  return [=](double x) { return (x + a/x)/2; };
}

export const int n = 5;         // OK, \tcode{n} has external linkage
\end{codeblocktu}

\begin{codeblocktu}{Implementation unit of \tcode{M}}
module M;
struct A {
  int value;
};
\end{codeblocktu}

\begin{codeblocktu}{Main program}
import M;
int main() {
  X::f();                       // OK, \tcode{X} is exported and definition of \tcode{X} is reachable
  X::Y y;                       // OK, \tcode{X::Y} is exported as a complete type
  auto f = rootFinder(2);       // OK
  return A{45}.value;           // error: \tcode{A} is incomplete
}
\end{codeblocktu}
\end{example}

\pnum
\begin{note}
Declarations in an exported \grammarterm{namespace-definition}
or in an exported \grammarterm{linkage-specification}\iref{dcl.link}
are exported and subject to the rules of exported declarations.
\begin{example}
\begin{codeblock}
export module M;
int g;
export namespace N {
  int x;                        // OK
  using ::g;                    // error: \tcode{::g} has module linkage
}
\end{codeblock}
\end{example}
\end{note}

\rSec1[module.import]{Import declaration}%

\begin{bnf}
\nontermdef{module-import-declaration}\br
    import-keyword module-name \opt{attribute-specifier-seq} \terminal{;}\br
    import-keyword module-partition \opt{attribute-specifier-seq} \terminal{;}\br
    import-keyword header-name \opt{attribute-specifier-seq} \terminal{;}
\end{bnf}

\pnum
A \grammarterm{module-import-declaration}
shall inhabit the global namespace scope.
In a module unit, all \grammarterm{module-import-declaration}{s}
and \grammarterm{export-declaration}{s} exporting
\grammarterm{module-import-declaration}{s}
shall appear before all other \grammarterm{declaration}{s} in
the \grammarterm{declaration-seq} of the
\grammarterm{translation-unit}
and of the \grammarterm{private-module-fragment} (if any).
The optional \grammarterm{attribute-specifier-seq}
appertains to the \grammarterm{module-import-declaration}.

\pnum
A \grammarterm{module-import-declaration} \defnx{imports}{import} a set of
translation units determined as described below.
\begin{note}
Namespace-scope declarations exported by the imported translation units
can be found by name lookup\iref{basic.lookup}
in the importing translation unit
and declarations within the imported translation units
become reachable\iref{module.reach}
in the importing translation unit
after the import declaration.
\end{note}

\pnum
A \grammarterm{module-import-declaration} that specifies
a \grammarterm{module-name} \tcode{M}
imports all module interface units of \tcode{M}.

\pnum
A \grammarterm{module-import-declaration} that specifies
a \grammarterm{module-partition} shall only appear after
the \grammarterm{module-declaration} in a module unit of
some module \tcode{M}.
Such a declaration imports the so-named
module partition of \tcode{M}.

\pnum
A \grammarterm{module-import-declaration} that specifies
a \grammarterm{header-name} \tcode{H} imports
a synthesized \defn{header unit},
which is a translation unit formed by applying
phases 1 to 7 of translation\iref{lex.phases}
to the source file or header nominated by \tcode{H},
which shall not contain a \grammarterm{module-declaration}.
\begin{note}
A header unit is a separate translation unit with
an independent set of defined macros.
All declarations within a header unit are implicitly
exported\iref{module.interface},
and are attached to the global module\iref{module.unit}.
\end{note}
An \defnadj{importable}{header} is a member of an
\impldef{how the set of importable headers is determined}
set of headers that
includes all importable \Cpp{} library headers\iref{headers}.
\tcode{H} shall identify an importable header.
Given two such \grammarterm{module-import-declaration}{s}:
\begin{itemize}
\item
if their \grammarterm{header-name}{s} identify
different headers or source files\iref{cpp.include},
they import distinct header units;
\item
otherwise, if they appear in the same translation unit,
they import the same header unit;
\item
otherwise, it is unspecified whether they import the same header unit.
\begin{note}
It is therefore possible that multiple copies exist of entities
declared with internal linkage in an importable header.
\end{note}
\end{itemize}
\begin{note}
A \grammarterm{module-import-declaration} nominating
a \grammarterm{header-name} is also recognized by the
preprocessor, and results in macros defined at the
end of phase 4 of translation of the header unit
being made visible as described in \ref{cpp.import}.
Any other \grammarterm{module-import-declaration}
does not make macros visible.
\end{note}

\pnum
A declaration of a name with internal linkage is
permitted within a header unit despite all
declarations being implicitly exported\iref{module.interface}.
\begin{note}
A definition that appears in multiple translation units
cannot in general refer to such names\iref{basic.def.odr}.
\end{note}
A header unit shall not contain
a definition of a non-inline function or variable
whose name has external linkage.

\pnum
When a \grammarterm{module-import-declaration} imports
a translation unit $T$, it also imports
all translation units imported by
exported \grammarterm{module-import-declaration}{s}
in $T$; such translation units are
said to be \defnx{exported}{module!exported} by $T$.
Additionally, when a \grammarterm{module-import-declaration}
in a module unit of some module $M$ imports another module unit $U$ of $M$,
it also imports all translation units imported by
non-exported \grammarterm{module-import-declaration}{s}
in the module unit purview of $U$.
\begin{footnote}
This is consistent
with the lookup rules for imported names\iref{basic.lookup}.
\end{footnote}
These rules can in turn lead to the importation of yet more
translation units.
\begin{note}
Such indirect importation does not make macros available,
because a translation unit is
a sequence of tokens in translation phase 7\iref{lex.phases}.
Macros can be made available by directly importing header units
as described in \ref{cpp.import}.
\end{note}

\pnum
A module implementation unit shall not be exported.
\begin{example}
\begin{codeblocktu}{Translation unit \#1}
module M:Part;
\end{codeblocktu}

\begin{codeblocktu}{Translation unit \#2}
export module M;
export import :Part;    // error: exported partition \tcode{:Part} is an implementation unit
\end{codeblocktu}
\end{example}

\pnum
A module implementation unit of a module \tcode{M}
that is not a module partition
shall not contain a \grammarterm{module-import-declaration}
nominating \tcode{M}.
\begin{example}
\begin{codeblock}
module M;
import M;               // error: cannot import \tcode{M} in its own unit
\end{codeblock}
\end{example}

\pnum
A translation unit has an \defn{interface dependency} on a translation unit \tcode{U}
if it contains a declaration (possibly a \grammarterm{module-declaration})
that imports \tcode{U} or if it has
an interface dependency on a translation unit that has an interface dependency on \tcode{U}.
A translation unit shall not have an interface dependency on itself.
\begin{example}
\begin{codeblocktu}{Interface unit of \tcode{M1}}
export module M1;
import M2;
\end{codeblocktu}

\begin{codeblocktu}{Interface unit of \tcode{M2}}
export module M2;
import M3;
\end{codeblocktu}

\begin{codeblocktu}{Interface unit of \tcode{M3}}
export module M3;
import M1;              // error: cyclic interface dependency $\mathtt{M3} \rightarrow \mathtt{M1} \rightarrow \mathtt{M2} \rightarrow \mathtt{M3}$
\end{codeblocktu}
\end{example}

\rSec1[module.global.frag]{Global module fragment}

\begin{bnf}
\nontermdef{global-module-fragment}\br
    module-keyword \terminal{;} \opt{declaration-seq}
\end{bnf}

\pnum
\begin{note}
Prior to phase 4 of translation,
only preprocessing directives can appear
in the \grammarterm{declaration-seq}\iref{cpp.pre}.
\end{note}

\pnum
A \grammarterm{global-module-fragment} specifies the contents of the
\defn{global module fragment} for a module unit.
The global module fragment can be used to provide declarations
that are attached to the global module and usable within the module unit.

\pnum
A declaration $D$ is \defn{decl-reachable} from a declaration $S$
in the same translation unit if:
\begin{itemize}
\item
$D$ does not declare a function or function template and
$S$ contains an
\grammarterm{id-expression},
\grammarterm{namespace-name},
\grammarterm{type-name},
\grammarterm{template-name}, or
\grammarterm{concept-name}
naming $D$, or

\item
$D$ declares a function or function template that
is named by an expression\iref{basic.def.odr}
appearing in $S$, or

\item
$S$ contains a dependent call \tcode{E}\iref{temp.dep}
and $D$ is found by any name lookup performed
for an expression synthesized from \tcode{E}
by replacing each type-dependent argument or operand
with a value of a placeholder type
with no associated namespaces or entities, or
\begin{note}
This includes the lookup for \tcode{\keyword{operator}==} performed
when considering rewriting an \tcode{!=} expression,
the lookup for \tcode{\keyword{operator}<=>}
performed when considering rewriting a relational comparison, and
the lookup for \tcode{\keyword{operator}!=}
when considering whether an \tcode{\keyword{operator}==} is a rewrite target.
\end{note}

\item
$S$ contains an expression that
takes the address of an overload set\iref{over.over}
that contains $D$ and
for which the target type is dependent, or

\item
there exists a declaration $M$ that is not a \grammarterm{namespace-definition}
for which $M$ is decl-reachable from $S$ and either
\begin{itemize}
\item
$D$ is decl-reachable from $M$, or
\item
$D$ and $M$ declare the same entity,
and $D$ neither is a friend declaration
nor inhabits a block scope, or
\item
$D$ declares a namespace $N$ and $M$ is a member of $N$, or
\item
one of $D$ and $M$ declares a class or class template $C$
and the other declares a member or friend of $C$, or
\item
one of $D$ and $M$ declares an enumeration $E$
and the other declares an enumerator of $E$, or
\item
$D$ declares a function or variable and $M$ is declared in $D$,
\begin{footnote}
A declaration can appear within a \grammarterm{lambda-expression}
in the initializer of a variable.
\end{footnote}
or
\item
one of $D$ and $M$ declares a template and the other declares
a partial or explicit specialization or
an implicit or explicit instantiation of that template, or
\item
$M$ declares a class template
and $D$ is a deduction guide for that template, or
\item
one of $D$ and $M$ declares a class or enumeration type
and the other introduces a typedef name for linkage purposes for that type.
\end{itemize}
\end{itemize}
In this determination, it is unspecified
\begin{itemize}
\item
whether a reference to an
\grammarterm{alias-declaration},
\tcode{typedef} declaration,
\grammarterm{using-declaration}, or
\grammarterm{namespace-alias-definition}
is replaced by the declarations they name
prior to this determination,

\item
whether a \grammarterm{simple-template-id}
that does not denote a dependent type
and whose \grammarterm{template-name} names an alias template
is replaced by its denoted type
prior to this determination,

\item
whether a \grammarterm{decltype-specifier}
that does not denote a dependent type
is replaced by its denoted type
prior to this determination,
and

\item
whether a non-value-dependent constant expression
is replaced by the result of constant evaluation
prior to this determination.
\end{itemize}

\pnum
A declaration \tcode{D} in a global module fragment of a module unit
is \defnx{discarded}{discarded!declaration} if \tcode{D}
is not decl-reachable from any \grammarterm{declaration}
in the \grammarterm{declaration-seq}
of the \grammarterm{translation-unit}.
\begin{note}
A discarded declaration is neither reachable
nor visible to name lookup outside the module unit,
nor in template instantiations whose points of instantiation\iref{temp.point}
are outside the module unit,
even when the instantiation context\iref{module.context}
includes the module unit.
\end{note}

\pnum
\begin{example}
\begin{codeblock}
const int size = 2;
int ary1[size];                 // unspecified whether \tcode{size} is decl-reachable from \tcode{ary1}
constexpr int identity(int x) { return x; }
int ary2[identity(2)];          // unspecified whether \tcode{identity} is decl-reachable from \tcode{ary2}

template<typename> struct S;
template<typename, int> struct S2;
constexpr int g(int);

template<typename T, int N>
S<S2<T, g(N)>> f();             // \tcode{S}, \tcode{S2}, \tcode{g}, and \tcode{::} are decl-reachable from \tcode{f}

template<int N>
void h() noexcept(g(N) == N);   // \tcode{g} and \tcode{::} are decl-reachable from \tcode{h}
\end{codeblock}
\end{example}

\pnum
\begin{example}
\begin{codeblocktu}{Source file \tcode{"foo.h"}}
namespace N {
  struct X {};
  int d();
  int e();
  inline int f(X, int = d()) { return e(); }
  int g(X);
  int h(X);
}
\end{codeblocktu}

\begin{codeblocktu}{Module \tcode{M} interface}
module;
#include "foo.h"
export module M;
template<typename T> int use_f() {
  N::X x;                       // \tcode{N::X}, \tcode{N}, and \tcode{::} are decl-reachable from \tcode{use_f}
  return f(x, 123);             // \tcode{N::f} is decl-reachable from \tcode{use_f},
                                // \tcode{N::e} is indirectly decl-reachable from \tcode{use_f}
                                //   because it is decl-reachable from \tcode{N::f}, and
                                // \tcode{N::d} is decl-reachable from \tcode{use_f}
                                //   because it is decl-reachable from \tcode{N::f}
                                //   even though it is not used in this call
}
template<typename T> int use_g() {
  N::X x;                       // \tcode{N::X}, \tcode{N}, and \tcode{::} are decl-reachable from \tcode{use_g}
  return g((T(), x));           // \tcode{N::g} is not decl-reachable from \tcode{use_g}
}
template<typename T> int use_h() {
  N::X x;                       // \tcode{N::X}, \tcode{N}, and \tcode{::} are decl-reachable from \tcode{use_h}
  return h((T(), x));           // \tcode{N::h} is not decl-reachable from \tcode{use_h}, but
                                // \tcode{N::h} is decl-reachable from \tcode{use_h<int>}
}
int k = use_h<int>();
  // \tcode{use_h<int>} is decl-reachable from \tcode{k}, so
  // \tcode{N::h} is decl-reachable from \tcode{k}
\end{codeblocktu}

\begin{codeblocktu}{Module \tcode{M} implementation}
module M;
int a = use_f<int>();           // OK
int b = use_g<int>();           // error: no viable function for call to \tcode{g};
                                // \tcode{g} is not decl-reachable from purview of
                                // module \tcode{M}{'s} interface, so is discarded
int c = use_h<int>();           // OK
\end{codeblocktu}
\end{example}

\rSec1[module.private.frag]{Private module fragment}

\begin{bnf}
\nontermdef{private-module-fragment}\br
    module-keyword \terminal{:} \keyword{private} \terminal{;} \opt{declaration-seq}
\end{bnf}

\pnum
A \grammarterm{private-module-fragment} shall appear only
in a primary module interface unit\iref{module.unit}.
A module unit with a \grammarterm{private-module-fragment}
shall be the only module unit of its module;
no diagnostic is required.

\pnum
\begin{note}
A \grammarterm{private-module-fragment} ends
the portion of the module interface unit
that can affect the behavior of other translation units.
A \grammarterm{private-module-fragment} allows a module
to be represented as a single translation unit
without making all of the contents of the module reachable to importers.
The presence of a \grammarterm{private-module-fragment} affects:
\begin{itemize}
\item
the point by which the definition of
an inline function or variable
is required\iref{dcl.inline},

\item
the point by which the definition of
an exported function with a placeholder return type
is required\iref{dcl.spec.auto},

\item
whether a declaration is required not to be an exposure\iref{basic.link},

\item
where definitions for inline functions and templates
must appear\iref{basic.def.odr,dcl.inline,temp.pre},

\item
the instantiation contexts of templates
instantiated before it\iref{module.context}, and

\item
the reachability of declarations within it\iref{module.reach}.
\end{itemize}
\end{note}

\pnum
\begin{example}
\begin{codeblock}
export module A;
export inline void fn_e();      // error: exported inline function \tcode{fn_e} not defined
                                // before private module fragment
inline void fn_m();             // error: non-exported inline function \tcode{fn_m} not defined
static void fn_s();
export struct X;
export void g(X *x) {
  fn_s();                       // OK, call to static function in same translation unit
}
export X *factory();            // OK

module :private;
struct X {};                    // definition not reachable from importers of \tcode{A}
X *factory() {
  return new X ();
}
void fn_e() {}
void fn_m() {}
void fn_s() {}
\end{codeblock}
\end{example}

\rSec1[module.context]{Instantiation context}

\pnum
The \defn{instantiation context} is a set of points within the program
that determines which declarations are found by
argument-dependent name lookup\iref{basic.lookup.argdep}
and which are reachable\iref{module.reach}
in the context of a particular declaration or template instantiation.

\pnum
During the implicit definition of
a defaulted function\iref{special,class.compare.default},
the instantiation context is the union of
the instantiation context from the definition of the class and
the instantiation context of the program construct that
resulted in the implicit definition of the defaulted function.

\pnum
During the implicit instantiation of a template
whose point of instantiation is specified as
that of an enclosing specialization\iref{temp.point},
the instantiation context is the union of
the instantiation context of the enclosing specialization and,
if the template is defined in a module interface unit of a module $M$
and the point of instantiation is not in a module interface unit of $M$,
the point at the end of the
\grammarterm{declaration-seq} of the
primary module interface unit of $M$
(prior to the \grammarterm{private-module-fragment}, if any).

\pnum
During the implicit instantiation of a template
that is implicitly instantiated because it is referenced
from within the implicit definition of a defaulted function,
the instantiation context is the instantiation context of
the defaulted function.

\pnum
During the instantiation of any other template specialization,
the instantiation context comprises the point of instantiation
of the template.

\pnum
In any other case, the instantiation context
at a point within the program
comprises that point.

\pnum
\begin{example}
\begin{codeblocktu}{Translation unit \#1}
export module stuff;
export template<typename T, typename U> void foo(T, U u) { auto v = u; }
export template<typename T, typename U> void bar(T, U u) { auto v = *u; }
\end{codeblocktu}

\begin{codeblocktu}{Translation unit \#2}
export module M1;
import "defn.h";        // provides \tcode{struct X \{\};}
import stuff;
export template<typename T> void f(T t) {
  X x;
  foo(t, x);
}
\end{codeblocktu}

\begin{codeblocktu}{Translation unit \#3}
export module M2;
import "decl.h";        // provides \tcode{struct X;} (not a definition)
import stuff;
export template<typename T> void g(T t) {
  X *x;
  bar(t, x);
}
\end{codeblocktu}

\begin{codeblocktu}{Translation unit \#4}
import M1;
import M2;
void test() {
  f(0);
  g(0);
}
\end{codeblocktu}
The call to \tcode{f(0)} is valid;
the instantiation context of \tcode{foo<int, X>} comprises
\begin{itemize}
\item the point at the end of translation unit \#1,
\item the point at the end of translation unit \#2, and
\item the point of the call to \tcode{f(0)},
\end{itemize}
so the definition of \tcode{X} is reachable\iref{module.reach}.

It is unspecified whether the call to \tcode{g(0)} is valid:
the instantiation context of \tcode{bar<int, X>} comprises
\begin{itemize}
\item the point at the end of translation unit \#1,
\item the point at the end of translation unit \#3, and
\item the point of the call to \tcode{g(0)},
\end{itemize}
so the definition of \tcode{X} need not be reachable,
as described in \ref{module.reach}.
\end{example}

\rSec1[module.reach]{Reachability}

\indextext{necessarily reachable|see{reachable, necessarily}}
\pnum
A translation unit $U$ is
\defnx{necessarily reachable}{reachable!necessarily!translation unit}
from a point $P$ if
$U$ is a module interface unit on which the translation unit containing $P$
has an interface dependency, or
the translation unit containing $P$ imports $U$,
in either case prior to $P$\iref{module.import}.
\begin{note}
While module interface units are reachable even when they are only
transitively imported via a non-exported import declaration,
namespace-scope names from such module interface units are not found by
name lookup\iref{basic.lookup}.
\end{note}

\pnum
All translation units that are necessarily reachable are
\defnx{reachable}{reachable!translation unit}.
Additional translation units on which the
point within the program has an interface dependency may be considered reachable,
but it is unspecified which are and under what circumstances.
\begin{footnote}
Implementations are therefore not required to prevent the semantic
effects of additional translation units involved in the compilation from being
observed.
\end{footnote}
\begin{note}
It is advisable to avoid
depending on the reachability of any additional translation units
in programs intending to be portable.
\end{note}

\pnum
A declaration $D$ is
\defnx{reachable from}{reachable from!declaration} a point $P$ if
\begin{itemize}
\item $D$ appears prior to $P$ in the same translation unit, or
\item $D$ is not discarded\iref{module.global.frag},
appears in a translation unit that is
reachable from $P$,
and
does not appear within a \grammarterm{private-module-fragment}.
\end{itemize}
A declaration is \defnx{reachable}{reachable!declaration}
if it is reachable from
any point in the instantiation context\iref{module.context}.
\begin{note}
Whether a declaration is exported has no bearing on whether it is reachable.
\end{note}

\pnum
The accumulated properties of all reachable declarations of
an entity within a context
determine the behavior of the entity within that context.
\begin{note}
These reachable semantic properties include type completeness,
type definitions, initializers,
default arguments of functions or template declarations, attributes,
names bound, etc.
Since default arguments are evaluated in the context of the call expression,
the reachable semantic properties of the corresponding parameter types apply in
that context.
\begin{example}
\begin{codeblocktu}{Translation unit \#1}
export module M:A;
export struct B;
\end{codeblocktu}

\begin{codeblocktu}{Translation unit \#2}
module M:B;
struct B {
  operator int();
};
\end{codeblocktu}

\begin{codeblocktu}{Translation unit \#3}
module M:C;
import :A;
B b1;                           // error: no reachable definition of \tcode{struct B}
\end{codeblocktu}

\begin{codeblocktu}{Translation unit \#4}
export module M;
export import :A;
import :B;
B b2;
export void f(B b = B());
\end{codeblocktu}

\begin{codeblocktu}{Translation unit \#5}
import M;
B b3;                           // error: no reachable definition of \tcode{struct B}
void g() { f(); }               // error: no reachable definition of \tcode{struct B}
\end{codeblocktu}
\end{example}
\end{note}

\pnum
\begin{note}
Declarations of an entity can be reachable
even where they cannot be found by name lookup.
\end{note}
\begin{example}
\begin{codeblocktu}{Translation unit \#1}
export module A;
struct X {};
export using Y = X;
\end{codeblocktu}

\begin{codeblocktu}{Translation unit \#2}
import A;
Y y;                // OK, definition of \tcode{X} is reachable
X x;                // error: \tcode{X} not visible to unqualified lookup
\end{codeblocktu}
\end{example}
