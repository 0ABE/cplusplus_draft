%!TEX root = std.tex

\rSec0[intro.scope]{Scope}

\pnum
\indextext{scope|(}%
This document specifies requirements for implementations
of the \Cpp{} programming language. The first such requirement is that
they implement the language, so this document also
defines \Cpp{}. Other requirements and relaxations of the first
requirement appear at various places within this document.

\pnum
\Cpp{} is a general purpose programming language based on the C
programming language as described in ISO/IEC 9899:2018
\doccite{Programming languages --- C} (hereinafter referred to as the
\defnx{C standard}{C!standard}). \Cpp{} provides many facilities
beyond those provided by C, including additional data types,
classes, templates, exceptions, namespaces, operator
overloading, function name overloading, references, free store
management operators, and additional library facilities.%
\indextext{scope|)}

\rSec0[intro.refs]{Normative references}%
\indextext{normative references|see{references, normative}}%

\pnum
\indextext{references!normative|(}%
The following documents are referred to in the text
in such a way that some or all of their content
constitutes requirements
of this document. For dated references, only the edition cited applies.
For undated references, the latest edition of the referenced document
(including any amendments) applies.
\begin{itemize}
\item Ecma International, \doccite{ECMAScript%
\footnote{ECMAScript\textregistered\ is a registered trademark of Ecma
International.
This information is given for the convenience of users of this document and
does not constitute an endorsement by ISO or IEC of this product.}
Language Specification},
Standard Ecma-262, third edition, 1999.
\item ISO/IEC 2382 (all parts), \doccite{Information technology ---
Vocabulary}
\item ISO 8601:2004, \doccite{Data elements and interchange formats ---
Information interchange --- Representation of dates and times}
\item ISO/IEC 9899:2018, \doccite{Programming languages --- C}
\item ISO/IEC 9945:2003, \doccite{Information Technology --- Portable
Operating System Interface (POSIX%
\footnote{POSIX\textregistered\ is a registered trademark of
the Institute of Electrical and Electronic Engineers, Inc.
This information is given for the convenience of users of this document and
does not constitute an endorsement by ISO or IEC of this product.})}
\item ISO/IEC 10646, \doccite{Information technology ---
Universal Coded Character Set (UCS)}
\item ISO/IEC 10646-1:1993, \doccite{Information technology ---
Universal Multiple-Octet Coded Character Set (UCS) --- Part 1:
Architecture and Basic Multilingual Plane}
\item ISO/IEC/IEEE 60559:2011, \doccite{Information technology ---
Microprocessor Systems --- Floating-Point arithmetic}
\item ISO 80000-2:2009, \doccite{Quantities and units ---
Part 2: Mathematical signs and symbols
to be used in the natural sciences and technology}
%%% Format for the following entry is based on that specified at
%%% http://www.iec.ch/standardsdev/resources/draftingpublications/directives/principles/referencing.htm
\item The Unicode Consortium. Unicode%
\footnote{Unicode\textregistered\ is a registered trademark of Unicode, Inc.
This information is given for the convenience of users of this document and
does not constitute an endorsement by ISO or IEC of this product.}
Standard Annex, UAX \#29, \doccite{Unicode Text Segmentation} [online].
Edited by Mark Davis. Revision 35; issued for Unicode 12.0.0. 2019-02-15 [viewed 2020-02-23].
Available at \url{http://www.unicode.org/reports/tr29/tr29-35.html}
\end{itemize}

\pnum
The library described in Clause 7 of ISO/IEC 9899:2018
is hereinafter called the
\defnx{C standard library}{C!standard library}.%
\footnote{With the qualifications noted in \ref{\firstlibchapter}
through \ref{\lastlibchapter} and in \ref{diff.library}, the C standard
library is a subset of the \Cpp{} standard library.}

\pnum
The operating system interface described in ISO/IEC 9945:2003 is
hereinafter called \defn{POSIX}.

\pnum
The ECMAScript Language Specification described in Standard Ecma-262 is
hereinafter called \defn{ECMA-262}.
\indextext{references!normative|)}

\pnum
\begin{note}
References to ISO/IEC 10646-1:1993 are used only
to support deprecated features\iref{depr.locale.stdcvt}.
\end{note}

\rSec0[intro.defs]{Terms and definitions}

\pnum
\indextext{definitions|(}%
For the purposes of this document,
the terms and definitions
given in ISO/IEC 2382-1:1993,
the terms, definitions, and symbols
given in ISO 80000-2:2009,
and the following apply.

\pnum
ISO and IEC maintain terminological databases
for use in standardization
at the following addresses:
\begin{itemize}
\item ISO Online browsing platform: available at \url{https://www.iso.org/obp}
\item IEC Electropedia: available at \url{http://www.electropedia.org/}
\end{itemize}

\pnum
\ref{definitions}
defines additional terms that are used only in \ref{library}
through \ref{\lastlibchapter} and \ref{depr}.

\pnum
Terms that are used only in a small portion of this document
are defined where they are used and italicized where they are
defined.

\indexdefn{access}%
\definition{access}{defns.access}
\defncontext{execution-time action}
read\iref{conv.lval} or
modify (\ref{expr.ass}, \ref{expr.post.incr}, \ref{expr.pre.incr})
the value of an object

\begin{defnote}
Only objects of scalar type can be accessed.
Attempts to read or modify an object of class type
typically invoke a constructor\iref{class.ctor}
or assignment operator\iref{class.copy.assign};
such invocations do not themselves constitute accesses,
although they may involve accesses of scalar subobjects.
\end{defnote}

\indexdefn{argument}%
\indexdefn{argument!function call expression}
\definition{argument}{defns.argument}
\defncontext{function call expression} expression in the
comma-separated list bounded by the parentheses\iref{expr.call}

\indexdefn{argument}%
\indexdefn{argument!function-like macro}%
\definition{argument}{defns.argument.macro}
\defncontext{function-like macro} sequence of preprocessing tokens in the
comma-separated list bounded by the parentheses\iref{cpp.replace}

\indexdefn{argument}%
\indexdefn{argument!throw expression}%
\definition{argument}{defns.argument.throw}
\defncontext{throw expression} operand of \tcode{throw}\iref{expr.throw}

\indexdefn{argument}%
\indexdefn{argument!template instantiation}%
\definition{argument}{defns.argument.templ}
\defncontext{template instantiation}
\grammarterm{constant-expression},
\grammarterm{type-id}, or
\grammarterm{id-expression} in the comma-separated
list bounded by the angle brackets\iref{temp.arg}

\indexdefn{block (execution)}%
\definition{block}{defns.block}
\defncontext{execution}
wait for some condition (other than for the implementation to execute
the execution steps of the thread of execution) to be satisfied before
continuing execution past the blocking operation

\indexdefn{block (statement)}%
\definition{block}{defns.block.stmt}
\defncontext{statement}
compound statement\iref{stmt.block}

\indexdefn{behavior!conditionally-supported}%
\definition{conditionally-supported}{defns.cond.supp}
program construct that an implementation is not required to support

\begin{defnote}
Each implementation documents all conditionally-supported
constructs that it does not support.
\end{defnote}

\indexdefn{message!diagnostic}%
\definition{diagnostic message}{defns.diagnostic}
message belonging to an \impldef{diagnostic message} subset of the
implementation's output messages

\indexdefn{type!dynamic}%
\definition{dynamic type}{defns.dynamic.type}
\defncontext{glvalue} type of the most derived object\iref{intro.object} to which the
glvalue refers

\begin{example}
If a pointer\iref{dcl.ptr} \tcode{p} whose static type is ``pointer to
class \tcode{B}'' is pointing to an object of class \tcode{D}, derived
from \tcode{B}\iref{class.derived}, the dynamic type of the
expression \tcode{*p} is ``\tcode{D}''. References\iref{dcl.ref} are
treated similarly.
\end{example}

\indexdefn{type!dynamic}%
\definition{dynamic type}{defns.dynamic.type.prvalue}
\defncontext{prvalue} static type of the prvalue expression

\indexdefn{program!ill-formed}%
\definition{ill-formed program}{defns.ill.formed}
program that is not well-formed\iref{defns.well.formed}

\indexdefn{behavior!implementation-defined}%
\definition{implementation-defined behavior}{defns.impl.defined}
behavior, for a well-formed program construct and correct data, that
depends on the implementation and that each implementation documents

\indexdefn{limits!implementation}%
\definition{implementation limits}{defns.impl.limits}
restrictions imposed upon programs by the implementation

\indexdefn{behavior!locale-specific}%
\definition{locale-specific behavior}{defns.locale.specific}
behavior that depends on local conventions of nationality, culture, and
language that each implementation documents

\indexdefn{character!multibyte}%
\definition{multibyte character}{defns.multibyte}
sequence of one or more bytes representing a member of the extended
character set of either the source or the execution environment

\begin{defnote}
The extended character set is a superset of the basic character
set\iref{lex.charset}.
\end{defnote}

\indexdefn{parameter}%
\indexdefn{parameter!function}%
\indexdefn{parameter!catch clause}%
\definition{parameter}{defns.parameter}
\defncontext{function or catch clause} object or reference declared as part of a function declaration or
definition or in the catch clause of an exception handler that
acquires a value on entry to the function or handler

\indexdefn{parameter}%
\indexdefn{parameter!function-like macro}%
\definition{parameter}{defns.parameter.macro}
\defncontext{function-like macro} identifier from
the comma-separated list bounded by the parentheses immediately
following the macro name

\indexdefn{parameter}%
\indexdefn{parameter!template}%
\definition{parameter}{defns.parameter.templ}
\defncontext{template} member of a \grammarterm{template-parameter-list}

\indexdefn{signature}%
\definition{signature}{defns.signature}
\defncontext{function}
name,
parameter-type-list\iref{dcl.fct},
and enclosing namespace (if any)

\begin{defnote}
Signatures are used as a basis for
name mangling and linking.
\end{defnote}

\indexdefn{signature}%
\definition{signature}{defns.signature.friend}
\defncontext{non-template friend function with trailing \grammarterm{requires-clause}}
name,
parameter-type-list\iref{dcl.fct},
enclosing class,
and
trailing \grammarterm{requires-clause}\iref{dcl.decl}

\indexdefn{signature}%
\definition{signature}{defns.signature.templ}
\defncontext{function template}
name,
parameter-type-list\iref{dcl.fct},
enclosing namespace (if any),
return type,
\grammarterm{template-head},
and
trailing \grammarterm{requires-clause}\iref{dcl.decl} (if any)

\indexdefn{signature}%
\definition{signature}{defns.signature.templ.friend}
\defncontext{friend function template with constraint involving enclosing template parameters}
name,
parameter-type-list\iref{dcl.fct},
return type,
enclosing class,
\grammarterm{template-head},
and
trailing \grammarterm{requires-clause}\iref{dcl.decl} (if any)

\indexdefn{signature}%
\definition{signature}{defns.signature.spec}
\defncontext{function template specialization} signature of the template of which it is a specialization
and its template arguments (whether explicitly specified or deduced)

\indexdefn{signature}%
\definition{signature}{defns.signature.member}
\defncontext{class member function}
name,
parameter-type-list\iref{dcl.fct},
class of which the function is a member,
\cv-qualifiers (if any),
\grammarterm{ref-qualifier} (if any),
and
trailing \grammarterm{requires-clause}\iref{dcl.decl} (if any)

\indexdefn{signature}%
\definition{signature}{defns.signature.member.templ}
\defncontext{class member function template}
name,
parameter-type-list\iref{dcl.fct},
class of which the function is a member,
\cv-qualifiers (if any),
\grammarterm{ref-qualifier} (if any),
return type (if any),
\grammarterm{template-head},
and
trailing \grammarterm{requires-clause}\iref{dcl.decl} (if any)

\indexdefn{signature}%
\definition{signature}{defns.signature.member.spec}
\defncontext{class member function template specialization} signature of the member function template
of which it is a specialization and its template arguments (whether explicitly specified or deduced)

\indexdefn{type!static}%
\definition{static type}{defns.static.type}
type of an expression\iref{basic.types} resulting from
analysis of the program without considering execution semantics

\begin{defnote}
The static type of an expression depends only on the form of the program in
which the expression appears, and does not change while the program is
executing.
\end{defnote}

\indexdefn{unblock}%
\definition{unblock}{defns.unblock}
satisfy a condition that one or more blocked threads of execution are waiting for

\indexdefn{behavior!undefined}%
\definition{undefined behavior}{defns.undefined}
behavior for which this document
imposes no requirements

\begin{defnote}
Undefined behavior may be expected when
this document omits any explicit
definition of behavior or when a program uses an erroneous construct or erroneous data.
Permissible undefined behavior ranges
from ignoring the situation completely with unpredictable results, to
behaving during translation or program execution in a documented manner
characteristic of the environment (with or without the issuance of a
diagnostic message), to terminating a translation or execution (with the
issuance of a diagnostic message). Many erroneous program constructs do
not engender undefined behavior; they are required to be diagnosed.
Evaluation of a constant expression never exhibits behavior explicitly
specified as undefined in \ref{intro} through \ref{cpp} of this document\iref{expr.const}.
\end{defnote}

\indexdefn{behavior!unspecified}%
\definition{unspecified behavior}{defns.unspecified}
behavior, for a well-formed program construct and correct data, that
depends on the implementation

\begin{defnote}
The implementation is not required to
document which behavior occurs. The range of
possible behaviors is usually delineated by this document.
\end{defnote}

\indexdefn{program!well-formed}%
\definition{well-formed program}{defns.well.formed}
\Cpp{}  program constructed according to the syntax rules, diagnosable
semantic rules, and the one-definition rule\iref{basic.def.odr}%
\indextext{definitions|)}

\rSec0[intro]{General principles}

\rSec1[intro.compliance]{Implementation compliance}%
\indextext{diagnostic message|see{message, diagnostic}}%
\indexdefn{conditionally-supported behavior|see{behavior, con\-ditionally-supported}}%
\indextext{dynamic type|see{type, dynamic}}%
\indextext{static type|see{type, static}}%
\indextext{ill-formed program|see{program, ill-formed}}%
\indextext{well-formed program|see{program, well-formed}}%
\indextext{implementation-defined behavior|see{behavior, im\-plementation-defined}}%
\indextext{undefined behavior|see{behavior, undefined}}%
\indextext{unspecified behavior|see{behavior, unspecified}}%
\indextext{implementation limits|see{limits, implementation}}%
\indextext{locale-specific behavior|see{behavior, locale-spe\-cific}}%
\indextext{multibyte character|see{character, multibyte}}%
\indextext{object|seealso{object model}}%
\indextext{subobject|seealso{object model}}%
\indextext{derived class!most|see{most derived class}}%
\indextext{derived object!most|see{most derived object}}%
\indextext{program execution!as-if rule|see{as-if rule}}%
\indextext{observable behavior|see{behavior, observable}}%
\indextext{precedence of operator|see{operator, precedence of}}%
\indextext{order of evaluation in expression|see{expression, order of evaluation of}}%
\indextext{multiple threads|see{threads, multiple}}%

\rSec2[intro.compliance.general]{General}

\pnum
\indextext{conformance requirements|(}%
\indextext{conformance requirements!general|(}%
The set of
\defn{diagnosable rules}
consists of all syntactic and semantic rules in this document
except for those rules containing an explicit notation that
``no diagnostic is required'' or which are described as resulting in
``undefined behavior''.

\pnum
\indextext{conformance requirements!method of description}%
Although this document states only requirements on \Cpp{}
implementations, those requirements are often easier to understand if
they are phrased as requirements on programs, parts of programs, or
execution of programs. Such requirements have the following meaning:
\begin{itemize}
\item
If a program contains no violations of the rules in this
document, a conforming implementation shall,
within its resource limits, accept and correctly execute\footnote{``Correct execution'' can include undefined behavior, depending on
the data being processed; see \ref{intro.defs} and~\ref{intro.execution}.}
that program.
\item
\indextext{message!diagnostic}%
If a program contains a violation of any diagnosable rule or an occurrence
of a construct described in this document as ``conditionally-supported'' when
the implementation does not support that construct, a conforming implementation
shall issue at least one diagnostic message.
\item
\indextext{behavior!undefined}%
If a program contains a violation of a rule for which no diagnostic
is required, this document places no requirement on
implementations with respect to that program.
\end{itemize}
\begin{note}
During template argument deduction and substitution,
certain constructs that in other contexts require a diagnostic
are treated differently;
see~\ref{temp.deduct}.
\end{note}

\pnum
\indextext{conformance requirements!library|(}%
\indextext{conformance requirements!classes}%
\indextext{conformance requirements!class templates}%
For classes and class templates, the library Clauses specify partial
definitions. Private members\iref{class.access} are not
specified, but each implementation shall supply them to complete the
definitions according to the description in the library Clauses.

\pnum
For functions, function templates, objects, and values, the library
Clauses specify declarations. Implementations shall supply definitions
consistent with the descriptions in the library Clauses.

\pnum
The names defined in the library have namespace
scope\iref{basic.namespace}. A \Cpp{}  translation
unit\iref{lex.phases} obtains access to these names by including the
appropriate standard library header or importing
the appropriate standard library named header unit\iref{using.headers}.

\pnum
The templates, classes, functions, and objects in the library have
external linkage\iref{basic.link}. The implementation provides
definitions for standard library entities, as necessary, while combining
translation units to form a complete \Cpp{}  program\iref{lex.phases}.%
\indextext{conformance requirements!library|)}

\pnum
Two kinds of implementations are defined: a \defn{hosted implementation} and a
\defn{freestanding implementation}. For a hosted implementation, this
document defines the set of available libraries. A freestanding
implementation is one in which execution may take place without the benefit of
an operating system, and has an \impldef{required libraries for freestanding
implementation} set of libraries that includes certain language-support
libraries\iref{compliance}.

\pnum
A conforming implementation may have extensions (including
additional library functions), provided they do not alter the
behavior of any well-formed program.
Implementations are required to diagnose programs that use such
extensions that are ill-formed according to this document.
Having done so, however, they can compile and execute such programs.

\pnum
Each implementation shall include documentation that identifies all
conditionally-supported constructs\indextext{behavior!conditionally-supported}
that it does not support and defines all locale-specific characteristics.\footnote{This documentation also defines implementation-defined behavior;
see~\ref{intro.abstract}.}%
\indextext{conformance requirements!general|)}%
\indextext{conformance requirements|)}%

\rSec2[intro.abstract]{Abstract machine}

\pnum
\indextext{program execution|(}%
\indextext{program execution!abstract machine}%
The semantic descriptions in this document define a
parameterized nondeterministic abstract machine. This document
places no requirement on the structure of conforming
implementations. In particular, they need not copy or emulate the
structure of the abstract machine.
\indextext{as-if rule}%
\indextext{behavior!observable}%
Rather, conforming implementations are required to emulate (only) the observable
behavior of the abstract machine as explained below.\footnote{This provision is
sometimes called the ``as-if'' rule, because an implementation is free to
disregard any requirement of this document as long as the result
is \emph{as if} the requirement had been obeyed, as far as can be determined
from the observable behavior of the program. For instance, an actual
implementation need not evaluate part of an expression if it can deduce that its
value is not used and that no
\indextext{side effects}%
side effects affecting the
observable behavior of the program are produced.}

\pnum
\indextext{behavior!implementation-defined}%
Certain aspects and operations of the abstract machine are described in this
document as implementation-defined (for example,
\tcode{sizeof(int)}). These constitute the parameters of the abstract machine.
Each implementation shall include documentation describing its characteristics
and behavior in these respects.\footnote{This documentation also includes
conditionally-supported constructs and locale-specific behavior.
See~\ref{intro.compliance}.} Such documentation shall define the instance of the
abstract machine that corresponds to that implementation (referred to as the
``corresponding instance'' below).

\pnum
\indextext{behavior!unspecified}%
Certain other aspects and operations of the abstract machine are
described in this document as unspecified (for example,
order of evaluation of arguments in a function call\iref{expr.call}).
Where possible, this
document defines a set of allowable behaviors. These
define the nondeterministic aspects of the abstract machine. An instance
of the abstract machine can thus have more than one possible execution
for a given program and a given input.

\pnum
\indextext{behavior!undefined}%
Certain other operations are described in this document as
undefined (for example, the effect of
attempting to modify a const object).
\begin{note}
This document imposes no requirements on the
behavior of programs that contain undefined behavior.
\end{note}

\pnum
\indextext{program!well-formed}%
\indextext{behavior!observable}%
A conforming implementation executing a well-formed program shall
produce the same observable behavior as one of the possible executions
of the corresponding instance of the abstract machine with the
same program and the same input.
\indextext{behavior!undefined}%
However, if any such execution contains an undefined operation, this document places no
requirement on the implementation executing that program with that input
(not even with regard to operations preceding the first undefined
operation).

\pnum
\indextext{conformance requirements}%
The least requirements on a conforming implementation are:
\begin{itemize}
\item
Accesses through volatile glvalues are evaluated strictly according to the
rules of the abstract machine.
\item
At program termination, all data written into files shall be
identical to one of the possible results that execution of the program
according to the abstract semantics would have produced.
\item
The input and output dynamics of interactive devices shall take
place in such a fashion that prompting output is actually delivered before a program waits for input. What constitutes an interactive device is
\impldef{interactive device}.
\end{itemize}

These collectively are referred to as the
\defnx{observable behavior}{behavior!observable} of the program.
\begin{note}
More stringent correspondences between abstract and actual
semantics may be defined by each implementation.
\end{note}

\rSec1[intro.structure]{Structure of this document}

\pnum
\indextext{standard!structure of|(}%
\indextext{standard!structure of}%
\ref{lex} through \ref{cpp} describe the \Cpp{} programming
language. That description includes detailed syntactic specifications in
a form described in~\ref{syntax}. For convenience, \ref{gram}
repeats all such syntactic specifications.

\pnum
\ref{\firstlibchapter} through \ref{\lastlibchapter} and \ref{depr}
(the \defn{library clauses}) describe the \Cpp{} standard library.
That description includes detailed descriptions of the
entities and macros
that constitute the library, in a form described in \ref{library}.

\pnum
\ref{implimits} recommends lower bounds on the capacity of conforming
implementations.

\pnum
\ref{diff} summarizes the evolution of \Cpp{} since its first
published description, and explains in detail the differences between
\Cpp{} and C\@. Certain features of \Cpp{} exist solely for compatibility
purposes; \ref{depr} describes those features.

\pnum
Throughout this document, each example is introduced by
``\noteintro{Example}'' and terminated by ``\noteoutro{example}''. Each note is
introduced by ``\noteintro{Note}'' or ``\noteintro{Note $n$ to entry}'' and
terminated by ``\noteoutro{note}''. Examples
and notes may be nested.%
\indextext{standard!structure of|)}

\rSec1[syntax]{Syntax notation}

\pnum
\indextext{notation!syntax|(}%
In the syntax notation used in this document, syntactic
categories are indicated by \fakegrammarterm{italic} type, and literal words
and characters in \tcode{constant} \tcode{width} type. Alternatives are
listed on separate lines except in a few cases where a long set of
alternatives is marked by the phrase ``one of''. If the text of an alternative is too long to fit on a line, the text is continued on subsequent lines indented from the first one.
An optional terminal or non-terminal symbol is indicated by the subscript
``\opt{\relax}'', so
\begin{ncbnf}
\terminal{\{} \opt{expression} \terminal{\}}
\end{ncbnf}
indicates an optional expression enclosed in braces.%

\pnum
Names for syntactic categories have generally been chosen according to
the following rules:
\begin{itemize}
\item \fakegrammarterm{X-name} is a use of an identifier in a context that
determines its meaning (e.g., \grammarterm{class-name},
\grammarterm{typedef-name}).
\item \fakegrammarterm{X-id} is an identifier with no context-dependent meaning
(e.g., \grammarterm{qualified-id}).
\item \fakegrammarterm{X-seq} is one or more \fakegrammarterm{X}'s without intervening
delimiters (e.g., \grammarterm{declaration-seq} is a sequence of
declarations).
\item \fakegrammarterm{X-list} is one or more \fakegrammarterm{X}'s separated by
intervening commas (e.g., \grammarterm{identifier-list} is a sequence of
identifiers separated by commas).
\end{itemize}%
\indextext{notation!syntax|)}
