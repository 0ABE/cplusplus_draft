%!TEX root = std.tex
\rSec0[lex]{Lexical conventions}

\gramSec[gram.lex]{Lexical conventions}

\indextext{lexical conventions|see{conventions, lexical}}
\indextext{translation!separate|see{compilation, separate}}
\indextext{separate translation|see{compilation, separate}}
\indextext{separate compilation|see{compilation, separate}}
\indextext{phases of translation|see{translation, phases}}
\indextext{source file character|see{character, source file}}
\indextext{alternative token|see{token, alternative}}
\indextext{digraph|see{token, alternative}}
\indextext{integer literal|see{literal, integer}}
\indextext{character literal|see{literal, character}}
\indextext{floating-point literal|see{literal, floating-point}}
\indextext{string literal|see{literal, string}}
\indextext{boolean literal|see{literal, boolean}}
\indextext{pointer literal|see{literal, pointer}}
\indextext{user-defined literal|see{literal, user-defined}}
\indextext{file, source|see{source file}}
\indextext{null character|see{character, null}}
\indextext{null wide character|see{wide-character, null}}

\rSec1[lex.separate]{Separate translation}

\pnum
\indextext{conventions!lexical|(}%
\indextext{compilation!separate|(}%
The text of the program is kept in units called
\defnx{source files}{source file} in this document.
A source file together with all the headers\iref{headers}
and source files included\iref{cpp.include} via the preprocessing
directive \tcode{\#include}, less any source lines skipped by any of the
conditional inclusion\iref{cpp.cond} preprocessing directives, is
called a \defn{translation unit}.
\begin{note}
A \Cpp{} program need not all be translated at the same time.
\end{note}

\pnum
\begin{note}
Previously translated translation units and instantiation
units can be preserved individually or in libraries. The separate
translation units of a program communicate\iref{basic.link} by (for
example)
calls to functions whose identifiers have external or module linkage,
manipulation of objects whose identifiers have external or module linkage, or
manipulation of data files. Translation units can be separately
translated and then later linked to produce an executable
program\iref{basic.link}.
\end{note}
\indextext{compilation!separate|)}

\rSec1[lex.phases]{Phases of translation}%

\pnum
\indextext{translation!phases|(}%
The precedence among the syntax rules of translation is specified by the
following phases.\footnote{Implementations behave as if these separate phases
occur, although in practice different phases can be folded together.}

\begin{enumerate}
\item
\indextext{character!source file}%
\indextext{character set!basic source}%
Physical source file characters are mapped, in an
\impldef{mapping physical source file characters to basic source character set} manner,
to the basic source character set (introducing new-line characters for end-of-line
indicators) if necessary.
The set of physical source file characters accepted is \impldef{physical source file
characters}.
Any
source file character not in the basic source character
set\iref{lex.charset} is replaced by the
\indextext{universal character name}\grammarterm{universal-character-name} that
designates that character. An implementation may use any internal
encoding, so long as an actual extended character encountered in the
source file, and the same extended character expressed in the source
file as a \grammarterm{universal-character-name} (e.g., using the \tcode{\textbackslash
uXXXX} notation), are handled equivalently
except where this replacement is reverted\iref{lex.pptoken} in a raw string literal.

\item
\indextext{line splicing}%
Each instance of a backslash character (\textbackslash)
immediately followed by a new-line character is deleted, splicing
physical source lines to form logical source lines. Only the last
backslash on any physical source line shall be eligible for being part
of such a splice.
Except for splices reverted in a raw string literal, if a splice results in
a character sequence that matches the
syntax of a \grammarterm{universal-character-name}, the behavior is
undefined. A source file that is not empty and that does not end in a new-line
character, or that ends in a new-line character immediately preceded by a
backslash character before any such splicing takes place,
shall be processed as if an additional new-line character were appended
to the file.

\item The source file is decomposed into preprocessing
tokens\iref{lex.pptoken} and sequences of white-space characters
(including comments). A source file shall not end in a partial
preprocessing token or in a partial comment.\footnote{A partial preprocessing
token would arise from a source file
ending in the first portion of a multi-character token that requires a
terminating sequence of characters, such as a \grammarterm{header-name}
that is missing the closing \tcode{"}
or \tcode{>}. A partial comment
would arise from a source file ending with an unclosed \tcode{/*}
comment.}
Each comment is replaced by one space character. New-line characters are
retained. Whether each nonempty sequence of white-space characters other
than new-line is retained or replaced by one space character is
unspecified. The process of dividing a source file's
characters into preprocessing tokens is context-dependent.
\begin{example}
See the handling of \tcode{<} within a \tcode{\#include} preprocessing
directive.
\end{example}

\item Preprocessing directives are executed, macro invocations are
expanded, and \tcode{_Pragma} unary operator expressions are executed.
If a character sequence that matches the syntax of a
\grammarterm{universal-character-name} is produced by token
concatenation\iref{cpp.concat}, the behavior is undefined. A
\tcode{\#include} preprocessing directive causes the named header or
source file to be processed from phase 1 through phase 4, recursively.
All preprocessing directives are then deleted.

\item Each basic source character set member in a \grammarterm{character-literal} or a
\grammarterm{string-literal}, as well as each escape sequence and \grammarterm{universal-character-name} in a
\grammarterm{character-literal} or a non-raw string literal, is converted to the corresponding
member of the execution character set~(\ref{lex.ccon}, \ref{lex.string}); if
there is no corresponding member, it is converted to an \impldef{converting
characters from source character set to execution character set} member other
than the null (wide) character.\footnote{An implementation need not convert all
non-corresponding source characters to the same execution character.}

\item Adjacent string literal tokens are concatenated.

\item White-space characters separating tokens are no longer
significant. Each preprocessing token is converted into a
token\iref{lex.token}. The resulting tokens are syntactically and
semantically analyzed and translated as a translation unit.
\begin{note}
The process of analyzing and translating the tokens can occasionally
result in one token being replaced by a sequence of other
tokens\iref{temp.names}.
\end{note}
It is
\impldef{whether the sources for
module units and header units
on which the current translation unit has an interface
dependency are required to be available during translation}
whether the sources for
module units and header units
on which the current translation unit has an interface
dependency (\ref{module.unit}, \ref{module.import})
are required to be available.
\begin{note}
Source files, translation
units and translated translation units need not necessarily be stored as
files, nor need there be any one-to-one correspondence between these
entities and any external representation. The description is conceptual
only, and does not specify any particular implementation.
\end{note}

\item Translated translation units and instantiation units are combined
as follows:
\begin{note}
Some or all of these can be supplied from a
library.
\end{note}
Each translated translation unit is examined to
produce a list of required instantiations.
\begin{note}
This can include
instantiations which have been explicitly
requested\iref{temp.explicit}.
\end{note}
The definitions of the
required templates are located. It is \impldef{whether source of translation units must
be available to locate template definitions} whether the
source of the translation units containing these definitions is required
to be available.
\begin{note}
An implementation could encode sufficient
information into the translated translation unit so as to ensure the
source is not required here.
\end{note}
All the required instantiations
are performed to produce
\defn{instantiation units}.
\begin{note}
These are similar
to translated translation units, but contain no references to
uninstantiated templates and no template definitions.
\end{note}
The
program is ill-formed if any instantiation fails.

\item All external entity references are resolved. Library
components are linked to satisfy external references to
entities not defined in the current translation. All such translator
output is collected into a program image which contains information
needed for execution in its execution environment.%
\indextext{translation!phases|)}
\end{enumerate}

\rSec1[lex.charset]{Character sets}

\pnum
\indextext{character set|(}%
The \defnx{basic source character set}{character set!basic source} consists of 96 characters: the space character,
the control characters representing horizontal tab, vertical tab, form feed, and
new-line, plus the following 91 graphical characters:\footnote{The glyphs for
the members of the basic source character set are intended to
identify characters from the subset of ISO/IEC 10646 which corresponds to the ASCII
character set. However, because the mapping from source file characters to the source
character set (described in translation phase 1) is specified as
\impldef{mapping from physical source file characters to basic source character set},
an implementation is required to document how the basic source characters are
represented in source files.}
\begin{codeblock}
a b c d e f g h i j k l m n o p q r s t u v w x y z
A B C D E F G H I J K L M N O P Q R S T U V W X Y Z
0 1 2 3 4 5 6 7 8 9
_ { } [ ] # ( ) < > % : ; . ? * + - / ^ & | ~ ! = , @\textbackslash@ " '
\end{codeblock}

\pnum
The \grammarterm{universal-character-name} construct provides a way to name
other characters.

\begin{bnf}
\nontermdef{hex-quad}\br
    hexadecimal-digit hexadecimal-digit hexadecimal-digit hexadecimal-digit
\end{bnf}

\begin{bnf}
\nontermdef{universal-character-name}\br
    \terminal{\textbackslash u} hex-quad\br
    \terminal{\textbackslash U} hex-quad hex-quad
\end{bnf}

A \grammarterm{universal-character-name}
designates the character in ISO/IEC 10646 (if any)
whose code point is the hexadecimal number represented by
the sequence of \grammarterm{hexadecimal-digit}s
in the \grammarterm{universal-character-name}.
The program is ill-formed if that number is not a code point
or if it is a surrogate code point.
Noncharacter code points and reserved code points
are considered to designate separate characters distinct from
any ISO/IEC 10646 character.
If a \grammarterm{universal-character-name} outside
the \grammarterm{c-char-sequence}, \grammarterm{s-char-sequence}, or
\grammarterm{r-char-sequence} of
a \grammarterm{character-literal} or \grammarterm{string-literal}
(in either case, including within a \grammarterm{user-defined-literal})
corresponds to a control character or
to a character in the basic
source character set, the program is ill-formed.\footnote{A sequence of characters resembling a \grammarterm{universal-character-name} in an
\grammarterm{r-char-sequence}\iref{lex.string} does not form a
\grammarterm{universal-character-name}.}
\begin{note}
ISO/IEC 10646 code points are integers in the range $[0, \mathrm{10FFFF}]$ (hexadecimal).
A surrogate code point is a value in the range $[\mathrm{D800}, \mathrm{DFFF}]$ (hexadecimal).
A control character is a character whose code point is
in either of the ranges $[0, \mathrm{1F}]$ or $[\mathrm{7F}, \mathrm{9F}]$ (hexadecimal).
\end{note}

\pnum
The \defnx{basic execution character set}{character set!basic execution} and the
\defnx{basic execution wide-character set}{wide-character set!basic execution}
shall each contain all the members of the
basic source character set, plus control characters representing alert,
backspace, and carriage return, plus a \defnx{null character}{character!null}
(respectively, \defnx{null wide character}{wide-character!null}), whose value is 0.
For each basic execution character set, the values of the
members shall be non-negative and distinct from one another. In both the
source and execution basic character sets, the value of each character
after \tcode{0} in the above list of decimal digits shall be one greater
than the value of the previous. The \defnx{execution character set}{character set!execution}
and the \defnx{execution wide-character set}{wide-character set!execution} are
\impldef{execution character set and execution wide-character set}
supersets of the
basic execution character set and the basic execution wide-character
set, respectively. The values of the members of the execution character sets
and the sets of additional members
are locale-specific.%
\indextext{character set|)}

\rSec1[lex.pptoken]{Preprocessing tokens}

\indextext{token!preprocessing|(}%
\begin{bnf}
\nontermdef{preprocessing-token}\br
    header-name\br
    import-keyword\br
    module-keyword\br
    export-keyword\br
    identifier\br
    pp-number\br
    character-literal\br
    user-defined-character-literal\br
    string-literal\br
    user-defined-string-literal\br
    preprocessing-op-or-punc\br
    \textnormal{each non-white-space character that cannot be one of the above}
\end{bnf}

\pnum
Each preprocessing token that is converted to a token\iref{lex.token}
shall have the lexical form of a keyword, an identifier, a literal,
or an operator or punctuator.

\pnum
A preprocessing token is the minimal lexical element of the language in translation
phases 3 through 6. The categories of preprocessing token are: header names,
placeholder tokens produced by preprocessing \tcode{import} and \tcode{module} directives
(\grammarterm{import-keyword}, \grammarterm{module-keyword}, and \grammarterm{export-keyword}),
identifiers, preprocessing numbers, character literals (including user-defined character
literals), string literals (including user-defined string literals), preprocessing
operators and punctuators, and single non-white-space characters that do not lexically
match the other preprocessing token categories. If a \tcode{'} or a \tcode{"} character
matches the last category, the behavior is undefined. Preprocessing tokens can be
separated by
\indextext{space!white}%
white space;
\indextext{comment}%
this consists of comments\iref{lex.comment}, or white-space
characters (space, horizontal tab, new-line, vertical tab, and
form-feed), or both. As described in \ref{cpp}, in certain
circumstances during translation phase 4, white space (or the absence
thereof) serves as more than preprocessing token separation. White space
can appear within a preprocessing token only as part of a header name or
between the quotation characters in a character literal or
string literal.

\pnum
If the input stream has been parsed into preprocessing tokens up to a
given character:
\begin{itemize}
\item
\indextext{literal!string!raw}%
If the next character begins a sequence of characters that could be the prefix
and initial double quote of a raw string literal, such as \tcode{R"}, the next preprocessing
token shall be a raw string literal. Between the initial and final
double quote characters of the raw string, any transformations performed in phases
1 and 2 (\grammarterm{universal-character-name}{s} and line splicing) are reverted; this reversion
shall apply before any \grammarterm{d-char}, \grammarterm{r-char}, or delimiting
parenthesis is identified. The raw string literal is defined as the shortest sequence
of characters that matches the raw-string pattern
\begin{ncbnf}
\opt{encoding-prefix} \terminal{R} raw-string
\end{ncbnf}

\item Otherwise, if the next three characters are \tcode{<::} and the subsequent character
is neither \tcode{:} nor \tcode{>}, the \tcode{<} is treated as a preprocessing token by
itself and not as the first character of the alternative token \tcode{<:}.

\item Otherwise,
the next preprocessing token is the longest sequence of
characters that could constitute a preprocessing token, even if that
would cause further lexical analysis to fail,
except that a \grammarterm{header-name}\iref{lex.header} is only formed
\begin{itemize}
\item
after the \tcode{include} or \tcode{import} preprocessing token in an
\tcode{\#include}\iref{cpp.include} or
\tcode{import}\iref{cpp.import} directive, or

\item
within a \grammarterm{has-include-expression}.

\end{itemize}
\end{itemize}

\begin{example}
\begin{codeblock}
#define R "x"
const char* s = R"y";           // ill-formed raw string, not \tcode{"x" "y"}
\end{codeblock}
\end{example}

\pnum
The \grammarterm{import-keyword} is produced
by processing an \tcode{import} directive\iref{cpp.import},
the \grammarterm{module-keyword} is produced
by preprocessing a \tcode{module} directive\iref{cpp.module}, and
the \grammarterm{export-keyword} is produced
by preprocessing either of the previous two directives.
\begin{note}
None has any observable spelling.
\end{note}

\pnum
\begin{example}
The program fragment \tcode{0xe+foo} is parsed as a
preprocessing number token (one that is not a valid
\grammarterm{integer-literal} or \grammarterm{floating-point-literal} token),
even though a parse as three preprocessing tokens
\tcode{0xe}, \tcode{+}, and \tcode{foo} might produce a valid expression (for example,
if \tcode{foo} were a macro defined as \tcode{1}). Similarly, the
program fragment \tcode{1E1} is parsed as a preprocessing number (one
that is a valid \grammarterm{floating-point-literal} token),
whether or not \tcode{E} is a macro name.
\end{example}

\pnum
\begin{example}
The program fragment \tcode{x+++++y} is parsed as \tcode{x
++ ++ + y}, which, if \tcode{x} and \tcode{y} have integral types,
violates a constraint on increment operators, even though the parse
\tcode{x ++ + ++ y} might yield a correct expression.
\end{example}
\indextext{token!preprocessing|)}

\rSec1[lex.digraph]{Alternative tokens}

\pnum
\indextext{token!alternative|(}%
Alternative token representations are provided for some operators and
punctuators.\footnote{\indextext{digraph}%
These include ``digraphs'' and additional reserved words. The term
``digraph'' (token consisting of two characters) is not perfectly
descriptive, since one of the alternative \grammarterm{preprocessing-token}s is
\tcode{\%:\%:} and of course several primary tokens contain two
characters. Nonetheless, those alternative tokens that aren't lexical
keywords are colloquially known as ``digraphs''. }

\pnum
In all respects of the language, each alternative token behaves the
same, respectively, as its primary token, except for its spelling.\footnote{Thus the ``stringized'' values\iref{cpp.stringize} of
\tcode{[} and \tcode{<:} will be different, maintaining the source
spelling, but the tokens can otherwise be freely interchanged. }
The set of alternative tokens is defined in
\tref{lex.digraph}.

\begin{tokentable}{Alternative tokens}{lex.digraph}{Alternative}{Primary}
\tcode{<\%}             &   \tcode{\{}         &
\tcode{and}             &   \tcode{\&\&}       &
\tcode{and_eq}          &   \tcode{\&=}        \\ \rowsep
\tcode{\%>}             &   \tcode{\}}         &
\tcode{bitor}           &   \tcode{|}          &
\tcode{or_eq}           &   \tcode{|=}         \\ \rowsep
\tcode{<:}              &   \tcode{[}          &
\tcode{or}              &   \tcode{||}         &
\tcode{xor_eq}          &   \tcode{\caret=}    \\ \rowsep
\tcode{:>}              &   \tcode{]}          &
\tcode{xor}             &   \tcode{\caret}     &
\tcode{not}             &   \tcode{!}          \\ \rowsep
\tcode{\%:}             &   \tcode{\#}         &
\tcode{compl}           &   \tcode{\~}         &
\tcode{not_eq}          &   \tcode{!=}         \\ \rowsep
\tcode{\%:\%:}          &   \tcode{\#\#}       &
\tcode{bitand}          &   \tcode{\&}         &
                        &                      \\
\end{tokentable}%
\indextext{token!alternative|)}

\rSec1[lex.token]{Tokens}

\indextext{token|(}%
\begin{bnf}
\nontermdef{token}\br
    identifier\br
    keyword\br
    literal\br
    operator-or-punctuator
\end{bnf}

\pnum
\indextext{\idxgram{token}}%
There are five kinds of tokens: identifiers, keywords, literals,\footnote{Literals include strings and character and numeric literals.
}
operators, and other separators.
\indextext{white space}%
Blanks, horizontal and vertical tabs, newlines, formfeeds, and comments
(collectively, ``white space''), as described below, are ignored except
as they serve to separate tokens.
\begin{note}
Some white space is
required to separate otherwise adjacent identifiers, keywords, numeric
literals, and alternative tokens containing alphabetic characters.
\end{note}
\indextext{token|)}

\rSec1[lex.comment]{Comments}

\pnum
\indextext{comment|(}%
\indextext{comment!\tcode{/*} \tcode{*/}}%
\indextext{comment!\tcode{//}}%
The characters \tcode{/*} start a comment, which terminates with the
characters \tcode{*/}. These comments do not nest.
\indextext{comment!\tcode{//}}%
The characters \tcode{//} start a comment, which terminates immediately before the
next new-line character. If there is a form-feed or a vertical-tab
character in such a comment, only white-space characters shall appear
between it and the new-line that terminates the comment; no diagnostic
is required.
\begin{note}
The comment characters \tcode{//}, \tcode{/*},
and \tcode{*/} have no special meaning within a \tcode{//} comment and
are treated just like other characters. Similarly, the comment
characters \tcode{//} and \tcode{/*} have no special meaning within a
\tcode{/*} comment.
\end{note}
\indextext{comment|)}

\rSec1[lex.header]{Header names}

\indextext{header!name|(}%
\begin{bnf}
\nontermdef{header-name}\br
    \terminal{<} h-char-sequence \terminal{>}\br
    \terminal{"} q-char-sequence \terminal{"}
\end{bnf}

\begin{bnf}
\nontermdef{h-char-sequence}\br
    h-char\br
    h-char-sequence h-char
\end{bnf}

\begin{bnf}
\nontermdef{h-char}\br
    \textnormal{any member of the source character set except new-line and \terminal{>}}
\end{bnf}

\begin{bnf}
\nontermdef{q-char-sequence}\br
    q-char\br
    q-char-sequence q-char
\end{bnf}

\begin{bnf}
\nontermdef{q-char}\br
    \textnormal{any member of the source character set except new-line and \terminal{"}}
\end{bnf}

\pnum
\begin{note}
Header name preprocessing tokens only appear within
a \tcode{\#include} preprocessing directive,
a \tcode{__has_include} preprocessing expression, or
after certain occurrences of an \tcode{import} token
(see~\ref{lex.pptoken}).
\end{note}
The sequences in both forms of \grammarterm{header-name}{s} are mapped in an
\impldef{mapping header name to header or external source file} manner to headers or to
external source file names as specified in~\ref{cpp.include}.

\pnum
The appearance of either of the characters \tcode{'} or \tcode{\textbackslash} or of
either of the character sequences \tcode{/*} or \tcode{//} in a
\grammarterm{q-char-sequence} or an \grammarterm{h-char-sequence}
is conditionally-supported with \impldef{meaning of \tcode{'}, \tcode{\textbackslash},
\tcode{/*}, or \tcode{//} in a \grammarterm{q-char-sequence} or an
\grammarterm{h-char-sequence}} semantics, as is the appearance of the character
\tcode{"} in an \grammarterm{h-char-sequence}.\footnote{Thus, a sequence of characters
that resembles an escape sequence might result in an error, be interpreted as the
character corresponding to the escape sequence, or have a completely different meaning,
depending on the implementation.}%
\indextext{header!name|)}

\rSec1[lex.ppnumber]{Preprocessing numbers}

\indextext{number!preprocessing|(}%
\begin{bnf}
\nontermdef{pp-number}\br
    digit\br
    \terminal{.} digit\br
    pp-number digit\br
    pp-number identifier-nondigit\br
    pp-number \terminal{'} digit\br
    pp-number \terminal{'} nondigit\br
    pp-number \terminal{e} sign\br
    pp-number \terminal{E} sign\br
    pp-number \terminal{p} sign\br
    pp-number \terminal{P} sign\br
    pp-number \terminal{.}
\end{bnf}

\pnum
Preprocessing number tokens lexically include
all \grammarterm{integer-literal} tokens\iref{lex.icon} and
all \grammarterm{floating-point-literal} tokens\iref{lex.fcon}.

\pnum
A preprocessing number does not have a type or a value; it acquires both
after a successful conversion to
an \grammarterm{integer-literal} token or
a \grammarterm{floating-point-literal} token.%
\indextext{number!preprocessing|)}

\rSec1[lex.name]{Identifiers}

\indextext{identifier|(}%
\begin{bnf}
\nontermdef{identifier}\br
    identifier-nondigit\br
    identifier identifier-nondigit\br
    identifier digit
\end{bnf}

\begin{bnf}
\nontermdef{identifier-nondigit}\br
    nondigit\br
    universal-character-name
\end{bnf}

\begin{bnf}
\nontermdef{nondigit} \textnormal{one of}\br
    \terminal{a b c d e f g h i j k l m}\br
    \terminal{n o p q r s t u v w x y z}\br
    \terminal{A B C D E F G H I J K L M}\br
    \terminal{N O P Q R S T U V W X Y Z _}
\end{bnf}

\begin{bnf}
\nontermdef{digit} \textnormal{one of}\br
    \terminal{0 1 2 3 4 5 6 7 8 9}
\end{bnf}

\pnum
\indextext{name!length of}%
\indextext{name}%
An identifier is an arbitrarily long sequence of letters and digits.
Each \grammarterm{universal-character-name} in an identifier shall designate a
character whose encoding in ISO/IEC 10646 falls into one of the ranges
specified in \tref{lex.name.allowed}.
The initial element shall not be a \grammarterm{universal-character-name}
designating a character whose encoding falls into one of the ranges
specified in \tref{lex.name.disallowed}.
Upper- and lower-case letters are
different. All characters are significant.\footnote{On systems in which linkers cannot accept extended
characters, an encoding of the \grammarterm{universal-character-name} may be used in
forming valid external identifiers. For example, some otherwise unused
character or sequence of characters may be used to encode the
\tcode{\textbackslash u} in a \grammarterm{universal-character-name}. Extended
characters may produce a long external identifier, but \Cpp{} does not
place a translation limit on significant characters for external
identifiers. In \Cpp{}, upper- and lower-case letters are considered
different for all identifiers, including external identifiers. }

\begin{floattable}{Ranges of characters allowed}{lex.name.allowed}
{lllll}
\topline
\tcode{00A8} &
\tcode{00AA} &
\tcode{00AD} &
\tcode{00AF} &
\tcode{00B2-00B5} \\
\tcode{00B7-00BA} &
\tcode{00BC-00BE} &
\tcode{00C0-00D6} &
\tcode{00D8-00F6} &
\tcode{00F8-00FF} \\
\tcode{0100-167F} &
\tcode{1681-180D} &
\tcode{180F-1FFF} &&\\
\tcode{200B-200D} &
\tcode{202A-202E} &
\tcode{203F-2040} &
\tcode{2054} &
\tcode{2060-206F} \\
\tcode{2070-218F} &
\tcode{2460-24FF} &
\tcode{2776-2793} &
\tcode{2C00-2DFF} &
\tcode{2E80-2FFF} \\
\tcode{3004-3007} &
\tcode{3021-302F} &
\tcode{3031-D7FF} && \\
\tcode{F900-FD3D} &
\tcode{FD40-FDCF} &
\tcode{FDF0-FE44} &
\tcode{FE47-FFFD} & \\
\tcode{10000-1FFFD} &
\tcode{20000-2FFFD} &
\tcode{30000-3FFFD} &
\tcode{40000-4FFFD} &
\tcode{50000-5FFFD} \\
\tcode{60000-6FFFD} &
\tcode{70000-7FFFD} &
\tcode{80000-8FFFD} &
\tcode{90000-9FFFD} &
\tcode{A0000-AFFFD} \\
\tcode{B0000-BFFFD} &
\tcode{C0000-CFFFD} &
\tcode{D0000-DFFFD} &
\tcode{E0000-EFFFD} &
\\
\end{floattable}

\begin{floattable}{Ranges of characters disallowed initially (combining characters)}{lex.name.disallowed}
{llll}
\topline
\tcode{0300-036F} &
% FIXME: Unicode v7 adds 1AB0-1AFF
\tcode{1DC0-1DFF} &
\tcode{20D0-20FF} &
\tcode{FE20-FE2F} \\
\end{floattable}

\pnum
\indextext{\idxcode{import}}%
\indextext{\idxcode{final}}%
\indextext{\idxcode{module}}%
\indextext{\idxcode{override}}%
The identifiers in \tref{lex.name.special} have a special meaning when
appearing in a certain context. When referred to in the grammar, these identifiers
are used explicitly rather than using the \grammarterm{identifier} grammar production.
Unless otherwise specified, any ambiguity as to whether a given
\grammarterm{identifier} has a special meaning is resolved to interpret the
token as a regular \grammarterm{identifier}.

\begin{multicolfloattable}{Identifiers with special meaning}{lex.name.special}
{llll}
\keyword{final}           \\
\columnbreak
\keyword{import}          \\
\columnbreak
\keyword{module}          \\
\columnbreak
\keyword{override}        \\
\end{multicolfloattable}

\pnum
\indextext{\idxcode{_}|see{character, underscore}}%
\indextext{character!underscore!in identifier}%
\indextext{reserved identifier}%
In addition, some identifiers are reserved for use by \Cpp{}
implementations and shall
not be used otherwise; no diagnostic is required.
\begin{itemize}
\item
Each identifier that contains a double underscore
\tcode{\unun}
\indextext{character!underscore}%
or begins with an underscore followed by
an uppercase letter
\indextext{uppercase}%
is reserved to the implementation for any use.
\item
Each identifier that begins with an underscore is
\indextext{character!underscore}%
reserved to the implementation for use as a name in the global namespace.%
\indextext{namespace!global}
\end{itemize}%
\indextext{identifier|)}

\rSec1[lex.key]{Keywords}

\begin{bnf}
\nontermdef{keyword}\br
    \textnormal{any identifier listed in \tref{lex.key}}\br
    \grammarterm{import-keyword}\br
    \grammarterm{module-keyword}\br
    \grammarterm{export-keyword}
\end{bnf}

\pnum
\indextext{keyword|(}%
The identifiers shown in \tref{lex.key} are reserved for use
as keywords (that is, they are unconditionally treated as keywords in
phase 7) except in an \grammarterm{attribute-token}\iref{dcl.attr.grammar}.
\begin{note}
The \keyword{register} keyword is unused but
is reserved for future use.
\end{note}

\begin{multicolfloattable}{Keywords}{lex.key}
{lllll}
\keyword{alignas} \\
\keyword{alignof} \\
\keyword{asm} \\
\keyword{auto} \\
\keyword{bool} \\
\keyword{break} \\
\keyword{case} \\
\keyword{catch} \\
\keyword{char} \\
\keyword{char8_t} \\
\keyword{char16_t} \\
\keyword{char32_t} \\
\keyword{class} \\
\keyword{concept} \\
\keyword{const} \\
\keyword{consteval} \\
\keyword{constexpr} \\
\columnbreak
\keyword{constinit} \\
\keyword{const_cast} \\
\keyword{continue} \\
\keyword{co_await} \\
\keyword{co_return} \\
\keyword{co_yield} \\
\keyword{decltype} \\
\keyword{default} \\
\keyword{delete} \\
\keyword{do} \\
\keyword{double} \\
\keyword{dynamic_cast} \\
\keyword{else} \\
\keyword{enum} \\
\keyword{explicit} \\
\keyword{export} \\
\keyword{extern} \\
\columnbreak
\keyword{false} \\
\keyword{float} \\
\keyword{for} \\
\keyword{friend} \\
\keyword{goto} \\
\keyword{if} \\
\keyword{inline} \\
\keyword{int} \\
\keyword{long} \\
\keyword{mutable} \\
\keyword{namespace} \\
\keyword{new} \\
\keyword{noexcept} \\
\keyword{nullptr} \\
\keyword{operator} \\
\keyword{private} \\
\keyword{protected} \\
\columnbreak
\keyword{public} \\
\keyword{register} \\
\keyword{reinterpret_cast} \\
\keyword{requires} \\
\keyword{return} \\
\keyword{short} \\
\keyword{signed} \\
\keyword{sizeof} \\
\keyword{static} \\
\keyword{static_assert} \\
\keyword{static_cast} \\
\keyword{struct} \\
\keyword{switch} \\
\keyword{template} \\
\keyword{this} \\
\keyword{thread_local} \\
\keyword{throw} \\
\columnbreak
\keyword{true} \\
\keyword{try} \\
\keyword{typedef} \\
\keyword{typeid} \\
\keyword{typename} \\
\keyword{union} \\
\keyword{unsigned} \\
\keyword{using} \\
\keyword{virtual} \\
\keyword{void} \\
\keyword{volatile} \\
\keyword{wchar_t} \\
\keyword{while} \\
\end{multicolfloattable}

\pnum
Furthermore, the alternative representations shown in
\tref{lex.key.digraph} for certain operators and
punctuators\iref{lex.digraph} are reserved and shall not be used
otherwise.

\begin{floattable}{Alternative representations}{lex.key.digraph}
{llllll}
\topline
\keyword{and}     &   \keyword{and_eq}  &   \keyword{bitand}  &   \keyword{bitor}   &   \keyword{compl}   &   \keyword{not} \\
\keyword{not_eq}  &   \keyword{or}      &   \keyword{or_eq}   &   \keyword{xor}     &   \keyword{xor_eq}  &       \\
\end{floattable}%
\indextext{keyword|)}%


\rSec1[lex.operators]{Operators and punctuators}

\pnum
\indextext{operator|(}%
\indextext{punctuator|(}%
The lexical representation of \Cpp{} programs includes a number of
preprocessing tokens that are used in the syntax of the preprocessor or
are converted into tokens for operators and punctuators:

\begin{bnf}
\nontermdef{preprocessing-op-or-punc}\br
    preprocessing-operator\br
    operator-or-punctuator
\end{bnf}

\begin{bnf}
%% Ed. note: character protrusion would misalign various operators.
\microtypesetup{protrusion=false}\obeyspaces
\nontermdef{preprocessing-operator} \textnormal{one of}\br
    \terminal{\#        \#\#       \%:       \%:\%:}
\end{bnf}

\begin{bnf}
\microtypesetup{protrusion=false}\obeyspaces
\nontermdef{operator-or-punctuator} \textnormal{one of}\br
    \terminal{\{        \}        [        ]        (        )}\br
    \terminal{<:       :>       <\%       \%>       ;        :        ...}\br
    \terminal{?        ::       .        .*       ->       ->*      \~}\br
    \terminal{!        +        -        *        /        \%        \caret{}        \&        |}\br
    \terminal{=        +=       -=       *=       /=       \%=       \caret{}=       \&=       |=}\br
    \terminal{==       !=       <        >        <=       >=       <=>      \&\&       ||}\br
    \terminal{<<       >>       <<=      >>=      ++       --       ,}\br
    \terminal{and      or       xor      not      bitand   bitor    compl}\br
    \terminal{and_eq   or_eq    xor_eq   not_eq}
\end{bnf}

Each \grammarterm{operator-or-punctuator} is converted to a single token
in translation phase 7\iref{lex.phases}.%
\indextext{punctuator|)}%
\indextext{operator|)}

\rSec1[lex.literal]{Literals}%
\indextext{literal|(}

\rSec2[lex.literal.kinds]{Kinds of literals}

\pnum
\indextext{constant}%
\indextext{literal!constant}%
There are several kinds of literals.\footnote{The term ``literal'' generally designates, in this
document, those tokens that are called ``constants'' in
ISO C. }

\begin{bnf}
\nontermdef{literal}\br
    integer-literal\br
    character-literal\br
    floating-point-literal\br
    string-literal\br
    boolean-literal\br
    pointer-literal\br
    user-defined-literal
\end{bnf}

\rSec2[lex.icon]{Integer literals}

\indextext{literal!integer}%
\begin{bnf}
\nontermdef{integer-literal}\br
    binary-literal \opt{integer-suffix}\br
    octal-literal \opt{integer-suffix}\br
    decimal-literal \opt{integer-suffix}\br
    hexadecimal-literal \opt{integer-suffix}
\end{bnf}

\begin{bnf}
\nontermdef{binary-literal}\br
    \terminal{0b} binary-digit\br
    \terminal{0B} binary-digit\br
    binary-literal \opt{\terminal{'}} binary-digit
\end{bnf}

\begin{bnf}
\nontermdef{octal-literal}\br
    \terminal{0}\br
    octal-literal \opt{\terminal{'}} octal-digit
\end{bnf}

\begin{bnf}
\nontermdef{decimal-literal}\br
    nonzero-digit\br
    decimal-literal \opt{\terminal{'}} digit
\end{bnf}

\begin{bnf}
\nontermdef{hexadecimal-literal}\br
    hexadecimal-prefix hexadecimal-digit-sequence
\end{bnf}

\begin{bnf}
\nontermdef{binary-digit} \textnormal{one of}\br
    \terminal{0  1}
\end{bnf}

\begin{bnf}
\nontermdef{octal-digit} \textnormal{one of}\br
    \terminal{0  1  2  3  4  5  6  7}
\end{bnf}

\begin{bnf}
\nontermdef{nonzero-digit} \textnormal{one of}\br
    \terminal{1  2  3  4  5  6  7  8  9}
\end{bnf}

\begin{bnf}
\nontermdef{hexadecimal-prefix} \textnormal{one of}\br
    \terminal{0x  0X}
\end{bnf}

\begin{bnf}
\nontermdef{hexadecimal-digit-sequence}\br
    hexadecimal-digit\br
    hexadecimal-digit-sequence \opt{\terminal{'}} hexadecimal-digit
\end{bnf}

\begin{bnf}
\nontermdef{hexadecimal-digit} \textnormal{one of}\br
    \terminal{0  1  2  3  4  5  6  7  8  9}\br
    \terminal{a  b  c  d  e  f}\br
    \terminal{A  B  C  D  E  F}
\end{bnf}

\begin{bnf}
\nontermdef{integer-suffix}\br
    unsigned-suffix \opt{long-suffix} \br
    unsigned-suffix \opt{long-long-suffix} \br
    long-suffix \opt{unsigned-suffix} \br
    long-long-suffix \opt{unsigned-suffix}
\end{bnf}

\begin{bnf}
\nontermdef{unsigned-suffix} \textnormal{one of}\br
    \terminal{u  U}
\end{bnf}

\begin{bnf}
\nontermdef{long-suffix} \textnormal{one of}\br
    \terminal{l  L}
\end{bnf}

\begin{bnf}
\nontermdef{long-long-suffix} \textnormal{one of}\br
    \terminal{ll  LL}
\end{bnf}

\pnum
\indextext{literal!\idxcode{unsigned}}%
\indextext{literal!\idxcode{long}}%
\indextext{literal!base of integer}%
In an \grammarterm{integer-literal},
the sequence of
\grammarterm{binary-digit}s,
\grammarterm{octal-digit}s,
\grammarterm{digit}s, or
\grammarterm{hexadecimal-digit}s
is interpreted as a base $N$ integer as shown in table \tref{lex.icon.base};
the lexically first digit of the sequence of digits is the most significant.
\begin{note}
The prefix and any optional separating single quotes are ignored
when determining the value.
\end{note}

\begin{simpletypetable}
{Base of \grammarterm{integer-literal}{s}}
{lex.icon.base}
{lr}
\topline
\lhdr{Kind of \grammarterm{integer-literal}} & \rhdr{base $N$} \\ \capsep
\grammarterm{binary-literal} & 2 \\
\grammarterm{octal-literal} & 8 \\
\grammarterm{decimal-literal} & 10 \\
\grammarterm{hexadecimal-literal} & 16 \\
\end{simpletypetable}

\pnum
The \grammarterm{hexadecimal-digit}s
\tcode{a} through \tcode{f} and \tcode{A} through \tcode{F}
have decimal values ten through fifteen.
\begin{example}
The number twelve can be written \tcode{12}, \tcode{014},
\tcode{0XC}, or \tcode{0b1100}. The \grammarterm{integer-literal}s \tcode{1048576},
\tcode{1'048'576}, \tcode{0X100000}, \tcode{0x10'0000}, and
\tcode{0'004'000'000} all have the same value.
\end{example}

\pnum
\indextext{literal!\idxcode{long}}%
\indextext{literal!\idxcode{unsigned}}%
\indextext{literal!integer}%
\indextext{literal!type of integer}%
\indextext{suffix!\idxcode{L}}%
\indextext{suffix!\idxcode{U}}%
\indextext{suffix!\idxcode{l}}%
\indextext{suffix!\idxcode{u}}%
The type of an \grammarterm{integer-literal} is
the first type in the list in \tref{lex.icon.type}
corresponding to its optional \grammarterm{integer-suffix}
in which its value can be represented.
An \grammarterm{integer-literal} is a prvalue.

\begin{floattable}{Types of \grammarterm{integer-literal}s}{lex.icon.type}{l|l|l}
\topline
\lhdr{\grammarterm{integer-suffix}} & \chdr{\grammarterm{decimal-literal}}  & \rhdr{\grammarterm{integer-literal} other than \grammarterm{decimal-literal}}   \\  \capsep
none    &
  \tcode{int} &
  \tcode{int}\\
        &
  \tcode{long int} &
  \tcode{unsigned int}\\
        &
  \tcode{long long int} &
  \tcode{long int}\\
        &
        &
  \tcode{unsigned long int}\\
        &
        &
  \tcode{long long int}\\
        &
        &
  \tcode{unsigned long long int}\\\hline
\tcode{u} or \tcode{U}  &
  \tcode{unsigned int}  &
  \tcode{unsigned int}\\
                              &
  \tcode{unsigned long int}   &
  \tcode{unsigned long int}\\
                              &
  \tcode{unsigned long long int}   &
  \tcode{unsigned long long int}\\\hline
\tcode{l} or \tcode{L}  &
  \tcode{long int}  &
  \tcode{long int}\\
                              &
  \tcode{long long int}       &
  \tcode{unsigned long int}\\
                              &
                              &
  \tcode{long long int}\\
                              &
                              &
  \tcode{unsigned long long int}\\\hline
Both \tcode{u} or \tcode{U}   &
  \tcode{unsigned long int}  &
  \tcode{unsigned long int}\\
and \tcode{l} or \tcode{L}  &
  \tcode{unsigned long long int}  &
  \tcode{unsigned long long int}\\\hline
\tcode{ll} or \tcode{LL}  &
  \tcode{long long int}       &
  \tcode{long long int}\\
                              &
                              &
  \tcode{unsigned long long int}\\\hline
Both \tcode{u} or \tcode{U}   &
  \tcode{unsigned long long int}  &
  \tcode{unsigned long long int}\\
and \tcode{ll} or \tcode{LL}  &
                              &
                              \\
\end{floattable}

\pnum
If an \grammarterm{integer-literal}
cannot be represented by any type in its list and
an extended integer type\iref{basic.fundamental} can represent its value,
it may have that extended integer type.
If all of the types in the list for the \grammarterm{integer-literal}
are signed,
the extended integer type shall be signed.
If all of the types in the list for the \grammarterm{integer-literal}
are unsigned,
the extended integer type shall be unsigned.
If the list contains both signed and unsigned types,
the extended integer type may be signed or unsigned.
A program is ill-formed
if one of its translation units contains an \grammarterm{integer-literal}
that cannot be represented by any of the allowed types.

\rSec2[lex.ccon]{Character literals}

\indextext{literal!character}%
\begin{bnf}
\nontermdef{character-literal}\br
    \opt{encoding-prefix} \terminal{'} c-char-sequence \terminal{'}
\end{bnf}

\begin{bnf}
\nontermdef{encoding-prefix} \textnormal{one of}\br
    \terminal{u8}\quad\terminal{u}\quad\terminal{U}\quad\terminal{L}
\end{bnf}

\begin{bnf}
\nontermdef{c-char-sequence}\br
    c-char\br
    c-char-sequence c-char
\end{bnf}

\begin{bnf}
\nontermdef{c-char}\br
    \textnormal{any member of the basic source character set except the single-quote \terminal{'}, backslash \terminal{\textbackslash}, or new-line character}\br
    escape-sequence\br
    universal-character-name
\end{bnf}

\begin{bnf}
\nontermdef{escape-sequence}\br
    simple-escape-sequence\br
    octal-escape-sequence\br
    hexadecimal-escape-sequence
\end{bnf}

\begin{bnf}
\nontermdef{simple-escape-sequence} \textnormal{one of}\br
    \terminal{\textbackslash'}\quad\terminal{\textbackslash"}\quad\terminal{\textbackslash ?}\quad\terminal{\textbackslash\textbackslash}\br
    \terminal{\textbackslash a}\quad\terminal{\textbackslash b}\quad\terminal{\textbackslash f}\quad\terminal{\textbackslash n}\quad\terminal{\textbackslash r}\quad\terminal{\textbackslash t}\quad\terminal{\textbackslash v}
\end{bnf}

\begin{bnf}
\nontermdef{octal-escape-sequence}\br
    \terminal{\textbackslash} octal-digit\br
    \terminal{\textbackslash} octal-digit octal-digit\br
    \terminal{\textbackslash} octal-digit octal-digit octal-digit
\end{bnf}

\begin{bnf}
\nontermdef{hexadecimal-escape-sequence}\br
    \terminal{\textbackslash x} hexadecimal-digit\br
    hexadecimal-escape-sequence hexadecimal-digit
\end{bnf}

\pnum
\indextext{literal!character}%
\indextext{literal!\idxcode{char16_t}}%
\indextext{literal!\idxcode{char32_t}}%
\indextext{literal!type of character}%
\indextext{literal!character!ordinary}%
A \grammarterm{character-literal} that does not begin with
\tcode{u8}, \tcode{u}, \tcode{U}, or \tcode{L}
is an \defn{ordinary character literal}.
An ordinary character literal that contains a
single \grammarterm{c-char} representable in the execution character
set has type \tcode{char}, with value equal to the
numerical value of the encoding of the \grammarterm{c-char} in the
execution character set. An ordinary character literal that contains
more than one \grammarterm{c-char} is a
\indextext{multicharacter literal|see{literal, multicharacter}}%
\defnadj{multicharacter}{literal}.
A multicharacter literal, or an ordinary character literal containing
a single \grammarterm{c-char} not representable in the execution
character set, is conditionally-supported, has type \tcode{int},
\indextext{literal!multicharacter!implementation-defined value of}%
and has an \impldef{value of multicharacter literal} value.

\pnum
\indextext{literal!character!UTF-8}%
A \grammarterm{character-literal} that
begins with \tcode{u8}, such as \tcode{u8'w'},
\indextext{prefix!\idxcode{u8}}%
is a \grammarterm{character-literal} of type \tcode{char8_t},
known as a \defn{UTF-8 character literal}.
The value of a UTF-8 character literal
is equal to its ISO/IEC 10646 code point value,
provided that the code point value
can be encoded as a single UTF-8 code unit.
\begin{note}
That is, provided the code point value is in the range $[0, \mathrm{7F}]$ (hexadecimal).
\end{note}
If the value is not representable with a single UTF-8 code unit,
the program is ill-formed.
A UTF-8 character literal containing multiple \grammarterm{c-char}{s} is ill-formed.

\pnum
\indextext{literal!character!UTF-16}%
\indextext{type!\idxcode{char16_t}}%
A \grammarterm{character-literal} that
begins with the letter \tcode{u}, such as \tcode{u'x'},
\indextext{prefix!\idxcode{u}}%
is a \grammarterm{character-literal} of type \tcode{char16_t},
known as a \defn{UTF-16 character literal}.
The value of a UTF-16 character literal
is equal to its ISO/IEC 10646 code point value,
provided that the code point value is
representable with a single 16-bit code unit.
\begin{note}
That is, provided the code point value is in the range $[0, \mathrm{FFFF}]$ (hexadecimal).
\end{note}
If the value is not representable
with a single 16-bit code unit, the program is ill-formed.
A UTF-16 character literal
containing multiple \grammarterm{c-char}{s} is ill-formed.

\pnum
\indextext{literal!character!UTF-32}%
\indextext{type!\idxcode{char32_t}}%
A \grammarterm{character-literal} that
begins with the letter \tcode{U}, such as \tcode{U'y'},
\indextext{prefix!\idxcode{U}}%
is a \grammarterm{character-literal} of type \tcode{char32_t},
known as a \defn{UTF-32 character literal}.
The value of a
UTF-32 character literal containing a single \grammarterm{c-char} is equal
to its ISO/IEC 10646 code point value.
A UTF-32 character literal containing
multiple \grammarterm{c-char}{s} is ill-formed.

\pnum
\indextext{literal!character!wide}%
\indextext{wide-character}%
\indexhdr{stddef.h}%
\indextext{type!\idxcode{wchar_t}}%
A \grammarterm{character-literal} that
begins with the letter \tcode{L}, such as \tcode{L'z'},
\indextext{prefix!\idxcode{L}}%
is a \defn{wide-character literal}. A wide-character literal has type
\tcode{wchar_t}.\footnote{They are intended for character sets where a character does
not fit into a single byte. }
The value of a wide-character literal containing a single
\grammarterm{c-char} has value equal to the numerical value of the encoding
of the \grammarterm{c-char} in the execution wide-character set, unless the
\grammarterm{c-char} has no representation in the execution wide-character set, in which
case the value is \impldef{value of wide-character literal with single c-char that is
not in execution wide-character set}.
\begin{note}
The type \tcode{wchar_t} is able to
represent all members of the execution wide-character set (see~\ref{basic.fundamental}).
\end{note}
The value
of a wide-character literal containing multiple \grammarterm{c-char}{s} is
\impldef{value of wide-character literal containing multiple characters}.

\pnum
Certain non-graphic characters, the single quote \tcode{'}, the double quote \tcode{"},
the question mark \tcode{?},\footnote{Using an escape sequence for a question mark
is supported for compatibility with ISO \CppXIV{} and ISO C.}
and the backslash
\indextext{backslash character}%
\indextext{\idxcode{\textbackslash}|see{backslash character}}%
\indextext{escape character|see{backslash character}}%
\tcode{\textbackslash}, can be represented according to
\tref{lex.ccon.esc}.
\indextext{escape sequence!undefined}%
The double quote \tcode{"}  and the question mark \tcode{?}, can be
represented as themselves or by the escape sequences
\tcode{\textbackslash "} and \tcode{\textbackslash ?} respectively, but
the single quote \tcode{'} and the backslash \tcode{\textbackslash}
shall be represented by the escape sequences \tcode{\textbackslash'} and
\tcode{\textbackslash\textbackslash} respectively. Escape sequences in
which the character following the backslash is not listed in
\tref{lex.ccon.esc} are conditionally-supported, with \impldef{semantics of
non-standard escape sequences} semantics. An escape sequence specifies a single
character.

\begin{floattable}{Escape sequences}{lex.ccon.esc}
{lll}
\topline
new-line        &   NL(LF)          &   \tcode{\textbackslash n}                \\
horizontal tab  &   HT              &   \tcode{\textbackslash t}                \\
vertical tab    &   VT              &   \tcode{\textbackslash v}                \\
backspace       &   BS              &   \tcode{\textbackslash b}                \\
carriage return &   CR              &   \tcode{\textbackslash r}                \\
form feed       &   FF              &   \tcode{\textbackslash f}                \\
alert           &   BEL             &   \tcode{\textbackslash a}                \\
backslash       &   \textbackslash  &   \tcode{\textbackslash\textbackslash}    \\
question mark   &   ?               &   \tcode{\textbackslash ?}                \\
single quote    &   \tcode{'}       &   \tcode{\textbackslash '}                \\
double quote    &   \tcode{"}       &   \tcode{\textbackslash "}                \\
octal number    &   \numconst{ooo}  &   \tcode{\textbackslash\numconst{ooo}}    \\
hex number      &   \numconst{hhh}  &   \tcode{\textbackslash x\numconst{hhh}}  \\
\end{floattable}

\pnum
The escape
\indextext{number!octal}%
\tcode{\textbackslash\numconst{ooo}} consists of the backslash followed by one,
two, or three octal digits that are taken to specify the value of the
desired character. The escape
\indextext{number!hex}%
\tcode{\textbackslash x\numconst{hhh}}
consists of the backslash followed by \tcode{x} followed by one or more
hexadecimal digits that are taken to specify the value of the desired
character. There is no limit to the number of digits in a hexadecimal
sequence. A sequence of octal or hexadecimal digits is terminated by the
first character that is not an octal digit or a hexadecimal digit,
respectively.
\indextext{literal!implementation-defined value of char@implementation-defined value of \tcode{char}}%
The value of a \grammarterm{character-literal} is \impldef{value of \grammarterm{character-literal} outside range of
corresponding type} if it falls outside of the \impldef{range defined for \grammarterm{character-literal}s}
range defined for \tcode{char} (for \grammarterm{character-literal}s with no prefix) or
\tcode{wchar_t} (for \grammarterm{character-literal}s prefixed by \tcode{L}).
\begin{note}
If the value of a \grammarterm{character-literal} prefixed by
\tcode{u}, \tcode{u8}, or \tcode{U}
is outside the range defined for its type,
the program is ill-formed.
\end{note}

\pnum
A \grammarterm{universal-character-name} is translated to the encoding, in the appropriate
execution character set, of the character named. If there is no such
encoding, the \grammarterm{universal-character-name} is translated to an
\impldef{encoding of universal character name not in execution character set} encoding.
\begin{note}
In translation phase 1, a \grammarterm{universal-character-name} is introduced whenever an
actual extended
character is encountered in the source text. Therefore, all extended
characters are described in terms of \grammarterm{universal-character-name}{s}. However,
the actual compiler implementation can use its own native character set,
so long as the same results are obtained.
\end{note}

\rSec2[lex.fcon]{Floating-point literals}

\indextext{literal!floating-point}%
\begin{bnf}
\nontermdef{floating-point-literal}\br
    decimal-floating-point-literal\br
    hexadecimal-floating-point-literal
\end{bnf}

\begin{bnf}
\nontermdef{decimal-floating-point-literal}\br
    fractional-constant \opt{exponent-part} \opt{floating-point-suffix}\br
    digit-sequence exponent-part \opt{floating-point-suffix}
\end{bnf}

\begin{bnf}
\nontermdef{hexadecimal-floating-point-literal}\br
    hexadecimal-prefix hexadecimal-fractional-constant binary-exponent-part \opt{floating-point-suffix}\br
    hexadecimal-prefix hexadecimal-digit-sequence binary-exponent-part \opt{floating-point-suffix}
\end{bnf}

\begin{bnf}
\nontermdef{fractional-constant}\br
    \opt{digit-sequence} \terminal{.} digit-sequence\br
    digit-sequence \terminal{.}
\end{bnf}

\begin{bnf}
\nontermdef{hexadecimal-fractional-constant}\br
    \opt{hexadecimal-digit-sequence} \terminal{.} hexadecimal-digit-sequence\br
    hexadecimal-digit-sequence \terminal{.}
\end{bnf}

\begin{bnf}
\nontermdef{exponent-part}\br
    \terminal{e} \opt{sign} digit-sequence\br
    \terminal{E} \opt{sign} digit-sequence
\end{bnf}

\begin{bnf}
\nontermdef{binary-exponent-part}\br
    \terminal{p} \opt{sign} digit-sequence\br
    \terminal{P} \opt{sign} digit-sequence
\end{bnf}

\begin{bnf}
\nontermdef{sign} \textnormal{one of}\br
    \terminal{+  -}
\end{bnf}

\begin{bnf}
\nontermdef{digit-sequence}\br
    digit\br
    digit-sequence \opt{\terminal{'}} digit
\end{bnf}

\begin{bnf}
\nontermdef{floating-point-suffix} \textnormal{one of}\br
    \terminal{f  l  F  L}
\end{bnf}

\pnum
\indextext{literal!type of floating-point}%
\indextext{literal!\idxcode{float}}%
\indextext{suffix!\idxcode{F}}%
\indextext{suffix!\idxcode{f}}%
\indextext{suffix!\idxcode{L}}%
\indextext{suffix!\idxcode{l}}%
\indextext{literal!\idxcode{long double}}%
The type of a \grammarterm{floating-point-literal} is determined by
its \grammarterm{floating-point-suffix} as specified in \tref{lex.fcon.type}.
\begin{simpletypetable}
{Types of \grammarterm{floating-point-literal}{s}}
{lex.fcon.type}
{ll}
\topline
\lhdr{\grammarterm{floating-point-suffix}} & \rhdr{type} \\ \capsep
none & \keyword{double} \\
\tcode{f} or \tcode{F} & \keyword {float} \\
\tcode{l} or \tcode{L} & \keyword{long} \keyword{double} \\
\end{simpletypetable}

\pnum
\indextext{literal!floating-point}%
The \defn{significand} of a \grammarterm{floating-point-literal}
is the \grammarterm{fractional-constant} or \grammarterm{digit-sequence}
of a \grammarterm{decimal-floating-point-literal}
or the \grammarterm{hexadecimal-fractional-constant}
or \grammarterm{hexadecimal-digit-sequence}
of a \grammarterm{hexadecimal-floating-point-literal}.
In the significand,
the sequence of \grammarterm{digit}s or \grammarterm{hexadecimal-digit}s
and optional period are interpreted as a base $N$ real number $s$,
where $N$ is 10 for a \grammarterm{decimal-floating-point-literal} and
16 for a \grammarterm{hexadecimal-floating-point-literal}.
\begin{note}
Any optional separating single quotes are ignored when determining the value.
\end{note}
If an \grammarterm{exponent-part} or \grammarterm{binary-exponent-part}
is present,
the exponent $e$ of the \grammarterm{floating-point-literal}
is the result of interpreting
the sequence of an optional \grammarterm{sign} and the \grammarterm{digit}s
as a base 10 integer.
Otherwise, the exponent $e$ is 0.
The scaled value of the literal is
$s \times 10^e$ for a \grammarterm{decimal-floating-point-literal} and
$s \times 2^e$ for a \grammarterm{hexadecimal-floating-point-literal}.
\begin{example}
The \grammarterm{floating-point-literal}{s}
\tcode{49.625} and \tcode{0xC.68p+2} have the same value.
The \grammarterm{floating-point-literal}{s}
\tcode{1.602'176'565e-19} and \tcode{1.602176565e-19}
have the same value.
\end{example}

\pnum
If the scaled value is not in the range of representable
values for its type, the program is ill-formed.
Otherwise, the value of a \grammarterm{floating-point-literal}
is the scaled value if representable,
else the larger or smaller representable value nearest the scaled value,
chosen in an \impldef{choice of larger or smaller value of
\grammarterm{floating-point-literal}} manner.

\rSec2[lex.string]{String literals}

\indextext{literal!string}%
\begin{bnf}
\nontermdef{string-literal}\br
    \opt{encoding-prefix} \terminal{"} \opt{s-char-sequence} \terminal{"}\br
    \opt{encoding-prefix} \terminal{R} raw-string
\end{bnf}

\begin{bnf}
\nontermdef{s-char-sequence}\br
    s-char\br
    s-char-sequence s-char
\end{bnf}

\begin{bnf}
\nontermdef{s-char}\br
    \textnormal{any member of the basic source character set except the double-quote \terminal{"}, backslash \terminal{\textbackslash}, or new-line character}\br
    escape-sequence\br
    universal-character-name
\end{bnf}

\begin{bnf}
\nontermdef{raw-string}\br
    \terminal{"} \opt{d-char-sequence} \terminal{(} \opt{r-char-sequence} \terminal{)} \opt{d-char-sequence} \terminal{"}
\end{bnf}

\begin{bnf}
\nontermdef{r-char-sequence}\br
    r-char\br
    r-char-sequence r-char
\end{bnf}

\begin{bnf}
\nontermdef{r-char}\br
    \textnormal{any member of the source character set, except a right parenthesis \terminal{)} followed by}\br
    \bnfindent\textnormal{the initial \grammarterm{d-char-sequence} (which may be empty) followed by a double quote \terminal{"}.}
\end{bnf}

\begin{bnf}
\nontermdef{d-char-sequence}\br
    d-char\br
    d-char-sequence d-char
\end{bnf}

\begin{bnf}
\nontermdef{d-char}\br
    \textnormal{any member of the basic source character set except:}\br
    \bnfindent\textnormal{space, the left parenthesis \terminal{(}, the right parenthesis \terminal{)}, the backslash \terminal{\textbackslash}, and the control characters}\br
    \bnfindent\textnormal{representing horizontal tab, vertical tab, form feed, and newline.}
\end{bnf}

\pnum
\indextext{literal!string!raw}%
A \grammarterm{string-literal} that has an \tcode{R}
\indextext{prefix!\idxcode{R}}%
in the prefix is a \defn{raw string literal}. The
\grammarterm{d-char-sequence} serves as a delimiter. The terminating
\grammarterm{d-char-sequence} of a \grammarterm{raw-string} is the same sequence of
characters as the initial \grammarterm{d-char-sequence}. A \grammarterm{d-char-sequence}
shall consist of at most 16 characters.

\pnum
\begin{note}
The characters \tcode{'('} and \tcode{')'} are permitted in a
\grammarterm{raw-string}. Thus, \tcode{R"delimiter((a|b))delimiter"} is equivalent to
\tcode{"(a|b)"}.
\end{note}

\pnum
\begin{note}
A source-file new-line in a raw string literal results in a new-line in the
resulting execution string literal. Assuming no
whitespace at the beginning of lines in the following example, the assert will succeed:
\begin{codeblock}
const char* p = R"(a\
b
c)";
assert(std::strcmp(p, "a\\\nb\nc") == 0);
\end{codeblock}
\end{note}

\pnum
\begin{example}
The raw string
\begin{codeblock}
R"a(
)\
a"
)a"
\end{codeblock}
is equivalent to \tcode{"\textbackslash n)\textbackslash \textbackslash \textbackslash na\textbackslash"\textbackslash n"}. The raw string
\begin{codeblock}
R"(x = "\"y\"")"
\end{codeblock}
is equivalent to \tcode{"x = \textbackslash "\textbackslash\textbackslash\textbackslash "y\textbackslash\textbackslash\textbackslash "\textbackslash ""}.
\end{example}

\pnum
\indextext{literal!string}%
\indextext{character string}%
\indextext{string!type of}%
\indextext{literal!string!narrow}%
After translation phase 6, a \grammarterm{string-literal}
that does not begin with an \grammarterm{encoding-prefix} is an
\defn{ordinary string literal}.
An ordinary string literal
has type ``array of \placeholder{n} \tcode{const char}''
where \placeholder{n} is the size of the string as defined below,
has static storage duration\iref{basic.stc}, and
is initialized with the given characters.

\pnum
\indextext{literal!string!UTF-8}%
A \grammarterm{string-literal} that begins with \tcode{u8},
\indextext{prefix!\idxcode{u8}}%
such as \tcode{u8"asdf"}, is a \defn{UTF-8 string literal}.
A UTF-8 string literal
has type ``array of \placeholder{n} \tcode{const char8_t}'',
where \placeholder{n} is the size of the string as defined below;
each successive element of the object representation\iref{basic.types} has
the value of the corresponding code unit of the UTF-8 encoding of the string.

\pnum
\indextext{literal!narrow-character}%
Ordinary string literals and UTF-8 string literals are
also referred to as narrow string literals.

\pnum
\indextext{literal!string!UTF-16}%
\indextext{literal!string!\idxcode{char16_t}}%
\indextext{type!\idxcode{char16_t}}%
A \grammarterm{string-literal} that begins with \tcode{u},
\indextext{prefix!\idxcode{u}}%
such as \tcode{u"asdf"}, is
a \defn{UTF-16 string literal}.
A UTF-16 string literal has
type ``array of \placeholder{n} \tcode{const char16_t}'', where \placeholder{n} is the
size of the string as defined below;
each successive element of the array
has the value of the corresponding code unit of
the UTF-16 encoding of the string.
\begin{note}
A single \grammarterm{c-char} may
produce more than one \tcode{char16_t} character in the form of
surrogate pairs.
A surrogate pair is a representation for a single code point
as a sequence of two 16-bit code units.
\end{note}

\pnum
\indextext{literal!string!UTF-32}%
\indextext{literal!string!\idxcode{char32_t}}%
\indextext{type!\idxcode{char32_t}}%
A \grammarterm{string-literal} that begins with \tcode{U},
\indextext{prefix!\idxcode{U}}%
such as \tcode{U"asdf"}, is
a \defn{UTF-32 string literal}.
A UTF-32 string literal has
type ``array of \placeholder{n} \tcode{const char32_t}'', where \placeholder{n} is the
size of the string as defined below;
each successive element of the array
has the value of the corresponding code unit of
the UTF-32 encoding of the string.

\pnum
\indextext{literal!string!wide}%
A \grammarterm{string-literal} that begins with \tcode{L},
\indextext{prefix!\idxcode{L}}%
such as \tcode{L"asdf"}, is a \defn{wide string literal}.
\indexhdr{stddef.h}%
\indextext{type!\idxcode{wchar_t}}%
\indextext{literal!string!wide}%
\indextext{prefix!\idxcode{L}}%
A wide string literal has type ``array of \placeholder{n} \tcode{const
wchar_t}'', where \placeholder{n} is the size of the string as defined below; it
is initialized with the given characters.

\pnum
\indextext{concatenation!string}%
In translation phase 6\iref{lex.phases}, adjacent \grammarterm{string-literal}{s} are concatenated. If
both \grammarterm{string-literal}{s} have the same \grammarterm{encoding-prefix}, the resulting concatenated \grammarterm{string-literal} has
that \grammarterm{encoding-prefix}. If one \grammarterm{string-literal} has no \grammarterm{encoding-prefix}, it is treated as a \grammarterm{string-literal} of
the same \grammarterm{encoding-prefix} as the other operand. If a UTF-8 string literal token is adjacent to a
wide string literal token, the program is ill-formed. Any other concatenations are
conditionally-supported with \impldef{concatenation of some types of \grammarterm{string-literal}s}
behavior.
\begin{note}
This
concatenation is an interpretation, not a conversion.
Because the interpretation happens in translation phase 6 (after each character from a
\grammarterm{string-literal} has been translated into a value from the appropriate character set), a
\grammarterm{string-literal}'s initial rawness has no effect on the interpretation or
well-formedness of the concatenation.
\end{note}
\tref{lex.string.concat} has some examples of valid concatenations.

\begin{floattable}{String literal concatenations}{lex.string.concat}
{lll|lll|lll}
\topline
\multicolumn{2}{|c}{Source} &
Means &
\multicolumn{2}{c}{Source} &
Means &
\multicolumn{2}{c}{Source} &
Means \\
\tcode{u"a"} & \tcode{u"b"} & \tcode{u"ab"} &
\tcode{U"a"} & \tcode{U"b"} & \tcode{U"ab"} &
\tcode{L"a"} & \tcode{L"b"} & \tcode{L"ab"} \\
\tcode{u"a"} & \tcode{"b"}  & \tcode{u"ab"} &
\tcode{U"a"} & \tcode{"b"}  & \tcode{U"ab"} &
\tcode{L"a"} & \tcode{"b"}  & \tcode{L"ab"} \\
\tcode{"a"}  & \tcode{u"b"} & \tcode{u"ab"} &
\tcode{"a"}  & \tcode{U"b"} & \tcode{U"ab"} &
\tcode{"a"}  & \tcode{L"b"} & \tcode{L"ab"} \\
\end{floattable}

Characters in concatenated strings are kept distinct.

\begin{example}
\begin{codeblock}
"\xA" "B"
\end{codeblock}
contains the two characters \tcode{'\textbackslash xA'} and \tcode{'B'}
after concatenation (and not the single hexadecimal character
\tcode{'\textbackslash xAB'}).
\end{example}

\pnum
\indextext{\idxcode{0}|seealso{zero, null}}%
\indextext{\idxcode{0}!string terminator}%
\indextext{\idxcode{0}!null character|see {character, null}}%
After any necessary concatenation, in translation phase
7\iref{lex.phases}, \tcode{'\textbackslash 0'} is appended to every
\grammarterm{string-literal} so that programs that scan a string can find its end.

\pnum
Escape sequences and \grammarterm{universal-character-name}{s} in non-raw string literals
have the same meaning as in \grammarterm{character-literal}s\iref{lex.ccon}, except that
the single quote \tcode{'} is representable either by itself or by the escape sequence
\tcode{\textbackslash'}, and the double quote \tcode{"} shall be preceded by a
\tcode{\textbackslash},
and except that a \grammarterm{universal-character-name} in a
UTF-16 string literal may yield a surrogate pair.
\indextext{string!\idxcode{sizeof}}%
In a narrow string literal, a \grammarterm{universal-character-name} may map to more
than one \tcode{char} or \tcode{char8_t} element due to \defnadj{multibyte}{encoding}. The
size of a \tcode{char32_t} or wide string literal is the total number of
escape sequences, \grammarterm{universal-character-name}{s}, and other characters, plus
one for the terminating \tcode{U'\textbackslash 0'} or
\tcode{L'\textbackslash 0'}. The size of a UTF-16 string
literal is the total number of escape sequences,
\grammarterm{universal-character-name}{s}, and other characters, plus one for each
character requiring a surrogate pair, plus one for the terminating
\tcode{u'\textbackslash 0'}.
\begin{note}
The size of a \tcode{char16_t}
string literal is the number of code units, not the number of
characters.
\end{note}
\begin{note}
Any \grammarterm{universal-character-name}{s} are required to
correspond to a code point in the range
$[0,$ $\mathrm{D800})$ or $[\mathrm{E000},$ $\mathrm{10FFFF}]$ (hexadecimal)\iref{lex.charset}.
\end{note}
The size of a narrow string literal is
the total number of escape sequences and other characters, plus at least
one for the multibyte encoding of each \grammarterm{universal-character-name}, plus
one for the terminating \tcode{'\textbackslash 0'}.

\pnum
Evaluating a \grammarterm{string-literal} results in a string literal object
with static storage duration, initialized from the given characters as
specified above.
\indextext{string!distinct}%
Whether all \grammarterm{string-literal}s are distinct (that is, are stored in
nonoverlapping objects) and whether successive evaluations of a
\grammarterm{string-literal} yield the same or a different object is
unspecified.
\begin{note}
\indextext{literal!string!undefined change to}%
The effect of attempting to modify a \grammarterm{string-literal} is undefined.
\end{note}

\rSec2[lex.bool]{Boolean literals}

\indextext{literal!boolean}%
\begin{bnf}
\nontermdef{boolean-literal}\br
    \terminal{false}\br
    \terminal{true}
\end{bnf}

\pnum
\indextext{Boolean literal}%
The Boolean literals are the keywords \tcode{false} and \tcode{true}.
Such literals are prvalues and have type \tcode{bool}.

\rSec2[lex.nullptr]{Pointer literals}

\indextext{literal!pointer}%
\begin{bnf}
\nontermdef{pointer-literal}\br
    \terminal{nullptr}
\end{bnf}

\pnum
The pointer literal is the keyword \tcode{nullptr}. It is a prvalue of type
\tcode{std::nullptr_t}.
\begin{note}
\tcode{std::nullptr_t} is a distinct type that is neither a pointer type nor a pointer-to-member type;
rather, a prvalue of this type is a null pointer constant and can be
converted to a null pointer value or null member pointer value. See~\ref{conv.ptr}
and~\ref{conv.mem}.
\end{note}

\rSec2[lex.ext]{User-defined literals}

\indextext{literal!user-defined}%
\begin{bnf}
\nontermdef{user-defined-literal}\br
    user-defined-integer-literal\br
    user-defined-floating-point-literal\br
    user-defined-string-literal\br
    user-defined-character-literal
\end{bnf}

\begin{bnf}
\nontermdef{user-defined-integer-literal}\br
    decimal-literal ud-suffix\br
    octal-literal ud-suffix\br
    hexadecimal-literal ud-suffix\br
    binary-literal ud-suffix
\end{bnf}

\begin{bnf}
\nontermdef{user-defined-floating-point-literal}\br
    fractional-constant \opt{exponent-part} ud-suffix\br
    digit-sequence exponent-part ud-suffix\br
    hexadecimal-prefix hexadecimal-fractional-constant binary-exponent-part ud-suffix\br
    hexadecimal-prefix hexadecimal-digit-sequence binary-exponent-part ud-suffix
\end{bnf}

\begin{bnf}
\nontermdef{user-defined-string-literal}\br
    string-literal ud-suffix
\end{bnf}

\begin{bnf}
\nontermdef{user-defined-character-literal}\br
    character-literal ud-suffix
\end{bnf}

\begin{bnf}
\nontermdef{ud-suffix}\br
    identifier
\end{bnf}

\pnum
If a token matches both \grammarterm{user-defined-literal} and another \grammarterm{literal} kind, it
is treated as the latter.
\begin{example}
\tcode{123_km}
is a \grammarterm{user-defined-literal}, but \tcode{12LL} is an
\grammarterm{integer-literal}.
\end{example}
The syntactic non-terminal preceding the \grammarterm{ud-suffix} in a
\grammarterm{user-defined-literal} is taken to be the longest sequence of
characters that could match that non-terminal.

\pnum
A \grammarterm{user-defined-literal} is treated as a call to a literal operator or
literal operator template\iref{over.literal}. To determine the form of this call for a
given \grammarterm{user-defined-literal} \placeholder{L} with \grammarterm{ud-suffix} \placeholder{X},
the \grammarterm{literal-operator-id} whose literal suffix identifier is \placeholder{X} is
looked up in the context of \placeholder{L} using the rules for unqualified name
lookup\iref{basic.lookup.unqual}. Let \placeholder{S} be the set of declarations found by
this lookup. \placeholder{S} shall not be empty.

\pnum
If \placeholder{L} is a \grammarterm{user-defined-integer-literal}, let \placeholder{n} be the literal
without its \grammarterm{ud-suffix}. If \placeholder{S} contains a literal operator with
parameter type \tcode{unsigned long long}, the literal \placeholder{L} is treated as a call of
the form
\begin{codeblock}
operator "" @\placeholder{X}@(@\placeholder{n}@ULL)
\end{codeblock}
Otherwise, \placeholder{S} shall contain a raw literal operator
or a numeric literal operator template\iref{over.literal} but not both.
If \placeholder{S} contains a raw literal operator,
the literal \placeholder{L} is treated as a call of the form
\begin{codeblock}
operator "" @\placeholder{X}@("@\placeholder{n}@")
\end{codeblock}
Otherwise (\placeholder{S} contains a numeric literal operator template),
\placeholder{L} is treated as a call of the form
\begin{codeblock}
operator "" @\placeholder{X}@<'@$c_1$@', '@$c_2$@', ... '@$c_k$@'>()
\end{codeblock}
where \placeholder{n} is the source character sequence $c_1c_2...c_k$.
\begin{note}
The sequence
$c_1c_2...c_k$ can only contain characters from the basic source character set.
\end{note}

\pnum
If \placeholder{L} is a \grammarterm{user-defined-floating-point-literal}, let \placeholder{f} be the
literal without its \grammarterm{ud-suffix}. If \placeholder{S} contains a literal operator
with parameter type \tcode{long double}, the literal \placeholder{L} is treated as a call of
the form
\begin{codeblock}
operator "" @\placeholder{X}@(@\placeholder{f}@L)
\end{codeblock}
Otherwise, \placeholder{S} shall contain a raw literal operator
or a numeric literal operator template\iref{over.literal} but not both.
If \placeholder{S} contains a raw literal operator,
the \grammarterm{literal} \placeholder{L} is treated as a call of the form
\begin{codeblock}
operator "" @\placeholder{X}@("@\placeholder{f}@")
\end{codeblock}
Otherwise (\placeholder{S} contains a numeric literal operator template),
\placeholder{L} is treated as a call of the form
\begin{codeblock}
operator "" @\placeholder{X}@<'@$c_1$@', '@$c_2$@', ... '@$c_k$@'>()
\end{codeblock}
where \placeholder{f} is the source character sequence $c_1c_2...c_k$.
\begin{note}
The sequence
$c_1c_2...c_k$ can only contain characters from the basic source character set.
\end{note}

\pnum
If \placeholder{L} is a \grammarterm{user-defined-string-literal},
let \placeholder{str} be the literal without its \grammarterm{ud-suffix}
and let \placeholder{len} be the number of code units in \placeholder{str}
(i.e., its length excluding the terminating null character).
If \placeholder{S} contains a literal operator template with
a non-type template parameter for which \placeholder{str} is
a well-formed \grammarterm{template-argument},
the literal \placeholder{L} is treated as a call of the form
\begin{codeblock}
operator "" @\placeholder{X}@<@\placeholder{str}{}@>()
\end{codeblock}
Otherwise, the literal \placeholder{L} is treated as a call of the form
\begin{codeblock}
operator "" @\placeholder{X}@(@\placeholder{str}{}@, @\placeholder{len}{}@)
\end{codeblock}

\pnum
If \placeholder{L} is a \grammarterm{user-defined-character-literal}, let \placeholder{ch} be the
literal without its \grammarterm{ud-suffix}.
\placeholder{S} shall contain a literal operator\iref{over.literal} whose only parameter has
the type of \placeholder{ch} and the
literal \placeholder{L} is treated as a call
of the form
\begin{codeblock}
operator "" @\placeholder{X}@(@\placeholder{ch}{}@)
\end{codeblock}

\pnum
\begin{example}
\begin{codeblock}
long double operator "" _w(long double);
std::string operator "" _w(const char16_t*, std::size_t);
unsigned operator "" _w(const char*);
int main() {
  1.2_w;            // calls \tcode{operator "" _w(1.2L)}
  u"one"_w;         // calls \tcode{operator "" _w(u"one", 3)}
  12_w;             // calls \tcode{operator "" _w("12")}
  "two"_w;          // error: no applicable literal operator
}
\end{codeblock}
\end{example}

\pnum
In translation phase 6\iref{lex.phases}, adjacent \grammarterm{string-literal}s are concatenated and
\grammarterm{user-defined-string-literal}{s} are considered \grammarterm{string-literal}s for that
purpose. During concatenation, \grammarterm{ud-suffix}{es} are removed and ignored and
the concatenation process occurs as described in~\ref{lex.string}. At the end of phase
6, if a \grammarterm{string-literal} is the result of a concatenation involving at least one
\grammarterm{user-defined-string-literal}, all the participating
\grammarterm{user-defined-string-literal}{s} shall have the same \grammarterm{ud-suffix}
and that suffix is applied to the result of the concatenation.

\pnum
\begin{example}
\begin{codeblock}
int main() {
  L"A" "B" "C"_x;   // OK: same as \tcode{L"ABC"_x}
  "P"_x "Q" "R"_y;  // error: two different \grammarterm{ud-suffix}{es}
}
\end{codeblock}
\end{example}
\indextext{literal|)}%
\indextext{conventions!lexical|)}
