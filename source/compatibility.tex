%!TEX root = std.tex
\infannex{diff}{Compatibility}

\rSec1[diff.cpp23]{\Cpp{} and ISO \CppXXIII{}}

\rSec2[diff.cpp23.general]{General}

\pnum
\indextext{summary!compatibility with ISO \CppXXIII{}}%
Subclause \ref{diff.cpp23} lists the differences between \Cpp{} and
ISO \CppXXIII{} (ISO/IEC 14882:2023, \doccite{Programming Languages --- \Cpp{}}),
by the chapters of this document.

\rSec2[diff.cpp23.expr]{\ref{expr}: expressions}

\diffref{expr.arith.conv}
\change
Operations mixing a value of an enumeration type and a value of a different
enumeration type or of a floating-point type are no longer valid.
\rationale
Reinforcing type safety.
\effect
A valid \CppXXIII{} program that performs operations mixing a value of an
enumeration type and a value of a different enumeration type or of a
floating-point type is ill-formed.
For example:
\begin{codeblock}
enum E1 { e };
enum E2 { f };
bool b = e <= 3.7;      // ill-formed; previously well-formed
int  k = f - e;         // ill-formed; previously well-formed
auto x = true ? e : f;  // ill-formed; previously well-formed
\end{codeblock}

\rSec2[diff.cpp23.dcl.dcl]{\ref{dcl.dcl}: Declarations}

\diffref{dcl.init.list}
\change
Pointer comparisons between \tcode{initializer_list} objects' backing arrays
are unspecified.
\rationale
Permit the implementation to store backing arrays in static read-only memory.
\effect
Valid \CppXXIII{} code
that relies on the result of pointer comparison between backing arrays
may change behavior.
For example:
\begin{codeblock}
bool ne(std::initializer_list<int> a, std::initializer_list<int> b) {
  return a.begin() != b.begin() + 1;
}
bool b = ne({2,3}, {1,2,3});    // unspecified result; previously \tcode{false}
\end{codeblock}

\diffref{dcl.array}
\change
Previously, \tcode{T...[n]} would declare a pack of function parameters.
\tcode{T...[n]} is now a \grammarterm{pack-index-specifier}.
\rationale
Improve the handling of packs.
\effect
Valid \CppXXIII{} code that declares a pack of parameters
without specifying a \grammarterm{declarator-id} becomes ill-formed.
For example:
\begin{codeblock}
template <typename... T>
void f(T... [1]);
template <typename... T>
void g(T... ptr[1]);
int main() {
  f<int, double>(nullptr, nullptr);     // ill-formed, previously \tcode{void f<int, double>(int [1], double [1])}
  g<int, double>(nullptr, nullptr);     // ok
}
\end{codeblock}

\rSec2[diff.cpp23.library]{\ref{library}: library introduction}

\diffref{headers}
\change
New headers.
\rationale
New functionality.
\effect
The following \Cpp{} headers are new:
\libheaderref{debugging},
\libheaderrefx{hazard_pointer}{hazard.pointer.syn},
\libheaderref{linalg},
\libheaderref{rcu}, and
\libheaderrefx{text_encoding}{text.encoding.syn}.
Valid \CppXXIII{} code that \tcode{\#include}{s} headers with these names may be
invalid in this revision of \Cpp{}.

\rSec2[diff.cpp23.strings]{\ref{strings}: strings library}

\diffref{string.conversions}
\change
Output of floating-point overloads of \tcode{to_string} and \tcode{to_wstring}.
\rationale
Prevent loss of information and improve consistency with other formatting
facilities.
\effect
\tcode{to_string} and \tcode{to_wstring} function calls that take
floating-point arguments may produce a different output.
For example:
\begin{codeblock}
auto s = std::to_string(1e-7);  // \tcode{"1e-07"}
                                // previously \tcode{"0.000000"} with \tcode{'.'} possibly
                                // changed according to the global C locale
\end{codeblock}

\rSec2[diff.cpp23.containers]{\ref{containers}: containers library}

\diffref{span.overview}
\change
\tcode{span<const T>} is constructible from \tcode{initializer_list<T>}.
\rationale
Permit passing a braced initializer list to a function taking \tcode{span}.
\effect
Valid \CppXXIII{} code that relies on the lack of this constructor
may refuse to compile, or change behavior in this revision of \Cpp{}.
For example:
\begin{codeblock}
void one(pair<int, int>);       // \#1
void one(span<const int>);      // \#2
void t1() { one({1, 2}); }      // ambiguous between \#1 and \#2; previously called \#1

void two(span<const int, 2>);
void t2() { two({{1, 2}}); }    // ill-formed; previously well-formed

void *a[10];
int x = span<void* const>{a, 0}.size();     // \tcode{x} is \tcode{2}; previously \tcode{0}
any b[10];
int y = span<const any>{b, b + 10}.size();  // \tcode{y} is \tcode{2}; previously \tcode{10}
\end{codeblock}

\rSec2[diff.cpp23.depr]{\ref{depr}: compatibility features}

\nodiffref
\change
Remove the type alias \tcode{allocator<T>::is_always_equal}.
\rationale
Non-empty allocator classes derived from \tcode{allocator} needed to explicitly
define an \tcode{is_always_equal} member type so that \tcode{allocator_traits}
would not use the one from the allocator base class.
\effect
It is simpler to correctly define an allocator class with an allocator base
class. For example:
\begin{codeblock}
template <class T>
struct MyAlloc : allocator<T> {
  int tag;
};

static_assert(!allocator_traits<MyAlloc<int>>::is_always_equal);        // Error in \CppXXIII{},
                                                                        // OK in \CppXXVI{}
\end{codeblock}

\nodiffref
\change
Remove the \tcode{basic_string::reserve()} overload with no parameters.
\rationale
The overload of \tcode{reserve} with no parameters is redundant.
The \tcode{shrink_to_fit} member function can be used instead.
\effect
A valid \CppXXIII{} program that calls \tcode{reserve()}
on a \tcode{basic_string} object may fail to compile.
The old functionality can be achieved by calling \tcode{shrink_to_fit()} instead,
or the function call can be safely eliminated with no side effects.

\nodiffref
\change
Remove header \libnoheader{codecvt} and all its contents.
\rationale
The header has been deprecated for the previous three editions of this standard
and no longer implements the current Unicode standard, supporting only the
obsolete UCS-2 encoding.
Ongoing support is at implementer's discretion,
exercising freedoms granted by \ref{zombie.names}.
\effect
A valid \CppXXIII{} program \tcode{\#include}-ing the header or importing the
header unit may fail to compile. Code that uses any of the following names by
importing the standard library modules may fail to compile:
\begin{itemize}
\item \tcode{codecvt_mode},
\item \tcode{codecvt_utf16},
\item \tcode{codecvt_utf8},
\item \tcode{codecvt_utf8_utf16},
\item \tcode{consume_header},
\item \tcode{generate_header}, and
\item \tcode{little_endian}.
\end{itemize}

\rSec1[diff.cpp20]{\Cpp{} and ISO \CppXX{}}

\rSec2[diff.cpp20.general]{General}

\pnum
\indextext{summary!compatibility with ISO \CppXX{}}%
Subclause \ref{diff.cpp20} lists the differences between \Cpp{} and
ISO \CppXX{} (ISO/IEC 14882:2020, \doccite{Programming Languages --- \Cpp{}}),
by the chapters of this document.

\rSec2[diff.cpp20.lex]{\ref{lex}: lexical conventions}

\diffref{lex.name}
\change
Previously valid identifiers containing characters
not present in \UAX{44} properties XID_Start or XID_Continue, or
not in Normalization Form C, are now rejected.
\rationale
Prevent confusing characters in identifiers.
Requiring normalization of names ensures consistent linker behavior.
\effect
Some identifiers are no longer well-formed.

\diffref{lex.string}
\change
Concatenated \grammarterm{string-literal}s can no longer have
conflicting \grammarterm{encoding-prefix}es.
\rationale
Removal of unimplemented conditionally-supported feature.
\effect
Concatenation of \grammarterm{string-literal}s
with different \grammarterm{encoding-prefix}es
is now ill-formed.
For example:
\begin{codeblock}
auto c = L"a" U"b";             // was conditionally-supported; now ill-formed
\end{codeblock}

\rSec2[diff.cpp20.expr]{\ref{expr}: expressions}

\diffref{expr.prim.id.unqual}
\change
Change move-eligible \grammarterm{id-expression}s from lvalues to xvalues.
\rationale
Simplify the rules for implicit move.
\effect
Valid \CppXX{} code that relies on a returned \grammarterm{id-expression}'s
being an lvalue may change behavior or fail to compile.
For example:
\begin{codeblock}
decltype(auto) f(int&& x) { return (x); }       // returns \tcode{int\&\&}; previously returned \tcode{int\&}
int& g(int&& x) { return x; }                   // ill-formed; previously well-formed
\end{codeblock}

\diffref{expr.sub}
\change
Change the meaning of comma in subscript expressions.
\rationale
Enable repurposing a deprecated syntax to support multidimensional indexing.
\effect
Valid \CppXX{} code that uses a comma expression within a
subscript expression may fail to compile.
For example:
\begin{codeblock}
arr[1, 2]               // was equivalent to \tcode{arr[(1, 2)]},
                        // now equivalent to \tcode{arr.operator[](1, 2)} or ill-formed
\end{codeblock}

\rSec2[diff.cpp20.stmt]{\ref{stmt.stmt}: statements}

\diffref{stmt.ranged}
\change
The lifetime of temporary objects in the \grammarterm{for-range-initializer}
is extended until the end of the loop\iref{class.temporary}.
\rationale
Improve usability of the range-based \keyword{for} statement.
\effect
Destructors of some temporary objects are invoked later.
For example:
\begin{codeblock}
void f() {
  std::vector<int> v = { 42, 17, 13 };
  std::mutex m;

  for (int x :
       static_cast<void>(std::lock_guard<std::mutex>(m)), v) {  // lock released in \CppXX
    std::lock_guard<std::mutex> guard(m);                       // OK in \CppXX, now deadlocks
  }
}
\end{codeblock}

\rSec2[diff.cpp20.dcl]{\ref{dcl.dcl}: declarations}

\diffref{dcl.init.string}
\change
UTF-8 string literals may initialize arrays of \keyword{char} or
\tcode{\keyword{unsigned} \keyword{char}}.
\rationale
Compatibility with previously written code that conformed to previous versions of this document.
\effect
Arrays of \keyword{char} or \tcode{\keyword{unsigned} \keyword{char}}
may now be initialized with a UTF-8 string literal.
This can affect initialization that includes arrays
that are directly initialized within class types, typically aggregates.
For example:
\begin{codeblock}
struct A {
  char8_t s[10];
};
struct B {
  char s[10];
};

void f(A);
void f(B);

int main() {
  f({u8""});            // ambiguous
}
\end{codeblock}

\rSec2[diff.cpp20.temp]{\ref{temp}: templates}

\diffref{temp.deduct.type}
\change
Deducing template arguments from exception specifications.
\rationale
Facilitate generic handling of throwing and non-throwing functions.
\effect
Valid ISO \CppXX{} code may be ill-formed in this revision of \Cpp{}.
For example:
\begin{codeblock}
template<bool> struct A { };
template<bool B> void f(void (*)(A<B>) noexcept(B));
void g(A<false>) noexcept;
void h() {
  f(g);                         // ill-formed; previously well-formed
}
\end{codeblock}

\rSec2[diff.cpp20.library]{\ref{library}: library introduction}

\diffref{headers}
\change
New headers.
\rationale
New functionality.
\effect
The following \Cpp{} headers are new:
\libheaderref{expected},
\libheaderrefx{flat_map}{flat.map.syn},
\libheaderrefx{flat_set}{flat.set.syn},
\libheaderref{generator},
\libheaderref{mdspan},
\libheaderref{print},
\libheaderref{spanstream},
\libheaderref{stacktrace},
\libheaderref{stdatomic.h}, and
\libheaderref{stdfloat}.
Valid \CppXX{} code that \tcode{\#include}{s} headers with these names may be
invalid in this revision of \Cpp{}.

\rSec2[diff.cpp20.concepts]{\ref{concepts}: concepts library}

\diffref{cmp.concept,concept.equalitycomparable,concept.totallyordered}
\change
Replace \tcode{common_reference_with} in \tcode{three_way_comparable_with},
\tcode{equality_comparable_with}, and \tcode{totally_ordered_with}
with an exposition-only concept.
\rationale
Allow uncopyable, but movable, types to model these concepts.
\effect
Valid \CppXX{} code relying on subsumption
with \tcode{common_reference_with}
may fail to compile in this revision of \Cpp{}.
For example:
\begin{codeblock}
template<class T, class U>
  requires @\libconcept{equality_comparable_with}@<T, U>
bool attempted_equals(const T&, const U& u);    // previously selected overload

template<class T, class U>
  requires @\libconcept{common_reference_with}@<const remove_reference_t<T>&, const remove_reference_t<U>&>
bool attempted_equals(const T& t, const U& u);  // ambiguous overload; previously
                                                // rejected by partial ordering
bool test(shared_ptr<int> p) {
  return attempted_equals(p, nullptr);          // ill-formed; previously well-formed
}
\end{codeblock}

\rSec2[diff.cpp20.memory]{\ref{mem}: memory management library}

\diffref{allocator.traits.general}
\change
Forbid partial and explicit program-defined specializations
of \tcode{allocator_traits}.
\rationale
Allow addition of \tcode{allocate_at_least} to \tcode{allocator_traits},
and potentially other members in the future.
\effect
Valid \CppXX{} code
that partially or explicitly specializes \tcode{allocator_traits}
is ill-formed with no diagnostic required in this revision of \Cpp{}.

\rSec2[diff.cpp20.utilities]{\ref{utilities}: general utilities library}

\diffref{format}
\change
Signature changes: \tcode{format}, \tcode{format_to}, \tcode{vformat_to},
\tcode{format_to_n}, \tcode{formatted_size}.
Removal of \tcode{format_args_t}.
\rationale
Improve safety via compile-time format string checks,
avoid unnecessary template instantiations.
\effect
Valid \CppXX{} code that
contained errors in format strings or
relied on previous format string signatures or
\tcode{format_args_t} may become ill-formed.
For example:
\begin{codeblock}
auto s = std::format("{:d}", "I am not a number");      // ill-formed,
                                                        // previously threw \tcode{format_error}
\end{codeblock}

\diffref{format}
\change
Signature changes: \tcode{format}, \tcode{format_to}, \tcode{format_to_n},
\tcode{formatted_size}.
\rationale
Enable formatting of views
that do not support iteration when const-qualified and
that are not copyable.
\effect
Valid \CppXX{} code that passes bit-fields to formatting functions
may become ill-formed.
For example:
\begin{codeblock}
struct tiny {
  int bit: 1;
};

auto t = tiny();
std::format("{}", t.bit);       // ill-formed, previously returned \tcode{"0"}
\end{codeblock}

\diffref{format.string.std}
\change
Restrict types of formatting arguments
used as \fmtgrammarterm{width} or \fmtgrammarterm{precision} in
a \fmtgrammarterm{std-format-spec}.
\rationale
Disallow types that do not have useful or portable semantics as
a formatting width or precision.
\effect
Valid \CppXX{} code that passes a boolean or character type as
\fmtgrammarterm{arg-id} becomes invalid.
For example:
\begin{codeblock}
std::format("{:*^{}}", "", true);   // ill-formed, previously returned \tcode{"*"}
std::format("{:*^{}}", "", '1');    // ill-formed, previously returned an
                                    // implementation-defined number of \tcode{'*'} characters
\end{codeblock}

\diffref{format.formatter.spec}
\change
Removed the \tcode{formatter} specialization:
\begin{codeblock}
template<size_t N> struct formatter<const charT[N], charT>;
\end{codeblock}
\rationale
The specialization is inconsistent with the design of \tcode{formatter},
which is intended to be instantiated only with cv-unqualified object types.
\effect
Valid \CppXX{} code that instantiated the removed specialization
can become ill-formed.

\rSec2[diff.cpp20.strings]{\ref{strings}: strings library}

\diffref{string.classes}
\change
Additional rvalue overload for the \tcode{substr} member function and
the corresponding constructor.
\rationale
Improve efficiency of operations on rvalues.
\effect
Valid \CppXX{} code that created a substring
by calling \tcode{substr} (or the corresponding constructor)
on an xvalue expression with type \tcode{S}
that is a specialization of \tcode{basic_string}
may change meaning in this revision of \Cpp{}.
For example:
\begin{codeblock}
std::string s1 = "some long string that forces allocation", s2 = s1;
std::move(s1).substr(10, 5);
assert(s1 == s2);       // unspecified, previously guaranteed to be \tcode{true}
std::string s3(std::move(s2), 10, 5);
assert(s1 == s2);       // unspecified, previously guaranteed to be \tcode{true}
\end{codeblock}

\rSec2[diff.cpp20.containers]{\ref{containers}: containers library}

\diffref{associative.reqmts,unord.req}
\change
Heterogeneous \tcode{extract} and \tcode{erase} overloads
for associative containers.
\rationale
Improve efficiency of erasing elements from associative containers.
\effect
Valid \CppXX{} code may fail to compile in this revision of \Cpp{}.
For example:
\begin{codeblock}
struct B {
  auto operator<=>(const B&) const = default;
};

struct D : private B {
  void f(std::set<B, std::less<>>& s) {
    s.erase(*this);             // ill-formed; previously well-formed
  }
};
\end{codeblock}

\rSec2[diff.cpp20.thread]{\ref{thread}: concurrency support library}

\diffref{thread.barrier}
\change
In this revision of \Cpp{},
it is implementation-defined whether a barrier's phase completion step runs
if no thread calls \tcode{wait}.
Previously the phase completion step was guaranteed to run on the last thread that calls \tcode{arrive} or \tcode{arrive_and_drop} during the phase.
In this revision of \Cpp{},
it can run on any of the threads that arrived or waited at the barrier
during the phase.
\rationale
Correct contradictory wording and
improve implementation flexibility for performance.
\effect
Valid \CppXX{} code using a barrier might have
different semantics in this revision of \Cpp{}
if it depends on a completion function's side effects occurring exactly once,
on a specific thread running the phase completion step, or
on a completion function's side effects occurring
without \tcode{wait} having been called.
For example:
\begin{codeblock}
auto b0 = std::barrier(1);
b0.arrive();
b0.arrive();            // implementation-defined; previously well-defined

int data = 0;
auto b1 = std::barrier(1, [&] { data++; });
b1.arrive();
assert(data == 1);      // implementation-defined; previously well-defined
b1.arrive();            // implementation-defined; previously well-defined
\end{codeblock}

\rSec1[diff.cpp17]{\Cpp{} and ISO \CppXVII{}}

\rSec2[diff.cpp17.general]{General}

\pnum
\indextext{summary!compatibility with ISO \CppXVII{}}%
Subclause \ref{diff.cpp17} lists the differences between \Cpp{} and
ISO \CppXVII{} (ISO/IEC 14882:2017, \doccite{Programming Languages --- \Cpp{}}),
by the chapters of this document.

\rSec2[diff.cpp17.lex]{\ref{lex}: lexical conventions}

\diffref{lex.pptoken,module.unit,module.import,cpp.pre,cpp.module,cpp.import}
\change
New identifiers with special meaning.
\rationale
Required for new features.
\effect
Logical lines beginning with
\tcode{module} or \tcode{import} may
be interpreted differently
in this revision of \Cpp{}.
For example:
\begin{codeblock}
class module {};
module m1;          // was variable declaration; now \grammarterm{module-declaration}
module *m2;         // variable declaration

class import {};
import j1;          // was variable declaration; now \grammarterm{module-import-declaration}
::import j2;        // variable declaration
\end{codeblock}

\diffref{lex.header}
\change
\grammarterm{header-name} tokens are formed in more contexts.
\rationale
Required for new features.
\effect
When the identifier \tcode{import}
is followed by a \tcode{<} character,
a \grammarterm{header-name} token may be formed.
For example:
\begin{codeblock}
template<typename> class import {};
import<int> f();                // ill-formed; previously well-formed
::import<int> g();              // OK
\end{codeblock}

\diffref{lex.key}
\change
New keywords.
\rationale
Required for new features.
\begin{itemize}
\item
\indextext{UTF-8}%
The \keyword{char8_t} keyword is added to differentiate
the types of ordinary and UTF-8 literals\iref{lex.string}.
\item
The \tcode{concept} keyword is
added to enable the definition of concepts\iref{temp.concept}.
\item
The \keyword{consteval} keyword is added to
declare immediate functions\iref{dcl.constexpr}.
\item
The \keyword{constinit} keyword is added to
prevent unintended dynamic initialization\iref{dcl.constinit}.
\item
The \keyword{co_await}, \keyword{co_yield}, and \keyword{co_return} keywords are added
to enable the definition of coroutines \iref{dcl.fct.def.coroutine}.
\item
The \tcode{requires} keyword is added
to introduce constraints through a \grammarterm{requires-clause}\iref{temp.pre}
or a \grammarterm{requires-expression}\iref{expr.prim.req}.
\end{itemize}
\effectafteritemize
Valid \CppXVII{} code using
\keyword{char8_t},
\tcode{concept},
\keyword{consteval},
\keyword{constinit},
\keyword{co_await}, \keyword{co_yield}, \keyword{co_return},
or \tcode{requires}
as an identifier is not valid in this revision of \Cpp{}.

\diffref{lex.operators}
\change
New operator \tcode{<=>}.
\rationale
Necessary for new functionality.
\effect
Valid \CppXVII{} code that contains a \tcode{<=} token
immediately followed by a \tcode{>} token
may be ill-formed or have different semantics in this revision of \Cpp{}.
For example:
\begin{codeblock}
namespace N {
  struct X {};
  bool operator<=(X, X);
  template<bool(X, X)> struct Y {};
  Y<operator<=> y;              // ill-formed; previously well-formed
}
\end{codeblock}

\diffref{lex.literal}
\indextext{UTF-8}%
\change
Type of UTF-8 string and character literals.
\rationale
Required for new features.
The changed types enable function overloading, template specialization, and
type deduction to distinguish ordinary and UTF-8 string and character literals.
\effect
Valid \CppXVII{} code that depends on
UTF-8 string literals having type ``array of \tcode{const char}'' and
UTF-8 character literals having type ``\tcode{char}''
is not valid in this revision of \Cpp{}.
For example:
\begin{codeblock}
const auto *u8s = u8"text";     // \tcode{u8s} previously deduced as \tcode{const char*}; now deduced as \tcode{const char8_t*}
const char *ps = u8s;           // ill-formed; previously well-formed

auto u8c = u8'c';               // \tcode{u8c} previously deduced as \tcode{char}; now deduced as \keyword{char8_t}
char *pc = &u8c;                // ill-formed; previously well-formed

std::string s = u8"text";       // ill-formed; previously well-formed

void f(const char *s);
f(u8"text");                    // ill-formed; previously well-formed

template<typename> struct ct;
template<> struct ct<char> {
  using type = char;
};
ct<decltype(u8'c')>::type x;    // ill-formed; previously well-formed.
\end{codeblock}

\rSec2[diff.cpp17.basic]{\ref{basic}: basics}

\diffref{basic.life}
\change
A pseudo-destructor call ends the lifetime of
the object to which it is applied.
\rationale
Increase consistency of the language model.
\effect
Valid ISO \CppXVII{} code may be ill-formed or
have undefined behavior in this revision of \Cpp{}.
For example:
\begin{codeblock}
int f() {
  int a = 123;
  using T = int;
  a.~T();
  return a;         // undefined behavior; previously returned 123
}
\end{codeblock}

\diffref{intro.races}
\change
Except for the initial release operation,
a release sequence consists solely of atomic read-modify-write operations.
\rationale
Removal of rarely used and confusing feature.
\effect
If a \tcode{memory_order_release} atomic store is followed
by a \tcode{memory_order_relaxed} store to the same variable by the same thread,
then reading the latter value with a \tcode{memory_order_acquire} load
no longer provides any ``happens before'' guarantees,
even in the absence of intervening stores by another thread.

\rSec2[diff.cpp17.expr]{\ref{expr}: expressions}

\diffref{expr.prim.lambda.capture}
\change
Implicit lambda capture may capture additional entities.
\rationale
Rule simplification, necessary to resolve interactions with constexpr if.
\effect
Lambdas with a \grammarterm{capture-default}
may capture local entities
that were not captured in \CppXVII{}
if those entities are only referenced in contexts
that do not result in an odr-use.

\rSec2[diff.cpp17.dcl.dcl]{\ref{dcl.dcl}: declarations}

\diffref{dcl.typedef}
\change
Unnamed classes with a typedef name for linkage purposes
can contain only C-compatible constructs.
\rationale
Necessary for implementability.
\effect
Valid \CppXVII{} code may be ill-formed in this revision of \Cpp{}.
For example:
\begin{codeblock}
typedef struct {
  void f() {}           // ill-formed; previously well-formed
} S;
\end{codeblock}

\diffref{dcl.fct.default}
\change
A function cannot have different default arguments
in different translation units.
\rationale
Required for modules support.
\effect
Valid \CppXVII{} code may be ill-formed in this revision of \Cpp{},
with no diagnostic required.
For example:
\begin{codeblock}
// Translation unit 1
int f(int a = 42);
int g() { return f(); }

// Translation unit 2
int f(int a = 76) { return a; }         // ill-formed, no diagnostic required; previously well-formed
int g();
int main() { return g(); }              // used to return 42
\end{codeblock}

\diffref{dcl.init.aggr}
\change
A class that has user-declared constructors is never an aggregate.
\rationale
Remove potentially error-prone aggregate initialization
which may apply notwithstanding the declared constructors of a class.
\effect
Valid \CppXVII{} code that aggregate-initializes
a type with a user-declared constructor
may be ill-formed or have different semantics
in this revision of \Cpp{}.
For example:
\begin{codeblock}
struct A {              // not an aggregate; previously an aggregate
  A() = delete;
};

struct B {              // not an aggregate; previously an aggregate
  B() = default;
  int i = 0;
};

struct C {              // not an aggregate; previously an aggregate
  C(C&&) = default;
  int a, b;
};

A a{};                  // ill-formed; previously well-formed
B b = {1};              // ill-formed; previously well-formed
auto* c = new C{2, 3};  // ill-formed; previously well-formed

struct Y;

struct X {
  operator Y();
};

struct Y {              // not an aggregate; previously an aggregate
  Y(const Y&) = default;
  X x;
};

Y y{X{}};               // copy constructor call; previously aggregate-initialization
\end{codeblock}

\diffref{dcl.init.list}
\change
Boolean conversion from a pointer or pointer-to-member type
is now a narrowing conversion.
\rationale
Catches bugs.
\effect
Valid \CppXVII{} code may fail to compile
in this revision of \Cpp{}.
For example:
\begin{codeblock}
bool y[] = { "bc" };    // ill-formed; previously well-formed
\end{codeblock}

\rSec2[diff.cpp17.class]{\ref{class}: classes}

\diffref{class.ctor,class.conv.fct}
\change
The class name can no longer be used parenthesized
immediately after an \keyword{explicit} \grammarterm{decl-specifier}
in a constructor declaration.
The \grammarterm{conversion-function-id} can no longer be used parenthesized
immediately after an \keyword{explicit} \grammarterm{decl-specifier}
in a conversion function declaration.
\rationale
Necessary for new functionality.
\effect
Valid \CppXVII{} code may fail to compile
in this revision of \Cpp{}.
For example:
\begin{codeblock}
struct S {
  explicit (S)(const S&);       // ill-formed; previously well-formed
  explicit (operator int)();    // ill-formed; previously well-formed
  explicit(true) (S)(int);      // OK
};
\end{codeblock}

\diffref{class.ctor,class.dtor}
\change
A \grammarterm{simple-template-id}
is no longer valid as the \grammarterm{declarator-id} of a constructor or destructor.
\rationale
Remove potentially error-prone option for redundancy.
\effect
Valid \CppXVII{} code may fail to compile
in this revision of \Cpp{}.
For example:
\begin{codeblock}
template<class T>
struct A {
  A<T>();           // error: \grammarterm{simple-template-id} not allowed for constructor
  A(int);           // OK, injected-class-name used
  ~A<T>();          // error: \grammarterm{simple-template-id} not allowed for destructor
};
\end{codeblock}

\diffref{class.copy.elision}
\change
A function returning an implicitly movable entity
may invoke a constructor taking an rvalue reference to a type
different from that of the returned expression.
Function and catch-clause parameters can be thrown using move constructors.
\rationale
Side effect of making it easier to write
more efficient code that takes advantage of moves.
\effect
Valid \CppXVII{} code may fail to compile or have different semantics
in this revision of \Cpp{}.
For example:
\begin{codeblock}
struct base {
  base();
  base(base const &);
private:
  base(base &&);
};

struct derived : base {};

base f(base b) {
  throw b;                      // error: \tcode{base(base \&\&)} is private
  derived d;
  return d;                     // error: \tcode{base(base \&\&)} is private
}

struct S {
  S(const char *s) : m(s) { }
  S(const S&) = default;
  S(S&& other) : m(other.m) { other.m = nullptr; }
  const char * m;
};

S consume(S&& s) { return s; }

void g() {
  S s("text");
  consume(static_cast<S&&>(s));
  char c = *s.m;                // undefined behavior; previously ok
}
\end{codeblock}

\rSec2[diff.cpp17.over]{\ref{over}: overloading}

\diffref{over.match.oper}
\change
Equality and inequality expressions can now find
reversed and rewritten candidates.
\rationale
Improve consistency of equality with three-way comparison
and make it easier to write the full complement of equality operations.
\effect
For certain pairs of types where one is convertible to the other,
equality or inequality expressions between an object of one type
and an object of the other type invoke a different operator.
Also, for certain types, equality or inequality expressions
between two objects of that type become ambiguous.
For example:
\begin{codeblock}
struct A {
  operator int() const;
};

bool operator==(A, int);        // \#1
// \#2 is built-in candidate: \tcode{bool operator==(int, int);}
// \#3 is built-in candidate: \tcode{bool operator!=(int, int);}

int check(A x, A y) {
  return (x == y) +             // ill-formed; previously well-formed
    (10 == x) +                 // calls \#1, previously selected \#2
    (10 != x);                  // calls \#1, previously selected \#3
}
\end{codeblock}

\diffref{over.match.oper}
\change
Overload resolution may change for equality operators\iref{expr.eq}.
\rationale
Support calling \tcode{operator==} with reversed order of arguments.
\effect
Valid \CppXVII{} code that uses equality operators with conversion functions
may be ill-formed or have different semantics in this revision of \Cpp{}.
For example:
\begin{codeblock}
struct A {
  operator int() const { return 10; }
};

bool operator==(A, int);        // \#1
// \#2 is built-in candidate: \tcode{bool operator==(int, int);}
bool b = 10 == A();             // calls \#1 with reversed order of arguments; previously selected \#2

struct B {
  bool operator==(const B&);    // member function with no cv-qualifier
};
B b1;
bool eq = (b1 == b1);           // ambiguous; previously well-formed
\end{codeblock}

\rSec2[diff.cpp17.temp]{\ref{temp}: templates}

\diffref{temp.names}
\change
An \grammarterm{unqualified-id}
that is followed by a \tcode{<}
and for which name lookup
finds nothing or finds a function
will be treated as a \grammarterm{template-name}
in order to potentially cause argument-dependent lookup to be performed.
\rationale
It was problematic to call a function template
with an explicit template argument list
via argument-dependent lookup
because of the need to have a template with the same name
visible via normal lookup.
\effect
Previously valid code that uses a function name
as the left operand of a \tcode{<} operator
would become ill-formed.
For example:
\begin{codeblock}
struct A {};
bool operator<(void (*fp)(), A);
void f() {}
int main() {
  A a;
  f < a;    // ill-formed; previously well-formed
  (f) < a;  // still well-formed
}
\end{codeblock}

\rSec2[diff.cpp17.except]{\ref{except}: exception handling}

\diffref{except.spec}
\change
Remove \tcode{throw()} exception specification.
\rationale
Removal of obsolete feature that has been replaced by \keyword{noexcept}.
\effect
A valid \CppXVII{} function declaration, member function declaration, function
pointer declaration, or function reference declaration that uses \tcode{throw()}
for its exception specification will be rejected as ill-formed in this
revision of \Cpp{}. It should simply be replaced with \keyword{noexcept} for no
change of meaning since \CppXVII{}.
\begin{note}
There is no way to write a function declaration
that is non-throwing in this revision of \Cpp{}
and is also non-throwing in \CppIII{}
except by using the preprocessor to generate
a different token sequence in each case.
\end{note}

\rSec2[diff.cpp17.library]{\ref{library}: library introduction}

\diffref{headers}
\change
New headers.
\rationale
New functionality.
\effect
The following \Cpp{} headers are new:
\libheaderref{barrier},
\libheaderref{bit},
\libheaderref{charconv},
\libheaderref{compare},
\libheaderref{concepts},
\libheaderref{coroutine},
\libheaderref{format},
\libheaderref{latch},
\libheaderref{numbers},
\libheaderref{ranges},
\libheaderref{semaphore},
\libheaderrefx{source_location}{source.location.syn},
\libheaderref{span},
\libheaderrefx{stop_token}{thread.stoptoken.syn},
\libheaderref{syncstream}, and
\libheaderrefx{version}{support.limits.general}.
Valid \CppXVII{} code that \tcode{\#include}{s} headers with these names may be
invalid in this revision of \Cpp{}.

\diffref{headers}
\change
Remove vacuous \Cpp{} header files.
\rationale
The empty headers implied a false requirement to achieve C compatibility with the \Cpp{} headers.
\effect
A valid \CppXVII{} program that \tcode{\#include}{s} any of the following headers may fail to compile:
\libnoheader{ccomplex},
\libnoheader{ciso646},
\libnoheader{cstdalign},
\libnoheader{cstdbool}, and
\libnoheader{ctgmath}.
To retain the same behavior:
\begin{itemize}
\item
a \tcode{\#include} of \libnoheader{ccomplex} can be replaced by
a \tcode{\#include} of \libheaderref{complex},
\item
a \tcode{\#include} of \libnoheader{ctgmath} can be replaced by
a \tcode{\#include} of \libheaderref{cmath} and
a \tcode{\#include} of \libheader{complex},
and
\item
a \tcode{\#include} of
\libnoheader{ciso646},
\libnoheader{cstdalign}, or
\libnoheader{cstdbool}
can simply be removed.
\end{itemize}

\rSec2[diff.cpp17.containers]{\ref{containers}: containers library}

\diffref{forward.list,list}
\change
Return types of \tcode{remove}, \tcode{remove_if}, and \tcode{unique}
changed from \keyword{void} to \tcode{container::size_type}.
\rationale
Improve efficiency and convenience of finding number of removed elements.
\effect
Code that depends on the return types might have different semantics in this revision of \Cpp{}.
Translation units compiled against this version of \Cpp{} may be incompatible with
translation units compiled against \CppXVII{}, either failing to link or having undefined behavior.

\rSec2[diff.cpp17.iterators]{\ref{iterators}: iterators library}

\diffref{iterator.traits}
\change
The specialization of \tcode{iterator_traits} for \tcode{void*} and
for function pointer types no longer contains any nested typedefs.
\rationale
Corrects an issue misidentifying pointer types that are not incrementable
as iterator types.
\effect
A valid \CppXVII{} program that relies on the presence of the typedefs
may fail to compile, or have different behavior.

\rSec2[diff.cpp17.alg.reqs]{\ref{algorithms}: algorithms library}

\diffref{algorithms.requirements}
\change
The number and order of deducible template parameters for algorithm declarations
is now unspecified, instead of being as-declared.
\rationale
Increase implementor freedom and allow some function templates
to be implemented as function objects with templated call operators.
\effect
A valid \CppXVII{} program that passes explicit template arguments to
algorithms not explicitly specified to allow such in this version of \Cpp{}
may fail to compile or have undefined behavior.

\rSec2[diff.cpp17.input.output]{\ref{input.output}: input/output library}

\diffref{istream.extractors}
\change
Character array extraction only takes array types.
\rationale
Increase safety via preventing buffer overflow at compile time.
\effect
Valid \CppXVII{} code may fail to compile in this revision of \Cpp{}.
For example:
\begin{codeblock}
auto p = new char[100];
char q[100];
std::cin >> std::setw(20) >> p;         // ill-formed; previously well-formed
std::cin >> std::setw(20) >> q;         // OK
\end{codeblock}

\diffref{ostream.inserters.character}
\indextext{UTF-8}%
\change
Overload resolution for ostream inserters used with UTF-8 literals.
\rationale
Required for new features.
\effect
Valid \CppXVII{} code that passes UTF-8 literals to
\tcode{basic_ostream<char, ...>::operator<<} or
\tcode{basic_ostream<wchar_t, ...>::operator<<} is now ill-formed.
For example:
\begin{codeblock}
std::cout << u8"text";          // previously called \tcode{operator<<(const char*)} and printed a string;
                                // now ill-formed
std::cout << u8'X';             // previously called \tcode{operator<<(char)} and printed a character;
                                // now ill-formed
\end{codeblock}

\diffref{ostream.inserters.character}
\change
Overload resolution for ostream inserters
used with \keyword{wchar_t}, \keyword{char16_t}, or \keyword{char32_t} types.
\rationale
Removal of surprising behavior.
\effect
Valid \CppXVII{} code that passes
\keyword{wchar_t}, \keyword{char16_t}, or \keyword{char32_t} characters or strings
to \tcode{basic_ostream<char, ...>::operator<<} or
that passes \keyword{char16_t} or \keyword{char32_t} characters or strings
to \tcode{basic_ostream<wchar_t, ...>::operator<<} is now ill-formed.
For example:
\begin{codeblock}
std::cout << u"text";           // previously formatted the string as a pointer value;
                                // now ill-formed
std::cout << u'X';              // previously formatted the character as an integer value;
                                // now ill-formed
\end{codeblock}

\diffref{fs.class.path}
\change
Return type of filesystem path format observer member functions.
\rationale
Required for new features.
\effect
Valid \CppXVII{} code that depends on the \tcode{u8string()} and
\tcode{generic_u8string()} member functions of \tcode{std::filesystem::path}
returning \tcode{std::string} is not valid in this revision of \Cpp{}.
For example:
\begin{codeblock}
std::filesystem::path p;
std::string s1 = p.u8string();          // ill-formed; previously well-formed
std::string s2 = p.generic_u8string();  // ill-formed; previously well-formed
\end{codeblock}

\rSec2[diff.cpp17.depr]{\ref{depr}: compatibility features}

\nodiffref
\change
Remove \tcode{uncaught_exception}.
\rationale
The function did not have a clear specification when multiple exceptions were
active, and has been superseded by \tcode{uncaught_exceptions}.
\effect
A valid \CppXVII{} program that calls \tcode{std::uncaught_exception} may fail
to compile. It can be revised to use \tcode{std::uncaught_exceptions} instead,
for clear and portable semantics.

\nodiffref
\change
Remove support for adaptable function API.
\rationale
The deprecated support relied on a limited convention that could not be
extended to support the general case or new language features. It has been
superseded by direct language support with \keyword{decltype}, and by the
\tcode{std::bind} and \tcode{std::not_fn} function templates.
\effect
A valid \CppXVII{} program that relies on the presence of \tcode{result_type},
\tcode{argument_type}, \tcode{first_argument_type}, or
\tcode{second_argument_type} in a standard library class may fail to compile. A
valid \CppXVII{} program that calls \tcode{not1} or \tcode{not2}, or uses the
class templates \tcode{unary_negate} or \tcode{binary_negate}, may fail to
compile.

\nodiffref
\change
Remove redundant members from \tcode{std::allocator}.
\rationale
\tcode{std::allocator} was overspecified, encouraging direct usage in user containers
rather than relying on \tcode{std::allocator_traits}, leading to poor containers.
\effect
A valid \CppXVII{} program that directly makes use of the \tcode{pointer},
\tcode{const_pointer}, \tcode{reference}, \tcode{const_reference},
\tcode{rebind}, \tcode{address}, \tcode{construct}, \tcode{destroy}, or
\tcode{max_size} members of \tcode{std::allocator}, or that directly calls
\tcode{allocate} with an additional hint argument, may fail to compile.

\nodiffref
\change
Remove \tcode{raw_storage_iterator}.
\rationale
The iterator encouraged use of potentially-throwing algorithms, but did
not return the number of elements successfully constructed,
as would be necessary to destroy them.
\effect
A valid \CppXVII{} program that uses this iterator class may fail to compile.

\nodiffref
\change
Remove temporary buffers API.
\rationale
The temporary buffer facility was intended to provide an efficient optimization
for small memory requests, but there is little evidence this was achieved in
practice, while requiring the user to provide their own exception-safe wrappers
to guard use of the facility in many cases.
\effect
A valid \CppXVII{} program that calls \tcode{get_temporary_buffer} or
\tcode{return_temporary_buffer} may fail to compile.

\nodiffref
\change
Remove \tcode{shared_ptr::unique}.
\rationale
The result of a call to this member function is not reliable in the presence of
multiple threads and weak pointers. The member function \tcode{use_count} is
similarly unreliable, but has a clearer contract in such cases, and remains
available for well-defined use in single-threaded cases.
\effect
A valid \CppXVII{} program that calls \tcode{unique} on a \tcode{shared_ptr}
object may fail to compile.

\diffref{depr.meta.types}
\change
Remove deprecated type traits.
\rationale
The traits had unreliable or awkward interfaces. The \tcode{is_literal_type}
trait provided no way to detect which subset of constructors and member
functions of a type were declared \keyword{constexpr}. The \tcode{result_of}
trait had a surprising syntax that did not directly support function types.
It has been superseded by the \tcode{invoke_result} trait.
\effect
A valid \CppXVII{} program that relies on the \tcode{is_literal_type} or
\tcode{result_of} type traits, on the \tcode{is_literal_type_v} variable template,
or on the \tcode{result_of_t} alias template may fail to compile.

\rSec1[diff.cpp14]{\Cpp{} and ISO \CppXIV{}}

\rSec2[diff.cpp14.general]{General}

\pnum
\indextext{summary!compatibility with ISO \CppXIV{}}%
Subclause \ref{diff.cpp14} lists the differences between \Cpp{} and
ISO \CppXIV{} (ISO/IEC 14882:2014, \doccite{Programming Languages --- \Cpp{}}),
in addition to those listed above,
by the chapters of this document.

\rSec2[diff.cpp14.lex]{\ref{lex}: lexical conventions}

\diffref{lex.phases}
\indextext{trigraph sequence}%
\change
Removal of trigraph support as a required feature.
\rationale
Prevents accidental uses of trigraphs in non-raw string literals and comments.
\effect
Valid \CppXIV{} code that uses trigraphs may not be valid or may have different
semantics in this revision of \Cpp{}. Implementations may choose to
translate trigraphs as specified in \CppXIV{} if they appear outside of a raw
string literal, as part of the \impldef{mapping input source file characters
to translation character set} mapping from input source file characters to
the translation character set.

\diffref{lex.ppnumber}
\change
\grammarterm{pp-number} can contain \tcode{p} \grammarterm{sign} and
\tcode{P} \grammarterm{sign}.
\rationale
Necessary to enable \grammarterm{hexadecimal-floating-point-literal}s.
\effect
Valid \CppXIV{} code may fail to compile or produce different results in
this revision of \Cpp{}. Specifically, character sequences like \tcode{0p+0}
and \tcode{0e1_p+0} are three separate tokens each in \CppXIV{}, but one single token
in this revision of \Cpp{}.
For example:
\begin{codeblock}
#define F(a) b ## a
int b0p = F(0p+0);  // ill-formed; equivalent to ``\tcode{int b0p = b0p + 0;}\!'' in \CppXIV{}
\end{codeblock}

\rSec2[diff.cpp14.expr]{\ref{expr}: expressions}

\diffref{expr.post.incr,expr.pre.incr}
\change
Remove increment operator with \tcode{bool} operand.
\rationale
Obsolete feature with occasionally surprising semantics.
\effect
A valid \CppXIV{} expression utilizing the increment operator on
a \tcode{bool} lvalue is ill-formed in this revision of \Cpp{}.

\diffref{expr.new,expr.delete}
\change
Dynamic allocation mechanism for over-aligned types.
\rationale
Simplify use of over-aligned types.
\effect
In \CppXIV{} code that uses a \grammarterm{new-expression}
to allocate an object with an over-aligned class type,
where that class has no allocation functions of its own,
\tcode{::operator new(std::size_t)}
is used to allocate the memory.
In this revision of \Cpp{},
\tcode{::operator new(std::size_t, std::align_val_t)}
is used instead.

\rSec2[diff.cpp14.dcl.dcl]{\ref{dcl.dcl}: declarations}

\diffref{dcl.stc}
\indextext{\idxcode{register} storage class}%
\change
Removal of \keyword{register} \grammarterm{storage-class-specifier}.
\rationale
Enable repurposing of deprecated keyword in future revisions of \Cpp{}.
\effect
A valid \CppXIV{} declaration utilizing the \keyword{register}
\grammarterm{storage-class-specifier} is ill-formed in this revision of \Cpp{}.
The specifier can simply be removed to retain the original meaning.

\diffref{dcl.spec.auto}
\change
\keyword{auto} deduction from \grammarterm{braced-init-list}.
\rationale
More intuitive deduction behavior.
\effect
Valid \CppXIV{} code may fail to compile or may change meaning
in this revision of \Cpp{}.
For example:
\begin{codeblock}
auto x1{1};         // was \tcode{std::initializer_list<int>}, now \tcode{int}
auto x2{1, 2};      // was \tcode{std::initializer_list<int>}, now ill-formed
\end{codeblock}

\diffref{dcl.fct}
\change
Make exception specifications be part of the type system.
\rationale
Improve type-safety.
\effect
Valid \CppXIV{} code may fail to compile or change meaning in this
revision of \Cpp{}.
For example:
\begin{codeblock}
void g1() noexcept;
void g2();
template<class T> int f(T *, T *);
int x = f(g1, g2);              // ill-formed; previously well-formed
\end{codeblock}

\diffref{dcl.init.aggr}
\change
Definition of an aggregate is extended
to apply to user-defined types with base classes.
\rationale
To increase convenience of aggregate initialization.
\effect
Valid \CppXIV{} code may fail to compile or produce different results in this
revision of \Cpp{}; initialization from an empty initializer list will
perform aggregate initialization instead of invoking a default constructor
for the affected types.
For example:
\begin{codeblock}
struct derived;
struct base {
  friend struct derived;
private:
  base();
};
struct derived : base {};

derived d1{};       // error; the code was well-formed in \CppXIV{}
derived d2;         // still OK
\end{codeblock}

\rSec2[diff.cpp14.class]{\ref{class}: classes}

\diffref{class.inhctor.init}
\change
Inheriting a constructor no longer injects a constructor into the derived class.
\rationale
Better interaction with other language features.
\effect
Valid \CppXIV{} code that uses inheriting constructors may not be valid
or may have different semantics. A \grammarterm{using-declaration}
that names a constructor now makes the corresponding base class constructors
visible to initializations of the derived class
rather than declaring additional derived class constructors.
For example:
\begin{codeblock}
struct A {
  template<typename T> A(T, typename T::type = 0);
  A(int);
};
struct B : A {
  using A::A;
  B(int);
};
B b(42L);           // now calls \tcode{B(int)}, used to call \tcode{B<long>(long)},
                    // which called \tcode{A(int)} due to substitution failure
                    // in \tcode{A<long>(long)}.
\end{codeblock}

\rSec2[diff.cpp14.temp]{\ref{temp}: templates}

\diffref{temp.deduct.type}
\change
Allowance to deduce from the type of a non-type template argument.
\rationale
In combination with the ability to declare
non-type template arguments with placeholder types,
allows partial specializations to decompose
from the type deduced for the non-type template argument.
\effect
Valid \CppXIV{} code may fail to compile
or produce different results in this revision of \Cpp{}.
For example:
\begin{codeblock}
template <int N> struct A;
template <typename T, T N> int foo(A<N> *) = delete;
void foo(void *);
void bar(A<0> *p) {
  foo(p);           // ill-formed; previously well-formed
}
\end{codeblock}

\rSec2[diff.cpp14.except]{\ref{except}: exception handling}

\diffref{except.spec}
\change
Remove dynamic exception specifications.
\rationale
Dynamic exception specifications were a deprecated feature
that was complex and brittle in use.
They interacted badly with the type system,
which became a more significant issue in this revision of \Cpp{}
where (non-dynamic) exception specifications are part of the function type.
\effect
A valid \CppXIV{} function declaration,
member function declaration,
function pointer declaration,
or function reference declaration,
if it has a potentially throwing dynamic exception specification,
is rejected as ill-formed in this revision of \Cpp{}.
Violating a non-throwing dynamic exception specification
calls \tcode{terminate} rather than \tcode{unexpected},
and it is unspecified whether stack unwinding is performed
prior to such a call.

\rSec2[diff.cpp14.library]{\ref{library}: library introduction}

\diffref{headers}
\change
New headers.
\rationale
New functionality.
\effect
The following \Cpp{} headers are new:
\libheaderrefx{any}{any.synop},
\libheaderref{charconv},
\libheaderref{execution},
\libheaderrefx{filesystem}{fs.filesystem.syn},
\libheaderrefx{memory_resource}{mem.res.syn},
\libheaderref{optional},\\
\libheaderrefx{string_view}{string.view.synop},
and
\libheaderref{variant}.
Valid \CppXIV{} code that \tcode{\#include}{s} headers with these names may be
invalid in this revision of \Cpp{}.

\diffref{namespace.future}
\change
New reserved namespaces.
\rationale
Reserve namespaces for future revisions of the standard library
that might otherwise be incompatible with existing programs.
\effect
The global namespaces \tcode{std}
followed by an arbitrary sequence of \grammarterm{digit}{s}\iref{lex.name}
are reserved for future standardization.
Valid \CppXIV{} code that uses such a top-level namespace,
e.g., \tcode{std2}, may be invalid in this revision of \Cpp{}.

\rSec2[diff.cpp14.utilities]{\ref{utilities}: general utilities library}

\diffref{func.wrap}
\change
Constructors taking allocators removed.
\rationale
No implementation consensus.
\effect
Valid \CppXIV{} code may fail to compile or may change meaning in this
revision of \Cpp{}. Specifically, constructing a \tcode{std::function} with
an allocator is ill-formed and uses-allocator construction will not pass an
allocator to \tcode{std::function} constructors in this revision of \Cpp{}.

\diffref{util.smartptr.shared}
\change
Different constraint on conversions from \tcode{unique_ptr}.
\rationale
Adding array support to \tcode{shared_ptr},
via the syntax \tcode{shared_ptr<T[]>} and \tcode{shared_ptr<T[N]>}.
\effect
Valid \CppXIV{} code may fail to compile or may change meaning in this
revision of \Cpp{}.
For example:
\begin{codeblock}
#include <memory>
std::unique_ptr<int[]> arr(new int[1]);
std::shared_ptr<int> ptr(std::move(arr));   // error: \tcode{int(*)[]} is not compatible with \tcode{int*}
\end{codeblock}

\rSec2[diff.cpp14.string]{\ref{strings}: strings library}

\diffref{basic.string}
\change
Non-const \tcode{.data()} member added.
\rationale
The lack of a non-const \tcode{.data()}
differed from the similar member of \tcode{std::vector}.
This change regularizes behavior.
\effect
Overloaded functions which have differing code paths
for \tcode{char*} and \tcode{const char*} arguments
will execute differently
when called with a non-const string's \tcode{.data()} member
in this revision of \Cpp{}.
For example:
\begin{codeblock}
int f(char *) = delete;
int f(const char *);
string s;
int x = f(s.data());            // ill-formed; previously well-formed
\end{codeblock}

\rSec2[diff.cpp14.containers]{\ref{containers}: containers library}

\diffref{associative.reqmts}
\change
Requirements change:
\rationale
Increase portability, clarification of associative container requirements.
\effect
Valid \CppXIV{} code that attempts to use associative containers
having a comparison object with non-const function call operator
may fail to compile in this revision of \Cpp{}.
For example:
\begin{codeblock}
#include <set>

struct compare
{
  bool operator()(int a, int b)
  {
    return a < b;
  }
};

int main() {
  const std::set<int, compare> s;
  s.find(0);
}
\end{codeblock}

\rSec2[diff.cpp14.depr]{\ref{depr}: compatibility features}

\nodiffref
\change
The class templates
\tcode{auto_ptr},
\tcode{unary_function}, and
\tcode{binary_function},
the function templates
\tcode{random_shuffle},
and the function templates (and their return types)
\tcode{ptr_fun},
\tcode{mem_fun},
\tcode{mem_fun_ref},
\tcode{bind1st}, and
\tcode{bind2nd}
are not defined.
\rationale
Superseded by new features.
\effect
Valid \CppXIV{} code that uses these class templates
and function templates may fail to compile in this revision of \Cpp{}.

\nodiffref
\change
Remove old iostreams members [depr.ios.members].
\rationale
Redundant feature for compatibility with pre-standard code
has served its time.
\effect
A valid \CppXIV{} program using these identifiers
may be ill-formed in this revision of \Cpp{}.

\rSec1[diff.cpp11]{\Cpp{} and ISO \CppXI{}}

\rSec2[diff.cpp11.general]{General}

\pnum
\indextext{summary!compatibility with ISO \CppXI{}}%
Subclause \ref{diff.cpp11} lists the differences between \Cpp{} and
ISO \CppXI{} (ISO/IEC 14882:2011, \doccite{Programming Languages --- \Cpp{}}),
in addition to those listed above,
by the chapters of this document.

\rSec2[diff.cpp11.lex]{\ref{lex}: lexical conventions}

\diffref{lex.ppnumber}
\change
\grammarterm{pp-number} can contain one or more single quotes.
\rationale
Necessary to enable single quotes as digit separators.
\effect
Valid \CppXI{} code may fail to compile or may change meaning in this
revision of \Cpp{}. For example, the following code is valid both in \CppXI{} and in
this revision of \Cpp{}, but the macro invocation produces different outcomes
because the single quotes delimit a \grammarterm{character-literal} in \CppXI{}, whereas they are digit
separators in this revision of \Cpp{}.
For example:
\begin{codeblock}
#define M(x, ...) __VA_ARGS__
int x[2] = { M(1'2,3'4, 5) };
// \tcode{int x[2] = \{ 5 \};\ \ \ \ \ } --- \CppXI{}
// \tcode{int x[2] = \{ 3'4, 5 \};} --- this revision of \Cpp{}
\end{codeblock}

\rSec2[diff.cpp11.basic]{\ref{basic}: basics}

\diffref{basic.stc.dynamic.deallocation}
\change
New usual (non-placement) deallocator.
\rationale
Required for sized deallocation.
\effect
Valid \CppXI{} code can declare a global placement allocation function and
deallocation function as follows:
\begin{codeblock}
void* operator new(std::size_t, std::size_t);
void operator delete(void*, std::size_t) noexcept;
\end{codeblock}
In this revision of \Cpp{}, however, the declaration of \tcode{operator delete}
might match a predefined usual (non-placement)
\tcode{operator delete}\iref{basic.stc.dynamic}. If so, the
program is ill-formed, as it was for class member allocation functions and
deallocation functions\iref{expr.new}.

\rSec2[diff.cpp11.expr]{\ref{expr}: expressions}

\diffref{expr.cond}
\change
A conditional expression with a throw expression as its second or third
operand keeps the type and value category of the other operand.
\rationale
Formerly mandated conversions (lvalue-to-rvalue\iref{conv.lval},
array-to-pointer\iref{conv.array}, and function-to-pointer\iref{conv.func}
standard conversions), especially the creation of the temporary due to
lvalue-to-rvalue conversion, were considered gratuitous and surprising.
\effect
Valid \CppXI{} code that relies on the conversions may behave differently
in this revision of \Cpp{}.
For example:
\begin{codeblock}
struct S {
  int x = 1;
  void mf() { x = 2; }
};
int f(bool cond) {
  S s;
  (cond ? s : throw 0).mf();
  return s.x;
}
\end{codeblock}
In \CppXI{}, \tcode{f(true)} returns \tcode{1}. In this revision of \Cpp{},
it returns \tcode{2}.
\begin{codeblock}
sizeof(true ? "" : throw 0)
\end{codeblock}
In \CppXI{}, the expression yields \tcode{sizeof(const char*)}. In this
revision of \Cpp{}, it yields \tcode{sizeof(const char[1])}.

\rSec2[diff.cpp11.dcl.dcl]{\ref{dcl.dcl}: declarations}

\diffref{dcl.constexpr}
\change
\keyword{constexpr} non-static member functions are not implicitly
\keyword{const} member functions.
\rationale
Necessary to allow \keyword{constexpr} member functions to mutate
the object.
\effect
Valid \CppXI{} code may fail to compile in this revision of \Cpp{}.
For example:
\begin{codeblock}
struct S {
  constexpr const int &f();
  int &f();
};
\end{codeblock}
This code is valid in \CppXI{}
but invalid in this revision of \Cpp{} because it declares the same member
function twice with different return types.

\diffref{dcl.init.aggr}
\change
Classes with default member initializers can be aggregates.
\rationale
Necessary to allow default member initializers to be used
by aggregate initialization.
\effect
Valid \CppXI{} code may fail to compile or may change meaning in this revision of \Cpp{}.
For example:
\begin{codeblock}
struct S {          // Aggregate in \CppXIV{} onwards.
  int m = 1;
};
struct X {
  operator int();
  operator S();
};
X a{};
S b{a};             // uses copy constructor in \CppXI{},
                    // performs aggregate initialization in this revision of \Cpp{}
\end{codeblock}

\rSec2[diff.cpp11.library]{\ref{library}: library introduction}

\diffref{headers}
\change
New header.
\rationale
New functionality.
\effect
The \Cpp{} header \libheaderrefx{shared_mutex}{shared.mutex.syn} is new.
Valid \CppXI{} code that \tcode{\#include}{s} a header with that name may be
invalid in this revision of \Cpp{}.

\rSec2[diff.cpp11.input.output]{\ref{input.output}: input/output library}

\diffref{c.files}
\change
\tcode{gets} is not defined.
\rationale
Use of \tcode{gets} is considered dangerous.
\effect
Valid \CppXI{} code that uses the \tcode{gets} function may fail to compile
in this revision of \Cpp{}.

\rSec1[diff.cpp03]{\Cpp{} and ISO \CppIII{}}

\rSec2[diff.cpp03.general]{General}

\pnum
\indextext{summary!compatibility with ISO \CppIII{}}%
Subclause \ref{diff.cpp03} lists the differences between \Cpp{} and
ISO \CppIII{} (ISO/IEC 14882:2003, \doccite{Programming Languages --- \Cpp{}}),
in addition to those listed above,
by the chapters of this document.

\rSec2[diff.cpp03.lex]{\ref{lex}: lexical conventions}

\diffref{lex.pptoken}
\change
New kinds of \grammarterm{string-literal}s.
\rationale
Required for new features.
\effect
Valid \CppIII{} code may fail to compile or produce different results in
this revision of \Cpp{}. Specifically, macros named \tcode{R}, \tcode{u8},
\tcode{u8R}, \tcode{u}, \tcode{uR}, \tcode{U}, \tcode{UR}, or \tcode{LR} will
not be expanded when adjacent to a \grammarterm{string-literal} but will be interpreted as
part of the \grammarterm{string-literal}.
For example:
\begin{codeblock}
#define u8 "abc"
const char* s = u8"def";        // Previously \tcode{"abcdef"}, now \tcode{"def"}
\end{codeblock}

\diffref{lex.pptoken}
\change
User-defined literal string support.
\rationale
Required for new features.
\effect
Valid \CppIII{} code may fail to compile or produce different results in
this revision of \Cpp{}.
For example:
\begin{codeblock}
#define _x "there"
"hello"_x           // \#1
\end{codeblock}

Previously, \#1 would have consisted of two separate preprocessing tokens and
the macro \tcode{_x} would have been expanded. In this revision of \Cpp{},
\#1 consists of a single preprocessing token, so the macro is not expanded.

\diffref{lex.key}
\change
New keywords.
\rationale
Required for new features.
\effect
Added to \tref{lex.key}, the following identifiers are new keywords:
\tcode{alignas},
\tcode{alignof},
\keyword{char16_t},
\keyword{char32_t},
\keyword{constexpr},
\keyword{decltype},
\keyword{noexcept},
\keyword{nullptr},
\keyword{static_assert},
and
\keyword{thread_local}.
Valid \CppIII{} code using these identifiers is invalid in this revision of \Cpp{}.

\diffref{lex.icon}
\change
Type of integer literals.
\rationale
C99 compatibility.
\effect
Certain integer literals larger than can be represented by \tcode{long} could
change from an unsigned integer type to \tcode{signed long long}.

\rSec2[diff.cpp03.expr]{\ref{expr}: expressions}

\diffref{conv.ptr}
\change
Only literals are integer null pointer constants.
\rationale
Removing surprising interactions with templates and constant
expressions.
\effect
Valid \CppIII{} code may fail to compile or produce different results in
this revision of \Cpp{}.
For example:
\begin{codeblock}
void f(void *);     // \#1
void f(...);        // \#2
template<int N> void g() {
  f(0*N);           // calls \#2; used to call \#1
}
\end{codeblock}

\diffref{expr.mul}
\change
Specify rounding for results of integer \tcode{/} and \tcode{\%}.
\rationale
Increase portability, C99 compatibility.
\effect
Valid \CppIII{} code that uses integer division rounds the result toward 0 or
toward negative infinity, whereas this revision of \Cpp{} always rounds
the result toward 0.

\diffref{expr.log.and}
\change
\tcode{\&\&} is valid in a \grammarterm{type-name}.
\rationale
Required for new features.
\effect
Valid \CppIII{} code may fail to compile or produce different results in
this revision of \Cpp{}.
For example:
\begin{codeblock}
bool b1 = new int && false;             // previously \tcode{false}, now ill-formed
struct S { operator int(); };
bool b2 = &S::operator int && false;    // previously \tcode{false}, now ill-formed
\end{codeblock}

\rSec2[diff.cpp03.dcl.dcl]{\ref{dcl.dcl}: declarations}

\diffref{dcl.spec}
\change
Remove \keyword{auto} as a storage class specifier.
\rationale
New feature.
\effect
Valid \CppIII{} code that uses the keyword \keyword{auto} as a storage class
specifier
may be invalid in this revision of \Cpp{}.
In this revision of \Cpp{},
\keyword{auto} indicates that the type of a variable is to be deduced
from its initializer expression.

\diffref{dcl.init.list}
\change
Narrowing restrictions in aggregate initializers.
\rationale
Catches bugs.
\effect
Valid \CppIII{} code may fail to compile in this revision of \Cpp{}.
For example:
\begin{codeblock}
int x[] = { 2.0 };
\end{codeblock}
This code is valid in \CppIII{} but invalid in this
revision of \Cpp{} because \tcode{double} to \tcode{int} is a narrowing
conversion.

\diffref{dcl.link}
\change
Names declared in an anonymous namespace
changed from external linkage to internal linkage;
language linkage applies to names with external linkage only.
\rationale
Alignment with user expectations.
\effect
Valid \CppIII{} code may violate the one-definition rule\iref{basic.def.odr}
in this revision of \Cpp{}.
For example:
\begin{codeblock}
namespace { extern "C" { extern int x; } }  // \#1, previously external linkage and C language linkage,
                                            // now internal linkage and \Cpp{} language linkage
namespace A { extern "C" int x = 42; }      // \#2, external linkage and C language linkage
int main(void) { return x; }
\end{codeblock}
This code is valid in \CppIII{},
but \tcode{\#2} is not a definition for \tcode{\#1}
in this revision of \Cpp{}, violating the one-definition rule.

\rSec2[diff.cpp03.class]{\ref{class}: classes}

\diffref{class.default.ctor,class.dtor,class.copy.ctor,class.copy.assign}
\change
Implicitly-declared special member functions are defined as deleted
when the implicit definition would have been ill-formed.
\rationale
Improves template argument deduction failure.
\effect
A valid \CppIII{} program that uses one of these special member functions in a
context where the definition is not required (e.g., in an expression that is
not potentially evaluated) becomes ill-formed.

\diffref{class.dtor}
\change
User-declared destructors have an implicit exception specification.
\rationale
Clarification of destructor requirements.
\effect
Valid \CppIII{} code may execute differently in this revision of \Cpp{}. In
particular, destructors that throw exceptions will call \tcode{std::terminate}
(without calling \tcode{std::unexpected}) if their exception specification is
non-throwing.

\rSec2[diff.cpp03.temp]{\ref{temp}: templates}

\diffref{temp.param}
\change
Repurpose \keyword{export} for modules\iref{module,cpp.module,cpp.import}.
\rationale
No implementation consensus for the \CppIII{} meaning of \keyword{export}.
\effect
A valid \CppIII{} program containing \tcode{export} is ill-formed in this
revision of \Cpp{}.

\diffref{temp.arg}
\change
Remove whitespace requirement for nested closing template right angle
brackets.
\rationale
Considered a persistent but minor annoyance. Template aliases
representing non-class types would exacerbate whitespace issues.
\effect
Change to semantics of well-defined expression. A valid \CppIII{} expression
containing a right angle bracket (``\tcode{>}'') followed immediately by
another right angle bracket may now be treated as closing two templates.
For example:
\begin{codeblock}
template <class T> struct X { };
template <int N> struct Y { };
X< Y< 1 >> 2 > > x;
\end{codeblock}
This code is valid in \CppIII{} because ``\tcode{>>}''
is a right-shift operator, but invalid in this revision of \Cpp{} because
``\tcode{>>}'' closes two templates.

\diffref{temp.dep.candidate}
\change
Allow dependent calls of functions with internal linkage.
\rationale
Overly constrained, simplify overload resolution rules.
\effect
A valid \CppIII{} program can get a different result in this
revision of \Cpp{}.

\rSec2[diff.cpp03.library]{\ref{library}: library introduction}

\pnum
\textbf{Affected:} \ref{library} -- \ref{\lastlibchapter}
\change
New reserved identifiers.
\rationale
Required by new features.
\effect
Valid \CppIII{} code that uses any identifiers added to the \Cpp{} standard
library by later revisions of \Cpp{} may fail to compile or produce different
results in this revision of \Cpp{}. A comprehensive list of identifiers used
by the \Cpp{} standard library can be found in the Index of Library Names in this
document.

\diffref{headers}
\change
New headers.
\rationale
New functionality.
\effect
The following \Cpp{} headers are new:
\libheaderref{array},
\libheaderrefx{atomic}{atomics.syn},
\libheaderrefx{chrono}{time.syn},
\libheaderrefx{condition_variable}{condition.variable.syn},
\libheaderrefx{forward_list}{forward.list.syn},
\libheaderref{future},
\libheaderrefx{initiali\-zer_list}{initializer.list.syn},
\libheaderref{mutex},
\libheaderrefx{random}{rand.synopsis},
\libheaderref{ratio},
\libheaderrefx{regex}{re.syn},
\libheaderrefx{scoped_allocator}{allocator.adaptor.syn},
\libheaderrefx{system_error}{system.error.syn},
\libheaderref{thread},
\libheaderref{tuple},
\libheaderrefx{type\-index}{type.index.synopsis},
\libheaderrefx{type_traits}{meta.type.synop},
\libheaderrefx{unordered_map}{unord.map.syn},
and
\libheaderrefx{unordered_set}{unord.set.syn}.
In addition the following C compatibility headers are new:
\libheaderref{cfenv},
\libheaderref{cinttypes},
\libheaderref{cstdint},
and
\libheaderref{cuchar}.
Valid \CppIII{} code that \tcode{\#include}{s} headers with these names may be
invalid in this revision of \Cpp{}.

\diffref{swappable.requirements}
\effect
Function \tcode{swap} moved to a different header
\rationale
Remove dependency on \libheaderref{algorithm} for \tcode{swap}.
\effect
Valid \CppIII{} code that has been compiled expecting swap to be in
\libheaderref{algorithm} may have to instead include \libheaderref{utility}.

\diffref{namespace.posix}
\change
New reserved namespace.
\rationale
New functionality.
\effect
The global namespace \tcode{posix} is now reserved for standardization. Valid
\CppIII{} code that uses a top-level namespace \tcode{posix} may be invalid in
this revision of \Cpp{}.

\diffref{res.on.macro.definitions}
\change
Additional restrictions on macro names.
\rationale
Avoid hard to diagnose or non-portable constructs.
\effect
Names of attribute identifiers may not be used as macro names. Valid \CppIII{}
code that defines \tcode{override}, \tcode{final},
\tcode{carries_dependency}, or \tcode{noreturn} as macros is invalid in this
revision of \Cpp{}.

\rSec2[diff.cpp03.language.support]{\ref{support}:
language support library}

\diffref{new.delete.single}
\change
\tcode{operator new} may throw exceptions other than
\tcode{std::bad_alloc}.
\rationale
Consistent application of \keyword{noexcept}.
\effect
Valid \CppIII{} code that assumes that global \tcode{operator new} only
throws \tcode{std::bad_alloc} may execute differently in this revision of \Cpp{}.
Valid \CppIII{} code that replaces the global replaceable \tcode{operator new}
is ill-formed in this revision of \Cpp{},
because the exception specification of \tcode{throw(std::bad_alloc)}
was removed.

\rSec2[diff.cpp03.diagnostics]{\ref{diagnostics}: diagnostics library}

\diffref{errno}
\change
Thread-local error numbers.
\rationale
Support for new thread facilities.
\effect
Valid but implementation-specific \CppIII{} code that relies on
\tcode{errno} being the same across threads may change behavior in this
revision of \Cpp{}.

\rSec2[diff.cpp03.utilities]{\ref{utilities}: general utilities library}

\diffref{refwrap,arithmetic.operations,comparisons,logical.operations,bitwise.operations}
\change
Standard function object types no longer derived from
\tcode{std::unary_function} or \tcode{std::binary_function}.
\rationale
Superseded by new feature; \tcode{unary_function} and
\tcode{binary_function} are no longer defined.
\effect
Valid \CppIII{} code that depends on function object types being derived from
\tcode{unary_function} or \tcode{binary_function} may fail to compile
in this revision of \Cpp{}.

\rSec2[diff.cpp03.strings]{\ref{strings}: strings library}

\diffref{string.classes}
\change
\tcode{basic_string} requirements no longer allow reference-counted
strings.
\rationale
Invalidation is subtly different with reference-counted strings.
This change regularizes behavior.
\effect
Valid \CppIII{} code may execute differently in this revision of \Cpp{}.

\diffref{string.require}
\change
Loosen \tcode{basic_string} invalidation rules.
\rationale
Allow small-string optimization.
\effect
Valid \CppIII{} code may execute differently in this revision of \Cpp{}.
Some \keyword{const} member functions, such as \tcode{data} and \tcode{c_str},
no longer invalidate iterators.

\rSec2[diff.cpp03.containers]{\ref{containers}: containers library}

\diffref{container.requirements}
\change
Complexity of \tcode{size()} member functions now constant.
\rationale
Lack of specification of complexity of \tcode{size()} resulted in
divergent implementations with inconsistent performance characteristics.
\effect
Some container implementations that conform to \CppIII{} may not conform to the
specified \tcode{size()} requirements in this revision of \Cpp{}. Adjusting
containers such as \tcode{std::list} to the stricter requirements may require
incompatible changes.

\diffref{container.requirements}
\change
Requirements change: relaxation.
\rationale
Clarification.
\effect
Valid \CppIII{} code that attempts to meet the specified container requirements
may now be over-specified. Code that attempted to be portable across containers
may need to be adjusted as follows:
\begin{itemize}
\item not all containers provide \tcode{size()}; use \tcode{empty()} instead
of \tcode{size() == 0};
\item not all containers are empty after construction (\tcode{array});
\item not all containers have constant complexity for \tcode{swap()} (\tcode{array}).
\end{itemize}

\diffref{container.requirements}
\change
Requirements change: default constructible.
\rationale
Clarification of container requirements.
\effect
Valid \CppIII{} code that attempts to explicitly instantiate a container using
a user-defined type with no default constructor may fail to compile.

\diffref{sequence.reqmts,associative.reqmts}
\change
Signature changes: from \keyword{void} return types.
\rationale
Old signature threw away useful information that may be expensive
to recalculate.
\effect
The following member functions have changed:
\begin{itemize}
\item \tcode{erase(iter)} for \tcode{set}, \tcode{multiset}, \tcode{map}, \tcode{multimap}
\item \tcode{erase(begin, end)} for \tcode{set}, \tcode{multiset}, \tcode{map}, \tcode{multimap}
\item \tcode{insert(pos, num, val)} for \tcode{vector}, \tcode{deque}, \tcode{list}, \tcode{forward_list}
\item \tcode{insert(pos, beg, end)} for \tcode{vector}, \tcode{deque}, \tcode{list}, \tcode{forward_list}
\end{itemize}

Valid \CppIII{} code that relies on these functions returning \keyword{void}
(e.g., code that creates a pointer to member function that points to one
of these functions) will fail to compile with this revision of \Cpp{}.

\diffref{sequence.reqmts,associative.reqmts}
\change
Signature changes: from \tcode{iterator} to \tcode{const_iterator}
parameters.
\rationale
Overspecification.
\effect
The signatures of the following member functions changed from taking an
\tcode{iterator} to taking a \tcode{const_iterator}:

\begin{itemize}
\item \tcode{insert(iter, val)} for \tcode{vector}, \tcode{deque}, \tcode{list},
\tcode{set}, \tcode{multiset}, \tcode{map}, \tcode{multimap}
\item \tcode{insert(pos, beg, end)} for \tcode{vector}, \tcode{deque}, \tcode{list},
\tcode{forward_list}
\item \tcode{erase(begin, end)} for \tcode{set}, \tcode{multiset}, \tcode{map}, \tcode{multimap}
\item all forms of \tcode{list::splice}
\item all forms of \tcode{list::merge}
\end{itemize}

Valid \CppIII{} code that uses these functions may fail to compile with this
revision of \Cpp{}.

\diffref{sequence.reqmts,associative.reqmts}
\change
Signature changes: \tcode{resize}.
\rationale
Performance, compatibility with move semantics.
\effect
For \tcode{vector}, \tcode{deque}, and \tcode{list}
the fill value passed to \tcode{resize} is now passed by reference instead of
by value, and an additional overload of \tcode{resize} has been added. Valid
\CppIII{} code that uses this function may fail to compile with this revision of \Cpp{}.

\rSec2[diff.cpp03.algorithms]{\ref{algorithms}: algorithms library}

\diffref{algorithms.general}
\change
Result state of inputs after application of some algorithms.
\rationale
Required by new feature.
\effect
A valid \CppIII{} program may detect that an object with a valid but
unspecified state has a different valid but unspecified state with this
revision of \Cpp{}. For example, \tcode{std::remove} and
\tcode{std::remove_if} may leave the tail of the input sequence with a
different set of values than previously.

\rSec2[diff.cpp03.numerics]{\ref{numerics}: numerics library}

\diffref{complex.numbers}
\change
Specified representation of complex numbers.
\rationale
Compatibility with C99.
\effect
Valid \CppIII{} code that uses implementation-specific knowledge about the
binary representation of the required template specializations of
\tcode{std::complex} may not be compatible with this revision of \Cpp{}.

\rSec2[diff.cpp03.locale]{\ref{localization}: localization library}

\diffref{facet.num.get.virtuals}
\change
The \tcode{num_get} facet recognizes hexadecimal floating point values.
\rationale
Required by new feature.
\effect
Valid \CppIII{} code may have different behavior in this revision of \Cpp{}.

\rSec2[diff.cpp03.input.output]{\ref{input.output}: input/output library}

\diffref{istream.sentry,ostream.sentry,iostate.flags}
\change
Specify use of \keyword{explicit} in existing boolean conversion functions.
\rationale
Clarify intentions, avoid workarounds.
\effect
Valid \CppIII{} code that relies on implicit boolean conversions will fail to
compile with this revision of \Cpp{}. Such conversions occur in the
following conditions:

\begin{itemize}
\item passing a value to a function that takes an argument of type \tcode{bool};
\item using \tcode{operator==} to compare to \tcode{false} or \tcode{true};
\item returning a value from a function with a return type of \tcode{bool};
\item initializing members of type \tcode{bool} via aggregate initialization;
\item initializing a \tcode{const bool\&} which would bind to a temporary object.
\end{itemize}

\diffref{ios.failure}
\change
Change base class of \tcode{std::ios_base::failure}.
\rationale
More detailed error messages.
\effect
\tcode{std::ios_base::failure} is no longer derived directly from
\tcode{std::exception}, but is now derived from \tcode{std::system_error},
which in turn is derived from \tcode{std::runtime_error}. Valid \CppIII{} code
that assumes that \tcode{std::ios_base::failure} is derived directly from
\tcode{std::exception} may execute differently in this revision of \Cpp{}.

\diffref{ios.base}
\change
Flag types in \tcode{std::ios_base} are now bitmasks with values
defined as constexpr static members.
\rationale
Required for new features.
\effect
Valid \CppIII{} code that relies on \tcode{std::ios_base} flag types being
represented as \tcode{std::bitset} or as an integer type may fail to compile
with this revision of \Cpp{}.
For example:
\begin{codeblock}
#include <iostream>

int main() {
  int flag = std::ios_base::hex;
  std::cout.setf(flag);         // error: \tcode{setf} does not take argument of type \tcode{int}
}
\end{codeblock}

\rSec1[diff.iso]{\Cpp{} and ISO C}

\rSec2[diff.iso.general]{General}

\pnum
\indextext{summary!compatibility with ISO C}%
Subclause \ref{diff.iso} lists the differences between \Cpp{} and ISO C,
in addition to those listed above,
by the chapters of this document.

\rSec2[diff.lex]{\ref{lex}: lexical conventions}

\diffref{lex.key}
\change
New Keywords\\
New keywords are added to \Cpp{};
see \ref{lex.key}.
\rationale
These keywords were added in order to implement the new
semantics of \Cpp{}.
\effect
Change to semantics of well-defined feature.
Any ISO C programs that used any of these keywords as identifiers
are not valid \Cpp{} programs.
\difficulty
Syntactic transformation.
Converting one specific program is easy.
Converting a large collection
of related programs takes more work.
\howwide
Common.

\diffref{lex.ccon}
\change
Type of \grammarterm{character-literal} is changed from \tcode{int} to \tcode{char}.
\rationale
This is needed for improved overloaded function argument type
matching. For example:
\begin{codeblock}
int function( int i );
int function( char c );

function( 'x' );
\end{codeblock}
It is preferable that this call match the second version of
function rather than the first.
\effect
Change to semantics of well-defined feature.
ISO C programs which depend on
\begin{codeblock}
sizeof('x') == sizeof(int)
\end{codeblock}
will not work the same as \Cpp{} programs.
\difficulty
Simple.
\howwide
Programs which depend upon \tcode{sizeof('x')} are probably rare.

\diffref{lex.string}
\change
Concatenated \grammarterm{string-literal}s can no longer have
conflicting \grammarterm{encoding-prefix}es.
\rationale
Removal of non-portable feature.
\effect
Concatenation of \grammarterm{string-literal}s
with different \grammarterm{encoding-prefix}es
is now ill-formed.
\difficulty
Syntactic transformation.
\howwide
Seldom.

\diffref{lex.string}
\change
String literals made const.\\
The type of a \grammarterm{string-literal} is changed
from ``array of \tcode{char}''
to ``array of \tcode{const char}''.
\indextext{UTF-8}%
The type of a UTF-8 string literal is changed
from ``array of \tcode{char}''
to ``array of \tcode{const char8_t}''.
\indextext{UTF-16}%
The type of a UTF-16 string literal is changed
from ``array of \textit{some-integer-type}''
to ``array of \tcode{const char16_t}''.
\indextext{UTF-32}%
The type of a UTF-32 string literal is changed
from ``array of \textit{some-integer-type}''
to ``array of \tcode{const char32_t}''.
The type of a wide string literal is changed
from ``array of \keyword{wchar_t}''
to ``array of \tcode{const wchar_t}''.
\rationale
This avoids calling an inappropriate overloaded function,
which might expect to be able to modify its argument.
\effect
Change to semantics of well-defined feature.
\difficulty
Syntactic transformation. The fix is to add a cast:
\begin{codeblock}
char* p = "abc";                // valid in C, invalid in \Cpp{}
void f(char*) {
  char* p = (char*)"abc";       // OK, cast added
  f(p);
  f((char*)"def");              // OK, cast added
}
\end{codeblock}
\howwide
Programs that have a legitimate reason to treat string literal objects
as potentially modifiable memory are probably rare.

\rSec2[diff.basic]{\ref{basic}: basics}

\diffref{basic.def}
\change
\Cpp{} does not have ``tentative definitions'' as in C.\\
E.g., at file scope,
\begin{codeblock}
int i;
int i;
\end{codeblock}
is valid in C, invalid in \Cpp{}.
This makes it impossible to define
mutually referential file-local objects with static storage duration,
if initializers are restricted to the syntactic forms of C\@.
For example,
\begin{codeblock}
struct X { int i; struct X* next; };

static struct X a;
static struct X b = { 0, &a };
static struct X a = { 1, &b };
\end{codeblock}
\rationale
This avoids having different initialization rules for
fundamental types and user-defined types.
\effect
Deletion of semantically well-defined feature.
\difficulty
Semantic transformation.
In \Cpp{}, the initializer for one of a set of
mutually-referential file-local objects with static storage
duration must invoke a function
call to achieve the initialization.
\howwide
Seldom.

\diffref{basic.scope}
\change
A \keyword{struct} is a scope in \Cpp{}, not in C.
For example,
\begin{codeblock}
struct X {
  struct Y { int a; } b;
};
struct Y c;
\end{codeblock}
is valid in C but not in \Cpp{}, which would require \tcode{X::Y c;}.
\rationale
Class scope is crucial to \Cpp{}, and a struct is a class.
\effect
Change to semantics of well-defined feature.
\difficulty
Semantic transformation.
\howwide
C programs use \keyword{struct} extremely frequently, but the
change is only noticeable when \keyword{struct}, enumeration, or enumerator
names are referred to outside the \keyword{struct}.
The latter is probably rare.

\diffref{basic.link} [also \ref{dcl.type}]
\change
A name of file scope that is explicitly declared \keyword{const}, and not explicitly
declared \keyword{extern}, has internal linkage, while in C it would have external linkage.
\rationale
Because const objects may be used as values during translation in
\Cpp{}, this feature urges programmers to provide an explicit initializer
for each const object.
This feature allows the user to put const objects in source files that are included
in more than one translation unit.
\effect
Change to semantics of well-defined feature.
\difficulty
Semantic transformation.
\howwide
Seldom.

\diffref{basic.start.main}
\change
The \tcode{main} function cannot be called recursively and cannot have its address taken.
\rationale
The \tcode{main} function may require special actions.
\effect
Deletion of semantically well-defined feature.
\difficulty
Trivial: create an intermediary function such as
\tcode{mymain(argc, argv)}.
\howwide
Seldom.

\diffref{basic.types}
\change
C allows ``compatible types'' in several places, \Cpp{} does not.\\
For example,
otherwise-identical \keyword{struct} types with different tag names
are ``compatible'' in C but are distinctly different types
in \Cpp{}.
\rationale
Stricter type checking is essential for \Cpp{}.
\effect
Deletion of semantically well-defined feature.
\difficulty
Semantic transformation.
The ``typesafe linkage'' mechanism will find many, but not all,
of such problems.
Those problems not found by typesafe linkage will continue to
function properly,
according to the ``layout compatibility rules'' of this
document.
\howwide
Common.

\rSec2[diff.expr]{\ref{expr}: expressions}

\diffref{conv.ptr}
\change
Converting \tcode{\keyword{void}*} to a pointer-to-object type requires casting.
\begin{codeblock}
char a[10];
void* b=a;
void foo() {
  char* c=b;
}
\end{codeblock}

ISO C accepts this usage of pointer to \keyword{void} being assigned
to a pointer to object type.
\Cpp{} does not.
\rationale
\Cpp{} tries harder than C to enforce compile-time type safety.
\effect
Deletion of semantically well-defined feature.
\difficulty
Can be automated.
Violations will be diagnosed by the \Cpp{} translator.
The
fix is to add a  cast.
For example:
\begin{codeblock}
char* c = (char*) b;
\end{codeblock}

\howwide
This is fairly widely used but it is good
programming practice to add the cast when assigning pointer-to-void to pointer-to-object.
Some ISO C translators will give a warning
if the cast is not used.

\diffref{expr.arith.conv}
\change
Operations mixing a value of an enumeration type and a value of a different
enumeration type or of a floating-point type are not valid.
For example:
\begin{codeblock}
enum E1 { e };
enum E2 { f };
int b = e <= 3.7;       // valid in C; ill-formed in \Cpp{}
int k = f - e;          // valid in C; ill-formed in \Cpp{}
int x = 1 ? e : f;      // valid in C; ill-formed in \Cpp{}
\end{codeblock}
\rationale
Reinforcing type safety in \Cpp{}.
\effect
Well-formed C code will not compile with this International Standard.
\difficulty
Violations will be diagnosed by the \Cpp{} translator.
The original behavior can be restored with a cast or integral promotion.
For example:
\begin{codeblock}
enum E1 { e };
enum E2 { f };
int b = (int)e <= 3.7;
int k = +f - e;
\end{codeblock}
\howwide
Uncommon.

\diffref{expr.post.incr,expr.pre.incr}
\change
Decrement operator is not allowed with \keyword{bool} operand.
\rationale
Feature with surprising semantics.
\effect
A valid ISO C expression utilizing the decrement operator on
a \keyword{bool} lvalue
(for instance, via the C typedef in \libheaderref{stdbool.h})
is ill-formed in \Cpp{}.

\diffref{expr.sizeof,expr.cast}
\change
In \Cpp{}, types can only be defined in declarations, not in expressions.\\
In C, a \keyword{sizeof} expression or cast expression may define a new type.
For example,
\begin{codeblock}
p = (void*)(struct x {int i;} *)0;
\end{codeblock}
defines a new type, struct \tcode{x}.
\rationale
This prohibition helps to clarify the location of
definitions in the source code.
\effect
Deletion of semantically well-defined feature.
\difficulty
Syntactic transformation.
\howwide
Seldom.

\diffref{expr.cond,expr.ass,expr.comma}
\indextext{conversion!lvalue-to-rvalue}%
\indextext{rvalue!lvalue conversion to}%
\indextext{lvalue}%
\change
The result of a conditional expression, an assignment expression, or a comma expression may be an lvalue.
\rationale
\Cpp{} is an object-oriented language, placing relatively
more emphasis on lvalues.  For example, function calls may
yield lvalues.
\effect
Change to semantics of well-defined feature.  Some C
expressions that implicitly rely on lvalue-to-rvalue
conversions will yield different results.
For example,
\begin{codeblock}
char arr[100];
sizeof(0, arr)
\end{codeblock}
yields
\tcode{100}
in \Cpp{} and
\tcode{sizeof(char*)}
in C.
\difficulty
Programs must add explicit casts to the appropriate rvalue.
\howwide
Rare.

\rSec2[diff.stat]{\ref{stmt.stmt}: statements}

\diffref{stmt.switch,stmt.goto}
\change
It is now invalid to jump past a declaration with explicit or implicit initializer (except across entire block not entered).
\rationale
Constructors used in initializers may allocate
resources which need to be de-allocated upon leaving the
block.
Allowing jump past initializers would require
complicated runtime determination of allocation.
Furthermore, many operations on such an uninitialized object
have undefined behavior.
With this simple compile-time rule, \Cpp{} assures that
if an initialized variable is in scope, then it has assuredly been
initialized.
\effect
Deletion of semantically well-defined feature.
\difficulty
Semantic transformation.
\howwide
Seldom.

\diffref{stmt.return}
\change
It is now invalid to return (explicitly or implicitly) from a function which is
declared to return a value without actually returning a value.
\rationale
The caller and callee may assume fairly elaborate
return-value mechanisms for the return of class objects.
If
some flow paths execute a return without specifying any value,
the implementation must embody many more complications.
Besides,
promising to return a value of a given type, and then not returning
such a value, has always been recognized to be a questionable
practice, tolerated only because very-old C had no distinction between
functions with \keyword{void} and \keyword{int} return types.
\effect
Deletion of semantically well-defined feature.
\difficulty
Semantic transformation.
Add an appropriate return value to the source code, such as zero.
\howwide
Seldom.
For several years, many existing C implementations have produced warnings in
this case.

\rSec2[diff.dcl]{\ref{dcl.dcl}: declarations}

\diffref{dcl.stc}
\change
In \Cpp{}, the \keyword{static} or \keyword{extern} specifiers can only be applied to names of objects or functions.\\
Using these specifiers with type declarations is illegal in \Cpp{}.
In C, these specifiers are ignored when used on type declarations.

Example:
\begin{codeblock}
static struct S {               // valid C, invalid in \Cpp{}
  int i;
};
\end{codeblock}

\rationale
Storage class specifiers don't have any meaning when associated
with a type.
In \Cpp{}, class members can be declared with the \keyword{static} storage
class specifier.
Storage class specifiers on type
declarations can be confusing for users.
\effect
Deletion of semantically well-defined feature.
\difficulty
Syntactic transformation.
\howwide
Seldom.

\diffref{dcl.stc}
\change
In \Cpp{}, \keyword{register} is not a storage class specifier.
\rationale
The storage class specifier had no effect in \Cpp{}.
\effect
Deletion of semantically well-defined feature.
\difficulty
Syntactic transformation.
\howwide
Common.

\diffref{dcl.typedef}
\change
A \Cpp{} \grammarterm{typedef-name} must be different from any class type name declared
in the same scope (except if the typedef is a synonym of the class name with the
same name). In C, a \grammarterm{typedef-name} and a struct tag name declared in the same scope
can have the same name (because they have different name spaces).

Example:
\begin{codeblock}
typedef struct name1 { @\commentellip@ } name1;         // valid C and \Cpp{}
struct name { @\commentellip@ };
typedef int name;               // valid C, invalid \Cpp{}
\end{codeblock}

\rationale
For ease of use, \Cpp{} doesn't require that a type name be prefixed
with the keywords \keyword{class}, \keyword{struct} or \keyword{union} when used in object
declarations or type casts.

Example:
\begin{codeblock}
class name { @\commentellip@ };
name i;                         // \tcode{i} has type \tcode{class name}
\end{codeblock}

\effect
Deletion of semantically well-defined feature.
\difficulty
Semantic transformation.
One of the 2 types has to be renamed.
\howwide
Seldom.

\diffref{dcl.type} [see also \ref{basic.link}]
\change
Const objects must be initialized in \Cpp{} but can be left uninitialized in C.
\rationale
A const object cannot be assigned to so it must be initialized
to hold a useful value.
\effect
Deletion of semantically well-defined feature.
\difficulty
Semantic transformation.
\howwide
Seldom.

\diffref{dcl.spec.auto}
\change
The keyword \keyword{auto} cannot be used as a storage class specifier.

Example:
\begin{codeblock}
void f() {
  auto int x;       // valid C, invalid \Cpp{}
}
\end{codeblock}

\rationale
Allowing the use of \keyword{auto} to deduce the type
of a variable from its initializer results in undesired interpretations of
\keyword{auto} as a storage class specifier in certain contexts.
\effect
Deletion of semantically well-defined feature.
\difficulty
Syntactic transformation.
\howwide
Rare.

\diffref{dcl.fct}
\change
In \Cpp{}, a function declared with an empty parameter list takes no arguments.
In C, an empty parameter list means that the number and type of the function arguments are unknown.

Example:
\begin{codeblock}
int f();            // means   \tcode{int f(void)} in \Cpp{}
                    // \tcode{int f(} unknown \tcode{)} in C
\end{codeblock}

\rationale
This is to avoid function calls
with the wrong number or type of arguments.
\effect
Change to semantics of well-defined feature.
This feature was marked as ``obsolescent'' in C.
\difficulty
Syntactic transformation.
The function declarations using C incomplete declaration style must
be completed to become full prototype declarations.
A program may need to be updated further if different calls to the
same (non-prototype) function have different numbers of arguments or
if the type of corresponding arguments differed.
\howwide
Common.

\diffref{dcl.fct} [see \ref{expr.sizeof}]
\change
In \Cpp{}, types may not be defined in return or parameter types.
In C, these type definitions are allowed.

Example:
\begin{codeblock}
void f( struct S { int a; } arg ) {}    // valid C, invalid \Cpp{}
enum E { A, B, C } f() {}               // valid C, invalid \Cpp{}
\end{codeblock}

\rationale
When comparing types in different translation units, \Cpp{} relies
on name equivalence when C relies on structural equivalence.
Regarding parameter types: since the type defined in a parameter list
would be in the scope of the function, the only legal calls in \Cpp{}
would be from within the function itself.
\effect
Deletion of semantically well-defined feature.
\difficulty
Semantic transformation.
The type definitions must be moved to file scope, or in header files.
\howwide
Seldom.
This style of type definition is seen as poor coding style.

\diffref{dcl.fct.def}
\change
In \Cpp{}, the syntax for function definition excludes the ``old-style'' C function.
In C, ``old-style'' syntax is allowed, but deprecated as ``obsolescent''.
\rationale
Prototypes are essential to type safety.
\effect
Deletion of semantically well-defined feature.
\difficulty
Syntactic transformation.
\howwide
Common in old programs, but already known to be obsolescent.

\diffref{dcl.init.aggr}
\change
In \Cpp{}, designated initialization support is restricted
compared to the corresponding functionality in C\@.
In \Cpp{},
designators for non-static data members
must be specified in declaration order,
designators for array elements and nested designators
are not supported,
and
designated and non-designated initializers
cannot be mixed in the same initializer list.

Example:
\begin{codeblock}
struct A { int x, y; };
struct B { struct A a; };
struct A a = {.y = 1, .x = 2};  // valid C, invalid \Cpp{}
int arr[3] = {[1] = 5};         // valid C, invalid \Cpp{}
struct B b = {.a.x = 0};        // valid C, invalid \Cpp{}
struct A c = {.x = 1, 2};       // valid C, invalid \Cpp{}
\end{codeblock}
\rationale
In \Cpp{}, members are destroyed in reverse construction order
and the elements of an initializer list are evaluated in lexical order,
so member initializers must be specified in order.
Array designators conflict with \grammarterm{lambda-expression} syntax.
Nested designators are seldom used.
\effect
Deletion of feature that is incompatible with \Cpp{}.
\difficulty
Syntactic transformation.
\howwide
Out-of-order initializers are common.
The other features are seldom used.

\diffref{dcl.init.string}
\change
In \Cpp{}, when initializing an array of character with a string, the number of
characters in the string (including the terminating \tcode{'\textbackslash 0'}) must not exceed the
number of elements in the array. In C, an array can be initialized with a string even if
the array is not large enough to contain the string-terminating \tcode{'\textbackslash 0'}.

Example:
\begin{codeblock}
char array[4] = "abcd";         // valid C, invalid \Cpp{}
\end{codeblock}
\rationale
When these non-terminated arrays are manipulated by standard
string functions, there is potential for major catastrophe.
\effect
Deletion of semantically well-defined feature.
\difficulty
Semantic transformation.
The arrays must be declared one element bigger to contain the
string terminating \tcode{'\textbackslash 0'}.
\howwide
Seldom.
This style of array initialization is seen as poor coding style.

\diffref{dcl.enum}
\change
\Cpp{} objects of enumeration type can only be assigned values of the same enumeration type.
In C, objects of enumeration type can be assigned values of any integral type.

Example:
\begin{codeblock}
enum color { red, blue, green };
enum color c = 1;               // valid C, invalid \Cpp{}
\end{codeblock}

\rationale
The type-safe nature of \Cpp{}.
\effect
Deletion of semantically well-defined feature.
\difficulty
Syntactic transformation.
(The type error produced by the assignment can be automatically
corrected by applying an explicit cast.)
\howwide
Common.

\diffref{dcl.enum}
\change
In \Cpp{}, the type of an enumerator is its enumeration. In C, the type of an enumerator is \keyword{int}.

Example:
\begin{codeblock}
enum e { A };
sizeof(A) == sizeof(int)        // in C
sizeof(A) == sizeof(e)          // in \Cpp{}
/* and @sizeof(int)@ is not necessarily equal to @sizeof(e)@ */
\end{codeblock}

\rationale
In \Cpp{}, an enumeration is a distinct type.
\effect
Change to semantics of well-defined feature.
\difficulty
Semantic transformation.
\howwide
Seldom.
The only time this affects existing C code is when the size of an
enumerator is taken.
Taking the size of an enumerator is not a
common C coding practice.

\diffref{dcl.align}
\change
In \Cpp{},
an \grammarterm{alignment-specifier} is an \grammarterm{attribute-specifier}.
In C, an \grammarterm{alignment-specifier} is a \gterm{declaration-specifier}.

Example:
\begin{codeblock}
#include <stdalign.h>
unsigned alignas(8) int x;      // valid C, invalid \Cpp{}
unsigned int y alignas(8);      // valid \Cpp{}, invalid C
\end{codeblock}
\rationale
\Cpp{} requires unambiguous placement of the \grammarterm{alignment-specifier}.
\effect
Deletion of semantically well-defined feature.
\difficulty
Syntactic transformation.
\howwide
Seldom.

\rSec2[diff.class]{\ref{class}: classes}

\diffref{class.name} [see also \ref{dcl.typedef}]
\change
In \Cpp{}, a class declaration introduces the class name into the scope where it is
declared and hides any object, function or other declaration of that name in an enclosing
scope. In C, an inner scope declaration of a struct tag name never hides the name of an
object or function in an outer scope.

Example:
\begin{codeblock}
int x[99];
void f() {
  struct x { int a; };
  sizeof(x);  /* size of the array in C */
  /* size of the struct in @\textit{\textrm{\Cpp{}}}@ */
}
\end{codeblock}
\rationale
This is one of the few incompatibilities between C and \Cpp{} that
can be attributed to the new \Cpp{} name space definition where a
name can be declared as a type and as a non-type in a single scope
causing the non-type name to hide the type name and requiring that
the keywords \keyword{class}, \keyword{struct}, \keyword{union} or \keyword{enum} be used to refer to the type name.
This new name space definition provides important notational
conveniences to \Cpp{} programmers and helps making the use of the
user-defined types as similar as possible to the use of fundamental
types.
The advantages of the new name space definition were judged to
outweigh by far the incompatibility with C described above.
\effect
Change to semantics of well-defined feature.
\difficulty
Semantic transformation.
If the hidden name that needs to be accessed is at global scope,
the \tcode{::} \Cpp{} operator can be used.
If the hidden name is at block scope, either the type or the struct
tag has to be renamed.
\howwide
Seldom.

\diffref{class.copy.ctor}
\change
Copying volatile objects.

The implicitly-declared copy constructor and
implicitly-declared copy assignment operator
cannot make a copy of a volatile lvalue.
For example, the following is valid in ISO C:
\begin{codeblock}
struct X { int i; };
volatile struct X x1 = {0};
struct X x2 = x1;               // invalid \Cpp{}
struct X x3;
x3 = x1;                        // also invalid \Cpp{}
\end{codeblock}

\rationale
Several alternatives were debated at length.
Changing the parameter to
\keyword{volatile}
\keyword{const}
\tcode{X\&}
would greatly complicate the generation of
efficient code for class objects.
Discussion of
providing two alternative signatures for these
implicitly-defined operations raised
unanswered concerns about creating
ambiguities and complicating
the rules that specify the formation of
these operators according to the bases and
members.
\effect
Deletion of semantically well-defined feature.
\difficulty
Semantic transformation.
If volatile semantics are required for the copy,
a user-declared constructor or assignment must
be provided.
If non-volatile semantics are required,
an explicit
\keyword{const_cast}
can be used.
\howwide
Seldom.

\diffref{class.bit}
\change
\indextext{bit-field!implementation-defined sign of}%
Bit-fields of type plain \keyword{int} are signed.
\rationale
The signedness needs to be consistent among template specializations.
For consistency,
the implementation freedom was eliminated for non-dependent types,
too.
\effect
The choice is implementation-defined in C, but not so in \Cpp{}.
\difficulty
Syntactic transformation.
\howwide
Seldom.

\diffref{class.nest}
\change
In \Cpp{}, the name of a nested class is local to its enclosing class. In C
the name of the nested class belongs to the same scope as the name of the outermost enclosing class.

Example:
\begin{codeblock}
struct X {
  struct Y { @\commentellip@ } y;
};
struct Y yy;                    // valid C, invalid \Cpp{}
\end{codeblock}
\rationale
\Cpp{} classes have member functions which require that classes
establish scopes.
The C rule would leave classes as an incomplete scope mechanism
which would prevent \Cpp{} programmers from maintaining locality
within a class.
A coherent set of scope rules for \Cpp{} based on the C rule would
be very complicated and \Cpp{} programmers would be unable to predict
reliably the meanings of nontrivial examples involving nested or
local functions.
\effect
Change to semantics of well-defined feature.
\difficulty
Semantic transformation.
To make the struct type name visible in the scope of the enclosing
struct, the struct tag can be declared in the scope of the
enclosing struct, before the enclosing struct is defined.
Example:
\begin{codeblock}
struct Y;                       // \tcode{struct Y} and \tcode{struct X} are at the same scope
struct X {
  struct Y { @\commentellip@ } y;
};
\end{codeblock}

All the definitions of C struct types enclosed in other struct
definitions and accessed outside the scope of the enclosing
struct can be exported to the scope of the enclosing struct.
Note: this is a consequence of the difference in scope rules,
which is documented in \ref{basic.scope}.
\howwide
Seldom.

\diffref{class.member.lookup}
\change
In \Cpp{}, a \grammarterm{typedef-name} may not be redeclared in a class definition after being used in that definition.

Example:
\begin{codeblock}
typedef int I;
struct S {
  I i;
  int I;            // valid C, invalid \Cpp{}
};
\end{codeblock}
\rationale
When classes become complicated, allowing such a redefinition
after the type has been used can create confusion for \Cpp{}
programmers as to what the meaning of \tcode{I} really is.
\effect
Deletion of semantically well-defined feature.
\difficulty
Semantic transformation.
Either the type or the struct member has to be renamed.
\howwide
Seldom.

\rSec2[diff.cpp]{\ref{cpp}: preprocessing directives}

\diffref{cpp.predefined}
\change
Whether \mname{STDC} is defined and if so, what its value is, are
\impldef{definition and meaning of \mname{STDC}}.
\rationale
\Cpp{} is not identical to ISO C\@.
Mandating that \mname{STDC}
be defined would require that translators make an incorrect claim.
\effect
Change to semantics of well-defined feature.
\difficulty
Semantic transformation.
\howwide
Programs and headers that reference \mname{STDC} are
quite common.

%%%%%%%%%%%%%%%%%%%%%%%%%%%%%%%%%%%%%%%
\rSec1[diff.library]{C standard library}

\rSec2[diff.library.general]{General}
\indextext{library!C standard}%

\pnum
Subclause \ref{diff.library} summarizes the explicit changes in headers,
definitions, declarations, or behavior between the C standard library
in the C standard and the parts of the \Cpp{} standard library that were
included from the C standard library.

\rSec2[diff.mods.to.headers]{Modifications to headers}

\pnum
For compatibility with the C standard library\indextext{library!C standard},
the \Cpp{} standard library provides the C headers enumerated
in~\ref{support.c.headers}.

\pnum
There are no \Cpp{} headers for the C standard library's headers
\libnoheader{stdnoreturn.h} and \libnoheader{threads.h},
nor are these headers from the C standard library headers themselves part of \Cpp{}.

\pnum
The C headers \libheader{complex.h} and
\libheader{tgmath.h} do not contain any of the content from
the C standard library and instead merely include other headers from the \Cpp{}
standard library.

\rSec2[diff.mods.to.definitions]{Modifications to definitions}

\rSec3[diff.char16]{Types \tcode{char16_t} and \tcode{char32_t}}

\pnum
The types \keyword{char16_t} and \keyword{char32_t}
are distinct types rather than typedefs to existing integral types.
The tokens \keyword{char16_t} and \keyword{char32_t}
are keywords in \Cpp{}\iref{lex.key}.
They do not appear as macro or type names defined in
\libheaderref{cuchar}.

\rSec3[diff.wchar.t]{Type \tcode{wchar_t}}

\pnum
The type \keyword{wchar_t} is a distinct type rather than a typedef to an
existing integral type.
The token \keyword{wchar_t}
is a keyword in \Cpp{}\iref{lex.key}.
It does not appear as a macro or type name defined in any of
\libheaderref{cstddef},
\libheaderref{cstdlib},
or \libheaderref{cwchar}.

\rSec3[diff.header.assert.h]{Header \tcode{<assert.h>}}
\indexhdr{assert.h}%

\pnum
The token \keyword{static_assert} is a keyword in \Cpp{}.
It does not appear as a macro name defined
in \libheaderref{cassert}.

\rSec3[diff.header.iso646.h]{Header \tcode{<iso646.h>}}

\pnum
The tokens
\keyword{and},
\keyword{and_eq},
\keyword{bitand},
\keyword{bitor},
\keyword{compl},
\keyword{not},
\keyword{not_eq},
\keyword{or},
\keyword{or_eq},
\keyword{xor},
and
\keyword{xor_eq}
are keywords in \Cpp{}\iref{lex.key},
and are not introduced as macros
by \libheaderref{iso646.h}.

\rSec3[diff.header.stdalign.h]{Header \tcode{<stdalign.h>}}
\indexhdr{stdalign.h}%

\pnum
The token \keyword{alignas} is a keyword in \Cpp{}\iref{lex.key},
and is not introduced as a macro
by \libheaderref{stdalign.h}.

\rSec3[diff.header.stdbool.h]{Header \tcode{<stdbool.h>}}
\indexhdr{stdbool.h}%

\pnum
The tokens \keyword{bool}, \keyword{true}, and \keyword{false}
are keywords in \Cpp{}\iref{lex.key},
and are not introduced as macros
by \libheaderref{stdbool.h}.

\rSec3[diff.null]{Macro \tcode{NULL}}

\pnum
The macro
\tcode{NULL},
defined in any of
\libheaderref{clocale},
\libheaderref{cstddef},
\libheaderref{cstdio},
\libheaderref{cstdlib},
\libheaderref{cstring},
\libheaderref{ctime},
or \libheaderref{cwchar},
is an \impldef{definition of \tcode{NULL}} null pointer constant in
\Cpp{}\iref{support.types}.

\rSec2[diff.mods.to.declarations]{Modifications to declarations}

\pnum
Header \libheaderref{cstring}:
The following functions have different declarations:

\begin{itemize}
\item \tcode{strchr}
\item \tcode{strpbrk}
\item \tcode{strrchr}
\item \tcode{strstr}
\item \tcode{memchr}
\end{itemize}

Subclause \ref{cstring.syn} describes the changes.

\pnum
Header \libheaderref{cwchar}:
The following functions have different declarations:

\begin{itemize}
\item \tcode{wcschr}
\item \tcode{wcspbrk}
\item \tcode{wcsrchr}
\item \tcode{wcsstr}
\item \tcode{wmemchr}
\end{itemize}

Subclause \ref{cwchar.syn} describes the changes.

\pnum
Header \libheaderref{cstddef}
declares the names \tcode{nullptr_t}, \tcode{byte}, and \tcode{to_integer},
and the operators and operator templates in \iref{support.types.byteops},
in addition to the names declared in
\libheaderrefx{stddef.h}{support.c.headers} in the C standard library.

\rSec2[diff.mods.to.behavior]{Modifications to behavior}

\rSec3[diff.mods.to.behavior.general]{General}

\pnum
Header \libheaderref{cstdlib}:
The following functions have different behavior:

\begin{itemize}
\item \tcode{atexit}
\item \tcode{exit}
\item \tcode{abort}
\end{itemize}

Subclause \ref{support.start.term} describes the changes.

\pnum
Header \libheaderref{csetjmp}:
The following functions have different behavior:
\begin{itemize}
\item \tcode{longjmp}
\end{itemize}

Subclause \ref{csetjmp.syn} describes the changes.

\rSec3[diff.offsetof]{Macro \tcode{offsetof(\placeholder{type}, \placeholder{member-designator})}}
\indexlibraryglobal{offsetof}%

\pnum
The macro \tcode{offsetof}, defined in
\libheaderref{cstddef},
accepts a restricted set of \tcode{\placeholder{type}} arguments in \Cpp{}.
Subclause \ref{support.types.layout} describes the change.

\rSec3[diff.malloc]{Memory allocation functions}

\pnum
The functions
\indexlibraryglobal{aligned_alloc}\tcode{aligned_alloc},
\indexlibraryglobal{calloc}\tcode{calloc},
\indexlibraryglobal{malloc}\tcode{malloc},
and
\indexlibraryglobal{realloc}\tcode{realloc}
are restricted in \Cpp{}.
Subclause \ref{c.malloc} describes the changes.
